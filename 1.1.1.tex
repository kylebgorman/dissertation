\subsection{Structuralism}

Since, with the possible exception of Roman Jakobson (who might be labeled an early generativist), the mentalistic interpretation of structural linguistic analyses was anathema, one must take care to not assume that a description of MSCs is something that these linguists would posit as part of speakers' knowledge of language.

\subsubsection{Morpheme boundaries in phonology}

bloomfield

%While \citeauthor{Bloomfield1930} is known for introducing morphemic structure into phonology, a member of the Prague circle developed an arguably more sophisticated approach only two years later. 

\citet{Jakobson1932} attributes a number of phonological properites of the Russian imperative to morpheme juncture. The vast majority of Russian infinitives consist of the stem followed by a theme vowel and /t\pal/. The reflexive is marked wih a final /-sə/ which feeds a general process of regressive voicing assimilation. In the reflexive infinitive, /\ldots t\pal-s\ldots/ is realized [ts], with no palatalization. However, palatalization of a stem-final labial or coronal---including /t\pal/, as in (\nextx c)---is preserved in the reflexive imperative.

\ex Russian reflexives \citep[after][]{Jakobson1932}:\footnote{
I have taken a number of liberties with \citeauthor{Jakobson1932}'s presentation of the data, which uses an abstract phonemic transcription. 
The broad transcription below corresponds to the speech of Lev Blumenfeld, who I thank for help with this data.
} \\
\begin{tabular}{l l l l} %\toprule
   &  infinitive      & imperative \\ %\midrule
a. & [slav\pal itsə]  & [slaf\pal s\pal ə]  & `be glorious'    \\
   & [upram\pal itsə] & [upram\pal s\pal ə] & `be stubborn'    \\
b. & [kras\pal itsə]  & [kras\pal s\pal ə]  & `put on makeup'  \\
   & [ʒar\pal itsə]   & [ʒar\pal s\pal ə]   & `roast'          \\
c. & [zəbytsə]        & [zəbut\pal s\pal ə] & `forget' \\ %\bottomrule
%   & ab\'utsa      & ab\'ujsa     & `put on shoes'   \\
\end{tabular} \xe

\noindent
The forms in (\last c) have different outcomes for their /t\pal-s/ clusters. \citeauthor{Jakobson1932} proposes that the preservation of palatalization is a special property of the imperative. %A few years later, \citet{Trnka1936}, another member of the Prague Circle, makes the connection between \citeauthor{Jakobson1932}'s hypothesis and phonotactic generalizations.

\subsubsection{The morph as a constraint domain}

A corrolary of this hypothesis is that the phoneme sequences found within morphs may be distinct from those which span multiple morphs. 

This was apparent to structuralists 

the phonologists 
This is apparent from 
a discussion of constraints on vowel sequences in Mixteco given by \citet{Pike1947b}.
It is apparent that \citeauthor{Pike1947b}'s constraints on vowel sequences are not intended to hold across morph boundaries, 

according to the phonological analysis of a Mixteco text published a few years earlier \citep{Pike1944}.

\ex Mixteco MSCs \citep{Pike1947b} and complex words \citep{Pike1944}: \\
\begin{tabular}{r l l l} %\toprule
   & MSC & complex exception \\ %\midrule
%a. & *{C}a{C}e & [k\'a-\textsuperscript{n}dee] & `kept \ldots inside' \\
%b. & *{C}\textipa{@}{C}e & [n\`i-k\`ə bə-de] & `who entered'        \\
%c. & *{C}e{C}i & [te-n\'i-ke\textsuperscript{n}da] & `was walking         \\
%d. & *{C}i{C}e & [te-n\`i-kee-t\`ə] & `and went away'      \\ %\toprule
%e. & *{C}e{C}o & b\'e\textglotstopvari e-\v{z}\'o & `our house'          \\
%f. & *{C}eo & ke-o-d\'e & `we eat him'         \\
a. & *{C}a{C}e & [ká\textsuperscript{n}dee] & `kept \ldots inside' \\
b. & *{C}ə{C}e & [nìk\`əbəde] & `who entered'        \\
c. & *{C}e{C}i & [teníke\textsuperscript{n}da] & `was walking         \\
d. & *{C}i{C}e & [tenìkeet\`ə] & `and went away'      \\ 
\end{tabular} \xe

\noindent
\citet[][166]{Pike1947b} affirms that the morpheme is ``marked'' by the violation of morpheme-internal sequence restrictions, an idea further developed by \citet{Harris1955} as a morpheme discovery routine.

\subsubsection{Biuniqueness and neutralization}

joos
harris
chomsky
