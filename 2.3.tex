\section{Conclusions}

It is not the case that this evidence is inconsistent with the hypothesis that there are additional degrees of wordlikeness: it is just a demonstration that these data to not require it. 

\subsection{Future directions}

Imagine a run of a wordlikeness experiment in which a subject rates the nonce word \emph{dresp} higher than the nonce word \emph{srest}. This observation should indicates the need for many more.

Did the subject faithfully perceive the unattested [sr] onset of \emph{srest}, or was it perceived as, e.g., [ʃr]? Does the subject have intuitions about how to ``correct'' \emph{srest} as to make it more English-like?\footnote{This appears to be a prediction of constraint-based theories of phonotactic grammaticality. Thanks to Colin Wilson (p.c.) for some discussion on this point.} Does the subject rate the relative wordlikeness of \emph{dresp} and \emph{srest} the same a day, a week, a month, a year later? 
%Are relative judgements ``commutative'' in the sense of \citep{Sprouse2011}, so that if another word is \emph
Do other subjects agree with this relative judgement? And, if not, does this contrast correlate with other subject-specific properties such as age, level of education, exposure to foreign languages, and so on? 
