% 2.3: Conclusion

\citet{Borowsky1989} on peripherality.

productive \citet{Duanmu2008}

The above is a stark reminder.


A reasonable objection to the arguments presented in this chapter is to view the results as little more than an indictment of the results of \citealt{Pierrehumbert1994}.

However, the fact that state-of-the-art models are not capable of providing large improvements to the predictive accuracy indicates


 do not have a statistically significant effect on the shape of the English lexicon, but that experiments might turn up evidence that speakers have internalized 


shown to be aware of static constraints if they reached statistical significance. 



 statistically reliable static constraints could be identified 

The of



Regarding the historical developments,

\citet{Martin2007}

On the other hand, patterns created by sound change are not guaranteed to persist over time. 
One example of non-persistence is discussed by \citet{Iverson2005}.  
Around 1100 CE, Old English \emph{sk} became [ʃ]. 
This sound change introduced no alternations.
Since long vowels were not found before tautosyllabic syllable clusters at this time, there were no \emph{V\lm sk\#} words when the
 change was actuated, and \emph{V\lm sh\#} continues to be rare in Modern English. 
What \citeauthor{Iverson2005} observe, however, is that there is nothing apparently peripheral about words like \emph{leash} or \emph{whoosh}, and loanwords and coinages have readily filled the gap.

A third pattern is that a historically inherited pattern

\citet[][140]{Frisch2004} suggest that the strong tendency for the first and second consonants of the Arabic root to be non-identic
al is the ``a diachronic result of a processing constraint that disfavors repetition.'' 
Unfortunately, there is no evidence that this pattern is diachronic other than in the sense that it appears to be inherited from the proto-language: there is simply no Proto-Semitic verb roots with identical first and second consonants \citep[][178]{Greenberg1950}. 
In other Semitic languages, the inherited patern has experienced considerable erosion. 

\ex Tigrinya roots with identical first and second consonants \citep{Buckley1990a}: \\
\begin{tabular}{l l l l}
a. & lʌlʌw     & `scorch'                   & (< Ge'ez \emph{lʌwlʌw} `inflame')     \\
   & mʌmʌy     & `winnow'                   & (< Ge'ez \emph{mʌymʌy} `distinguish') \\
%   & mʌmʌt & `pick out loot' & (< 
b. & s’ʌs’ʌw   & `finish off a drink'       & (cf. \emph{s’ʌws’ʌw} `gulp down')           \\
   & t’ʌt’ʌf   & `prune tree'               & (cf. \emph{t’ʌft’ʌf} `smear wall with mud') \\
c. & kʷakʷkʷʌr & `waste away, be emaciated' & (cf. \emph{kʷarkʷʌr} `interrogate')         \\
   & kakʷkʷɨʕ  & `clean wax from ears'      & (cf. \emph{kaʕkʷɨʕ} `start to form pods')   \\
\end{tabular}
\xe

Similar exceptions are found in 
%Amharic (\citealp[][?]{Broselow1984}, \citealp[][?]{McCarthy1985}) and 
Hebrew \citep[][29]{Bat-El2005}.

The next two chapters return to the question of synchrony, addressing the relationship between statistical patterns in the lexicon and speakers' behaviors when presented with underrepresented sequences.
