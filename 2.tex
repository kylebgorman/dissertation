% 2: Accidental and structural gaps in English syllable contact

In the preceding chapter, it was proposed that 

Reconsider the null hypothesis presented in the preceding chapter:

\ex The null hypothesis: \\
Universal grammar does not countenance constraints on the contents of URs
\xe

The idea 

relationship to proposal

One primary way which this null hypothesis has been critiqued is by demonstrating that the lexicon of some language 
exhibits dispreferences or gaps corresponding to.

argue by trends in the lexicon

possibility of sparsity (C\&H, Fischer/Vogt and recent stats)

Of course, to obtain the desired results, we must guarantee that each sequence structure rule reflects a general systematic fact about the language, and not a fact which is due merely to the existence of accidental gaps in the lexicon. \citep[][401, fn.~8]{Stanley1967}

In the preceding chapter, it was proposed that 

Lexicon

This appears to be exceptionlessly true of Arabic

That even gradient patterns 

One of the first study of this type was an analysis of co-occurrence restrictions on consonants in Javanese roots by \citet{Mester1988}, recently revisited by \citet{Graff2011}. 
Another Austronesian language studied in this fashion is Muna \citep{Coetzee2008a,Anttila2008}.
This technique has been applied to consonants in Semitic, especially Arabic \citep{McCarthy1988,McCarthy1994,Pierrehumbert1993,Frisch1996,Frisch2004,Coetzee2008a} but also in Tigrinya \citep{Buckley1997}, Hebrew \citep{Berent2003}, and Amharic \citep{Colavin2010}, and to various co-occurrence restrictions in English \citep{Berkley1994b,Berkley1994a,Pierrehumbert1994,Dmitrieva2008a,Dmitrieva2008b,Coetzee2008b}, Russian \citep{Padgett1992}, Dutch \citep{Graff2011}, Navajo \citep{Martin2007,Martin2011} and Gitksan \citep{Brown2010}.

The absence of 
The claim here is that however stark the absence or underrepresentation of some form is, it 

This hypothesis admits the possibility that there may be accidental gaps in the lexicon, a possibility which predates generative thinking:

\begin{quotation}
\ldots the fact that some [clusters--KG] are not found must be due to accidental gaps in the inventory of signs, and cannot be explained by structural laws of the language. \citep[][16]{Fischer-Jorgensen1952}
\end{quotation}

\noindent
Hans \citeauthor{Vogt1954} makes a similar observation in his study of Georgian clusters

\begin{quotation}
Although my material is drawn from a fairly extensive corpus---all accessible dictionaries and vocabularies, printed texts of tens of thousands of pages as well as ordinary speech---there is every reason to believe, as experience has shown, that additional material would yield new clusters. The material will never be complete. It will always contain accidental gaps \ldots partly because some clusters by pure chance do not occur in the vocabulary. \citep[][30]{Vogt1954}
\end{quotation}

\citet{Chomsky1965}

This chaper demonstrates that apparently systematic gaps may arise even in the absence of phonotactic preferences, and that this possibility compromises attempts to show that such preferences shape the lexicon. 
