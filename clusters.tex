
ontogeny reveals ontology

\chapter{Structural and accidental phonotactic gaps in the lexicon}

Some (admittedly similar) 

Some evidence that this is active comes from a study by \citet[]{Wright1975}. \citeauthor{Wright1975} taught several members of a ``neighborhood gang'' a few words from an artificial language, and then asked them to repeat the words to each other in a game of ``telephone''. Crucially

A particularly 

\ex Artificial language repetitions \citep[from][]{Wright1975}: \\
\begin{tabular}{l l l l l l} %\toprule
%a. & [go\textupsilon np] & $>$ & [gu\lm mp] \\
%[g\textturnv np]  
%b. & [\dh \textturnv\textipa{N}d] & $>$ & [\dh \textopeno\textipa{N}g\textschwa] \\ % $>$ [t\textopeno\textipa{N}g\textschwa] 
%c. & [\textdyoghlig umg] & $>$ & [\textdyoghlig u\textipa{N}g\textschwa] \\
\end{tabular} \xe

\noindent
While a number of other adaptations there 
[np] becomes [mp], 


%[\textipa{N}d] becomes [nd], and [mg] becomes [\textipa{N}g].

[ɳd]

t
If this task is viewed as something like loanword adaptation, then there are certainly good reasons to think that extragrammatical pressures \citep{Dupoux1999,Ussishkin2003,Peperkamp2008} could be responsible for what is observed. That said, it is certainly noteable that 

\footnote{In this latter view, it is interesting that the (\lastx c) final nasal-[g] cluster is preserved. In standard }

\citep{Schleef2011}

%[\textipa{N}g] is certainly 

though it can be found in the speech fo 

There is good phonological eivd

In this latter view, it is interesting that the finIn this latter view, it is interesting that the finIn this latter view, it is interesting that the final [g] is preserved.

As \citeauthor{Wright1975} notes, it is particularly interesting that 

It is particularly interesting that 
%included a stimulus with [\textipa{N}] as a control, since


There is a lengthy literature arguing for a phonotactic component to speakers' phonological knowledge. The fundamental argument pursued in this chapter is that many of these studies are misguided, and fail to take into account a powerful null hypothesis.

The \emph{lexicon} has many senses: throughout, I use the term to refer to the set of underlying representations \citep[][269]{LANGUAGE}, but I depart from \citeauthor{LANGUAGE}'s understanding by further adopting the assumption that this set is merely a reflection of individual speakers' knowledge of their language, realized in the human; it has no further reality.

\section{Preliminaries}

\subsection{Some definitions}

\subsection{The null hypothesis}

\subsubsection{History of the null}

\subsection{Structuralist phonology}

A major result of the American structuralists was the finding that generalizations which hold of the sound sequences found in underlying forms may not hold across morpheme boundaries. If this is correct, then there must be some sense in which phonology is sensitive to the presence of morpheme boundaries. In his history of phonology, \citet[][267]{Anderson1985} credits this innovation to \citet{Bloomfield1930}, who proposes that a problem in the phonemic analysis of German can be resolved by allowing for allophony to make reference to morpheme boundaries. \citeauthor{Bloomfield1930} notes that [\c{c}] and [x] can be cast as allophones, despite the existence of apparent minimal pairs like \emph{Kuchen} [ku\lm x\textschwa n] `cake' $\sim$ \emph{Kuhchen} [ku\lm\c{c}\textschwa n] `cow-let'. \citeauthor{Bloomfield1930}'s claim is that these words differ not just in their medial consonsant but also that the latter is a compound, marked with the diminutive \emph{-chen} (cf. \emph{Kuh} `cow'). Under the assumption that [x] occurs in this position (before a back vowel) only when this preceding vowel belongs to the same word or morpheme, [\c{c}] and [x] can be cast as allophones of the same phoneme (likely /\c{c}/, the basic variant). 

A natural consequence of this view is that generalizations about sound sequences found within morphemes are necessarily different within and across morpheme boundaries, and \citet{Pike1947b} argues that the distributional analysis of sound sequences must take morphology into account. \citeauthor{Pike1947b} discusses limitations on possible vowel sequences in Mixteco monomorphs, but none of these generalizations hold across morpheme boundaries, at least according to the phonological analysis of a short Mixteco text in one of \citeauthor{Pike1944}'s earlier papers.

\ex Mixteco MSCs \citep{Pike1947b} and complex words \citep{Pike1944}: \\ 
\begin{tabular}{r l l l} %\toprule
   & MSC & complex exception \\ %\midrule
%a. & *{C}a{C}e & [k\'a-\textsuperscript{n}dee] & `kept \ldots inside' \\
%b. & *{C}\textipa{@}{C}e & [n\`i-k\`\textschwa b\textschwa-de] & `who entered'        \\
%c. & *{C}e{C}i & [te-n\'i-ke\textsuperscript{n}da] & `was walking         \\
%d. & *{C}i{C}e & [te-n\`i-kee-t\`\textschwa] & `and went away'      \\ %\toprule
a. & *{C}a{C}e & [k\'a\textsuperscript{n}dee] & `kept \ldots inside' \\
b. & *{C}\textipa{@}{C}e & [n\`ik\`\textschwa b\textschwa de] & `who entered'        \\
c. & *{C}e{C}i & [ten\'i ke\textsuperscript{n}da] & `was walking         \\
d. & *{C}i{C}e & [ten\`i keet\`\textschwa] & `and went away'      \\ %\toprule
%e. & *{C}e{C}o & b\'e\textglotstopvari e-\v{z}\'o & `our house'          \\
%f. & *{C}eo & ke-o-d\'e & `we eat him'         \\
\end{tabular} \xe

\noindent
\citet[][166]{Pike1947b} affirms that the morpheme is ``marked'' by the violation of morpheme-internal sequence restrictions, an idea further developed by \citet{Harris1955} as a morpheme discovery routine.

While \citeauthor{Bloomfield1930} is known for introducing morphemic structure into phonology, a member of the Prague circle developed an arguably more sophisticated approach only two years later. \citet{Jakobson1932} attributes a number of phonological properites of the Russian imperative to morpheme juncture. For instance, most Russian infinitives end with a final /t'/. The reflexive variant of these verbs is marked with a final /-sa/, which feeds a general process of regressive voicing assimilation, and this /\ldots t\textceltpal -s\ldots/ juncture triggers the loss of palatalization. However, palatalization of a stem-final labial or coronal---including /t'/, as in (\nextx c)---is preserved in reflexive imperatives, which are formed by attaching the reflexive /-sa/ to the bare stem.\footnote{I have taken a number of liberties with \citeauthor{Jakobson1932}'s presentation of the data, which uses a highly abstract phonemic transcription. The broad phonetic transcriptions below are the result of consultations with N linguistically na\"ive native Russian speakers. Palatalization is marked even where not constrastive.}

\ex Russian reflexives \citep[after][]{Jakobson1932}: \\
\begin{tabular}{l l l l} %\toprule
   &  infinitive & imperative \\ %\midrule
a. & [slav\textceltpal its\textschwa]         & [slaf\textceltpal s\textschwa]            & `be glorious'    \\
   & [pramits\textschwa]         & [upram\textceltpal s\textschwa]          & `be stubborn'    \\
b. & [kras\textceltpal its\textschwa]         & [kras\textceltpal s\textschwa]           & `put on makeup'  \\
   & [\textyogh arits\textschwa]       & [\textyogh ar\textceltpal s\textschwa]         & `roast'          \\ 
c. & [z\textschwa byts\textschwa] & [z\textschwa but\textceltpal s\textschwa] & `forget' \\ %\bottomrule
%   & ab\'utsa      & ab\'ujsa     & `put on shoes'   \\
\end{tabular} \xe

\noindent

Both forms in (\lastx c) contain a /t\textceltpal-s/ juncture, though only one is faithful. \citeauthor{Jakobson1932} proposes that the phonology treats them differently: the imperative does not undergo loss of palatalization. A few years later, \citet{Trnka1936}, another member of the Prague Circle, makes the connection between \citeauthor{Jakobson1932}'s hypothesis and phonotactic generalizations.

Arguably, this more powerful system might be appropriate for \citeauthor{Bloomfield1930}'s data, if it is the case that the final \emph{-en} in \emph{Kuchen} is also a morpheme (perhaps related to \emph{K\"uche} `kitchen; cuisine'). There are obvious parallels between \citeauthor{Jakobson1932}'s analysis and notions like readjustment in \emph{SPE}, or cyclic rules in lexical phonology or the \citet{Halle1987} model. 

Further, I promise that the only compelling reason to reconsider this view is no longer of force, and 

As reviewed below, this was the view of many phonologists of the 1960s and 1970s. While several classic documents indicated problems with this view, I will argue that the broader view of 

It is worthwhile to consider the appraisal offered by some key textbooks of the era.

\citet{Dell1973}

\citet{Anderson1974}

\citet{Kenstowicz1979}

phonological

At the same time, the hypothesis seems to have been ignored by work in the late 1980s and early 1990s, and that 

and no compelling reasons have been offered to revise it. If anything, it appears that this hypothesis was forgotten or ignored during the 1980s and early '90s, and that despite the fact that it was championed by the founding Optimality Theory


 founding document forgotten or ignored during the 1980s and early '90s, and that despite the fact that it was championed by the founding document forgotten or ignored during the 1980s and early '90s, and that despite the fact that it was championed by the founding document forgotten or ignored during the 1980s and early '90s, and that despite the fact that it was championed by the founding document 

 work was forgotten during the 1980s, briefly revived by the the earliset 

\subsection{Generative phonology}


\subsection{Early psycholinguistic evidence}

Linguists of this era were not ignorant of the possibilities of obtaining experimental evidence bearing on MSCs, and the experiments by \citet{Greenberg1964} and \citet{Scholes1966} will be a major topic of the following chapter. Here, I will discuss a study by \citet{Zimmer1969} concerning the synchronic status of three Turkish morpheme structure constraints and their relation to synchronic processes of vowel harmony, after touching on a few grammatical prerequisites.

Turksih vowel phonemes can be divided by height, roundness, and backness:\footnote{\citeauthor{Zimmer1969} uses place features to distinguish vowels, so that, for instance, \emph{i} is [$+$\textsc{palatal}, $-$\textsc{labial}]. I have taken the liberty of using a more familiar feature specification, though nothing hinges on this choice. I have also structured my discussion of the Turkish facts to avoid the need to discuss the details of roundness harmony. See \citealt{Zimmer1969} for additional details.}

\ex Feature specification for Turkish vowels: \par\nobreak
\begin{tabular}{l c c c c}
 & \multicolumn{2}{c}{[$-$\textsc{back}]} & \multicolumn{2}{c}{[$+$\textsc{back}]} \\
& [$-$\textsc{round}] & [$+$\textsc{round}] & [$-$\textsc{round}] & [$+$\textsc{round}] \\
\buf[$+$\textsc{high}] & \emph{i} [i]        & \emph{\"u} [y]      & \emph{\i} [\textturnm] & \emph{u} [u] \\
\buf[$-$\textsc{high}] & \emph{e} [e]        & \emph{\"o} [\o]     & \emph{a} [\textschwa]  & \emph{o} [o] \\
\end{tabular}
\xe

%into four [$+$\textsc{back}] vowels \emph{u}, \emph{\i}, \emph{o}, \emph{a}, denoting [u, \textturnm, o, a] and four [$-$\textsc{back}] vowels \emph{i}, \emph{\"u}, \emph{e}, \emph{\"o} denoting [i, y, e, \o]. 

Native roots generally contain only [$+$\textsc{back}] vowels (e.g., \emph{g\"orev} `duty') or [$-$\textsc{back}] vowels (e.g., \emph{odun} `wood'), but loanwords need not conform to this generalization (e.g., \emph{sosis} `sausage'). Despite this latter fact, there is some evidence that this constraint on roots is an induced one.

\ex Turkish declensional suffix allomorphy (after \citealt[][315]{Zimmer1969}): \par\nobreak
\begin{tabular}{l l l l l}
   & nom.sg.      & acc.sg.         & gen.sg.              \\
a. & \emph{it}    & \emph{iti}      & \emph{itin}      & `dog'  \\
%a. & \emph{g\"ul} & \emph{g\"ul\"u} & \emph{g\"ul\"un} & `rose' \\
c. & \emph{k\i z} & \emph{k\i z\i}  & \emph{k\i z\i n} & `girl' \\
%c. & \emph{tuz}   & \emph{tuzu}     & \emph{tuzun}     & `salt' \\
\end{tabular}
\xe

\noindent
Similar facts are indicated by the rest of the declensional paradigms. It is possible to analyze the forms of the acc.sg. 
(\emph{i}, \emph{\i})
%(\emph{\"u}, \emph{u})
%(\emph{i}, \emph{\"u}, \emph{\i}, \emph{u})
and gen.sg. 
(\emph{in}, \emph{\i n})
%(\emph{\"un}, \emph{un})
%(\emph{in}, \emph{\"un}, \emph{\i n}, and \emph{un}) 
without suppletion, using the following phonological rule: 

\ex \buf[\textsc{back}] \textsc{Harmony}: %\par\nobreak
V $\rightarrow$ [$\alpha$\textsc{back}] / V[$\alpha$\textsc{back}] C$_0$ \gap\gap \xe

In other words, a vowel takes on the [\textsc{back}] specifications of the preceding vowel. The acc.sg. suffix is then [$+$\textsc{high}] (and for these cases, [$-$\textsc{round}]); there is no evidence bearing on whether it is one of /i, \i/ or simply underspecified for [back]. The gen.sg. suffix is the same vowel followed by /n/. 

If (\lastx) is a correct synchronic generalization, it clearly can induce the aforementioned morpheme structure constraint requiring root vowels to agree on [\textsc{back}]. However, (\lastx) as stated must contend with loanword exceptions. These exceptions might be used to argue that it is not a phonologically general rule, and should be restricted, for instance, to apply only in morphologically derived environments. I will return to this question shortly. 

Another morpheme structure constraint discussed by \citeauthor{Zimmer1969} is that a vowel immediately preceded by \emph{a} and a labial consonant (\emph{p}, \emph{b}, \emph{f}, \emph{v}) must be [$+$\textsc{round}], though once again there are a few roots that are exceptions to this generalization. What is crucial here is that this labial consonant constraint is not involved in any suffix alternation, and in fact, does not hold of complex words. For instance, the acc.sg. of \emph{av} `hunt' is not *\emph{avu}, but rather \emph{av\i}. This suggests that this constraint cannot be considered an induced restriction.

\citeauthor{Zimmer1969} attempts to study whether speakers respond differently to induced restrictions and static restrictions in a wordlikeness task. The linking hypothesis for such a task and the grammatical system will be taken up in the following chapter: for the moment, I will simply assume, with \citeauthor{Zimmer1969}, that some reasonable linking hypothesis can be put forth. 

%In Zimmer's first experiment, native Turkish speakers were given a questionnaire where each line consisted of two nonce words written in the modern orthography. The subjects were asked to read the words allowed and to put a checkmark next to the word they preferred. Subjects were also permitted to check both words, indicating ``no preference''. Four of the pairs in the first study consisted of a pair of nonce words differing only whether or not they are consistent with the [\textsc{back}] \textsc{Harmony} generalization. These results are given in Table \ref{backharm}.
%\begin{table}
%\centering
%\begin{tabular}{l r l r r}
%\toprule
%hi \\
%\midrule
%\emph{temez}   & $19$ & \emph{temaz} & $3$ & $1$ \\ %back
%%\emph{tipez}   & $21$ & \emph{tipaz} & $1$ & $1$ \\ %back
%\emph{ter\"uz} & $20$ & \emph{teruz} & $1$ & $2$ \\ %back
%%t\"or\"uz & $19$ & t\"oriz & $1$ & $3$ \\ %round
%\emph{teriz}   & $23$ & ter\i z & $0$ & $0$ \\ 
%% tamuz   & $3$  & tam\I z & $16$ \\
%% tafuz   & $3$  & taf\I z & $17$ \\
%% tapuz   & $7$  & tap\I z & $9$ \\
%\bottomrule
%\end{tabular}
%\caption{This is a test, \citep[][313]{Zimmer1969}}
%\label{backharm}
%\end{table}

%\begin{tabular}{l r l r r}
%\toprule
%hi \\
%\midrule
%tamuz & $3$ & tam\i z & $16$ & $4$ \\
%tafuz & $3$ & taf\i z & $17$ & $3$ \\
%tapuz & $7$ & tap\i z & $9$  & $7$ \\
%tavuz & $9$ & tav\i z & $4$  & $10$ \\
%tabuz & $5$ & tab\i z & $12$ & $6$ \\
%\bottomrule
%\end{tabular}
%\caption{\citet[][313]{Zimmer1969}}
%\end{table}

In Zimmer's Experiment II, native Turkish speakers heard pairs of nonce words spoken, by a native speaker. Fifteen minutes later, they heard a second presentation of each of the pairs, and were asked to select which of the two words in the pair sounded more like a Turkish word. Four of the pairs differ solely in whether they obey or disobey the [\textsc{back}] \textsc{Harmony} generalization. These results are given in Table \ref{backharm}.

\begin{table}
\centering
\begin{tabular}{l r l r r}
\toprule
harmonic      & \#   & disharmonic & \#   \\
\midrule
\emph{pemez}  & $30$  & \emph{pemaz}     & $2$  \\
\emph{teriz}   & $28$ & \emph{ter\i z}   & $3$  \\
\emph{tokaz}   & $26$ & \emph{tokez}     & $6$  \\
\emph{tipez}   & $24$ & \emph{tipaz}     & $8$  \\
\emph{ter\"uz} & $19$ & \emph{teruz}     & $13$ \\
\bottomrule
\end{tabular}
\caption{this is at test \citet[][314]{Zimmer1969}}
\label{backharm}
\end{table}
% c = 127, d = 32, p = 5.233e-29

In each case, the root obeying [\textsc{back}] \textsc{Harmony} is strongly preferred. One way to quantify the preference for the harmonic stems is as follows. An observation from this experiment is called experiment can be thought of being either \emph{concordant} with the hypothesis that speakers will prefer harmonic nonce words if the harmonic nonce word is selected as better. If the non-harmonic nonce word is selected, this observation is called \emph{discordant}. ``No preference'' observations are ignored.  \citet[][749]{Goodman1954} use the count of concordant pairs $c$ and discordant pairs $d$ to define a metric $\gamma$:

\begin{equation}
\gamma = \frac{c - d}{c + d}
\end{equation}

This $\gamma$ ranges between $-1$, indicating complete disagreement, and $1$, indicating that all data is is consistent with the hypothesis. For the four nonce words in table one, $\gamma = 0.597$, which indicates that speakers do have a moderately preference for the harmonic nonce words. In the same experiment, subjects also encountered five pairs which differed on whether they obeyed the labial consonant generalization. The results are shown in Table \ref{labcons}.

\begin{table}
\centering
\begin{tabular}{l r l r}
\toprule
harmonic & \# & disharmonic & \# \\
\midrule
\emph{tafuz} & $21$ & \emph{taf\i z} & $11$ \\
\emph{pamuz} & $17$ & \emph{pam\i z} & $15$ \\
\emph{tapuz} & $17$ & \emph{tap\i z} & $15$ \\
\emph{mavuz} & $16$ & \emph{mav\i z} & $16$ \\
\emph{tabuz} & $16$ & \emph{tab\i z} & $16$ \\
\bottomrule
\end{tabular}
\caption{this is at test \citet[][314]{Zimmer1969}}
\label{labcons}
\end{table}
% c = 87, d = 73, p = 0.101

Here is it clear that the preferences, while going in the expected direction, are far smaller ($\gamma = 0.088$). There are ties for two of the lexical items, and only one of the five pairs, \emph{tafuz} $\sim$ \emph{taf\i z}, matches the stronger preferences seen with backness harmony. If we remove this pair, $\gamma = 0.032$.  It might be possible to though any failure to find significance here should surely be attributed to insufficient statistical power, not evidence for the null hypothesis. 




\subsection{Optimality Theory}

\subsection{The present day}

\section{English syllable contact}

The rest of this chapter concerns English syllable contact clusters in monomorphs. This phenomenon is chosen for two reasons: first, English syllable contact clusters are the subject of many prior phonological and phonotactic analyses, and second, the data is very accessible.

%c. & l'etS'        & l'\'akka     & `lie down' &
%for the verb l'\'e\v{c} `to lie down'. From two present active forms (among other parts of the paradigm), 1sg. \emph{l'\'ago}, 3pl., \emph{l'\'agot}, one can detect an argument for analyzing the root as /g/-final. The imperative, \emph{l'\'akka},

Degemination

\citet[][20]{Harris1994}

