
As reviewed below, this was the view of many phonologists of the 1960s and 1970s. While several classic documents indicated problesm with this view, I will argue that the broader view of 


phonological

At the same time, the hypothesis seems to have been ignored by work in the late 1980s and early 1990s, and that 

and no compelling reasons have been offered to revise it. If anything, it appears that this hypothesis was forgotten or ignored during the 1980s and early '90s, and that despite the fact that it was championed by the founding Optimality Theory


 founding document forgotten or ignored during the 1980s and early '90s, and that despite the fact that it was championed by the founding document forgotten or ignored during the 1980s and early '90s, and that despite the fact that it was championed by the founding document forgotten or ignored during the 1980s and early '90s, and that despite the fact that it was championed by the founding document 

 work was forgotten during the 1980s, briefly revived by the the earliset 

Linguists of this era were not ignorant of the possibilities of obtaining experimental evidence bearing on MSCs. A classic study, and one that is a  demonstration of the fundamental correctness of the classic view

\footnote{\citeauthor{Zimmer1969}'s paper uses place features to distinguish vowels, so that front vowels are [$+$\textsc{palatal}], and so on. I have taken the liberty of using more familiar vowel features, such as [$\pm$\textsc{high}, $\pm$\textsc{back}].}

[$+$\textsc{back}] \emph{u, o, \i, a}
[$-$\textsc{back}] \emph{i, e, \"u, \"o}

questionnaire

Subjects were also permitted to check both words, indicating ``no preference''. 
\begin{tabular}{l r l r}
\toprule
front harmony &      & not \\
\midrule
temez     & $19$ & temaz   & $3$ & $1$ \\
tipez     & $21$ & tipaz   & $1$ & $1$ \\
ter\"uz   & $20$ & teruz   & $1$ & $2$ \\
t\"or\"uz & $19$ & t\"oriz & $1$ & $3$ \\
teriz     & $23$ & ter\i z & $0$ & $0$ \\
% tamuz   & $3$  & tam\I z & $16$ \\
% tafuz   & $3$  & taf\I z & $17$ \\
% tapuz   & $7$  & tap\I z & $9$ \\
\bottomrule
\end{tabular}

\citet[][313]{Zimmer1969}

\citep{Goodman1954}[][749]

\begin{equation}
\gamma = \frac{c - d}{c + d}
\end{equation}

$c = 102$, $d = 6$, $\gamma = 0.888$

\begin{tabular}{l r l r}
\toprule
[\emph{a}[labial]\emph{u}  &      & *\emph{a}[labial]\emph{\i] & 
\midrule
pamuz    & $17$ & pam\i z & $15$ \\
tafuz    & $21$ & taf\i z & $11$ \\
tapuz    & $17$ & tap\i z & $15$ \\
mavuz    & $16$ & mav\i z & $16$ \\
tabuz    & $16$ & tab\i z & $16$ \\
\bottomrule
\end{tabular}
\citet[][314]{Zimmer1969}

$c = 87$, $d = 73$, $\gamma = 0.088$

\section{The phenomenon}

% why syllable contact clusters
I adopt English syllable contact clusters in monomorphs. Two considerations guide this choice. First, English syllable contact clusters are the subject of many prior phonological and phonotactic analyses. Second, the data should be accessible to any reader.

% lexicon
The \emph{lexicon} has many senses: throughout, I use the term to refer to the set of underlying representations \citep[][269]{LANGUAGE}. I depart from \citeauthor{LANGUAGE}'s characteriztion of the lexicon in two ways. First, I assume that this set is merely a reflection of individual speakers' knowledge of their language, realized in the human; it has no further reality. Second, in using the term, I do not intend to refer to any knowledge of morphological irregularities, which also reside in the lexicon for \citeauthor{LANGUAGE} (ibid., p.~273).

\section{Phonotactic gaps and the null hypothesis}

Before studying

\section{History}

A major result of the American structuralists was the finding that the set of sound sequences found in morphs and across morpheme boundaries are different.\footnote{The discussion in this paragraph is based on \citealt[][267]{Anderson1985}.} \citet{Bloomfield1930} proposes that a problem in the phonemic analysis of German can be resolved by allowing for allophony to make reference to morpheme boundaries. \citeauthor{Bloomfield1930} notes that [\c{c}] and [x] can be cast as allophones, despite the existence of apparent minimal pairs like \emph{Kuchen} [ku\lm x\textschwa n] `cake' $\sim$ \emph{Kuhchen} [ku\lm\c{c}\textschwa n] `cow-let'. \citeauthor{Bloomfield1930}'s claim is that these words differ not just in their medial consonsant but also that the latter is a compound, marked with the diminutive \emph{-chen} (cf. \emph{Kuh} `cow'). Under the assumption that [x] occurs in this position (before a back vowel) only when this preceding vowel belongs to the same word or morpheme, [\c{c}] and [x] can be cast as allophones of the same phoneme (likely /\c{c}/, the basic variant). A natural consequence of this view is that generalizations about sound sequences found within morphemes are necessarily different within and across morpheme boundaries, 
and thus a distributional analysis must take morphology into account, as argued forcefully by \citet{Pike1947b}. \citeauthor{Pike1947b} discusses limitations on possible vowel sequences in Mixteco monomorphs, but it is easy to find exceptions to these generalizations when a morpheme boundary intervenes in the short Mixteco text analyzed in \citealt{Pike1944}. 

\ex Mixteco MSCs \citep{Pike1947b} and complex words \citep{Pike1944}: \\ 
\begin{tabular}{r l l l} %\toprule
   & MSC & complex word \\ %\midrule
a. & *{C}a{C}e & k\'a-\textsuperscript{n}dee & `kept \ldots inside' \\
b. & *{C}\textipa{@}{C}e & n\`i-k\textipa{\`@b@-de} & `who entered'        \\
c. & *{C}e{C}i & te-n\'i-ke\textsuperscript{n}da & `was walking         \\
d. & *{C}i{C}e & te-n\`i-kee-t\`\textschwa & `and went away'      \\ %\toprule
%e. & *{C}e{C}o & b\'e\textglotstopvari e-\v{z}\'o & `our house'          \\
%f. & *{C}eo & ke-o-d\'e & `we eat him'         \\
\end{tabular} \xe

\noindent
\citet[][166]{Pike1947b} affirms that the morpheme is ``marked'' by the violation of morpheme-internal sequence restrictions, and this idea was further developed by \citet{Harris1955}, who proposes that this mechanism allows the linguist to identify morpheme boundary. 

While the above work is widely known for introducing ``grammatical'' (which in the sense current at the time just meant ``supraphonemic'') components into the theory of phonology, albeit indirectly, similar (and arguably more sophisticated) intuitions about the interaction between phonotactic generalizations and morpheme boundaries were made by members of the Prague Circle at roughly the same time, a fact which has not been noted to my knowledge. \citet{Jakobson1932} attributes a number of phonological properites of the Russian imperative to morpheme juncture. For instance, most Russian infinitives end with a final /t'/. The reflexive variant of these verbs is marked with a final /-sa/, which feeds a general process of regressive voicing assimilation, and this /\ldots t\textceltpal -s\ldots/ juncture triggers the loss of palatalization. However, palatalization of a stem-final labial or coronal---including /t'/, as in ?c ---is preserved in reflexive imperatives, which are formed by attaching the reflexive /-sa/ to the bare stem.\footnote{I have taken a number of liberties with \citeauthor{Jakobson1932}'s presentation of the data, which makes use of a highly abstract phonemic transcription. The broad phonetic transcriptions below are the result of consultations with N linguistically na\"ive native Russian speakers. Stress and palatalization are marked regardless of whether they are constrastive in that position.}

\ex Russian reflexives \citep[after][]{Jakobson1932}: \\
\begin{tabular}{l l l l} %\toprule
   &  infinitive & imperative \\ %\midrule
a. & sl\textceltpal av\textceltpal its\textschwa         & sl\'af\textceltpal s\textschwa            & `be glorious'    \\
   & upr\'amits\textschwa         & upr\'am\textceltpal s\textschwa          & `be stubborn'    \\
b. & kr\'as\textceltpal its\textschwa         & kr\'as\textceltpal s\textschwa            & `put on makeup'  \\
   & \v{z}\'arits\textschwa       & \v{z}\'ar\textceltpal s\textschwa         & `roast'          \\ 
c. & z\textschwa b\'yts\textschwa & z\textschwa b\'ut\textceltpal s\textschwa & `forget' \\ %\bottomrule
%   & ab\'utsa      & ab\'ujsa     & `put on shoes'   \\
\end{tabular} \xe

Unlike in \citeauthor{Bloomfield1930}'s analysis of German, the cluster of palatal coronal followed by /s/ has an intervening morpheme boundary in both infinitive and imperative. \citeauthor{Jakobson1932} proposes that the two /t'-sa/ junctures in c, in fact different, with the imperative not giving rise to the loess of palatalization. Obvious parallels of this more powerful notion can be found in Lexical Phonology or the model developed by \citet{Halle1987}, to name a few. A few years later, \citet{Trnka1936}, another member of the Prague Circle, makes the connection between \citeauthor{Jakobson1932}'s hypothesis and phonotactic generalizations.

Arguably, this more powerful system might be appropriate for \citeauthor{Bloomfield1930}'s data, if it is the case that the final \emph{-en} in \emph{Kuchen} is also a morpheme (perhaps related to \emph{K\"uche} `kitchen; cuisine').

%c. & l'etS'        & l'\'akka     & `lie down' &
%for the verb l'\'e\v{c} `to lie down'. From two present active forms (among other parts of the paradigm), 1sg. \emph{l'\'ago}, 3pl., \emph{l'\'agot}, one can detect an argument for analyzing the root as /g/-final. The imperative, \emph{l'\'akka},

