\subsection{Gradience in judgements}

at least as far back as \emph{LSLT} \citep{LSLT}

%and has become a recent feature of the discussion
is that non-gradient models are totally incompatible with psycholinguistic results.

\subsubsection{Task model}

\citet[][16]{Schutze1996}

\citet{Schutze2005}
\citet{Schutze2011}

\citet{Coleman1997}

\subsubsection{Falsifiability}

\citet{Armstrong1983}

% wordlikeness critique
\citet{Yang2008a}

a mix of ``declarative'' knowledge,
corresponding to rote facts and thus the lexicon,
and ``procedural'' knowledge, or knowledge of generative procedures 
(in the sense of J. \citealt{Anderson1993}). 

% math cog
\citet{Logan1988}
\citet{Sfard1991}
\citet{Delazer2005}
