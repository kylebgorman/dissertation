% 2.1.3: Inventory considerations

The syllables derived by the above procedure must further be divided into onset, nucleus, and coda, and mapped to underlying representation. In most cases, this is trivial: the nucleus is a pure vowel and can be easily separated from the consonants, which surface faithfully. However, the status of several surface phones in the syllable and the relationship of surface segments to underlying forms is not always obvious. The treatment of those segments is described and motivated below. 

\subsubsection{The velar nasal}
\label{velarnasal}

There is a long-standing debate regarding whether English [ŋ] is a phoneme in its own right, as demanded by Kiparsky's Alternation Condition \citep{Kiparsky1968} or Lexicon Optimization \citep[][53]{OT}, or simply the allophone of /n/ found before /k, g/ \citep[][65]{Borowsky1986}. The strongest piece of evidence for the pure allophonic analysis (and an attentant process of /g/-\textsc{Deletion}) is the general absence of onset [ŋ], a position where it can never be followed by a dorsal consonant needed to derive the velar allophone. \citet{Pierrehumbert1994} assumes the pure allophonic analysis in her study of syllable contact clusters, and it is adopted here and formalized below in Section \ref{cnpasection}.
%\citealt[][62]{Halle1985a}),

\subsubsection{[j] onglides}

I assume that the [j] onglide in words such as \emph{ass}[j]\emph{ume} is present in underlying representations (e.g., \citealt[][278]{Borowsky1986}, J. \citealt{Anderson1988b}, pace \emph{SPE}:196, \citealt[][89]{Halle1985a}, \citealt[][217]{McMahon1990}). The primary motivation for this assumption is the observation that presence or absence of this glide is contrastive (e.g. \emph{coup} $\sim$ \emph{queue}, \emph{booty} $\sim$ \emph{beauty}). I further assume that the front glide is not part of the onset except in the case where the onset would ortherwise be null (e.g., \emph{jun}[j]\emph{or}).
%\footnote{English glides are transcribed here as full segments, not as ``subsegments'', as this distinction does not appear to be meaningful for English, or empirically motivated for other languages \citep{Rubach2002}.}
When [j] is a simplex onset, it may be followed by any vowel \citep[][276]{Borowsky1986}. However, when [j] is immediately preceded by an onset consonant (e.g., [bj]\emph{ugle}), the following vowel is always [u\lm], and speakers judge other following vowels to be anomalous in nonce words \citep{Moreland2009}. This defective distribution of vowels following [j] follows if the onglide in this context is the first component of a phonological diphthong, and thus part of the nucleus (\citealp[][232]{Hayes1980}, \citealp[][61f.]{Harris1994}, \citealp{Davis1995}). \citet[][42]{Clements1983} note that /m, v/ do not appear in onset clusters except in words like \emph{muse} or \emph{view}; the exceptionality of [ju\lm] also suggests the glide is nuclear. There is a great deal of external support for this position as well. The [ju\lm] in words like \emph{spew} may pattern together in Pig Latin \citep{Davis1995,Idsardi2005} and \emph{shm}-reduplication \citep{Nevins2003}, to the exclusion of the rest of preceding tautosyllabic consonants. The same fusion of [ju\lm] at the expense of the onset is also found in speech errors, e.g., [kju\lm]\emph{mor homponent} for [hju\lm]\emph{mor component} \citep[][130]{Shattuck-Hufnagel1986}. Finally, \citet{Buchwald2005} considers [j] onglides in the speech of VBR, an aphasic patient who has difficulties producing complex onsets. 

\ex VBR's complex onsets \citep[79--80, his transcriptions]{Buchwald2005}: \\
\begin{tabular}{l l l}
a. & kəræb  & `crab'  \\
   & bəlid  & `bleed' \\
b. & kəwin  & `queen' \\
   & kəwoʊt & `quote' \\
c. & kut    & `cute'  \\
   & musɪk  & `music  \\ 
\end{tabular}
\xe

\noindent
VBR breaks up complex onsets, including back onglide clusters in (\nextx b). However, in (\nextx c), which contain the front onglide in standard English, no epenthesis occurs: the glide is simply absent. This also suggests that the front onglide is part of the onset, not the nucleus. 

%One potential problem with this account is noted by \citet{Kaye1996}, who obseres while [ju\lm] may follow any single tautosyllabic consonant, it never follows branching onsets unless they consist of [s] and a single consonant. 
%This is the only sign that [ju\lm] shows an affiliation for the onset. 

\subsubsection{[w] onglides}

The selective properties of the back onglide [w] contrast sharply with those of the front onglide, and I assume that it is assigned to the onset. Whereas the front onglide shows only limited selectivity for preceding tautosyllabic consonants \citep{Davis1995,Kaye1996}, the back onglide [w] is rarely preceded by tautosyllabic consonants other than [k] (e.g., \emph{tran}[kw]\emph{il}). Unlike the front glide, syllable-initial [kw] may be followed by nearly any vowel \citep[][161]{Davis1995}. Unlike [ju\lm], onglide [w] followed by a vowel does not pattern together in Pig Latin \citep[][166]{Davis1995}.

%Whereas [Cju\lm] syllables attracts stress, [kwV] syllables do not \citep[][162f.]{Davis1995}.

\subsubsection{Post-vocalic \emph{r}}

Like this study, \citet{Pierrehumbert1994} uses a dictionary that transcribes RP, in which word-medial post-vocalic \emph{r} has been lost. There are independent reasons to believe that \emph{r}-full dialects of English assign post-vocalic \emph{r} to the nucleus and is not part of syllable contact clusters in such dialects. First, \citet[][255]{Harris1994} notes that the number of vowel contrasts is greatly reduced before \emph{r} when compared to other coronal consonants. For instance, the majority of North America has lost the historical contrast between \emph{Mary}, \emph{marry}, and \emph{merry}, and one holdout, Philadelphia, is losing the the contrast between \emph{merry} and \emph{Murray} \citep[14f.]{ANAE}. Another source of evidence for the nuclear status of post-vocalic \emph{r} comes from its failure to trigger two processes in English variable phonology. \citeauthor{Harris1994} reports that a variable process of /t/-\textsc{Glottalization} in many dialects of British English is blocked when /t/ is preceded by any consonant except post-vocalic \emph{r}.

\ex /t/-\textsc{Glottalization} in British English \citep[after][195, 258]{Harris1994}: \\
\begin{tabular}{l l l@{} l}
a. & fis[t]   & * & fis[ʔ]   \\
   & mis[t]er & * & mis[ʔ]er \\
b. & par[t]   &   & par[ʔ]   \\
   & car[t]on &   & car[ʔ]on \\
\end{tabular}
\xe

\noindent
Similarly, while /t, d/ delete in word-final position when immediately preceded by a consonant, including sonorants /n, l/, as in (\nextx a), deletion of /t, d/ after post-vocalic \emph{r} in American English is ``rare or nonexistent'' \citep[][8]{Guy1980}.

\ex /t, d/-\textsc{Deletion} in American English: \\
\begin{tabular}{l l l@{} l}
a. & be[lt]  &   & be[l]  \\
%   & we[ld]  &   & we[l]  \\
%   & cha[nt] &   & cha[n] \\
   & me[nd]  &   & me[n]  \\
%b. & fl[ɜ˞]  & * & fl[ɜ˞]  \\
%b. & fl[ɝt]  & * & fl[ɝ]  \\
b. & sh[ɚt]  & * & sh[ɚ] \\
%   & w[ɚd]   & * & w[ɚ]  \\
   & c[ɚd]   & * & c[ɚ]  \\
\end{tabular}
\xe

%Finally, \citet[][251]{Fromkin1973} also presents evidence that post-vocalic \emph{r} may behave as if nuclear in speech errors. 
