\subsection{Inventory considerations}

To identify the inventory of syllable contact clusters, it is also necessary to parse syllables into onset, nucleus, and coda. In most cases this is trivial, but a few distributional heuristics for difficult cases are described below.

\subsubsection{The velar nasal}
\label{velarnasal}

There is a long-standing debate regarding whether English [ŋ] is a phoneme in its own right, as demanded by Kiparsky's Alternation Condition \citep{Kiparsky1968} or Lexicon Optimization \citep[][53]{OT}, or simply the allophone of /n/ found before /k, g/ \citep[][65]{Borowsky1986}. The strongest piece of evidence for treating the velar nasal as a pure allophone is the general absence of ?
nset [ŋ], a position where it can never be followed by a dorsal consonant needed to derive the velar allophone. \citet{Pierrehumbert1994} assumes the pure allophonic analysis in her study of syllable contact clusters, and it is assumed here. English nasal place allophony is formalized in Section \ref{cnpasection} below.
%\citealt[][62]{Halle1985a}),

\subsubsection{[j] onglides}

The [j] onglide in words such as in words such as \emph{ass}[j]\emph{ume} is assumed to be present in underlying representations (e.g., \citealt[][278]{Borowsky1986}, J. \citealt{Anderson1988b}, pace \emph{SPE}:196, \citealt[][89]{Halle1985a}, \citealt[][217]{McMahon1990}), since the presence or absence of the glide is at least marginally contrastive (e.g., \emph{coo}/\emph{coup} \alt{} \emph{queue}, \emph{booty} \alt{} \emph{beauty}). The front onglide is further assumed to be assigned to the nucleus, except when the onset would otherwise be null (e.g., \emph{jun}[j]\emph{or}). 
%\footnote{English glides are transcribed here as full segments, not as ``subsegments'', as this distinction does not appear to be meaningful for English, or empirically motivated for other languages \citep{Rubach2002}.}
There is extensive motivation for this principle. When [j] is a simplex onset, it may be followed by any vowel \citep[][276]{Borowsky1986}, but when [j] is immediately preceded by an onset consonant (e.g., [bj]\emph{ugle}), the following vowel is always [uː]. In fact, speakers judge words in which the front onglide is preceded by an onset consonant and followed by a vowel other than [uː] to be anomalous \citep{Moreland2009}. This defective distribution of vowels following [j] suggests that the onglide in this context is the first component of a phonological diphthong (\citealp[][232]{Hayes1980}, \citealp[][61f.]{Harris1994}, \citealp{Davis1995}). \citet[][42]{Clements1983} note that /m, v/ do not appear in onset clusters except in words like \emph{muse} or \emph{view}; the exceptionality of [juː] also suggests the glide is nuclear. There is also strong external support for this principle. The [juː] in words like \emph{spew} may pattern together in Pig Latin \citep{Davis1995,Idsardi2005} and \emph{shm}-reduplication \citep{Nevins2003} to the exclusion of the rest of preceding tautosyllabic consonants. The same fusion of [juː] at the expense of the onset is also found in speech errors, e.g., [kjuː]\emph{mor} [h]\emph{omponent} for [hjuː]\emph{mor component} \citep[][130]{Shattuck-Hufnagel1986}. Finally, \citet{Buchwald2005} considers [j] onglides in the speech of VBR, an aphasic patient who has difficulties producing complex onsets. 

\begin{example}[VBR's complex onsets (\citealp{Buchwald2005}:79--80, his transcriptions)]
%\protect\citep[79--80, his transcriptions]{Buchwald2005}]

\begin{tabular}{l l l}
a. & kəræb  & `crab'  \\
   & bəlid  & `bleed' \\
b. & kəwin  & `queen' \\
   & kəwoʊt & `quote' \\
c. & kut    & `cute'  \\
   & musɪk  & `music  \\ 
\end{tabular}
\label{VBR}
\end{example}

\noindent
VBR breaks up complex onsets with epenthesis, including those that consist of a consonant and a back onglide cluster (\ref{VBR}b). However, no epenthesis occurs between a consonant and a front onglide; rather, the glide is absent (\ref{VBR}c). The failure of the front onglide to pattern with other consonant clusters suggests once again that the glide is part of the nucleus. 

%One potential problem with this account is noted by \citet{Kaye1996}, who obseres while [juː] may follow any single tautosyllabic consonant, it never follows branching onsets unless they consist of [s] and a single consonant. 
%This is the only sign that [juː] shows an affiliation for the onset. 

\subsubsection{[w] onglides}

The selective properties of the back onglide [w] contrast sharply with those of the front onglide, and I assume that it is assigned to the onset. Whereas the front onglide shows only limited selectivity for preceding tautosyllabic consonants \citep{Davis1995,Kaye1996}, the back onglide [w] is rarely preceded by tautosyllabic consonants other than [k] (e.g., \emph{tran}[kw]\emph{il}). Unlike the front glide, syllable-initial [kw] may be followed by nearly any vowel \citep[][161]{Davis1995}. Unlike [juː], onglide [w] followed by a vowel does not pattern together in Pig Latin \citep[][166]{Davis1995}. And onsets followed by [w] pattern with other complex onsets in undergoing epenthesis in the patholical speech of VBR, discussed immediately above.

%Whereas [Cjuː] syllables attracts stress, [kwV] syllables do not \citep[][162f.]{Davis1995}.

\subsubsection{Post-vocalic \emph{r}}

Both CELEX and the dictionary used by \citet{Pierrehumbert1994} in her study transcribe Received Pronunciation, in which word-medial post-vocalic \emph{r} has been lost. In \emph{r}-full dialects, there is reason to believe that post-vocalic \emph{r} is in fact nuclear, and therefore its presence or absence is irrelevant to the constraints on syllable contact clusters. Many vowel contrasts are suspended before \emph{r} \citep[][255]{Harris1994}; for example, \emph{fern}, \emph{fir}, and \emph{fur} lack the contrasts found in \emph{pet}, \emph{pit}, and \emph{putt}. Further evidence for the nuclear status of post-vocalic \emph{r} comes from variable phonological processes in which post-vocalic \emph{r} patterns differently than internal coda consonants. \citeauthor{Harris1994} reports that a variable process of /t/-\textsc{Glottalization} in many dialects of British English is blocked when /t/ is preceded by any consonant except post-vocalic \emph{r}.

\begin{example}
/t/-\textsc{Glottalization} in British English \citep[after][195, 258]{Harris1994}: 

\vspace{0.5\baselineskip}
\begin{tabular}{l l l@{} l}
a. & fis[t]   & * & fis[ʔ]   \\
   & mis[t]er & * & mis[ʔ]er \\
b. & par[t]   &   & par[ʔ]   \\
   & car[t]on &   & car[ʔ]on \\
\end{tabular}
\end{example}

\noindent
Similarly, while /t, d/ delete in word-final position when immediately preceded by a consonant, including sonorants /n, l/, as in (\ref{td}a), deletion of /t, d/ after post-vocalic \emph{r} in American English is ``rare or nonexistent'' \citep[][8]{Guy1980}. 

\begin{example}
\label{td}
/t, d/-\textsc{Deletion} in American English: 

\vspace{0.5\baselineskip}
\begin{tabular}{l l l@{} l}
a. & be[lt]  &   & be[l]  \\
%   & we[ld]  &   & we[l]  \\
%   & cha[nt] &   & cha[n] \\
   & me[nd]  &   & me[n]  \\
%b. & fl[ɜ˞]  & * & fl[ɜ˞]  \\
%b. & fl[ɝt]  & * & fl[ɝ]  \\
b. & sh[ɚt]  & * & sh[ɚ] \\
%   & w[ɚd]   & * & w[ɚ]  \\
   & c[ɚd]   & * & c[ɚ]  \\
\end{tabular}
\end{example}

%Finally, \citet[][251]{Fromkin1973} also presents evidence that post-vocalic \emph{r} may behave as if nuclear in speech errors. 
