\subsection{Labial attraction}

I am using the dotted association line in a deliberately ambiguous fashion in this chapter. The autosegmental structures represent formalizations of the constraints, but I intend to remain ambiguous as to their ontology and do not intend them to be interpreted as derivational rules. These questions are taken up more carefully below.

\citet{Clements1982}

\citet{Zimmer1969}, \citet{Zuraw2000}, and \citet{Inkelas2001} all suggest

\subsubsection{Phonological description}

\citet[][36]{Lees1966a} describes \textsc{Labial Attraction} as a process by which ``a high, short harmonic vowel is rounded in the second syllable of a disyllabic word whose first vowel is /a/, and whose medial consonant cluster contains a labial /p, b, m, v/, and then it is de-harmonified''. This description is transalted into an autosegmental rule below.

\ex \textsc{Labial Attraction} (after \citealt[][286]{Lees1966b}, \citealt[][171]{Inkelas2001}): \\
\xymatrix@R=12pt{
\txt{[$-$\textsc{Round}]}\ar@{-}[d] &                            & \txt{[$+$\textsc{Round}]}\ar@{..}[d] \\
\txt{V}\ar@{-}[d]                    & \txt{C}\ar@{-}[d]          & \txt{V}\ar@{-}[d] \\
\txt{[$-$\textsc{High}, $+$\textsc{Back}]}             & \txt{[$+$\textsc{Labial}]} & \txt{[$+$\textsc{High}, $+$\textsc{Back}]}}
\xe

\noindent
A few examples of words conforming to this generalization are \emph{çapul} `raid', \emph{sabur} `patient', \emph{şaful} `wooden honey tub', \emph{avuç} `palm of hand', and \emph{samur} `sable' \citep[][285]{Lees1966b}; \citeauthor{Lees1966b} also notes the existence of exceptions like \emph{tavır} `mode', but writes they are ``surprisingly rare'' (ibid., 286); \citet[][225]{Clements1982} lists a number of additional exceptions.

\citeauthor{Clements1982} propose that there is an additional type of exception to \textsc{Labial Attraction}.

\begin{quotation}
Even more decisive evidence against a rule of Labial Attraction is the existence of a further, much larger set of roots containing /\ldots~aCu~\ldots/ sequences in which the intervening consonant or consonant cluster does not contain a labial\ldots We conclude that there is no systematic restriction on the set of consonants that may occur medially in rotos of the form /\ldots~aCu~\ldots/. \citep[][225]{Clements1982}
\end{quotation}

\noindent
\citeauthor{Clements1982} are suggesting that \textsc{Labial Attraction} implies that \emph{aTu}, where \emph{T} represents a non-labial consonant, should be infrequent, but this fact is not inconsistent with the formulation of the constraint by \citet{Lees1966a,Lees1966b} and \citet{Zimmer1969}; \emph{aTu} clusters do not meet \textsc{Labial Attraction}'s structural description.

There is reason to believe that \textsc{Labial Attraction} is at best a static constraint: it never applies across morpheme boundaries. If, contrary to fact, there was a \textsc{Labial Attraction} alternation, then one would expect, for example, that the gen.sg. of \emph{sap} `stalk' would be *\emph{sapun} instead of the observed \emph{sapın}. Additional potential evidence for the static nature of \textsc{Labial Attraction} is provided by \citeauthor{Clements1982}, who note a class of bisyllabic words in which the second vowel, always [$+$\textsc{High}], alternates with zero. 

\ex High-vowel/zero alternations \citep[][243]{Clements1982}: \\
\begin{tabular}{l l l l}
   & nom.sg. & 3.poss. \\
a. & fikir   & fikri & `idea' \\ 
   & hüküm   & hükmü & `judgement' \\ 
%   & filim   & filmi & `film' & \citep[][178]{Inkelas2001} \\
%   & kojun   & kojun & `bosom' \\
b. & vakit  & vakti  & `time' \\
   & rahim  & rahmi  & `womb' \\
\end{tabular}
\xe

\noindent 
The vowel is present in the nom.sg. form and absent in the 3.poss. form. The examples in (\lastx b) shows that it need not harmonize with the preceding vowel
\citeauthor{Inkelas2001} (ibid, 178) note that \textsc{Labial Attraction} does not affect this vowel (e.g., \emph{sabır}/\emph{sabrı} `patience'), citing this as evidence against \textsc{Labial Attraction}. However, it is not a foregone conclusion that \textsc{Labial Attraction} should apply to the ``unstable'' vowel. Since it is as a morpheme structure constraint, this vowel must be underlying present and deleted in the 3.poss.; \citeauthor{Clements1982} prefer the converse analysis, in which epenthesis applies in the nom.sg., though they note that either approach requires lexically arbitrary diacritics. I conclude that unstable vowels provide no evidence for or against \textsc{Labial Attraction}.

\begin{quotation}
Lee's rule of \textsc{Labial Attraction}\ldots is not a real generalization about the Turkish lexicon. It is not true synchronically, either of native or nonnative items; nor, according to the historical and dialectical literature, does \textsc{Labial Attraction} appear to have been true at any stage going back as far as Old Turkic. \citep[][196]{Inkelas2001}
\end{quotation}

\begin{quotation}
Vowel labialization following labials is not a synchronic alternation in Turkish, yet it (unlike \textsc{Labial Attraction} per se) \emph{is} a statistically supported tendency worthy of further research. \citep[][196, emphasis in original]{Inkelas2001}
\end{quotation}

\citet{Inkelas1997}
\citet{Inkelas2001}
\citet{TELL}

\subsubsection{Lexical statistics}

%##TELL: 31,236
%#Lexicon: 25,164

\ex Lexical effects of \textsc{Labial Attraction} \citep[][186]{Inkelas2001}: \vspace{6pt} \\
\begin{tabular}{l r r r | r r r}
\toprule
Corpus  & aPu & aPı & \% & aTu & aTı & $p$-value   \\ % & corpus size
\midrule
%TELL    & 378 & 248 & 446 & 1,140 & 2.83\e{-44} \\ % & 31,236 \\
%TELL    & 77  & 30 & 72.6 & 115 & 356 & ?\e{-20} \\
%Lexicon & 108 & 57 & 65.5 & 155 & 533 & 6.6\e{-25} \\
\bottomrule
\end{tabular} \xe


BROAD labial attraction stats:

TELL full:
585 aPu (favored)
230 aPI (disfavored)
1,137 aTu
1,461 aTI
71.8\% labial attraction
p = 3.4\e{-45}

Lexicon:
1,175 aPu (favored)
446 aPI (disfavored)
1,939 aTu
2,878 aTI
72.5\% labial attraction
p = 3.1\e{-114}

STUPID attraction stats:

TELL full:
267 aPu (favored)
546 aPI (disfavored)
346 aTu
327 aTI
67.1\% attraction
p = 5.2\e{-13}

Lexicon:
357 aPu (favored)
874 aPI (disfavored)
488 aTu
396 aTI
71% attraction
p = 9.7\e{-34}

zimmer: 319-320 
inkelas et al.: 196

\subsubsection{External evidence}

Beyond the experimental results of \citet{Zimmer1969} reviewed in the next section, there is no external evidence for or against \textsc{Labial Attraction}.

%% HERE THERE BE DRAGONS
%\ex Examples of \textsc{Labial Attraction} \citep[][285]{Lees1966b}: \\
%\begin{tabular}{l l l}
%a. & çapul &  `raid \\
%b. & sabur & `patience' \\
%c. & şaful & `wooden honey tub' \\
%d. & avuç  & `palm of hand' \\
%   & havuz & `hollow'       \\
%e. & samur & sable'         \\
%   & camus & `buffalo'      \\
%\end{tabular}
%\xe
