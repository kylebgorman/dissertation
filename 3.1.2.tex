\subsection{Roundness Harmony}

The rule of roundness harmony and the data that motivates it overlaps with the preceding evidence for \textsc{Backness Harmony}. 

\subsubsection{Phonological description}

The environment for roundness harmony differs from \textsc{Backness Harmony} in that it requiers the target to be [$+$\textsc{High}]. 

\begin{example}
\textsc{Roundness Harmony}:

\xymatrix@R=24pt@C=24pt{
\txt{[α \textsc{Round}]}\ar@{-}[d]\ar@{..}[drr] &             & \txt{[$+$\textsc{High}]} \\
\txt{V}                                         & \txt{C$_0$} & \txt{V}\ar@{-}[u] \\
}
\end{example}

This rule, in concert with \textsc{Backness Harmony}, accounts for the forms of the dative singular (dat.sg.) and genitive singular (gen.sg.), among other suffixes.

\begin{example}
Turkish nominal suffix allomorphy: 

\vspace{0.5\baselineskip}
\begin{tabular}{l l l l l l l}
   & \emph{nom.sg.} & \emph{dat.sg.} & \emph{gen.sg.}  \\
a. & ip             & ipi            & ipin           & `rope' & (CS:216) \\
   & el             & eli            & elin           & `hand'    \\
   & kız            & kızı           & kızın          & `girl'    \\
   & sap            & sapı           & sapın          & `stalk'   \\
   & yüz            & yüzü           & yüzün          & `face'    \\
   & köy            & köyü           & köyün          & `village' \\
   & pul            & pulu           & pulun          & `stamp'   \\
   & son            & sonu           & sonun          & `end'     \\
b. & boğaz          & boğazı         & boğazın        & `throat'  & (TELL) \\
   & pelür          & pelürü         & pelürün        & `tissue paper' \\
   & döviz          & dövizi         & dövizin        & `currency' \\
   & yamuk          & yamuğu         & yamuğun        & `trapezoid' \\
   & ümit           & ümiti          & ümitin         & `hope'     \\
\end{tabular}
\end{example}

\noindent
Much as was the case for \textsc{Backness Harmony}, there are disharmonic roots which participate in suffix harmony. Once again, underspecification of harmonic roots and full specification of disharmonic roots allows for the use of the \emph{SPE} exception convention. 

\subsubsection{Psycholinguistic evidence}

The only psycholinguistic evidence for \textsc{Roundness Harmony} comes from \citeauthor{Harrison2001}'s language game (\ref{redupgame}). As shown above, the second vowel in the reduplicated form of \emph{bütün} `whole' is \emph{bütün-batın}. The second vowel in the base, \emph{ü}, is reharmonized as [$+$\textsc{Back}, $-$\textsc{Round}], indicating that \textsc{Roundness Harmony} also participates in reharmonization.
