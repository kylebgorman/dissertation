\subsection{Backness harmony}

\citeauthor{Lees1966b} (\citeyear[][35]{Lees1966b}, \citeyear[][284]{Lees1966a}) models the Turkish vowel harmony systems by means of three feature spreading rules. Backness harmony is the the most general of these processes.  

\subsubsection{Phonological description}

\textsc{Backness Harmony} spreads the backness specification of a vowel rightward ignoring any intervening consonants.

%\citep[after][229]{Clements1982}: \\
\begin{example}
\textsc{Backness Harmony}:

\xymatrix@R=24pt@C=24pt{
\txt{V}                                        & \txt{C$_0$~~~~~} & \txt{V} & \txt{(condition: rightward application)} \\
\txt{[α \textsc{Back}]}\ar@{-}[u]\ar@{--}[urr] \\
}
\end{example}

This rule accounts for the tendency of polysyllablic roots to either contain all [$+$\textsc{Back}] or all [$-$\textsc{Back}];
%, as can be seen in the nominative singulars in (\ref{turknom}b); 
this tendency is quantified in \S\ref{lexstats} below. \textsc{Backness Harmony} also accounts for the allomorphs of several inflectional suffixes. For instance, the nominative plural (nom.pl.) suffix is \emph{-ler}, when the final root vowel is [$-$\textsc{Back}], and \emph{-lar} when it is [$+$\textsc{Back}]. 

\begin{example}
\label{turknom}
The Turkish nominative:

%\vspace{0.5\baselineskip}
\begin{tabular}{l l l l l}
   & \emph{nom.sg.} & \emph{nom.pl.} \\
a. & ip             & ipler          & `rope' & \citep[][216]{Clements1982} \\
   & el             & eller          & `hand'    \\
   & köy            & köyler         & `village' \\
   & yüz            & yüzler         & `face'    \\
   & kız            & kizlar         & `girl'    \\
   & sap            & saplar         & `stalk'   \\
   & son            & sonlar         & `end'     \\
   & pul            & pullar         & `stamp'   \\
b. & neden          & nedenler       & `reason'  & (TELL) \\ % front
   & kiler          & kilerler       & `pantry'           \\ % front
   & pelür          & pelürler       & `tissue paper'     \\ % back
   & boğaz          & boğazlar       & `throat'           \\ % back
   & sapık          & sapıklar       & `pervert'          \\ % back
\end{tabular}
\end{example}

There are a few complications, however. First, not all polysyllabic roots conform to \textsc{Backness Harmony}. In this case, as shown in (\ref{turkexcept}a), suffix vowels generally harmonize with the final root vowel. There is also a very small class of nouns, shown in (\ref{turkexcept}b), which take \emph{-ler} despite the fact that their final root vowel is [$+$\textsc{Back}].

\begin{example}
\label{turkexcept}
Exceptional Turkish nominatives:

%\vspace{0.5\baselineskip}
\begin{tabular}{l l l l l}
   & \emph{nom.sg.} & \emph{nom.pl.} \\
a. & mezar          & mezarlar       & `grave' & \citep{TELL} \\
   & model          & modeller       & `model' \\
   & silah          & silahlar       & `weapon'     \\
   & memur          & memurlar       & `bureaucrat' \\
   & sabun          & sabunlar       & `soap'       \\
b. & saat           & saatler        & `hour, clock' \\
   & harf           & harfler        & `(alphabetic) letter' \\ %& \citep{Goksel2005}
   & etol           & etoller        & `fur stole' \\
\end{tabular}
\end{example}

While it is uncontroversial that the disharmonic suffixes of (\ref{turkexcept}b) are no more than very sporadic exceptions to \textsc{Backness Harmony}, root disharmony has ben the subject of much debate. As disharmonic roots still trigger suffix harmony, \citet[][212, 289]{Anderson1974} and \citet{Iverson1978} propose to separate suffix harmony, an alternation, from a sequence structure constraint governing root harmony. 

The disadvantage of this account is that it introduces a sort of ``duplication'' (in the sense of \citealt{Kenstowicz1977}) of sequence structure and phonological generalizations, differning only in their patterns of exceptionality.
%\citep{Kisseberth1970b} (see also \citealp[][401]{Stanley1967} and \emph{SPE}:382)
However, \citet[][197f.]{Zonneveld1978} observes that the theory of exceptionality proposed in \emph{SPE}
% (p.~374f.)
can account for suffix harmony in disharmonic roots. \citeauthor{SPE} assume that the specification of the target (i.e., the segment or segments to be changed) of a rule \emph{R} must be marked [$+$\emph{R}] by convention. A root or affix which fails to undergo \emph{R} despite otherwise matching the structural description is simply said to be marked [$-$\emph{R}]. In other words, a form is never truly an ``exception'': it simply fails to match an extended structural description including the rule feature. If disharmonic roots are marked [$-$\textsc{Backness Harmony}], then the final vowel of disharmonic roots will still trigger \textsc{Backness Harmony}, since the [$-$\textsc{Backness Harmony}] root is no longer the target but rather the trigger, which is not subject to the [$+$\textsc{Backness Harmony}] requirement.\footnote{The definition of ``target of a rule'' is not obvious in autosegmental rules, but any useful formalization of this notion should, for example, identify a harmonic suffix vowel as the target of \textsc{Backness Harmony}.} With this in place, \textsc{Backness Harmony} is sufficient to account for both root and suffix harmony.

%Once additional source of evidence on root (dis)harmony is inconclusive. There is a small class of bisyllabic words in which the second vowel, always [$+$\textsc{High}, $-$\textsc{Back}], alternates with zero. 

%\ex High-vowel/zero alternations \citep[][243]{Clements1982}: \\
%\begin{tabular}{l l l l}
%   & nom.sg. & gen.sg. \\
%a. & fikir   & fikri  & `idea' \\
%   & hüküm   & hükmün & `judgement' \\
%%  & filim   & filmi & `film' & \citep[][178]{Inkelas2001} \\
%b. & vakit   & vaktin & `time' \\
%   & rahim   & rahmin & `womb' \\
%\end{tabular} \xe
%
%\noindent
%It is possible that \textsc{Backness Harmony} might produce a fluctuating \emph{ı} after root \emph{a}, but this does not obtain (\lastx b). However, this might simply indicate that the fluctuating vowel is epenthetic and that harmony applies before epenthesis (see \citealt{Clements1982} for both sides of this argument), making it less than a counterexample. 

Before moving on, it is necessary to dispense with an alternative analysis originally proposed by \citet{Clements1982} and further developed by \citet{Inkelas1997}. Root vowels exhibit a robust contrast for backness (e.g., \emph{kül} `ash' vs.  \emph{kul} `servant', \emph{kepek} `bran' vs. \emph{kapak} `lid'), whereas harmonic roots are those in which the backness of any remaining vowels is predictable. \citeauthor{Clements1982} propose that these vowels are underspecified for backness, whereas the non-initial vowels of disharmonic roots are fully specified. This is schematized below.

%This is exemplified below in (\ref{spec}).
%(e.g., \emph{deve} `camel' vs. \emph{deva} `medicine', \emph{sene} `year' vs. \emph{sena} `praise'). 
%There is some further evidence that individual vowels may differ in specification for this feature even within individual roots or affixes. For instance, the present continuous suffix has harmony-determined allomorphs \emph{-iyor}, \emph{-üyor}, \emph{-ıyor}, \emph{-uyor}, but the \emph{o} of the suffix is invariant. A similar situation might obtain in Turkish roots. 

\begin{example}
\label{spec}
Underspecification in harmonic roots (after \citealp{Clements1982}):

%\vspace{0.5\baselineskip}
\xymatrix@R=24pt@C=24pt{
\txt{a.} & \txt{harmonic root:~~~~} & \txt{C} & \txt{V} & \txt{C} & \txt{V} \\
&   &    & \txt{[$-$\textsc{Back}]}\ar@{-}[u]\ar@{--}[urr] \\
\txt{b.} & \txt{disharmonic root:} & \txt{C} & \txt{V} & \txt{C} & \txt{V} \\
    &    &         & \txt{[$-$\textsc{Back}]}\ar@{-}[u] & & \txt{[$+$\textsc{Back}]}\ar@{-}[u]
}
\end{example}

Harmonizing suffix vowels will also be underspecified for backness.

Two additional details are needed to complete this analysis. First, there must be some kind of constraint which prevents speakers from positing redundant backness specifications for non-initial vowels in harmonic roots, a constraint on identical adjacent specifications in underlying representations, the lexical ``obligatory contour principle'' \citep[][OCP]{Leben1973}, on surface tonal representations \citep{Goldsmith1976}, or both (\citealp{Leben1978}, \citealp{McCarthy1986}; see \citealp{Odden1986,Odden1988} for criticism). Second, \textsc{Backness Harmony} needs to be prevented from overwriting the [$+$\textsc{Back}] specification of disharmonic roots, one option being the use of a structure preservation condition \citep{Kiparsky1985}. However, any condition which prevents \textsc{Backness Harmony} from overwriting underlying backness specifications will reintroduce the duplication of sequence structure and phonological generalizations; under this analysis, disharmonic roots are no longer exceptional, despite considerable evidence (reviewed below) that they are formally marked in Turkish.\footnote{On the other hand, if one interprets ``harmony'', that is, a single backness specification per root, as the learner's default (as \citealp{Odden1986} proposes for the tonal obligatory contour principle), this account is a mere notational variant of the rule exceptionality account.}
%\footnote{Kie Zuraw (p.c.) suggests that the markedness of disharmonic roots might be attributed to the presence of more structure, namely, multiple specifications for backness.}
I reject the underspecification analysis on these grounds. Rather, I assume a unitary phonological rule of \textsc{Backness Harmony} and assign [$-$\textsc{Backness Harmony}] to disharmonic roots.

\subsubsection{External evidence}
\label{backharmexternal}

While suffix harmony can be inferred from alternations, no evidence has yet been presented to show that speakers internalize in any way tendency for roots to exhibit backness harmony. 

%\begin{quote}
%\ldots{}these generalizations\ldots{}have no observable consequences in the course of the normal use of the language. \citep[][320]{Zimmer1969}
%\end{quote}

Indeed, \citeauthor{PE} adopt the null hypothesis that speakers do not internalize any generalizations about possible or likely sequences in underlying representations.

\begin{quote}
Even if we, as linguists, find some generalizations in our description of the lexicon, there is no reason to posit these generalizations as part of the speaker's knowledge of their language, since they are computationally inert and thus irrelevant to the input-output mapping that the grammar is responsible for. \citep[][18]{PE}
\end{quote}

Several independent pieces of external evidence argue that this is not the case for root harmony in Turkish. 
The adaptation of non-native word-initial onset clusters, discussed by \citet{Clements1982} and \citet{Kaun1999}, is one such case.\footnote{Thanks to Kie Zuraw for bringing this data to my attention.} While some speakers are said to be capable of pronouncing these non-native clusters, in normal speech the cluster is resolved by epenthesis of a [$+$\textsc{High}] vowel. In most cases, the epenthetic vowel matches the following root vowel for backness, suggesting that adapt foreign words in a way that conforms to the root harmony generalization.

%the epenthetic vowel is always [$+$\textsc{High}],
%and in general, it conforms to \textsc{Backness Harmony}.
%For reasons that are 
%and when the first consonant of the cluster is non-dorsal, it shows backness harmony

\begin{example}
Adaptation of initial foreign clusters \citep[][247]{Clements1982}: 

%\vspace{0.5\baselineskip}
\begin{tabular}{l l l l l l}
a. & spiker  & \alt{} & sipiker  & `announcer' \\
   & fren    & \alt{} & firen    & `break'     \\
b. & brom    & \alt{} & burom    & `bromide'   \\
   & prusya  & \alt{} & purusya  & `Prussia'   \\
%b. & grip    & \alt{} & gırip    & `grippe'    \\ % unexpectedly back
%   & kredi   & \alt{} & kıredi   & `credit'    \\
%   & trablus & \alt{} & tırablus & `Tripoli'   \\
%b. & kral    & \alt{} & kıral    & `king'      \\
%   & grup    & \alt{} & gurup    & `group'     \\
\end{tabular}
\end{example}

Similar evidence comes from a language game discussed by \citet{Harrison2001}.\footnote{Thanks to Bert Vaux for bringing this study to my attention.} This game is not indigenous to Turkish, but is identical to a morphological process in the related language Tuvan, in which it conveys a sense of ``vagueness or jocularity'', and \citeauthor{Harrison2001} report that it can be quickly taught to Turkish speakers. The game consists of reduplication of the base and replacement of the first vowel in the reduplicant with the [$+$\textsc{Back}] vowels \emph{a} or \emph{u}. Both in the native Tuvan process and in the Turkish game, the second vowel of the reduplicant (shown in braces below) is affected by this process: when the base is harmonically [$-$\textsc{Back}], the insertion of a [$+$\textsc{Back}] results in what \citeauthor{Harrison2001} call ``reharmonization'', shown in  (\ref{redupgame}a).

\begin{example}
\label{redupgame}
Turkish reduplication game \citep[][231]{Harrison2001}:

\vspace{0.5\baselineskip}
\begin{tabular}{l l l l}
a. & kibrit & kibrit-\{kabrıt\} & `match'    \\
   & bütün  & bütün-\{batın\}   & `whole'    \\
b. & mali   & mali-\{muli\}     & `Mali'     \\
   & butik  & butik-\{batik\}   & `boutique' \\
\end{tabular}
\end{example}

\noindent \citeauthor{Harrison2001} analyze this in terms of the \citeauthor{Clements1982} underspecification analysis, but it is equally consistent with full specification. Reharmonization is simply the spread of the backness specification of the \emph{a} or \emph{u} in the reduplicant. On the other hand, (\ref{redupgame}b) shows that this does not obtain for disharmonic roots; presumably, the [$-$\textsc{Backness Harmony}] exception feature is copied under reduplication as well. Similar results obtain both in Tuvan and in an unrelated Finnish language game \citep{Campbell1986}.

A number of studies on Finnish suggest that speakers of that language, which has a vowel harmony system similar to that of Turkish, make use of (dis)harmony to processing running speech.\footnote{Thanks to Charles Yang for bringing these studies to my attention.} \citet{Suomi1997} and \citet{Vroomen1998} generate nonce trisyllabic words by adding a monosyllabic pseudo-prefix to real and nonce disyllabic words, all of which conform to backness harmony. These stimuli are auditorily presented to subjects who are asked to press a button when the nonce trisyllable ends with a target nonce disyllable, or a real disyllabic word. Speakers are quicker to press the button when the prefix and disyllabic word have different backness specifications. These results suggest that speakers are attuned to the fact that disharmonic transisitions are good predictors of word boundaries. If speakers have also internalized the converse generalization, that harmonic transistions are more likely to be root-internal, then there is additional evidence that harmony is active not just in Finnish affix alternations but also in roots. 

\citet{Kabak2010} report that Turkish \textsc{Backness Harmony} has the same effect in word-spotting tasks as it does in Finnish: speakers are quicker and more accurate at the task of spotting the nonce target word \emph{pavo} when preceded by the pseudo-prefix \emph{gölü-}, a disharmonic transistion, than when it is preceded by the pseudo-prefix \emph{golu-}, a harmonic transition. \citeauthor{Kabak2010} find that effect of harmony does not obtain for speakers of French, a language which lacks vowel harmony. As in Finnish, the results imply speakers have internalized the predominance of root-internal harmony.

It seems that the root-internal harmony bias is in fact learned by Turkish speakers very early. The pseudoword spotting experiment has been adapted for 9-month-old Turkish infants by \citet{Kampen2008}. Infants are familiarized with harmonic disyllabic pseudowords bearing a pseudo-prefix, which may be harmonic or disharmonic. At test time, the infants are played the disyllabic pseudowords in isolation using the head turn preference paradigm. Infants show a preference to listen to those pseudowords which were familiarized with a disharmonic pseudo-prefix over those which were familiarized with a harmonic pseudo-prefix. This preference is not observed in 9-month-old infants learning German, which also lacks vowel harmony. Similarly, \citeauthor{Kampen2008} report that Turkish 6-month-old infants prefer to listen to harmonic pseudowords such as \emph{paroz} over disharmonic pseudowords like \emph{nelok}, but German 6-month-old infants show no such preference.
