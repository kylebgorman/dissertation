\subsection{Backness harmony}

One of the most salient aspects of spoken Turkish is the presence of vowel harmony. To a first approximation, 
the high vowels of a Turkish prosodic word are all either all round or all unround. Here we will focus on backness harmony

vowels agree 

all the vowels in a Turkish prosodic words agree for backness, and in addition, high vowels agree in roundness.

\subsubsection{Phonological description}

\citeauthor{Lees1966a} (\citeyear[][284]{Lees1966b}, \citeyear[][35]{Lees1966a}) proposes a number of phonological rules to describe the harmony processes that act on roots and regularly spread rightward to suffixes (Turkish has no prefixes); one such rule consists of rightward spreading of the feature [\textsc{Back}].

\ex \textsc{Backness Harmony} \citep[e.g.,][229]{Clements1982}: \\
\xymatrix@R=12pt{
\txt{[α \textsc{Back]}}\ar@{-}[d]\ar@{..}[drr] &             \\
\txt{V}                                        & \txt{C$_0$} & \txt{V}}
\xe

%\citet{Clements1982}
%\citet{Inkelas1997}

This rule accounts for much of the suffix allomorphy found in Turkish. The dat.sg. (and the vowel of the gen.sg.) is always a [$+$\textsc{High}] vowel, and in general, it agrees with the final stem vowel for backness and roundness. Similarly, the vowel of the nom.pl. is [$-$\textsc{High}, $-$\textsc{Round}], and takes its backness from the stem-final vowel. Similar patterns obtain for many other Turkish suffixes.

\ex Turkish nominal suffix allomorphy: \vspace{6pt} \\
\begin{tabular}{l l l l l l l}
   & \emph{nom.sg.} & \emph{dat.sg.} & \emph{gen.sg.} & \emph{nom.pl.} \\
a. & ip             & ipi            & ipin           & ipler          & `rope' & (C\&S:216) \\
   & el             & eli            & elin           & eller          & `hand'    \\
   & kız            & kızı           & kızın          & kilzar         & `girl'    \\
   & sap            & sapı           & sapın          & saplar         & `stalk'   \\
   & yüz            & yüzü           & yüzün          & yüzler         & `face'    \\
   & köy            & köyü           & köyün          & köyler         & `village' \\
   & pul            & pulu           & pulun          & pullar         & `stamp'   \\
   & son            & sonu           & sonun          & sonlar         & `end'     \\
b. & neden          & nedeni         & nedenin        & nedenler       & `reason'  & (TELL) \\
   & boğaz          & boğazı         & boğazın        & boğazlar       & `throat'  \\
   & kiler          & kileri         & kilerin        & kilerler       & `pantry'  \\
   & pelür          & pelürü         & pelürün        & pelürler       & `tissue paper' \\
   & sapık          & sapığı         & sapığın        & sapıklar       & `pervert' \\
\end{tabular} \xe

\noindent
If the vowel of the nom.pl. suffix is specified as [$-$\textsc{High}, $-$\textsc{Round}], then the \emph{a} $\sim$ \emph{e} pattern follows naturally from \textsc{Backness Harmony}. If the dat.sg. suffix is a vowel specified only as [$+$\textsc{High}], then \textsc{Backness Harmony}, along with the rightward spreading of \textsc{Round}, will produce the correct results.

There are a few complications, however. First, while most polysyllabic roots conform to \textsc{Backness Harmony} (\lastx b), not all do. In this case, suffix vowels generally harmonize with the final root vowel (\nextx a). Finally, here is a small class of nouns which exhibit [$-$\textsc{Back}] suffix vowels despite the fact that their final root vowel is [$+$\textsc{Back}] (\nextx b).

\ex Turkish nominal suffix allomorphy: \vspace{6pt} \\
\begin{tabular}{l l l l l l l}
   & \emph{nom.sg.} & \emph{dat.sg.} & \emph{gen.sg.} & \emph{nom.pl.} \\
a. & mezar          & mezarı         & mezarın        & mezarlar       & `grave' & (TELL) \\
   & model          & modeli         & modelin        & modeller       & `model' \\
   & silah          & silahı         & silahın        & silahlar       & `weapon'     \\
   & memur          & memuru         & memurun        & memurlar       & `bureaucrat' \\
   & sabun          & sabunu         & sabunun        & sabunlar       & `soap'       \\
b. & saat           & saati          & saatin         & saatler        & `hour, clock' \\
   & harf           & harfi          & harfin         & harfler        & `(alphabetic) letter' \\ %& \citep{Goksel2005}
   & etol           & etolü          & etolün         & etoller        & `fur stole' \\
\end{tabular} \xe

While it is uncontroversial that disharmonic suffixes (\lastx b) constitute no more than sporadic exceptions to \textsc{Backness Harmony}, the status of root disharmony (\lastx a) has been the subject of much debate. 

1. \citet[][289]{Anderson1974} 
%2. \citet[][]{Kiparsky1981} makes a similar objection regarding Finnish vowel harmony.
3. \citet{Clements1982}
4. \citet{Inkelas1997}

have identified (\lastx b) as sporadic exceptions to 
A number of objections can be made for 

\subsubsection{Lexical statistics}

\citet{TELL}
%\citet{Inkelas2001}

\ex Lexical effects of \textsc{Obstruent Voice Assimilation}: \vspace{6pt} \\
\begin{tabular}{l r r r r}
\toprule
           & attested & unattested & saturation & $p$-value \\
\midrule
conforming & 80 & 370 & 0.178 & \multirow{2}{*}{1.1\e{-11}}\\
violating  &  6 & 264 & 0.022 \\
\bottomrule
\end{tabular} \xe 

\subsubsection{External evidence}

The data reviewed thus far leaves open the possibility that speakers do not internalize the generalization that roots conform to \textsc{Backness Harmony}. One piece of evidence against this hypothesis comes from a Turkic language game \citep{Harrison2001}.\footnote{Thanks to Bert Vaux for bringing this study to my attention.} This game is not indigenous to Turkish, but it corresponds to a morphological process native to the related langauge Tuvan, and is quickly taught to even young Turkish speakers. The process consists of reduplication of the base with the base-initial vowel replaced by \emph{a} or \emph{u}, which conveys a sense of ``vagueness'' or ``jocularity''. Reduplication interacts with root harmony in both Tuvan and Turkish. When the base is harmonically [$-$\textsc{Back}], the non-initial vowels of the reduplicant do not surface faithfully: they surface as [$+$\textsc{Back}] (\nextx a).

\ex Turkish reduplication game \citep[][231]{Harrison2001}: \\
\begin{tabular}{l l l l}
a. & kibrit & kibrit-kabrıt & `match'    \\
   & bütün  & bütün-batın   & `whole'    \\
b. & mali   & mali-muli     & `Mali'     \\
   & butik  & butik-batik   & `boutique' \\
\end{tabular} \xe

\noindent
Interestingly, reharmonization is absent in disharmonic bases (\lastx b). This is not an isolated fact about Turkic, as \citeauthor{Harrison2001} note similar outcomes in an unrelated Finnish language game \citep{Campbell1986}. The simplest explanation of this contrast is to treat \textsc{Backness Harmony} as a feature-filling process, as proposed by \citet{Clements1982} and \citet{Inkelas1997}. In harmonic roots, the non-initial vowels are underspecified for \textsc{Back} and are spelled out by \textsc{Backness Harmony}. In reduplication, this produces reharmonization. In disharmonic roots, the vowels are fully specified and do not meet the structural description of \textsc{Backness Harmony}. 

A number of studies have found that speakers of Finnish use disharmony as a clue to segment running speech.\footnote{Thanks to Charles Yang for pointing out the relevance of these studies to me.} \citet{Suomi1997} and \citet{Vroomen1998} generate nonce trisyllabic words by adding a monosyllabic pseudo-prefix to real and nonce disyllabic words, all of which are harmonic for the feature \textsc{Back}. These stimuli are auditorily presented to subjects who are asked to press a button when the nonce trisyllable ends with a target nonce disyllable, or a real disyllabic word. Speakers are quicker to press the button when the prefix and disyllabic word disagree for \textsc{Back}. This is taken to indicate that speakers are attuned to the fact that disharmonic transistions are good predictors of word boundaries. On the other hand, if this also indicates that speakers recognize that harmonic transistions are more likely to be root-internal, then harmony is active not just where it is triggered by alternations (i.e., in Finnish affixes) but also in roots. 

\citet{Kabak2010} report that Turkish \textsc{Backness Harmony} has the same effect on word-spotting as it does in Finnish: speakers are quicker and more accurate at the task of spotting the nonce target word \emph{pavo} when it preceded by the disharmonic pseudo-prefix \emph{gölü-} than when it is prefixed with harmonic nonce prefix \emph{golu-}. The effect of harmony does not obtain for speakers of French, which lacks vowel harmony. As in Finnish, the results imply speakers have internalized the predominance of root-internal harmony.

It seems that root-internal harmony is not only present in adults, but in fact is learned very early. The pseudoword spotting experiment has been adapted for 9-month-old Turkish infants by \citet{Kampen2008}. Infants are familiarized with harmonic disyllabic pseudowords bearing a pseudo-prefix, which may be harmonic or disharmonic. At test time, the infants are played the disyllabic pseudowords in isolation using the head turn preference paradigm. Infants show a preference to listen to those pseudowords which were familiarized with a disharmonic pseudo-prefix over those which were familiarized with a harmonic pseudo-prefix. This preference is not observed in 9-month-old infants learning German, which lacks vowel harmony. Similarly, \citeauthor{Kampen2008} report that Turkish 6-month-old infants prefer to listen to harmonic pseudowords such as \emph{paroz} over disharmonic pseudowords like \emph{nelok}, but German 6-month-old infants do not.

%We might suggest methods not based only on statistical significance, but rather substantive computational properties of the linguistic objects in question \citep[e.g.,][]{Yang2005a}.
