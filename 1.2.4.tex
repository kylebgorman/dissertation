% 1.2.4: Universal constraints?
% This may be cut.

(see \citealt[][323f.]{Hockett1947}, \citealt[][415f.]{Nida1948}, \citealt[][50f.]{Anderson1992}, \citealt[][36f.]{Stump2001a}). Irrespective of one's position on this debate, it does not seem correct to extend this hypothesis to lexical roots. For instance, there  

imperfect subjunctive  & īrem     & īrēs     & īret     & īrēmus     & īrētis     & īrent \\
pluperfect subjunctive & issem    & issēs    & isset    & issēmus    & issētis    & issent \\
imperfect subjunctive  & venīrem  & venīrēs  & venīret  & venīrēmus  & venīrētis  & venīrent \\
pluperfect subjunctive & vēnissem & vēnissēs & vēnisset & vēnissēmus & vēnissētis & vēnissent \\

That this is not the 

While it might be possible to analyze \emph{ueni:re} as ``athematic'' /weni:-re/ and \emph{i:re} as /i:-re/, the frequentive \emph{uentita:re} `come often, be wont to come', formed with the \emph{-tit-} frequentive suffix (see \citet[][\S263]{Allen1903}), which selects for the first conjugation (cf.~\emph{agere} `act, make' vs. \emph{actita:re} `act, make often/repeatedly') suggests /wen-/.

It is also interesting to note that the same pattern has emerged in the history of French for a set of verbs which do not share this property in Latin. and the future and conditional indicative forms of \emph{aller} `go' and \emph{cuire} `to cook' in French.\footnote{\emph{Cuire}, interestingly, is not descended from the same conjugation as Latin \emph{i:re}, but rather from Latin \emph{coquere}; thus this syncretism is not simply an etymological relic of Latin. The same holds for other French verbs that inflect in the same manner, such as \emph{conduire} `drive (a vehicle); behave' and \emph{d\'etruire} `destroy'.}

future indicative      & irai   & iras   & ira      & irons    & irez    & iront \\
conditional indicative & irais  & irais  & irait    & irions   & iriez   & iraient \\
future indicative      & cuirai & cuiras  & cuira   & cuirons  & cuirez  & cuiront \\
conditional indicative & curias & cuirais & cuirait & cuirions & cuiriez & cuiront \\

One could even conceive of a largely vacuous principle on morpheme structure which simply requires that some subset of morphs not be null. 

%Anttila 2008
%Dmitrieva et al. 2008a,b
%Duanmu 2009
%Berkley 1994a,b, 2000
%Buckley 1997
%Coetzee 2008, Coetzee and Pater 2008
%Goad 2011
%Graff & Jaeger in press,
%Hammond 1999
%Colavin 2010,
%Frisch 1996, Frisch et al. 2004
%Hayes & Wilson 2008
%Kessler et al. 1997
%Martin 2007
%McGowan 2011
%Mester 1986
%Fudge 1969
%Padgett 1992
%Pierrehumbert 1993, 1994
