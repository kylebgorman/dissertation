\section{Conclusions}

\citet{Harrison2001}

\ex Turkish reduplication game \citep[][231]{Harrison2001}: \\
\begin{tabular}{l l l l}
a. & kibrit & kibrit-kabrıt & `match'    \\
   & bütün  & bütün-batın   & `whole'    \\
b. & mali   & mali-muli     & `Mali'     \\
   & butik  & butik-batik   & `boutique' \\
\end{tabular} \xe

%Reduplicate and replace the reduplicant-initial vowel with [a]
this ludling, while not in present in Turkish, is found in Tuvan.

The failure of disharmonic stems to ``reharmonize'' in language games is not limited to this particular artificial Turkish langauge game, either: Tuvan disharmonic roots also fail to disharmonize \citep{Harrison2001}, and, as \citeauthor{Harrison2001} notes, also for a different language game in Finnish \citep{Campbell1986}. 

\citet{Suomi1997} and \citet{Vroomen1998} find that Finnish adults
% use harmony to segment

\citet{Kabak2010} % turksih adults use harmony to segment

\citet{Kampen2008} % Turkish L1 (german l2) speakers segment using harmony, but their L1 German colleagues do not

%We might suggest methods not based only on statistical significance, but rather substantive computational properties of the linguistic objects in question \citep[e.g.,][]{Yang2005a}.


% FORMER 3.3.1: Naturalness

\citet{Becker2011}

This has an ad hoc nature to it; 
There are two senses in which this objection is ad hoc. 
First, \citeauthor{Becker2011} appeal to no particular theory of the naturalness of processes or static constraints which excludes \textsc{Labial Attraction}. 
Secondly, this appears to be a minority view: \textsc{Labial Attraction} was considered a true generalization by early specialists
\citep[e.g.,][]{Lees1966a}, and despite \citeauthor{Zimmer1969}'s suggestive psycholinguistic results, reviewed above, it also been treated as a plausible constraint by later theorists \citep[e.g.,][]{NiChiosain1993,Ito1993,Ito1995a,Zuraw2000}.
Further, there is a real danger that if the label ``unnatural'' 
%Labial Attraction} 
describes an impossible structural change or structural description, that one will fail to account for earlier forms of Turkish or sound changes therein.

%Classical Arabic adjectives often have stative verbs in which the root is imposed onto the template CaCuCa:

%\ex Arabic verbs of coming into being: \\
%\begin{tabular}{r l l l}
%a. & kabura & `become big'       & (cf. \emph{kabiːr} `big') \\
%b. & saʁiir & `become small'     & (cf. \emph{saʁiːr} `small') \\
%c. & ħasuna & `become beautiful' & (cf. \emph{ħasan} `handsome') \\
%\end{tabular}
%\xe 


% FORMER 3.3.2: Diachronic factors 
 
% etymological issues
\citet{Inkelas2001}

\ex \textsc{Labial harmony} and etymology in TELL \citep[][187]{Inkelas2001}: \vspace{6pt} \\
\begin{tabular}{l r r r}
        & aPu & aPı & $p$-value \\
Native  & 12  & 11  & \multirow{2}{*}{0.042} \\
Foreign & 84  & 28  \\
\end{tabular}
\xe 

\citet{NiChiosain1993} and \citet{Ito1995b} 



``sonority projection''
Ther

\citet{Daland2011b} finds that this can be learned easily from positive data but a number of different psych
