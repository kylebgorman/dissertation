% 3.3.1: Naturalness
\subsection{Naturalness}

\citet{Becker2011}

This has an ad hoc nature to it; 
There are two senses in which this objection is ad hoc. 
First, \citeauthor{Becker2011} appeal to no particular theory of the naturalness of processes or static constraints which excludes \textsc{Labial Attraction}. 
Secondly, this appears to be a minority view: \textsc{Labial Attraction} was considered a true generalization by early specialists
\citep[e.g.,][]{Lees1966a}, and despite \citeauthor{Zimmer1969}'s suggestive psycholinguistic results, reviewed above, it also been treated as a plausible constraint by later theorists \citep[e.g.,][]{NiChiosain1993,Ito1993,Ito1995a,Zuraw2000}.
Further, there is a real danger that if the label ``unnatural'' 
%Labial Attraction} 
describes an impossible structural change or structural description, that one will fail to account for earlier forms of Turkish or sound changes therein.

%Classical Arabic adjectives often have stative verbs in which the root is imposed onto the template CaCuCa:

%\ex Arabic verbs of coming into being: \\
%\begin{tabular}{r l l l}
%a. & kabura & `become big'       & (cf. \emph{kabiːr} `big') \\
%b. & saʁiir & `become small'     & (cf. \emph{saʁiːr} `small') \\
%c. & ħasuna & `become beautiful' & (cf. \emph{ħasan} `handsome') \\
%\end{tabular}
%\xe 
