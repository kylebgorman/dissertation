\subsection{Static and derived constraints}

With this database of English syllable contact clusters it is now possible to evaluate constraints on sequences of underlying phonemes. 


Here, the metric used is simply 

The evaluation 

The foremost evidence for 


The evaluation used here is simply 

Having described the database of English syllable contact clusters, 



I now consider constraints on sequences of URs as predictors of attested and unattested contact clusters. First, I describe the statistical technique used, then apply this technique to both phonologically derived constraints and a set of ``static'' phonotactic constraints proposed by \citet{Pierrehumbert1994}. 

\subsubsection{Statistical technique}
\label{stattech}

Before the dawn of generative phonology, many linguists concerned themselves with documenting co-occurrence in various languages. \citet[][28]{Vogt1954} declares that a ``phonemic description of a language \ldots should also comprise a description of the phoneme combinations that occur or may be expected to occur in the language''. Studies in this tradition include the discussions of consononantal co-occurrence in Arabic and Javanese by \citet{Greenberg1950} and \citet{Uhlenbeck1950}, respectively. \citeauthor{Uhlenbeck1950}, for instance, puts forth co-occurrence dispreferences, rather than exceptionless generalizations of the sort given by \citet{Greenberg1950}. Years later, \citet{Mester1988} revisits \citeauthor{Uhlenbeck1950}'s study and proposes a statistical technique, based on the chi-square test, to distinguish between accidental and structural lexical dispreferences in the Javanese lexicon. 

\citeauthor{Mester1988}'s technique has been adopted by many subsequent studies (e.g., \citealt{Padgett1992,Padgett1995} on Russian; \citealt{Pierrehumbert1993}, \citealt{Frisch1996}, and \citealt{Frisch2004} on Arabic; \citealt{Berkley1994b,Berkley1994a,Berkley2000}, \citealt{Dmitrieva2008a}, \citealt{Dmitrieva2008b} on English). The null hypothesis $H_0$ is that some sequence occurs at no different a rate than the researcher expects, and the alternative hypothesis $H_1$ is that it is underrepresented. Given an observed frequency $O$ and also a frequency $E$ expected under the null hypothesis, the test is as follows.

\begin{example}
$\displaystyle \textrm{Reject } H_0 \textrm{ iff } \frac{(O E) ^ 2}{E} >
χ^2$
\end{example}

%\footnote{This is the one-tailed version of the test.}
  
In this study, I use the \citet{Fisher1934} exact test, which is isomorphic to this chi-square test. Compared to the chi-square test, the Fisher exact test $p$-value is somewhat more difficult to compute, but both are automated by modern statistical packages and can be computed rapidly. The major advantage of the Fisher test over the chi-square test is that it is appropriate for small samples or rare events, whereas the chi-square test depends on an approximation which is only exact in the limit (i.e., with a hypothetically infinite sample), and therefore is not appropriate for small samples \citep[see][]{Gorman2012a}.
%\citep[e.g.,][142]{Mester1988}.

On the other end of the spectrum, the Monte Carlo statistical procedure developed by \citet{Kessler2001} and applied to co-occurrence statistics by \citet{Martin2007,Martin2011} and \citet{Brown2010} does not scale to large samples. The Monte Carlo procedure is isomorphic to the one-tailed Fisher Exact test, and involves comparing observed co-occurrence counts to those produced by randomly generated samples.  This requires the researcher to generate random permutations, something which is particularly difficult for large samples, as I will explain; disinterested readers may wish to skip the following paragraph.

Any permutation of a sequence $L$ can be represented as a sequence $S$ of the same length, where the value of $S_i$ corresponds to the permuted position of $L_i$. Generating random permutations requires an unbiased method to generate all $N!$ of these lists with equal probability. Unfortunately, this is near impossible for $N$s of moderate size with current computational resources. Any pseudo-random number generator can be characterized by its ``period'', the number of pseduo-random numbers it can emit before repeating itself; this is also the upper limit for the number of random permutations that can be generated. The pseudo-random number generator used by both \citeauthor{Martin2011} and \citeauthor{Brown2010} has a period of approximately $2^{48}$, a number which turns out to be smaller than $17!$. For the large samples, sometimes in the thousands, used by these researchers, virtually all the possible permutations can never be generated by this method. As a consequence, the shuffling procedure is far from random and introducing the possibility that the statistical test is biased in unpredictable ways.

As an example of the Fisher exact test as it is applied here, consider some of the Javanese co-occurrence facts considered by \citet{Mester1988}. \citeauthor{Mester1988} is at pains to show that there is a gradient dispreference for first and second root consonants which share the same major place. However, perfectly identical first and second consonants are exempt. Since 


% APPLY THIS TO JAVANESE

\begin{example}
Javanese root co-occurrence (\citealp[][264]{Uhlenbeck1950}, \citealp[][139]{Mester1988}): 

\vspace{0.5\baselineskip}
\begin{tabular}{l r r r r}
\toprule
           & attested & unattested & saturation & $p$-value \\
\midrule
conforming & 134      & 12         & 0.918      & \multirow{2}{*}{6.095\e{-06}} \\
violating  & 21       & 15         & 0.583 \\
\bottomrule
\end{tabular}
\end{example}

\noindent
It is evident that roots with identical labial stops are far more common than those in which the roots disagree in voicing. 

This data is input to a chi-square test, which reports that this highly skewed distribution is unlikely to be due to chance 

That this is highly unlikely to be due to chance is apparent both from the chi-square ($\chi^2 = 25.572$, two-tailed $p = 4.3$\e{-07}) and Fisher exaact tests ($p =  6.095$\e{-06}). 

I use two-tailed statistical tests throughout. 
Under a one-tailed test, the alternative hypothesis is underrepresentation, whereas a two-tailed test also allows the possibility that a class of clusters would be overrepresented, i.e., more frequent than predicted by the null hypothesis. As 

\citet{Pierrehumbert1994}

It must be stressed that under current assumptions, this is independent of whether predictable prosodic representations (particularly coda and onset identity) are part of the memorized underlying representation \citep[e.g.,][]{Vaux2003}.
It is an undeniable fact that URs need to be 
The derivative nature of 
\citet{Ito1989a,Noske1992}

least some researchers (i.e., \citealt{Mester1988}) 

 \citep[e.g.,]{Brown2010} have argued for phonotactic constraints giving rise to overrepresentation, so I admit this possibility here and use two-tailed tests throughout. 

%\footnote{For example, \citet{Martin2007} applies this technique to a list of 4,758 noun-noun compounds. The number of permutations of a list of this length can be estimated using Stirling's approximation.

%\ex $ \lim_{n \rightarrow \infty} n! = e ^ {n \ln{n} - n} = 2 ^ {\log_2{e} (n \ln{n} - n)} $ \xe

%\noindent
%$4,758!$ is approximately $2 ^ {51260}$. The author is unaware of any system that provides 51,260 bits of entropy; and some popular programming languages, such as ANSI C, provide as few as 15 bits.}

%%% START OF SERIOUSNESS

I will first consider three well-known English phonological alternations, showing that these give rise to constraints on the sequence structure of morphs. Simple autosegmental formalizations of these processes are provided, and the Fisher exact test is used to demonstrate their effect on the English lexicon.

\subsubsection{Obstruent Voice Assimilation}

English has a process of voice assimilation which is found in two inflectional suffixes. The suffix /-d/ forms regular pasts and denominal adjectives, and /-z/ forms regular noun plurals and genitives, and the 3sg. present active indicative verbs.\footnote{The URs /-d, -z/ are argued for by \citet[][282]{Hockett1958}, \citet[][210]{SPE}, \citet{Basboll1972}, \citet{Shibatani1972}, S. \citet[][]{Anderson1973a}, \citet[][102]{Pinker1988}, and \citet[][284f.]{Bakovic2005b}, and alternative views are presented by \citet[][210f.]{LANGUAGE}, \citet[][426]{Nida1948}, \citet{Luelsdorff1969}, \citet{Lightner1970}, \citet{Hoard1971}, \citet[]{Miner1975}, \citet{Zwicky1975}, \citet{Kiparsky1985}, and \citet[][135]{Borowsky1986}.}

\begin{example}
Voice assimilation in the past tense and noun plural:

\vspace{0.5\baselineskip}
\begin{tabular}{l l l l l l}
a. & /næp-d/     & \goesto & [næpt]     & (cf. \emph{nab}[d], \emph{haul}[d]) \\
b. & /læp-z/     & \goesto & [læps]     & (cf. \emph{lab}[z], \emph{clan}[z]) \\
\end{tabular}
\end{example}

\noindent
The voiced obstruent suffixes devoice when preceded by a voiceless obstruent.
This is formalized as a rule spreading the [\textsc{Voice}] specification rightward over adjacent obstruents.

\begin{example}
\textsc{Obstruent Voice Assimilation} (\emph{SPE}:178): 

\xymatrix@R=24pt@C=24pt{
\txt{[α \textsc{Voice}]}\ar@{-}[d]\ar@{..}[dr] & \\
\txt{C}\ar@{-}[d]             & \txt{C}\ar@{-}[d] \\
\txt{[$+$\textsc{Obstruent}]} & \txt{[$+$\textsc{Obstruent}]}\\
}
\end{example}

\noindent
This rules out the possibility of surface clusters of obstruents with differing  [\textsc{Voice}] specifications. While the data make it clear \textsc{Obstruent Voice Assimilation} applies in morphologically derived environments, it is not obvious that it applies to obstruent clusters in non-derived environments (e.g., obstruent clusters in the URs of morphs) as well. \citet{Pierrehumbert1994} does not mention voice assimilation in her study of gaps in English syllable contact clusters. \citet[][74f.]{Hammond1999a} notes the existence of words like \emph{a}[b.s]\emph{inth} and \emph{a}[s.b]\emph{estos}, which contain obstruent voice clusters disagreeing in [\textsc{Voice}], and takes this as evidence that obstruent voicing is restricted to derived environments.

Despite these exceptions, I propose that \textsc{Obstruent Voice Assimilation} plays a major role in shaping the English lexicon. Only English syllable contact clusters which contain two adjacent obstruents speak to the status of \textsc{Obstruent Voice Assimilation}; the remaining 160 possible clusters are ignored. A $2 \times 2$ contingency table is constructed by dividing the remaining possible clusters into those which agree on obstruent voicing and those which do not, and into those which are attested and those which are not.

\begin{example}
Lexical effects of \textsc{Obstruent Voice Assimilation}: 

\vspace{0.5\baselineskip}
\begin{tabular}{l r r r r}
\toprule
           & attested & unattested & saturation & $p$-value \\
\midrule
conforming & 80 & 370 & 0.178 & \multirow{2}{*}{1.1\e{-11}}\\
violating  &  6 & 264 & 0.022 \\
\bottomrule
\end{tabular}
\end{example}

\noindent 
18\% of the possible clusters containing adjacent obstruents with the same [\textsc{Voice}] specification are found, whereas only 2\%  of those which disagree in [\textsc{Voice}], such as in \emph{a}[b.s]\emph{inth} and \emph{a}[s.b]\emph{estos}, CELEX also contains \emph{jo}[d.p]\emph{urs}, \emph{pi}[ntʃ.b]\emph{eck}, \emph{sa}[k.b]\emph{ut}, and \emph{ja}[k.d]\emph{aw}. As the $p$-value shows, the rarity of these disagreeing clusters is unlikely to be due to chance. 

I propose that this pattern should be attributed directly to \textsc{Obstruent Voice Assimilation} applying to non-derived environments, preventing learners from positing URs in which adjacent obstruents do not agree in voice. There is no duplication between constraints on sequence structure and surface forms if both are derived from the application of the phonological rule (\citealt[][401f.]{Stanley1967}, \citealt[][382]{SPE}). The handful of exceptions to this generalization can be marked with appropriate exception diacritics. Another possibility is suggested by recent psycholinguistic results. \citet{Mattys2001b} find that 9-month old infants are sensitive to the contrast between [vt], a cluster of obstruents disagreeing in [\textsc{Voice}] which is not found word-medially, and [ft], which conforms to the above generalization and is found medially (e.g. \emph{a}[f.t]\emph{er}). This ability to use phonotactic violations to segment continuous speech persists into adulthood \citep{McQueen1998}, suggesting that ``hetero-voiced'' obstruent clusters may be analyzed as semantically opaque compounds by native speakers (e.g., \emph{ja}[k\#d]\emph{aw}). 
%\footnote{An analogous account is the proposal that readjustment rules insert morphological boundaries for the purposes of assigning stress (\emph{SPE}:94).}

\subsubsection{Coda nasal place assimilation}
\label{cnpasection}

Above, it was assumed that the velar nasal is derived from underlying /n/. 
This analysis is not purely allophonic, as nasal place assimilation alternations target the final consonant of the \emph{im-}/\emph{in-} prefix. In (\ref{npa}ab), the prefix-final consonant takes on the same major place of articulation as the following obstruent, and in (\ref{npa}c), where the root is vowel-initial, the prefix-final consonant is realized as a default /n/.

\begin{example}
\label{npa}
Nasal place assimilation in Latinate prefixes:

\vspace{0.5\baselineskip}
\begin{tabular}{l l l l}
%a. & possible & i[m.p]ossible & & balance & i[m.b]alance \\
%b. & tangible & i[n.t]angible & & decent  & i[n.d]ecent  \\
%c. & elegant  & i[n.]elegant  & & ability & i[n.]ability \\
a. & possible & i[m.p]ossible \\
   & balance  & i[m.b]alance  \\
b. & tangible & i[n.t]angible \\
   & decent   & i[n.d]ecent   \\
c. & elegant  & i[n.]elegant  \\
   & ability  & i[n.]ability  \\
\end{tabular} 
\end{example}

\noindent
The shape of the prefix before other obstruents is somewhat more complex. 
%Before /f/, CELEX transcribes the nasal as either labial or alveolar, the latter being very rare in simplex words. 
%There are also additional complexities regarding the shape of the prefix when attached to /k, g/-initial roots. 
\citet[][62]{Halle1985a} and \citet[][90]{Borowsky1986} report that coda nasals assimilate to [ŋ] before dorsal consonants, but this assimilation is blocked by stress on the following syllable, citing contrasts like the one between \emph{í}[ŋ.k]\emph{ubate} and \emph{i}[n.k]\emph{lúde}. The prefix \emph{com-}/\emph{con-}, which patterns with \emph{im-}/\emph{in-} in showing assimilation (e.g., \emph{co}[m.b]\emph{ine}), participates in a similar contrast, between \emph{có}[ŋ.g]\emph{ress} and \emph{co}[n.g]\emph{réssional}. As \citeauthor{Borowsky1986} notes, stress does not block assimilation in simplex words (e.g., \emph{a}[ŋ.g]\emph{óra}), a fact which demonstrates the necessity of a partially morphological explanation for the blocking of assimliation in words like \emph{co}[n.g]\emph{réssional}. Regardless of these complexities, there is sufficient evidence to posit assimilation of the \textsc{Place} features of an obstruent to a preceding nasal consonant. 

\begin{example}
\textsc{Coda Nasal Place Assimilation} (\emph{SPE}:85):

\xymatrix@R=24pt@C=24pt{
                          & \txt{\textsc{Place}}\ar@{-}[d]\ar@{..}[dl] \\
\txt{C}\ar@{-}[d]         & \txt{C}\ar@{-}[d] \\
\txt{[$+$\textsc{Nasal}]} & \txt{[$+$\textsc{Obstruent}]} \\
}
\end{example}

\citet[][175]{Pierrehumbert1994} observes that in simplex English words, ``nasal-stop sequences agree in labiality''; presumably, this generalization is limited to [\textsc{Labial}] because \citeauthor{Pierrehumbert1994} also adopts the assumption that the velar nasal is derived from /n/. However, \citeauthor{Pierrehumbert1994} does not propose any connection between the alternation and the generalization that nasal consonants in URs tend to agree with the following consonant. While clusters like \emph{pi}[m.p]\emph{le}, \emph{sta}[n.z]\emph{a}, and \emph{mo}[ŋ.k]\emph{ey} are the norm, exceptions like \emph{pli}[m.s]\emph{oll}, \emph{da}[m.z]\emph{el}, \emph{scri}[m.ʃ]\emph{aw}, and \emph{ra}[m.k]\emph{in} are found.

In all, there are only 80 possible nasal-obstruent syllable contact clusters, the majority of which are attested. In addition, there are a small number of words which contain clusters like \emph{oi}[nt.m]\emph{ent}, in which a nasal-obstruent cluster are both parsed into a complex medial coda, but in this position, place agreement is exceptionless, and thus these clusters are ignored below.

\begin{example}
Lexical effects of \textsc{Coda Nasal Place Assimilation}: 

\vspace{0.5\baselineskip}
\begin{tabular}{l r r r r}
\toprule
           & attested & unattested & saturation & $p$-value \\
\midrule
conforming & 41 & 11 & 0.788 & \multirow{2}{*}{2.6\e{-05}}\\
violating  & 8  & 20 & 0.286 \\
\bottomrule
\end{tabular}
\end{example}

\noindent
Clusters which conform to this generalization are far more likely to be attested than those that violate it, and this is highly unlikely to be due to chance.

There is some evidence that even young infants show a preference for nonce words that conform to this generalization. %\citet{Jusczyk2003}
\citet{Davidson2004}, \citet{Mattys1999}, and \citet{Jusczyk2002} play nonce words containing nasal-obstruent clusters to 4.5, 9, and 10-month-old infants (respectively) acquiring English, All three studies find that infants prefer to listen to stimuli which conform to the above generalization (e.g., \emph{u}[m.b]\emph{o}) over those which do not (e.g., \emph{u}[n.b]\emph{o}). \citet{Wright1975} taught several nonce words to a group of adolescents, and asked them to repeat the words to each other in a game of ``Telephone''. In these nonse words, nasals do not agree in \textsc{Place} with the following obstruent in these nonsense words, and \citeauthor{Wright1975} observes that after a few rounds of the game, the nonce words had been adapted to conform to \textsc{Coda Nasal Place Assimilation}.

\begin{example}
Artificial language adaptations \citep[][394, his transcriptions]{Wright1975}: 

\vspace{0.5\baselineskip}
\begin{tabular}{l l l l l l}
a. & [gownp] & > & [gump] \\
b. & [ǰumg]  & > & [ǰúŋgə] \\
c. & [ðʌŋd]  & > & [tɔŋg] \\
\end{tabular}
\end{example}

\noindent
If this game is akin to the process of loanword adaptation, it is possible that various extragrammatical pressures are at play \citep[e.g.,][]{Halle1998b,Dupoux1999,Ussishkin2003,Peperkamp2008}. Yet, the independent evidence for a phonological process of nasal place assimilation provides the most parsiminious account of the adaptations seen in \citeauthor{Wright1975}'s study. J. \citet{Myers1993} also shows that \textsc{Coda Nasal Place Assimilation} is fed by speech errors:

\begin{example} 
Speech errors and \textsc{Coda Nasal Place Assimilation} \citep[][228]{Myers1993}:

\vspace{0.5\baselineskip}
\begin{tabular}{l l l}
   & \emph{error}  & \emph{target} \\
a. & primps        & prints        \\
b. & rand orker    & rank order    \\
%  & [sprɪg]time for [hɪ̃ntlɘr] & springtime for Hitler \\
c. & po[ŋ]cutation & computation   \\
\end{tabular} 
\end{example}

%\citet{Hay2004a} demonstrates that [n.p, mθ] clusters, which are not found in English, are in fact rated better than many attested clusters.
%report that 10-month-old infants prefer to listen to nonsense words which conform to \textsc{Coda Nasal Place Assimilation} (e.g., \emph{u}[m.b]\emph{o}) compared to those which do not (e.g., \emph{u}[n.b]\emph{o}). 

\subsubsection{Degemination}

A final well-known alternation in English is the simplification of geminates derived by certain suffixes. For instance, the suffix \emph{-ly} rarely forms a geminate when attaching to root-final /l/, whether forming denominal adjectives (e.g., \emph{sme}[l]\emph{y}) or deadjectival adverbs (e.g., \emph{norma}[l]\emph{y}). A small class of English verbs take a /-t/ suffix in the past; as this suffix does not trigger the \textsc{Epenthesis} found in the regular past (e.g., \citealt{Bakovic2005b}, \citealt{Fruehwald2011}), \textsc{Degemination} applies.

\begin{example}
Degemination in the /-t/ past (\citealp[][105]{Halle1985a}, S. \citealp[][492]{Myers1987}): 

\vspace{0.5\baselineskip}
\begin{tabular}{l l l l l}
a. & /bɛnd-t/ & \goesto & [bɛnt] & (cf. \emph{burn}[t]) \\
b. & /bɪld-t/ & \goesto & [bɪlt] & (cf. \emph{spil}[t]) \\
\end{tabular}
\end{example}

More complicated cases of degemination are found in English prefix morphology and described in \emph{SPE} (p.~148, 219f.) and \citealt[][]{Borowsky1986} (p. 102f.), and are not reviewed here.

%\ex \textsc{Coda Nasal Place Assimilation} feeds \textsc{Degemination} \citep[][116]{Borowsky1986}: \\
%\begin{tabular}{l l l l}
%a. & mature    & i[m]ature    & (cf. \emph{i}[m.b]\emph{alance})   \\
%b. & numerable & i[n]umerable & (cf. \emph{i}[n.d]\emph{ependent}) \\
%\end{tabular}
%\xe

The rule deletes the first of two adjacent consonants agreeing in \textsc{Place} and manner features, and is formalized using quantification over features \citep{Reiss2003b}.
% cf. \citep{Berent2012}

\begin{example}
\textsc{Degemination} (\emph{SPE}:253):

\xymatrix@R=24pt@C=24pt{
\textsc{C}_1       & \longrightarrow & \emptyset & / & A~\gap\gap~A & \textsc{C}_2 \\
}
Condition: $\forall F_i \in \{\textsc{Place}, \textsc{Sonorant}, \textsc{Nasal}, \textsc{Continuant}, \textsc{Lateral}\}$ . $(F_i)_1 = (F_i)_2$
\end{example}

\noindent
I assume that geminates crossing a syllable boundary are ``false geminates'', meaning that they are adjacent similar segments rather than a single feature matrix linked to two timing slots. Under the alternative account in which the absence of geminates is an inventory restriction (i.e., a \textsc{Segment Structure Condition}), an undesirable duplication of phonological derived surface constraints and constraints on URs would be unavoidable.

Neither identical consonants, nor those which differ only in voice (e.g., *[p.b]) occurs in simplex words in CELEX, and this absence is highly unlikely to be due to chance.

\begin{example} Lexical effects of \textsc{Degemination}: 

\vspace{0.5\baselineskip}
\begin{tabular}{l r r r r}
\toprule
           & attested & unattested & saturation & $p$-value \\
\midrule
conforming & 157 & 596 & 0.208 & \multirow{2}{*}{7.2\e{-09}}\\
violating  &   0 &  87 & 0.000 \\
\bottomrule
\end{tabular}
\end{example}

%\footnote{Note that it is not the case that the English lexicon as envisioned by \emph{SPE} is free of underlying geminates; \citeauthor{SPE} use underlying geminates to attract stress to a preceding nucleus (p.~148--151). In \emph{SPE}, the underlying geminate-singleton contrast is absolutely neutralized, but has consequences for surface stress. In addition to the usual objections to absolute neutralization (i.e., that it is a learning conundrum mascarading as a phonological solution) that emerged in the wake of the publication of \emph{SPE}, the phonological particulars of this account have also come under attack \citep[e.g.,][]{Ross1972a}. It would not seem tenable to use underlying geminates to attract stress even if absolutely neutralized contrasts were admitted.} Unsurprisingly, this is unlikely to be due to chance. 

\subsubsection{Dorsal/non-coronal clusters}

\citet{Pierrehumbert1994} surveys the English syllable contact cluster inventory, and discerns the effect of several ``static'' constraints, those which do not derive from neutralizing phonological rules. 

\citet[][173]{Pierrehumbert1994} writes that ``velar obstruents occurred only before coronals in the clusters studied, never before labials or other velars.'' Since \citeauthor{Pierrehumbert1994}'s study is restricted to triconsonantal clusters, so words like \emph{a}[k.m]\emph{e}, \emph{ru}[g.b]\emph{y}, or \emph{pi}[g.m]\emph{ent} do not violate this narrower generalization. Further, \citeauthor{Pierrehumbert1994} adopts the traditional assumption \citep[e.g.,][66f.]{Borowsky1986} that coda [ŋ] is an allophone of /n/ before dorsal consonants, so words such as \emph{mo}[ŋ.gr]\emph{el} are not exceptions. 

\citeauthor{Pierrehumbert1994} generalization is termed static simply because there is no evidence for any phonological process which changes clusters of dorsal obstruents followed by non-coronals into something else (perhaps a dorsal/coronal cluster). However, the statistical test, taking into account clusters of any length, does not indicate that this dispreference for dorsal/non-coronal clusters is so compelling that it could not be generated by chance, simply because possible dorsal/non-coronal clusters are about as likely to occur as are similar types of clusters such as those with a non-dorsal coda and a non-coronal onset (e.g., \emph{a}[t.m]\emph{osphere}) or a dorsal coda and coronal onset (e.g., \emph{ve}[k.t]\emph{or}). These clusters are marked as ``conforming'' to a constraint \textsc{*[+Dorsal][$-$Coronal]}, and dorsal/non-coronal consonants as ``violating'' it. 

\begin{example}
Lexical effects of \textsc{*[$+$Dorsal][$-$Coronal]}: 

\vspace{0.5\baselineskip}
\begin{tabular}{l r r r r}
\toprule
           & attested & unattested & saturation & $p$-value \\
\midrule
conforming & 76 & 359 & 0.175 & \multirow{2}{*}{0.145} \\
violating  &  6 &  57 & 0.095 \\
\bottomrule
\end{tabular}
\end{example}

There is no exceptions to \citeauthor{Pierrehumbert1994}'s narrower generalization concerning triconsonantal clusters, but it is no more statistically convincing. 

\begin{example}
Lexical effects of \textsc{*[$+$Dorsal][$-$Coronal]} on triconsonantal clusters: 

\vspace{0.5\baselineskip}
\begin{tabular}{l r r r r}
\toprule
           & attested & unattested & saturation & $p$-value \\
\midrule
conforming & 20 & 148 & 0.119 & \multirow{2}{*}{0.136} \\
violating  &  0 &  22 & 0.000 \\
\bottomrule
\end{tabular}
\end{example}

\noindent
The evidence for \citeauthor{Pierrehumbert1994}'s static constraint does not actually require that such a constraint exists at all, since it is quite possible that the apparent dispreference for dorsal/non-coronal clusters could result by accident.

\subsubsection{Coda coronal obstruents}

Another static generalization about triconsonantal syllable contact clusters proposed by \citet[][175]{Pierrehumbert1994} is that  ``clusters with a coronal obstruent in the coda do not occur.'' In an endnote (p. 186), \citet{Pierrehumbert1994} notes an exception in \emph{a}[nt.l]\emph{er}, and one can add the triconsonantal clusters in \emph{ke}[s.tr]\emph{el} and \emph{oi}[nt.m]\emph{ent}, as well as many biconsonantal clusters like \emph{a}[t.l\emph{as} or \emph{e}[s.t]\emph{er}. As before, applying the statistical test to clusters of any length reveals that coda coronal obstruent clusters are not significantly less likely to occur than clusters with non-coronal codas. 

\begin{example}
Lexical effects of \textsc{*Coda Coronal Obstruent}: 

\vspace{0.5\baselineskip}
\begin{tabular}{l r r r r}
\toprule
           & attested & unattested & saturation & $p$-value \\
\midrule
conforming &  48 & 312 & 0.133 & \multirow{2}{*}{0.301} \\
violating  &  38 & 322 & 0.106 \\
\bottomrule
\end{tabular}
\end{example}

Restricting \citeauthor{Pierrehumbert1994}'s generalization to triconsonantal clusters does not strengthen the case for this static constraint.

\begin{example}
Lexical effects of \textsc{*Coda Coronal Obstruent} on triconsonantal clusters: 

\vspace{0.5\baselineskip}
\begin{tabular}{l r r r r}
\toprule
           & attested & unattested & saturation & $p$-value \\
\midrule
conforming &  4 & 104 & 0.037 & \multirow{2}{*}{0.419} \\
violating  &  2 & 124 & 0.016 \\
\bottomrule
\end{tabular} 
\end{example}

%Note that [tʃ] has not been included since it so rarely occurs in clusters.

\subsubsection{ABA clusters}

Finally, \citet[][176]{Pierrehumbert1994} notes ``a lack of clusters with identical first and third elements''. For the determination of identity, I assume the same definition of geminate as used in the rule of \textsc{Degemination} above, namely consonants which share all the same \textsc{Place} and manner features, ignoring \textsc{Voice}. 
%and I also expand the definition of the constraint to also apply to the second and fourth consonant of quadraconsonantal clusters. 
\citeauthor{Pierrehumbert1994} notes one potential exception to this generalization, the cluster in words such as \emph{e}[ks.kl]\emph{ude}, which contains two /k/, does not arise here since such words have been marked morphologically complex. Despite the fact that the corpus contains no exceptions to this generalization, these \textsc{ABA} clusters are not significantly rarer than other triconsonaontal and quadraconsonantal clusters, of which only 7.8\% occur. 

\begin{example}
Lexical effects of \textsc{*ABA}: 

\vspace{0.5\baselineskip}
\begin{tabular}{l r r r r}
\toprule
           & attested & unattested & saturation & $p$-value \\
\midrule
conforming & 41 & 478 & 0.079 & \multirow{2}{*}{0.818} \\
violating  &  0 &  11 & 0.000 \\
\bottomrule
\end{tabular}
\end{example}

Whereas the three alternations targeting English syllable contact clusters have robust lexical reflexes as measured by the Fisher exact test, such tests reveal no evidence for the static dispreferences identified by \citeauthor{Pierrehumbert1994}.

%\subsubsection{Summary}
%
%The above results are summarized in Table \ref{constraints}.
%
%\begin{table}
%\centering
%\begin{tabular}{l r r r r r}
%\toprule
%                                       & \multicolumn{2}{c}{attested} & \multicolumn{2}{c}{unattested} & \multirow{2}{*}{$p$-value} \\
%                                       & conforming & violating & conforming & violating \\
%\midrule
%\textsc{O.V.A.} (\ref{ovar})        &  80 &   6 & 370 & 264 & 1.1\e{-11} \\
%\textsc{C.N.P.A} (\ref{CNPA})       &  41 &  25 &  12 &  42 & 1.7\e{-05} \\ 
%\textsc{Degemination} (\ref{degem}) & 161 & 632 &   0 &  87 & 1.4\e{-08} \\
%\midrule
%\textsc{*Dorsal-Labial} ()          &  68 &   4 & 246 &  40 & 0.069 \\
%\textsc{*C.C.O.} ()                 &  48 &  38 & 312 & 322 & 0.301 \\
%\textsc{*A,BA} ()                   & 161 &   0 & 708 &  11 & 0.231 \\
%\bottomrule
%\end{tabular}
%\caption{Counts and Fisher exact test $p$-values for the constraints discussed above.}
%\label{constraints}
%\end{table}
