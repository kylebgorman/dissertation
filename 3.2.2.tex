\subsection{Wordlikeness results}

There is some disagreement in the literature about just what \citeauthor{Zimmer1969}'s study reveals about \textsc{Labial Attraction}.

zimmer
clements
\citet{Ito1993} argue that the data shows that it holds of native words only
\citet{Inkelas2001} refute this point



\citet{Zuraw2000}:4



\citet{Inkelas2001} present a different variation on the original rule of \textsc{Labial Attraction}



\begin{quote}
Vowel labialization following labials is not a synchronic alternation in Turkish, yet it (unlike \textsc{Labial Attraction} per se) \emph{is} a statistically supported tendency worthy of further research. \citep[][196, emphasis in original]{Inkelas2001}
\end{quote}

``These results support our view that there is no rule of Labial Attraction'' \citep[][225]{Clements1982}.

%since the opposing are all broader versions than the original formulation by \citeauthor{Lees1966a}.

\citet{Goodman1954}

\begin{example}
$\displaystyle \gamma = \frac{C - D}{C + D}$
\end{example}

\subsubsection{Backness harmony}


\begin{example}
Backness harmony wordlikeness forced choices \citep[314]{Zimmer1969}: 

\vspace{0.5\baselineskip}
\begin{tabular}{l r l r}
\toprule
\multicolumn{2}{l}{harmonic} & \multicolumn{2}{l}{disharmonic} \\
\midrule
pemez & 30                   & pemaz & 2  \\
tipez & 24                   & tipaz & 8  \\ 
terüz & 19                   & teruz & 13 \\ % roundness violator
teriz & 28                   & terız & 3  \\
tokaz & 26                   & tokez & 6  \\ % roundness violator
\bottomrule
\end{tabular}
\end{example}

$\gamma = 0.597$, $p = 5.2$\e{-29}

\subsubsection{Roundness harmony}

\begin{example}
Roundness harmony forced choice wordlikeness judgements: 

\vspace{0.5\baselineskip}
\begin{tabular}{l r l r}
\toprule
\multicolumn{2}{l}{harmonic} & \multicolumn{2}{l}{disharmonic} \\
\midrule
pörüz & 32 & pöriz & 0  \\
tatız & 20 & tatuz & 12 \\
tüpüz & 31 & tüpiz & 1  \\
takız & 22 & takuz & 10 \\
\bottomrule
\end{tabular}
\end{example}

$\gamma = 0.641$, $p = 7.7$\e{-27}

%tamaz & 26 & tamoz & 6  \\
%putoz & 25 & putaz & 7  \\

\subsubsection{Labial harmony}

\begin{example}
Labial attraction wordlikeness forced choices \citep[314]{Zimmer1969}: 

\vspace{0.5\baselineskip}
\begin{tabular}{l r l r}
\toprule
\multicolumn{2}{l}{aPu} & \multicolumn{2}{l}{aPı} \\
\midrule
pamuz & 17              & pamız & 15 \\
tafuz & 21              & tafız & 11 \\
tapuz & 17              & tapız & 15 \\
mavuz & 16              & mavız & 16 \\
tabuz & 16              & tabız & 16 \\
\bottomrule
\end{tabular}
\end{example}

$\gamma = 0.087$, $p = 0.101$
