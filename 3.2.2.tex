\subsection{Wordlikeness results}

There is some disagreement in the literature about just what \citet{Zimmer1969} finds regarding \textsc{Labial Attraction}.

\citet{Zimmer1969}
(see also \citet[][4]{Zuraw2000})
claims however, Zimmer's evidence suggests that the speakers did internalize a version of the labial attraction generalization.

\citet{Inkelas2001} present a different variation on the original rule of \textsc{Labial Attraction}

\begin{quotation}
Vowel labialization following labials is not a synchronic alternation in Turkish, yet it (unlike \textsc{Labial Attraction} per se) \emph{is} a statistically supported tendency worthy of further research. \citep[][196, emphasis in original]{Inkelas2001}
\end{quotation}

``These results support our view that there is no rule of Labial Attraction'' \citep[][225]{Clements1982}.

since the opposing are all broader versions than the original formulation by \citeauthor{Lees1966a}.

\citet{Ito1993} argue that the data shows that it holds of native words only

\citet{Goodman1954}

\ex $\displaystyle \gamma = \frac{C - D}{C + D}$ \xe

\subsubsection{Roundness harmony}

($\gamma = 0.597$, $p = 5.2$\e{-29}) 

Roundness harmony wordlikeness forced choices \citep[314]{Zimmer1969}: \vspace{6pt} \\ 
\begin{tabular}{l r l r}
\toprule
\multicolumn{2}{l}{harmonic} & \multicolumn{2}{l}{disharmonic} \\
\midrule
pemez & 30                   & pemaz & 2 \\
teriz & 28                   & terız & 3 \\
tokaz & 26                   & tokez & 6 \\
tipez & 24                   & tipaz & 8 \\
terüz & 19                   & teruz & 13 \\
\bottomrule
\end{tabular}

\subsubsection{Labial harmony}

Labial attraction wordlikeness forced choices \citep[314]{Zimmer1969}: \vspace{6pt} \\ 
\begin{tabular}{l r l r}
\toprule
\multicolumn{2}{l}{aPu} & \multicolumn{2}{l}{aPı} \\
\midrule
tafuz & 21              & tafız & 11 \\
pamuz & 17              & pamız & 15 \\
tapuz & 17              & tapız & 15 \\
mavuz & 16              & mavız & 16 \\
tabuz & 16              & tabız & 16 \\
\bottomrule
\end{tabular}

($\gamma = 0.087$, $p = 0.101$)
