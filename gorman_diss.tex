%!TEX TS-program = xelatex
%!TEX encoding = UTF-8 Unicode

\documentclass{upenndiss}

\usepackage{longtable}
\usepackage{tipa}
\usepackage{multirow}
\usepackage{booktabs}
\usepackage{pifont}
\usepackage{amsmath}
\usepackage{latexsym}
\usepackage{colortbl}
\usepackage{arydshln} % colortbl >> arydshln is a crit. ordering
\usepackage{courier}
\usepackage{expex}
\usepackage[all]{xy}

\usepackage{mathspec}
\setmainfont[Mapping=tex-text]{Linux Libertine O}
\setmathfont(Digits,Greek,Latin){Linux Libertine O}

%\usepackage{tipx} % OT commands
%\usepackage{wasysym}  
%\usepackage{marvosym}
\usepackage{graphicx}
%\newcommand{\frwn}{\frownie}
%\newcommand{\hand}{\ding{43}} 
%\newcommand{\np}{\textObullseye}

%\usepackage[T1]{fontenc} % allows for dh, th, etc.
%\usepackage[utf8x]{inputenc}

% programming language syntax highlighting
\usepackage{listings}

% my shortcuts
\newcommand{\zr}{$\emptyset$}
\newcommand{\lm}{ː}
\newcommand{\buf}{\hspace{0pt}}
\newcommand{\gap}{\rule{.21cm}{0.4pt}}
\newcommand{\pal}{\textsuperscript{j}}
\newcommand{\shade}{\cellcolor[gray]{0.7}}
\newcommand{\arrow}{$\longrightarrow$}

% lorem ipsum FIXME
\usepackage{lipsum}

\usepackage[round]{natbib}
\bibpunct[:]{(}{)}{,}{a}{}{,}

%\providecommand{\e}[1]{\ensuremath{\times 10^{#1}}}
\providecommand{\e}[1]{\textsc{e}{$#1$}}

\title{Words \& Gaps}

\author{Kyle Gorman}
\supervisor{Charles Yang \\ Associate Professor, Linguistics, Computer and Information Science, \& Psychology}
\gradchair{Eugene Buckley \\ Associate Professor, Linguistics}
\committee{Stephen R. Anderson \\ Dorothy R. Diebold professor of Linguistics, Psychology, \& Cognitive Science, Yale University \and
Mark Liberman \\ Trustee Professor of Phonetics, Linguistics \& Computer and Information Science \and
Rolf Noyer \\ Associate Professor, Linguistics}

\copyrighttrue
%\copyrightyear{2011}  % Defaults to the current year

\department{Linguistics}

\abstractfile{abs} 
\acknowledgementsfile{ack}
%\dedicationfile{ded} % will probably not use this -- KG
%\prefacefile{pre}

\dedication{\center To my parents}

\begin{document}
\frenchspacing

%\FrontMatter
%
%\chapter{Morpheme structure constraints in generative grammar} \label{intro}


\section{A program for phonotactic theory}



Theories of phonotactic knowledge should be evaluated by the same stringentcriteria applied to other formal arenas: neither overgeneration nor undergeneration should be permitted. It will be shown that the orthodoxy that phonotactic knowledge is in some sense ``probabilistic'' suffers from quite severe over- and undergeneration. 

This heuristic has the greatest impact regarding debates about ``abstractness'' of underlying representations. 
It cannot be said, precisely, that it either excludes or requires any particular type of abstractness. 

Proposed restrictions on UR abstractness have been faulted for a number of reasons. In some cases, they preclude otherwise-desirable analyses of alternations. Another objection that could reasonably made is that any restriction on abstract URs that introduces an assymetry in URs is bad.
\citep[~chap.~1]{KK77}

I would like to suggest that there are two types of problems that arise in developing a theory of phonotactics free of duplication. The first type of problem consists of conflicts between theoretical assumptions and the desire to eliminate duplication. If we continue to view duplication as a sort of negative heuristic, then it may be the case that the theoretical assumptions are wrong. Such a case arises in Chapter \ref{turkish} in the discussion of archiphonemic underspecification analysis of Turkish vowel harmony proposed by \citet{Clements1982}. \citeauthor{Clemenst1982} propose that all but the first vowel of a harmonic root is underspecified for backness (and in high vowels, roundness) and is filled in by rule. Since there are disharmonic roots, this rule must be ``structure-filling''. Consequently, this rule cannot account for the apparent markedness of disharmonic roots revealed by wordlikeness judgements (among other psycholinguistic tasks). If duplication is a pathology, this analysis is wrong.

Another type of problem is the ``conundrum'', an apparent fact which poses a problem for virtually any generative theorists. 

pure allophony

There are a few contexts which this interacts:

- prosody
- exceptionality
- wordlikeness

Syntax can hardly be described as a theory of word ``arrangement''; it would be inconceivable for syntacticians to engage in a lengthy debate about whether some word sequence is permitted in some language. Yet it is possible for two reasonable phonologists to dispute whether [k.p] is a permissible medial consonant cluster in English. While lexical entries are thought to consist of segments, and segments to consist of feature specifications, the phonological form of a word like \emph{spectre} is not ``generated'', in any relevant sense, by the concatenation of segments /s/, /p/, and so on, or by the concatenation of syllables \{spɛk\}$_{\sigma}$ and \{tɚ\}$_{\sigma}$. It would be possible to posit ``lexical concatenation rules'', on analogy with the phrase structure rules of syntax, but it is unclear how the strings produced would be linked with meanings, or how such a system would distinguish between, e.g., \emph{brick} and meaningless \emph{blick}.

The more appealing alternative is that the phonological form of \emph{spectre}, whatever it is, is an atom of unit of lexical memory. It remains to account for any restrictions on the contents of underlying forms. The theory of \emph{phonotactics} (which is related to \emph{syntax} through the Greek root \emph{táxis} `order') is concerned with these restrictions and how they are related to the mappings between underlying and surface forms for which the phonological component is responsible. 

%\footnote{It is tempting, however, to describe binding principles as ``syntactics''.}

I believe that the boundaries of this knowledge

The boundaries of this knowledge
and what might be acquired.
These are not unrelated issues:

One question is: why bother?

It is certainly possible to imagine otherwise. \citet[][320]{Zimmer1969} writes that phonotactic generalizations ``have no observable consequences in the course of the normal use of the language'', and \citeauthor{PE} echo this sentiment more recently:

\begin{quote}
Even if we, as linguists, find some generalizations in our description of the lexicon, there is no reason to posit these generalizations as part of the speaker's knowledge of their language, since they are computationally inert and thus irrelevant to the input-output mapping that the grammar is responsible for. \citep[][18]{PE}
\end{quote}

This represents a principled null hypothesis, but can be quickly dismissed in light of speaker's ready judgements of possible and impossible words observed by \citeauthor{Chomsky1965}. Metalinguistic judgements of these sorts, reviewed in Chapter \ref{wordlikeness}, are not the only task which exemplify phonotactic knowledge; infants and adults are also thought to make use \emph{possible word constraint} in more quotidien tasks like recognizing words in running speech \citep[e.g.,][]{Brown1956,Mattys1999,McQueen1998b,Norris1997}.

\subsection{Units of phonotactic description}

\subsubsection{Segments}

Nothing will be said about what \citet{Stanley1967} calls \emph{segment structure rules}, or language-specific constraints on the inventory of (archi)phonemes, since there is a great deal of overlap between competing theories. Consider possible explanations for the apparent absense of ejectives in English, It might be that English does, in a certain sense, permit underlying ejectives, but such segments are uniformly realized as plain (i.e., pulmoni) voiceless stops: in other words, the inventory is an epiphenomenon of phonological processes. Or, perhaps English lacks the features needed to contrast ejectives with plain voiceless stops, in which case the inventory is an epiphenomenon of the system of contrast. Or, perhaps the absense of ejectives follows from no particular fact, and has the grammatical status of an accidental gap. Further assumptions are needed to distinguish these analyses.

\subsubsection{Underlying representations}

The term \emph{sequence structure constraint} is now somewhat imprecise given the apparent obsolescence of boundary segments: the ``sequence structure onstraint'' that rules out onset /bn/ in English must actually be stated as a constraint militating against /\#bn/. \citep[149]{Kenstowicz1977}

\subsubsection{Surface sequences}

% discussion of Ernestus & Baayen 2003, Becker et al. 2011

The notion of blocking in non-derived environments (\citealp[163]{Kiparsky1973a}, \citeyear[152]{Kiparsky1982a}, \citealp{Mascaro1976}) is well-known, but \citet{Hall2006} also amasses evidence for phonological processes restricted to non-derived environments. One of the most famous examples is the distribution of \emph{ich-laut} [ç] and \emph{ach-laut} [x] in German \citep{Bloomfield1930}. The dorsal fricative is [x] before back vowels and[ç] in other contexts.

\begin{example}[German \emph{ich}- and \emph{ach-laut}]
\begin{tabular}{l l l}
a. & [buːx]   & `book'           \\
   & [tɔxtər] & `daughter'       \\
   & [naxt]   & `night'          \\
b. & [siçt]   & `view'           \\
   & [ʃpeçt]  & `woodpecker'     \\
   & [ɡərʏçt] & `rumor'          \\
   & [knøçəl] & `ankle, knuckle' \\
   & [flɛçə]  & `surface'        \\
\end{tabular}
\end{example}

\emph{Umlaut}, the fronting (and raising) of back vowels in certain morphological contexts, produces the front variant of the dorsal fricative; e.g., [lɔx]-[løçər] `hole-holes'.
%\emph{B}[uːx]-\emph{B}[yːç]\emph{er} `book-books',

\emph{K}[uːx]\emph{en}
\emph{K}[uːç]\emph{en}

%\emph{Mas}[oːx]-\emph{Mas}
\footnote{Examples like \emph{Mas}[oːç]\emph{ist} `masochist', \emph{Eun}[uːç]\emph{ismus} `eunichism', first noted by \citet{Merchant1994}, suggest that this should also be restricted to assimilation within the same foot \citep[226f.]{Jensen2000}.}
%\emph{Mas}[oːx]-\emph{Mas}[oːç]\emph{ist}

\section{The model}

\subsection{Syllabification}

\subsection{Rule application and occultation}

\subsection{Detecting violations}

This is precisely the case for 

\subsection{Exceptionality}
% autosegments

\section{Predictions of the model}

\subsection{Gradience}
\subsubsection{Probabilistic well-formedness}
\subsubsection{External factors}

\subsection{Static phonotactics}
\subsubsection{Insensitivity to statistics}
\subsubsection{Saussurean arbitrariness}

\subsection{Order of acquisition}

A point of departure is a review of phonotactic acquisition by \citet{Hayes2004b}.

\emph{pure phonotactic learner}
%...
However, \citeauthor{Hayes2004b} ignores the considerable evidence that infants younger than nine months of age---the \emph{terminus ante quem} for the onset of phonotactic knowledge---already have relatively rich and detailed lexical knowledge

A word of caution is in order. An experiment which finds typically-developing infants of a certain age insensitive to an adult-like contrast using a certain task is by no means sufficient evidence to conclude that infants of this age lack this ability. A negative result may simply indicate that the task lacks ecological validity or requires domain-general cognitive resources beyond those of the infants. For instance, \citet{Werker2002} report that 14-month-old infants are insensitive to small phonological differences between novel words, but \citet{Fennell2006} shows that they are sensitive to the same contrasts when the novel words are situated in a more naturalistic naming scenario and cognitive demands are minimized. Other negative findings might be attributed to failure to control for properties of the stimulus, or to an experimental design which lacks sufficient statistical power.

\citet{Jusczyk1993b} native vs. non-native
\citet{Jusczyk1994} no preference for high prob
\citet{Friederici1993}

Finally, \citet{Mattys1999} find that 9-month-old infants treat hetero-organic nasal-stop clusters in nonce words as indications of word juncture, though this effect is small compared to the effect of other cues like primary stress: they do not test younger infants.

\subsubsection{Prosodic parsing}

Newborn infants are already able to distinguish between monosyllabic and bisyllabic words, but these infants are thought to recognize syllables ``holistically'' rather than as sequences at least until 4 months \citep[e.g.,]{Bertoncini1981,Eimas1999,Jusczyk1987}. The earliest evidence for segmental representations comes not from phonotactic preferences, but from infants' ability to dicriminate between familiar words like \emph{cup} and mispronunciations like \emph{*tup} at 7.5 months of age \citep{Jusczyk1995}.

\subsubsection{Phonological processes}

Very few studies have investigated young infants' knowledge of phonological alternations. One exception is a fascinating study by \citet{White2008}. Simplifying somewhat, the experimenters expose 8.5-month-old and 10-month-old infants to an artificial language in which the voicing of fricatives is contrastive, but voiced and voiceless variants of plosives are in a complementary distribution, appearing only after vowels (\emph{na-bevi}) and after voiceless consonants (\emph{rot-pevi}), respectively. After familiarization, infants at both ages prefer to listen to stimuli which preserve the complementary distribution over those which disrupt it (e.g., \emph{na-poli}, \emph{rot-boli}). As \citeauthor{White2008} note, this suggests that infants have both extracted the plosive voicing alternation and grouped \emph{pevi} and \emph{bevi} together.

\subsubsection{Lexical acquisition}

As first observed by \citet{Darwin1877}, the first words infants recognize are names, their own and those of caretakers. Recent studies suggest that infants learn these names as early as 4 months of age \citep{Bortfeld2005,Mandel1995,Tincoff1999}. Infants as young as 6 months of age have learned the meaning of familiar words \citep{Bergelson2012}. By 8 months of age, infants are able to locate familiar and novel words in utterances \citep{Jusczyk1997,Seidl2006}.

\section{Outline of the dissertation}

The remainder of the dissertation consists of three case studies which provide support for the novel and contentious predictions of the minimal phonotactic model sketched above.

Chapter \ref{gradience} considers evidence from wordlikeness rating tasks. It is argued that intermediate well-formedness ratings are obtained whether or not the categories in question are graded. A primitive categorical model of wordlikeness using prosodic representations is outlined, and shown to predict English speakers' wordlikeness judgements at least as acccurately as state-of-the-art gradient wellformedness models. Once categorical effects are controlled for, gradient models are uncorrelated with well-formedness ratings.

Chapter \ref{turkish} considers the relationship between lexical generalizations, phonological alternations, and speakers' nonce word judgements with a focus on Turkish vowel patterns. It is shown that even exception-filled phonological generalizations have a robust effect on wellformedness judgements, but that statistically reliable phonotactic generalizations go unlearned when they are not derived from phonological alternations.

Chapter \ref{clusters} investigates the role of phonological alternations in constraining lexical entries, focusing specifically on medial consonant clusters in English. Static phonotactic constraints previously proposed to describe gaps in the inventory of medial clusters are shown to be statistically unsound, whereas phonological alternations impose robust restrictions on the cluster inventory. The remaining gaps in the cluster inventory are attributed to the sparse nature of the lexicon, not static phonotactic restrictions.

%% NOTES HERE
%
%a. & \emph{B}[uːx]          & \emph{B}[yːç]\emph{er} & `book(s)' \\
%   & \emph{L}[ɔx]           & \emph{L}[øç]\emph{er}  & `hole(s)' \\
%   &\emph{T}[ɔx]\emph{ter} & \emph{T}[øç]\emph{ter}
%   & \emph{N}[ax]\emph{t}   & \emph{N}[ɛç]\emph{te} & `night' \\
%
% While the bulk of evidence for detailed phonological entries is found in infants over a year of age \citep[e.g.,]{Werker2002,Fennell2003,Fennell2006,Stager1997,Swingley2000,Zamuner2006},
%9-month-old infants already have learned the predominant stress pattern of their language \citep{Jusczyk1993a}.
%\emph{τάξις}
%Whether phonological computations or representations themselves are graded \citep[e.g.,][]{Lakoff1973} is besides the point, as metalinguistic judgements are behaviors, not mental states; they can no more be compared than can ``fear'' and ``flight response''.

\citet{Silverman2000}
\citet{Kiparsky1995}
\citet[352f.]{Hale2003a}
%$\begin{bmatrix} +\textsc{Dorsal} \\ +\textsc{Continuant \end{bmatrix}~\goesto~\begin{bmatrix} =\textsc{Back} \end{bmatrix}~/~\begin{bmatrix} =\textsc{Back} \end{bmatrix}~\gap$
\citeauthor{SPE} propose to derive gradience from the complexity (using a simple feature-counting metric) of the redundancy rules needed to map a nonce word to an attested word, and they show that this derives the increasing cline of wordlikeness running from ``consonant soup'' \emph{bnzk} to ill-formed \emph{bnick}, possible \emph{blick}, and finally lexical \emph{brick}.

Despite the considerable attention given to the proposals of \emph{SPE} in the wake of that book's publication in 1968, the \emph{SPE} wordlikeness model has received almost no further attention in the literature. At the risk of explaining what might be no more than an accidental gap in the literature, the novel aspects of \emph{SPE} model---gradience derived from similarity to existing lexical entries---may have been overshadowed by the many other contentious proposals in \emph{SPE}, and particularly by compelling arguments against the assumption that wordlikeness contrasts derive solely from properties of underlying forms. \citet{Shibatani1973} observes that there are some generalizations about surface forms which give rise to wordlikeness contrasts, but cannot be stated as constraints on underlying forms. An example from German is shown in (\ref{fd}) below.

\begin{example}[German final devoicing] 
\label{fd}
\begin{tabular}{l l l l}
   & nom.sg. & nom.pl.    \\
a. & [piːp]    & [piːpə]  & `cheep(s)'      \\
   & [diːp]    & [diːbə]  & `thief/thieves' \\
b. & [ɡʀaːt]   & [ɡʀaːtə] & `ridge(s)'      \\
   & [ɡʀaːt]   & [ɡʀaːdə] & `degree(s)'     \\
\end{tabular}
\end{example}

\noindent The plural appears to preserve a contrast in final obstruent voicing which is absent in the singular:\footnote{While there is a contentious debate as to whether devoicing in German is completely neutralizing \citep[e.g.,][]{Fourakis1984} or not \citep[e.g.,][]{Port1985}, it is irrelevant to the discussion at hand.} word-final voiced stops are never found in German, and \citet[95]{Shibatani1973} reports that ``it is easy to show that a native speaker of German rejects those forms \emph{on the ground that they end in voiced obstruents}'' (emphasis in original). However, given that root-final voicing is not predictable (e.g., /ɡʀaːt \alt{} ɡʀaːd/), the process of \textsc{Final Devoicing} cannot be any kind of lexical redundancy and is mysterious under the \emph{SPE} account. The alternative proposed by \citeauthor{Shibatani1973} and by \citet{Clayton1976}, is that to say that nonce words like [ɡʀaːd] are ill-formed not because of any underlying property, but since they fail to undergo an otherwise-exceptionless phonological process, \textsc{Final Devoicing}, or equivalently, the surface-true generalization it derives.

\citet{Sommerstein1974} further notes that the computation of hypothetical underlying forms from nonce surface forms which is implied by the \emph{SPE} theory is non-trivial in the presence of phonological opacity \citep[see][528f.]{Anderson1988a}, and the rejection of the biuniqueness principle means that there is not always a unique solution, as is illustrated by the two underlying forms corresponding to [ɡʀaːt] in (\ref{fd}) above.

\subsubsection{Autosegmental phonology and beyond}

The arguments of \citeauthor{Shibatani1973} and others led theorists to focus their attention on properties of surface representations as determinants of wordlikeness. Though syllabification plays no role in \emph{SPE}, it is crucial to many earlier studies \citep[for a review, see][]{Goldsmith2011b}, and it received particular attention in the 1970s. \citet{Hooper1973} and \citet{Kahn1976} argue that the syllable is useful for defining wordlikeness generalizations.\footnote{\citet{Steriade1999} and \citet{Blevins2003}, however, argue that a number of phonotactic generalizations previously stated in syllabic terms can be reanalyzed without making reference to syllables.} \citeauthor{Hooper1973} argues, for instance, that [bn], impossible as an English onset, is unobjectionable as a syllable contact cluster in nonce words like \emph{stabnik} (or in names like \emph{Abner}), and that this demonstrates the superiority of syllable-based wordlikeness generalizations. This already signals further trouble for alternative accounts which focus on underlying forms. Syllabification may span morphs, is generally predictable, and is universally non-contrastive, and as a consequence, few posit in to be present in underlying representations \citep[though see, e.g.,][]{Vaux2003}. \citeauthor{Hooper1973} also points to loanword adaptations which produce native syllable structure \citep[e.g.,][]{Carlisle1991} as evidence that syllabification is part of the phonological computation. Further enrichments to the theory are provided by the autosegmental theory of the syllable \citep{McCarthy1979b}, which envisions the syllable as an articulated tree structure \citep[as first envisioned by][]{Pike1947a}, and theories like prosodic licensing \citep{Ito1989a}, in which syllabification triggers phonological repairs.

While none of these authors discuss gradience, the syllable plays a role in the definition of a gradient measure, ``positional probability'', though to correlate closely with human judgements of wordlikeness. The positional probability of a nonce monosyllabic word is derived from the combined probabilities at which segments occur in onset, nucleus, and coda positions in the lexicon of a given language. 

\citet{Shibatani1973}, however, argues that morpheme structure constraints cannot account for all wordlikeness constraints.
In German, for instance, final obstruents devoice, and as a result, there a
re pairs such  [ɡʀaːt]-[ɡʀaːtə] `ridge(s)' and [ɡʀaːt]-[ɡʀaːdə] `degree(s)'
 differing only in the plural shape of the root.
Since the voicing of final obstruents is contrastive, the restriction on th
e voicing of obstruents cannot be a morpheme structure constraint. 
Yet, \citeauthor{Shibatani1973} claims that native speakers reject nonce wo
rds ending in final voiced obstruents ``\emph{on the ground that they end i
n voiced obstruents}'' (95). 
Thus, not all constrasts in possible wordhood can be stated as morpheme str
ucture constraints.

Whereas \citeauthor{Shibatani1973} maintains that morpheme structure constraints are insufficient to account for speakers' knowledge of possible wordhood, others argue that morpheme structure constraints are also unnecessary. \citet[297]{Hale1965}, \citet{Kisseberth1970b}, and \citet[212f.]{Postal1968} observe that the structural descriptions of phonological processes often are often reflected in the lexical redundancies in the same language. In Russian, there are alternations implicating a process of obstruent voice assimilation, and tautomorphemic obstruent clusters have uniform voicing \citep[283]{A74}. \citet[205f.]{Dell1973} and \citet[28f.]{Stampe1973} argue that morpheme structure constraints are otiose, as phonological rules triggering alternations impose restrictions on the contents of URs, an effect known as \emph{Stampean occultation}.

Since at least \citet{Pike1947b}, linguists have posited language-specific constraints on the contents of underlying representations.
These \emph{morpheme structure constraints} were at one time thought to fully account for speakers' knowledge of possible and impossible words (e.g., \citealt{Chomsky1965}, \citeyear[382]{SPE}, \citealt[22f.]{SPR}, \citeyear{
Halle1962}, \citealt{Stanley1967}).

%    \section{A brief history of MSCs} %%1.1: A brief history of MSCs

\subsection{Structuralism}               \subsection{Structuralism}

Since, with the possible exception of Roman Jakobson (who might be labeled an early generativist), the mentalistic interpretation of structural linguistic analyses was anathema, one must take care to not assume that a description of MSCs is something that these linguists would posit as part of speakers' knowledge of language.

\subsubsection{Morpheme boundaries in phonology}

bloomfield

%While \citeauthor{Bloomfield1930} is known for introducing morphemic structure into phonology, a member of the Prague circle developed an arguably more sophisticated approach only two years later. 

\citet{Jakobson1932} attributes a number of phonological properites of the Russian imperative to morpheme juncture. The vast majority of Russian infinitives consist of the stem followed by a theme vowel and /t\pal/. The reflexive is marked wih a final /-sə/ which feeds a general process of regressive voicing assimilation. In the reflexive infinitive, /\ldots t\pal-s\ldots/ is realized [ts], with no palatalization. However, palatalization of a stem-final labial or coronal---including /t\pal/, as in (\ref{russian}c)---is preserved in the reflexive imperative.

\begin{example}
\label{russian}
Russian reflexives \citep[after][]{Jakobson1932}:\footnote{I have taken a number of liberties with \citeauthor{Jakobson1932}'s presentation of the data, which uses an abstract phonemic transcription. Thanks to Lev Blumenfeld (p.c.) for help with the transcription of this data.}

\begin{tabular}{l l l l} %\toprule
   &  infinitive      & imperative \\ %\midrule
a. & [slav\pal itsə]  & [slaf\pal s\pal ə]  & `be glorious'    \\
   & [upram\pal itsə] & [upram\pal s\pal ə] & `be stubborn'    \\
b. & [kras\pal itsə]  & [kras\pal s\pal ə]  & `put on makeup'  \\
   & [ʒar\pal itsə]   & [ʒar\pal s\pal ə]   & `roast'          \\
c. & [zəbytsə]        & [zəbut\pal s\pal ə] & `forget' \\ %\bottomrule
%   & ab\'utsa      & ab\'ujsa     & `put on shoes'   \\
\end{tabular}
\end{example}

\noindent The forms in (\ref{russian}c) have different outcomes for their /t\pal-s/ clusters. \citeauthor{Jakobson1932} proposes that the preservation of palatalization is a special property of the imperative. %A few years later, \citet{Trnka1936}, another member of the Prague Circle, makes the connection between \citeauthor{Jakobson1932}'s hypothesis and phonotactic generalizations.

\subsubsection{The morph as a constraint domain}

A corrolary of this hypothesis is that the phoneme sequences found within morphs may be distinct from those which span multiple morphs. 

This was apparent to structuralists 

the phonologists 
This is apparent from 
a discussion of constraints on vowel sequences in Mixteco given by \citet{Pike1947b}.
It is apparent that \citeauthor{Pike1947b}'s constraints on vowel sequences are not intended to hold across morph boundaries, 

according to the phonological analysis of a Mixteco text published a few years earlier \citep{Pike1944}.

\begin{example}
Mixteco MSCs \citep{Pike1947b} and complex words \citep{Pike1944}:

\begin{tabular}{r l l l} %\toprule
   & MSC & complex exception \\ %\midrule
%a. & *{C}a{C}e & [k\'a-\textsuperscript{n}dee] & `kept \ldots inside' \\
%b. & *{C}\textipa{@}{C}e & [n\`i-k\`ə bə-de] & `who entered'        \\
%c. & *{C}e{C}i & [te-n\'i-ke\textsuperscript{n}da] & `was walking         \\
%d. & *{C}i{C}e & [te-n\`i-kee-t\`ə] & `and went away'      \\ %\toprule
%e. & *{C}e{C}o & b\'e\textglotstopvari e-\v{z}\'o & `our house'          \\
%f. & *{C}eo & ke-o-d\'e & `we eat him'         \\
a. & *{C}a{C}e & [ká\textsuperscript{n}dee] & `kept \ldots inside' \\
b. & *{C}ə{C}e & [nìk\`əbəde] & `who entered'        \\
c. & *{C}e{C}i & [teníke\textsuperscript{n}da] & `was walking         \\
d. & *{C}i{C}e & [tenìkeet\`ə] & `and went away'      \\ 
\end{tabular}
\end{example}

\noindent \citet[][166]{Pike1947b} affirms that the morpheme is ``marked'' by the violation of morpheme-internal sequence restrictions, an idea further developed by \citet{Harris1955} as a morpheme discovery routine.

\subsubsection{Biuniqueness and neutralization}

joos
harris
chomsky

\subsection{Early generativism}          % 1.1.2: Early generativism


\subsubsection{\citealt{Stanley1967}}

inventory constraints
sequence constraints

\subsubsection{Surface constraints and occultation}

shibatani
stampe
dell

\emph{A}[mt] `office', \emph{Anbau} `cultivation'
\emph{ei}[n.g]\emph{reifen} `to intervene',

\subsubsection{The duplication problem}

inventory problems 
dell, chung, but also problems

conspiracies

\subsection{Autosegmentalism}            \subsection{Autosegmentalism}

also ``supersegmentalism''

The proposals during this period are very diverse. 

\subsubsection{Syllable structure}

hooper
noske

\subsubsection{Morph features}

Leben
Kaye


\subsubsection{The Obligatory Contour Principle}

\subsection{Classical Optimality Theory} \subsection{Classic Optimality Theory}

%(page numbers here are taken from the published version)

\subsubsection{Richness of the Base}

some quotes from OT doc
\citet{PE}
\citet{Bye2001}

\subsubsection{Freedom of Analysis}

\citet{Smolensky1996}
\citet{PE}

\subsubsection{Undominated constraints}

% complaints about maceahern, gallagher
% bit on loanword adaptation

\subsection{Recent developments}         \subsection{Recent developments}

\subsubsection{Language acquisition}
% jusczyk forever
% white et al.
% early word learning
% syllables

\subsubsection{Psycholinguistic tasks in adults}

% wordlikeness
% spotting
% error correction, etc.

\subsubsection{Non-native word adaptation}

% perceptual factors


%        \subsection{Structuralism} %\subsection{Structuralism}

Since, with the possible exception of Roman Jakobson (who might be labeled an early generativist), the mentalistic interpretation of structural linguistic analyses was anathema, one must take care to not assume that a description of MSCs is something that these linguists would posit as part of speakers' knowledge of language.

\subsubsection{Morpheme boundaries in phonology}

bloomfield

%While \citeauthor{Bloomfield1930} is known for introducing morphemic structure into phonology, a member of the Prague circle developed an arguably more sophisticated approach only two years later. 

\citet{Jakobson1932} attributes a number of phonological properites of the Russian imperative to morpheme juncture. The vast majority of Russian infinitives consist of the stem followed by a theme vowel and /t\pal/. The reflexive is marked wih a final /-sə/ which feeds a general process of regressive voicing assimilation. In the reflexive infinitive, /\ldots t\pal-s\ldots/ is realized [ts], with no palatalization. However, palatalization of a stem-final labial or coronal---including /t\pal/, as in (\ref{russian}c)---is preserved in the reflexive imperative.

\begin{example}
\label{russian}
Russian reflexives \citep[after][]{Jakobson1932}:\footnote{I have taken a number of liberties with \citeauthor{Jakobson1932}'s presentation of the data, which uses an abstract phonemic transcription. Thanks to Lev Blumenfeld (p.c.) for help with the transcription of this data.}

\begin{tabular}{l l l l} %\toprule
   &  infinitive      & imperative \\ %\midrule
a. & [slav\pal itsə]  & [slaf\pal s\pal ə]  & `be glorious'    \\
   & [upram\pal itsə] & [upram\pal s\pal ə] & `be stubborn'    \\
b. & [kras\pal itsə]  & [kras\pal s\pal ə]  & `put on makeup'  \\
   & [ʒar\pal itsə]   & [ʒar\pal s\pal ə]   & `roast'          \\
c. & [zəbytsə]        & [zəbut\pal s\pal ə] & `forget' \\ %\bottomrule
%   & ab\'utsa      & ab\'ujsa     & `put on shoes'   \\
\end{tabular}
\end{example}

\noindent The forms in (\ref{russian}c) have different outcomes for their /t\pal-s/ clusters. \citeauthor{Jakobson1932} proposes that the preservation of palatalization is a special property of the imperative. %A few years later, \citet{Trnka1936}, another member of the Prague Circle, makes the connection between \citeauthor{Jakobson1932}'s hypothesis and phonotactic generalizations.

\subsubsection{The morph as a constraint domain}

A corrolary of this hypothesis is that the phoneme sequences found within morphs may be distinct from those which span multiple morphs. 

This was apparent to structuralists 

the phonologists 
This is apparent from 
a discussion of constraints on vowel sequences in Mixteco given by \citet{Pike1947b}.
It is apparent that \citeauthor{Pike1947b}'s constraints on vowel sequences are not intended to hold across morph boundaries, 

according to the phonological analysis of a Mixteco text published a few years earlier \citep{Pike1944}.

\begin{example}
Mixteco MSCs \citep{Pike1947b} and complex words \citep{Pike1944}:

\begin{tabular}{r l l l} %\toprule
   & MSC & complex exception \\ %\midrule
%a. & *{C}a{C}e & [k\'a-\textsuperscript{n}dee] & `kept \ldots inside' \\
%b. & *{C}\textipa{@}{C}e & [n\`i-k\`ə bə-de] & `who entered'        \\
%c. & *{C}e{C}i & [te-n\'i-ke\textsuperscript{n}da] & `was walking         \\
%d. & *{C}i{C}e & [te-n\`i-kee-t\`ə] & `and went away'      \\ %\toprule
%e. & *{C}e{C}o & b\'e\textglotstopvari e-\v{z}\'o & `our house'          \\
%f. & *{C}eo & ke-o-d\'e & `we eat him'         \\
a. & *{C}a{C}e & [ká\textsuperscript{n}dee] & `kept \ldots inside' \\
b. & *{C}ə{C}e & [nìk\`əbəde] & `who entered'        \\
c. & *{C}e{C}i & [teníke\textsuperscript{n}da] & `was walking         \\
d. & *{C}i{C}e & [tenìkeet\`ə] & `and went away'      \\ 
\end{tabular}
\end{example}

\noindent \citet[][166]{Pike1947b} affirms that the morpheme is ``marked'' by the violation of morpheme-internal sequence restrictions, an idea further developed by \citet{Harris1955} as a morpheme discovery routine.

\subsubsection{Biuniqueness and neutralization}

joos
harris
chomsky

%            \subsubsection{Morpheme boundaries in phonology} %\subsection{Structuralism}

Since, with the possible exception of Roman Jakobson (who might be labeled an early generativist), the mentalistic interpretation of structural linguistic analyses was anathema, one must take care to not assume that a description of MSCs is something that these linguists would posit as part of speakers' knowledge of language.

\subsubsection{Morpheme boundaries in phonology}

bloomfield

%While \citeauthor{Bloomfield1930} is known for introducing morphemic structure into phonology, a member of the Prague circle developed an arguably more sophisticated approach only two years later. 

\citet{Jakobson1932} attributes a number of phonological properites of the Russian imperative to morpheme juncture. The vast majority of Russian infinitives consist of the stem followed by a theme vowel and /t\pal/. The reflexive is marked wih a final /-sə/ which feeds a general process of regressive voicing assimilation. In the reflexive infinitive, /\ldots t\pal-s\ldots/ is realized [ts], with no palatalization. However, palatalization of a stem-final labial or coronal---including /t\pal/, as in (\ref{russian}c)---is preserved in the reflexive imperative.

\begin{example}
\label{russian}
Russian reflexives \citep[after][]{Jakobson1932}:\footnote{I have taken a number of liberties with \citeauthor{Jakobson1932}'s presentation of the data, which uses an abstract phonemic transcription. Thanks to Lev Blumenfeld (p.c.) for help with the transcription of this data.}

\begin{tabular}{l l l l} %\toprule
   &  infinitive      & imperative \\ %\midrule
a. & [slav\pal itsə]  & [slaf\pal s\pal ə]  & `be glorious'    \\
   & [upram\pal itsə] & [upram\pal s\pal ə] & `be stubborn'    \\
b. & [kras\pal itsə]  & [kras\pal s\pal ə]  & `put on makeup'  \\
   & [ʒar\pal itsə]   & [ʒar\pal s\pal ə]   & `roast'          \\
c. & [zəbytsə]        & [zəbut\pal s\pal ə] & `forget' \\ %\bottomrule
%   & ab\'utsa      & ab\'ujsa     & `put on shoes'   \\
\end{tabular}
\end{example}

\noindent The forms in (\ref{russian}c) have different outcomes for their /t\pal-s/ clusters. \citeauthor{Jakobson1932} proposes that the preservation of palatalization is a special property of the imperative. %A few years later, \citet{Trnka1936}, another member of the Prague Circle, makes the connection between \citeauthor{Jakobson1932}'s hypothesis and phonotactic generalizations.

\subsubsection{The morph as a constraint domain}

A corrolary of this hypothesis is that the phoneme sequences found within morphs may be distinct from those which span multiple morphs. 

This was apparent to structuralists 

the phonologists 
This is apparent from 
a discussion of constraints on vowel sequences in Mixteco given by \citet{Pike1947b}.
It is apparent that \citeauthor{Pike1947b}'s constraints on vowel sequences are not intended to hold across morph boundaries, 

according to the phonological analysis of a Mixteco text published a few years earlier \citep{Pike1944}.

\begin{example}
Mixteco MSCs \citep{Pike1947b} and complex words \citep{Pike1944}:

\begin{tabular}{r l l l} %\toprule
   & MSC & complex exception \\ %\midrule
%a. & *{C}a{C}e & [k\'a-\textsuperscript{n}dee] & `kept \ldots inside' \\
%b. & *{C}\textipa{@}{C}e & [n\`i-k\`ə bə-de] & `who entered'        \\
%c. & *{C}e{C}i & [te-n\'i-ke\textsuperscript{n}da] & `was walking         \\
%d. & *{C}i{C}e & [te-n\`i-kee-t\`ə] & `and went away'      \\ %\toprule
%e. & *{C}e{C}o & b\'e\textglotstopvari e-\v{z}\'o & `our house'          \\
%f. & *{C}eo & ke-o-d\'e & `we eat him'         \\
a. & *{C}a{C}e & [ká\textsuperscript{n}dee] & `kept \ldots inside' \\
b. & *{C}ə{C}e & [nìk\`əbəde] & `who entered'        \\
c. & *{C}e{C}i & [teníke\textsuperscript{n}da] & `was walking         \\
d. & *{C}i{C}e & [tenìkeet\`ə] & `and went away'      \\ 
\end{tabular}
\end{example}

\noindent \citet[][166]{Pike1947b} affirms that the morpheme is ``marked'' by the violation of morpheme-internal sequence restrictions, an idea further developed by \citet{Harris1955} as a morpheme discovery routine.

\subsubsection{Biuniqueness and neutralization}

joos
harris
chomsky

%            \subsubsection{The morph as a constraint domain} %\subsection{Structuralism}

Since, with the possible exception of Roman Jakobson (who might be labeled an early generativist), the mentalistic interpretation of structural linguistic analyses was anathema, one must take care to not assume that a description of MSCs is something that these linguists would posit as part of speakers' knowledge of language.

\subsubsection{Morpheme boundaries in phonology}

bloomfield

%While \citeauthor{Bloomfield1930} is known for introducing morphemic structure into phonology, a member of the Prague circle developed an arguably more sophisticated approach only two years later. 

\citet{Jakobson1932} attributes a number of phonological properites of the Russian imperative to morpheme juncture. The vast majority of Russian infinitives consist of the stem followed by a theme vowel and /t\pal/. The reflexive is marked wih a final /-sə/ which feeds a general process of regressive voicing assimilation. In the reflexive infinitive, /\ldots t\pal-s\ldots/ is realized [ts], with no palatalization. However, palatalization of a stem-final labial or coronal---including /t\pal/, as in (\ref{russian}c)---is preserved in the reflexive imperative.

\begin{example}
\label{russian}
Russian reflexives \citep[after][]{Jakobson1932}:\footnote{I have taken a number of liberties with \citeauthor{Jakobson1932}'s presentation of the data, which uses an abstract phonemic transcription. Thanks to Lev Blumenfeld (p.c.) for help with the transcription of this data.}

\begin{tabular}{l l l l} %\toprule
   &  infinitive      & imperative \\ %\midrule
a. & [slav\pal itsə]  & [slaf\pal s\pal ə]  & `be glorious'    \\
   & [upram\pal itsə] & [upram\pal s\pal ə] & `be stubborn'    \\
b. & [kras\pal itsə]  & [kras\pal s\pal ə]  & `put on makeup'  \\
   & [ʒar\pal itsə]   & [ʒar\pal s\pal ə]   & `roast'          \\
c. & [zəbytsə]        & [zəbut\pal s\pal ə] & `forget' \\ %\bottomrule
%   & ab\'utsa      & ab\'ujsa     & `put on shoes'   \\
\end{tabular}
\end{example}

\noindent The forms in (\ref{russian}c) have different outcomes for their /t\pal-s/ clusters. \citeauthor{Jakobson1932} proposes that the preservation of palatalization is a special property of the imperative. %A few years later, \citet{Trnka1936}, another member of the Prague Circle, makes the connection between \citeauthor{Jakobson1932}'s hypothesis and phonotactic generalizations.

\subsubsection{The morph as a constraint domain}

A corrolary of this hypothesis is that the phoneme sequences found within morphs may be distinct from those which span multiple morphs. 

This was apparent to structuralists 

the phonologists 
This is apparent from 
a discussion of constraints on vowel sequences in Mixteco given by \citet{Pike1947b}.
It is apparent that \citeauthor{Pike1947b}'s constraints on vowel sequences are not intended to hold across morph boundaries, 

according to the phonological analysis of a Mixteco text published a few years earlier \citep{Pike1944}.

\begin{example}
Mixteco MSCs \citep{Pike1947b} and complex words \citep{Pike1944}:

\begin{tabular}{r l l l} %\toprule
   & MSC & complex exception \\ %\midrule
%a. & *{C}a{C}e & [k\'a-\textsuperscript{n}dee] & `kept \ldots inside' \\
%b. & *{C}\textipa{@}{C}e & [n\`i-k\`ə bə-de] & `who entered'        \\
%c. & *{C}e{C}i & [te-n\'i-ke\textsuperscript{n}da] & `was walking         \\
%d. & *{C}i{C}e & [te-n\`i-kee-t\`ə] & `and went away'      \\ %\toprule
%e. & *{C}e{C}o & b\'e\textglotstopvari e-\v{z}\'o & `our house'          \\
%f. & *{C}eo & ke-o-d\'e & `we eat him'         \\
a. & *{C}a{C}e & [ká\textsuperscript{n}dee] & `kept \ldots inside' \\
b. & *{C}ə{C}e & [nìk\`əbəde] & `who entered'        \\
c. & *{C}e{C}i & [teníke\textsuperscript{n}da] & `was walking         \\
d. & *{C}i{C}e & [tenìkeet\`ə] & `and went away'      \\ 
\end{tabular}
\end{example}

\noindent \citet[][166]{Pike1947b} affirms that the morpheme is ``marked'' by the violation of morpheme-internal sequence restrictions, an idea further developed by \citet{Harris1955} as a morpheme discovery routine.

\subsubsection{Biuniqueness and neutralization}

joos
harris
chomsky

%            \subsubsection{Biuniqueness and neutralization} %\subsection{Structuralism}

Since, with the possible exception of Roman Jakobson (who might be labeled an early generativist), the mentalistic interpretation of structural linguistic analyses was anathema, one must take care to not assume that a description of MSCs is something that these linguists would posit as part of speakers' knowledge of language.

\subsubsection{Morpheme boundaries in phonology}

bloomfield

%While \citeauthor{Bloomfield1930} is known for introducing morphemic structure into phonology, a member of the Prague circle developed an arguably more sophisticated approach only two years later. 

\citet{Jakobson1932} attributes a number of phonological properites of the Russian imperative to morpheme juncture. The vast majority of Russian infinitives consist of the stem followed by a theme vowel and /t\pal/. The reflexive is marked wih a final /-sə/ which feeds a general process of regressive voicing assimilation. In the reflexive infinitive, /\ldots t\pal-s\ldots/ is realized [ts], with no palatalization. However, palatalization of a stem-final labial or coronal---including /t\pal/, as in (\ref{russian}c)---is preserved in the reflexive imperative.

\begin{example}
\label{russian}
Russian reflexives \citep[after][]{Jakobson1932}:\footnote{I have taken a number of liberties with \citeauthor{Jakobson1932}'s presentation of the data, which uses an abstract phonemic transcription. Thanks to Lev Blumenfeld (p.c.) for help with the transcription of this data.}

\begin{tabular}{l l l l} %\toprule
   &  infinitive      & imperative \\ %\midrule
a. & [slav\pal itsə]  & [slaf\pal s\pal ə]  & `be glorious'    \\
   & [upram\pal itsə] & [upram\pal s\pal ə] & `be stubborn'    \\
b. & [kras\pal itsə]  & [kras\pal s\pal ə]  & `put on makeup'  \\
   & [ʒar\pal itsə]   & [ʒar\pal s\pal ə]   & `roast'          \\
c. & [zəbytsə]        & [zəbut\pal s\pal ə] & `forget' \\ %\bottomrule
%   & ab\'utsa      & ab\'ujsa     & `put on shoes'   \\
\end{tabular}
\end{example}

\noindent The forms in (\ref{russian}c) have different outcomes for their /t\pal-s/ clusters. \citeauthor{Jakobson1932} proposes that the preservation of palatalization is a special property of the imperative. %A few years later, \citet{Trnka1936}, another member of the Prague Circle, makes the connection between \citeauthor{Jakobson1932}'s hypothesis and phonotactic generalizations.

\subsubsection{The morph as a constraint domain}

A corrolary of this hypothesis is that the phoneme sequences found within morphs may be distinct from those which span multiple morphs. 

This was apparent to structuralists 

the phonologists 
This is apparent from 
a discussion of constraints on vowel sequences in Mixteco given by \citet{Pike1947b}.
It is apparent that \citeauthor{Pike1947b}'s constraints on vowel sequences are not intended to hold across morph boundaries, 

according to the phonological analysis of a Mixteco text published a few years earlier \citep{Pike1944}.

\begin{example}
Mixteco MSCs \citep{Pike1947b} and complex words \citep{Pike1944}:

\begin{tabular}{r l l l} %\toprule
   & MSC & complex exception \\ %\midrule
%a. & *{C}a{C}e & [k\'a-\textsuperscript{n}dee] & `kept \ldots inside' \\
%b. & *{C}\textipa{@}{C}e & [n\`i-k\`ə bə-de] & `who entered'        \\
%c. & *{C}e{C}i & [te-n\'i-ke\textsuperscript{n}da] & `was walking         \\
%d. & *{C}i{C}e & [te-n\`i-kee-t\`ə] & `and went away'      \\ %\toprule
%e. & *{C}e{C}o & b\'e\textglotstopvari e-\v{z}\'o & `our house'          \\
%f. & *{C}eo & ke-o-d\'e & `we eat him'         \\
a. & *{C}a{C}e & [ká\textsuperscript{n}dee] & `kept \ldots inside' \\
b. & *{C}ə{C}e & [nìk\`əbəde] & `who entered'        \\
c. & *{C}e{C}i & [teníke\textsuperscript{n}da] & `was walking         \\
d. & *{C}i{C}e & [tenìkeet\`ə] & `and went away'      \\ 
\end{tabular}
\end{example}

\noindent \citet[][166]{Pike1947b} affirms that the morpheme is ``marked'' by the violation of morpheme-internal sequence restrictions, an idea further developed by \citet{Harris1955} as a morpheme discovery routine.

\subsubsection{Biuniqueness and neutralization}

joos
harris
chomsky

%        \subsection{Early generativism} %% 1.1.2: Early generativism


\subsubsection{\citealt{Stanley1967}}

inventory constraints
sequence constraints

\subsubsection{Surface constraints and occultation}

shibatani
stampe
dell

\emph{A}[mt] `office', \emph{Anbau} `cultivation'
\emph{ei}[n.g]\emph{reifen} `to intervene',

\subsubsection{The duplication problem}

inventory problems 
dell, chung, but also problems

conspiracies

%            \subsubsection{\citet{Stanley1967}} %% 1.1.2: Early generativism


\subsubsection{\citealt{Stanley1967}}

inventory constraints
sequence constraints

\subsubsection{Surface constraints and occultation}

shibatani
stampe
dell

\emph{A}[mt] `office', \emph{Anbau} `cultivation'
\emph{ei}[n.g]\emph{reifen} `to intervene',

\subsubsection{The duplication problem}

inventory problems 
dell, chung, but also problems

conspiracies

%            \subsubsection{Surface constraints and occultation} %% 1.1.2: Early generativism


\subsubsection{\citealt{Stanley1967}}

inventory constraints
sequence constraints

\subsubsection{Surface constraints and occultation}

shibatani
stampe
dell

\emph{A}[mt] `office', \emph{Anbau} `cultivation'
\emph{ei}[n.g]\emph{reifen} `to intervene',

\subsubsection{The duplication problem}

inventory problems 
dell, chung, but also problems

conspiracies

%            \subsubsection{The duplication problem} %% 1.1.2: Early generativism


\subsubsection{\citealt{Stanley1967}}

inventory constraints
sequence constraints

\subsubsection{Surface constraints and occultation}

shibatani
stampe
dell

\emph{A}[mt] `office', \emph{Anbau} `cultivation'
\emph{ei}[n.g]\emph{reifen} `to intervene',

\subsubsection{The duplication problem}

inventory problems 
dell, chung, but also problems

conspiracies

%        \subsection{Autosegmentalism} %\subsection{Autosegmentalism}

also ``supersegmentalism''

The proposals during this period are very diverse. 

\subsubsection{Syllable structure}

hooper
noske

\subsubsection{Morph features}

Leben
Kaye


\subsubsection{The Obligatory Contour Principle}

%            \subsubsection{Syllable structure} %\subsection{Autosegmentalism}

also ``supersegmentalism''

The proposals during this period are very diverse. 

\subsubsection{Syllable structure}

hooper
noske

\subsubsection{Morph features}

Leben
Kaye


\subsubsection{The Obligatory Contour Principle}

%            \subsubsection{Morph features} %\subsection{Autosegmentalism}

also ``supersegmentalism''

The proposals during this period are very diverse. 

\subsubsection{Syllable structure}

hooper
noske

\subsubsection{Morph features}

Leben
Kaye


\subsubsection{The Obligatory Contour Principle}

%            \subsubsection{The Obligatory Contour Principle} %\subsection{Autosegmentalism}

also ``supersegmentalism''

The proposals during this period are very diverse. 

\subsubsection{Syllable structure}

hooper
noske

\subsubsection{Morph features}

Leben
Kaye


\subsubsection{The Obligatory Contour Principle}

%        \subsection{Classical Optimality Theory} %\subsection{Classic Optimality Theory}

%(page numbers here are taken from the published version)

\subsubsection{Richness of the Base}

some quotes from OT doc
\citet{PE}
\citet{Bye2001}

\subsubsection{Freedom of Analysis}

\citet{Smolensky1996}
\citet{PE}

\subsubsection{Undominated constraints}

% complaints about maceahern, gallagher
% bit on loanword adaptation

%            \subsubsection{Richness of the Base} %\subsection{Classic Optimality Theory}

%(page numbers here are taken from the published version)

\subsubsection{Richness of the Base}

some quotes from OT doc
\citet{PE}
\citet{Bye2001}

\subsubsection{Freedom of Analysis}

\citet{Smolensky1996}
\citet{PE}

\subsubsection{Undominated constraints}

% complaints about maceahern, gallagher
% bit on loanword adaptation

%            \subsubsection{Freedom of Analysis} %\subsection{Classic Optimality Theory}

%(page numbers here are taken from the published version)

\subsubsection{Richness of the Base}

some quotes from OT doc
\citet{PE}
\citet{Bye2001}

\subsubsection{Freedom of Analysis}

\citet{Smolensky1996}
\citet{PE}

\subsubsection{Undominated constraints}

% complaints about maceahern, gallagher
% bit on loanword adaptation

%            \subsubsection{Undominated constraints} %\subsection{Classic Optimality Theory}

%(page numbers here are taken from the published version)

\subsubsection{Richness of the Base}

some quotes from OT doc
\citet{PE}
\citet{Bye2001}

\subsubsection{Freedom of Analysis}

\citet{Smolensky1996}
\citet{PE}

\subsubsection{Undominated constraints}

% complaints about maceahern, gallagher
% bit on loanword adaptation

%        \subsection{Recent developments} %\subsection{Recent developments}

\subsubsection{Language acquisition}
% jusczyk forever
% white et al.
% early word learning
% syllables

\subsubsection{Psycholinguistic tasks in adults}

% wordlikeness
% spotting
% error correction, etc.

\subsubsection{Non-native word adaptation}

% perceptual factors

%            \subsubsection{Language acquisition} %\subsection{Recent developments}

\subsubsection{Language acquisition}
% jusczyk forever
% white et al.
% early word learning
% syllables

\subsubsection{Psycholinguistic tasks in adults}

% wordlikeness
% spotting
% error correction, etc.

\subsubsection{Non-native word adaptation}

% perceptual factors

%            \subsubsection{Psycholinguistic tasks in adults} %\subsection{Recent developments}

\subsubsection{Language acquisition}
% jusczyk forever
% white et al.
% early word learning
% syllables

\subsubsection{Psycholinguistic tasks in adults}

% wordlikeness
% spotting
% error correction, etc.

\subsubsection{Non-native word adaptation}

% perceptual factors

%            \subsubsection{Loanword adaptation} %\subsection{Recent developments}

\subsubsection{Language acquisition}
% jusczyk forever
% white et al.
% early word learning
% syllables

\subsubsection{Psycholinguistic tasks in adults}

% wordlikeness
% spotting
% error correction, etc.

\subsubsection{Non-native word adaptation}

% perceptual factors

%    \section{The null hypothesis} %\section{The null hypothesis}

% 1.2.1: The empirical proposal


% 1.2.2: The acquisition principle

% 1.2.3: Falsifiability

%\subsection{Universal constraints?}    % 1.2.4: Universal constraints?
% This may be cut.

(see \citealt[][323f.]{Hockett1947}, \citealt[][415f.]{Nida1948}, \citealt[][50f.]{Anderson1992}, \citealt[][36f.]{Stump2001a}). Irrespective of one's position on this debate, it does not seem correct to extend this hypothesis to lexical roots. For instance, there  

imperfect subjunctive  & īrem     & īrēs     & īret     & īrēmus     & īrētis     & īrent \\
pluperfect subjunctive & issem    & issēs    & isset    & issēmus    & issētis    & issent \\
imperfect subjunctive  & venīrem  & venīrēs  & venīret  & venīrēmus  & venīrētis  & venīrent \\
pluperfect subjunctive & vēnissem & vēnissēs & vēnisset & vēnissēmus & vēnissētis & vēnissent \\

That this is not the 

While it might be possible to analyze \emph{ueni:re} as ``athematic'' /weni:-re/ and \emph{i:re} as /i:-re/, the frequentive \emph{uentita:re} `come often, be wont to come', formed with the \emph{-tit-} frequentive suffix (see \citet[][\S263]{Allen1903}), which selects for the first conjugation (cf.~\emph{agere} `act, make' vs. \emph{actita:re} `act, make often/repeatedly') suggests /wen-/.

It is also interesting to note that the same pattern has emerged in the history of French for a set of verbs which do not share this property in Latin. and the future and conditional indicative forms of \emph{aller} `go' and \emph{cuire} `to cook' in French.\footnote{\emph{Cuire}, interestingly, is not descended from the same conjugation as Latin \emph{i:re}, but rather from Latin \emph{coquere}; thus this syncretism is not simply an etymological relic of Latin. The same holds for other French verbs that inflect in the same manner, such as \emph{conduire} `drive (a vehicle); behave' and \emph{d\'etruire} `destroy'.}

future indicative      & irai   & iras   & ira      & irons    & irez    & iront \\
conditional indicative & irais  & irais  & irait    & irions   & iriez   & iraient \\
future indicative      & cuirai & cuiras  & cuira   & cuirons  & cuirez  & cuiront \\
conditional indicative & curias & cuirais & cuirait & cuirions & cuiriez & cuiront \\

One could even conceive of a largely vacuous principle on morpheme structure which simply requires that some subset of morphs not be null. 

%Anttila 2008
%Dmitrieva et al. 2008a,b
%Duanmu 2009
%Berkley 1994a,b, 2000
%Buckley 1997
%Coetzee 2008, Coetzee and Pater 2008
%Goad 2011
%Graff & Jaeger in press,
%Hammond 1999
%Colavin 2010,
%Frisch 1996, Frisch et al. 2004
%Hayes & Wilson 2008
%Kessler et al. 1997
%Martin 2007
%McGowan 2011
%Mester 1986
%Fudge 1969
%Padgett 1992
%Pierrehumbert 1993, 1994

%% FIXME probably cut this last one

%        \subsection{The empirical proposal} %% 1.2.1: The empirical proposal


%        \subsection{The acquisition principle} %% 1.2.2: The acquisition principle

%        \subsection{Falsifiability} %% 1.2.3: Falsifiability

%        \subsection{Universal constraints?} %% 1.2.4: Universal constraints?
% This may be cut.

(see \citealt[][323f.]{Hockett1947}, \citealt[][415f.]{Nida1948}, \citealt[][50f.]{Anderson1992}, \citealt[][36f.]{Stump2001a}). Irrespective of one's position on this debate, it does not seem correct to extend this hypothesis to lexical roots. For instance, there  

imperfect subjunctive  & īrem     & īrēs     & īret     & īrēmus     & īrētis     & īrent \\
pluperfect subjunctive & issem    & issēs    & isset    & issēmus    & issētis    & issent \\
imperfect subjunctive  & venīrem  & venīrēs  & venīret  & venīrēmus  & venīrētis  & venīrent \\
pluperfect subjunctive & vēnissem & vēnissēs & vēnisset & vēnissēmus & vēnissētis & vēnissent \\

That this is not the 

While it might be possible to analyze \emph{ueni:re} as ``athematic'' /weni:-re/ and \emph{i:re} as /i:-re/, the frequentive \emph{uentita:re} `come often, be wont to come', formed with the \emph{-tit-} frequentive suffix (see \citet[][\S263]{Allen1903}), which selects for the first conjugation (cf.~\emph{agere} `act, make' vs. \emph{actita:re} `act, make often/repeatedly') suggests /wen-/.

It is also interesting to note that the same pattern has emerged in the history of French for a set of verbs which do not share this property in Latin. and the future and conditional indicative forms of \emph{aller} `go' and \emph{cuire} `to cook' in French.\footnote{\emph{Cuire}, interestingly, is not descended from the same conjugation as Latin \emph{i:re}, but rather from Latin \emph{coquere}; thus this syncretism is not simply an etymological relic of Latin. The same holds for other French verbs that inflect in the same manner, such as \emph{conduire} `drive (a vehicle); behave' and \emph{d\'etruire} `destroy'.}

future indicative      & irai   & iras   & ira      & irons    & irez    & iront \\
conditional indicative & irais  & irais  & irait    & irions   & iriez   & iraient \\
future indicative      & cuirai & cuiras  & cuira   & cuirons  & cuirez  & cuiront \\
conditional indicative & curias & cuirais & cuirait & cuirions & cuiriez & cuiront \\

One could even conceive of a largely vacuous principle on morpheme structure which simply requires that some subset of morphs not be null. 

%Anttila 2008
%Dmitrieva et al. 2008a,b
%Duanmu 2009
%Berkley 1994a,b, 2000
%Buckley 1997
%Coetzee 2008, Coetzee and Pater 2008
%Goad 2011
%Graff & Jaeger in press,
%Hammond 1999
%Colavin 2010,
%Frisch 1996, Frisch et al. 2004
%Hayes & Wilson 2008
%Kessler et al. 1997
%Martin 2007
%McGowan 2011
%Mester 1986
%Fudge 1969
%Padgett 1992
%Pierrehumbert 1993, 1994

%    \section{Outline of Part I} %% 1.3: Outline of Part I

Chapter 2

Chapter 3

Chapter 4

\chapter[2]{Accidental and structural gaps in English syllable contact} %\chapter{Categorical and gradient aspects of wordlikeness judgements} 
\label{gradience}

Two nonce words, \emph{blick} [blɪk] and \emph{bnick} [bnɪk], underlie the claim that speakers can rapidly distinguish between possible and impossible words \citet{Halle1962}. In \emph{SPE}, \citet{SPE} use a third word, \emph{bnzk}, to argue that nonce words fall on a cline of well-formedness. Neither \emph{bnick} nor \emph{bnzk} are possible words of English, they are possible words in other languages: stop-nasal onsets are found in Moroccan Arabic (e.g., \emph{bniqa} `closet') and stop-nasal-fricative-stop words in Imdlawn Tashlhiyt Berber \citep{Dell1985}. However, there is some sense in which \emph{bnzk} is even less English-like than \emph{bnick}. Of this, \citeauthor{SPE} write:

\begin{quote}
Hence, a real solution to the problem of ``admissibility'' will not simply define a tripartite categorization of occurring, accidental gap, and inadmissible, but will define the `degree of admissibility' of each potential lexical matrix in such a way as to\ldots{}make numerous other distinctions of this sort (\emph{SPE}:416--417)
\end{quote}

\noindent
The position that the well-formedness of nonce words is consistent the view of syntactic grammaticality taken in \emph{LSLT} and \emph{Aspects} \citep{LSLT,ASPECTS}, where it is claimed that different syntactic vioaltions result in different degrees of ungrammaticality.

Most recent discussions of the ``problem of admissibility'' focus on this cline of wellformedness as it is evidenced in wordlikeness tasks.

\begin{quote}
When native speakers are asked to judge made-up (nonce) words, their intuitions are rarely all-or-nothing. In the usual case, novel items fall along a gradient cline of acceptability. \citep[][9]{Albright2009a}

In the particular domain of phonotactics gradient intuitions are pervasive: they have been found in every experiment that allowed participants to rate forms on a scale.
\citep[][382]{Hayes2008a}

\ldots{}when judgements are elicited in a controlled fashion from speakers, they always emerge as gradient, including all intermediate values. \citep[371]{Shademan2006} 
\end{quote}

At first blush, this would seem to argue against a naïve account of wellformedness judgements in which ill-formedness results when prosodic parsing fails (e.g., \citealt{Ito1989a}, \citealt{Noske1992}, \citealt{OT}). Consider the possibility that a nonce word is ill-formed if it cannot be syllabified without modification. There is a long history for the proposal that impossible words are those whose surface form cannot be parsed into prosodic structures like syllables without further phonological modification (e.g., \citealt[10f.]{Hooper1973}, \citealt[57f.]{Kahn1976}). There is no doubt that prosodic context is involved in determining where segments may occur. For instance, [ŋ] is a possible coda in English, but not a possible onset. Similarly, the [bn] of \emph{bnick} is judged to be admissible (though unattested) in medial position in English (e.g., \emph{sta}[b.n]\emph{ick}; \citealt[97]{Hooper1973}) but not as an onset. English is unable to sylalbify initial onset /bn/ without further modification, so one might expect underlying /bnɪk/ to be subject to deletion \citep[19f.]{Wolf2009}, prothesis, or anaptyxis. Indeed, \citet{Davidson2006b} finds that English speakers use all three of these repairs when asked to mimic foreign pronunciations of obstruent-nasal onsets impossible in English.\footnote{However, \citet{Davidson2005,Davidson2006a} finds acoustic and articulatory distinctions between the ``exscrescent'' schwa used to resolve non-native onset clusters, and the ``lexical'' schwa found in real words. For example, the anaptyctic [z${^\textrm{ə}}$k] produced by English speakers attempting to mimic word-initial [zk] more closely resembles the onset of \emph{scum} than the start of \emph{succumb}. This may undermine any link between non-native realizations and epenthesis of schwa in native words (e.g., in the syllabic allomorphs of the regular past tense and noun plural).}

However, prosodic well-formedness distinctions do not provide a ready explanation for the ``numerous other distinctions'' posited in \emph{SPE}. This is thought to rule out simple prosodic account of wordlikeness judgements.

\begin{quote}
A defect of current grammatical acounts of phonotactics is that they render simple up-or-down decisions concerning well-formedness and cannot account for gradient judgements. \citep[371]{Shademan2006}
\end{quote}

Implicit in this claim is the assumption that the graded judgements obtained from wordlikeness tasks which permit gradient responses require a gradient model of well-formedness. However, it is argued here this is not pertinent to the question of whether wordlikeness is gradient or categorical, because graded judgements are a property of all graded tasks, even when applied to judgements of discrete categories.

\citet[382]{Hayes2008a} ``consider the ability to model gradient intuitions to be an important criterion for evaluation phonotactic models''. However, they fail to ask consider whether categorical models provide a reasonable fit to gradient judgements. Rather than directly evaluating the categorical model of English phonotactics proposed by \citet{Clements1983}, \citeauthor{Hayes2008a} convert these constraints, many of which are exceptionless, into probabilitistic constraints. This suggests that \citeauthor{Hayes2008a} equate ``the ability to model gradient intuitions'' with the ability to make continuous predictions about well-formedness. But if gradient judgements are nothing more than a task effect, a model need not make gradient predictions to fit gradient judgements.

Despite many claims that categorical models are empirically insufficient, there is not a single quantitative investigation of categorical wordlikeness models, and no study which compares categorical and gradient wordlikeness models on an equal footing. This chapter represents a first attempt to fill that lacuna. 

OUTLINE HERE

\section{What wordlikeness might not be}

\subsection{Gradience in wordlikeness judgements}

\subsection{Gradience as a task effect}

\section{Evaluation} 
\label{2evaluation}

\subsection{Method}

\subsubsection{Materials and subjects}

\subsubsection{Computational models}

\subsection{Results}

\subsection{Discussion}

\section{Conclusions}

%\citet{Armstrong1983} cut the Gordian knot by asserting the possibility of observing a false consequent; they find the existence of all-or-nothing concepts like ``odd number'' or (more controversially) ``female'' to be apparent.
%
%\begin{quote}
%Are there definitional concepts? Of course. For example, consider the concept \emph{odd number}. This seems to have a clear definition, a precise description. [\ldots{}] No integer seems to sit on the fence, undecided as to whether it is quite even, or perhaps a bit odd. No odd number seems odder than any other odd number. \citep[274]{Armstrong1983}
%\end{quote}
%
%This proposition has the form of an indicative conditional. While the antecedent of this condition, the presence or absence of intermediate ratings in a graded judgement task, is readily observable, it has no bearing on the truth or falsity of the conditional proposition. However, whereas a false consequent (e.g., an all-or-nothing concept) would potentially falsify the proposition, the consequent, a mental state, cannot be directly observed. If this is the case, then both the proposition and the consequent (the proposition that wordlikeness is itself gradient) fail the test of falsification, placing it beyond the scope of scientific inquiry.
%
%\subsection{Gradience as a liability}
%
%One might reasonably ask whether 
%representations 
%
%
%\citeauthor{Armstrong1983} note that if subjects produce intermediate ratings for these categories, then the naïve proposition is false, and this is what \citeauthor{Armstrong1983} find in their experiments. They ask subjects to rate, on a seven-point scale, the extent to which, e.g., certain odd counting numbers represent the concept ``odd number'', and find that speakers do make use of intermediate values, as shown in Figure \ref{agg}. In summary, the proposition that intermediate ratings indicate mental gradience is either untestable, or it is false, and is thus rejected.
%
%\begin{figure} 
%\centering
%\includegraphics{agg.pdf}
%\caption{Subjects asked to rate the degree to which certain even and odd numbers represent the categories ``even'' and ``odd'', respectively, freely use intermediate ratings.}
%\label{agg} 
%\end{figure}
%
%\citeauthor{Armstrong1983} do not view their results as evidence that subjects have an internal model of odd numbers which is at odds with the formal, all-or-nothing definition, and in fact, several additional experiments show that subjects deploy the extension of the formal definition in other tasks. Reviewing this study, \citet[215]{Schutze2011} writes that the experiments show that judgements are ``sensitive to factors other than our underlying competence''. In the case of wordlikeness, a case can be made that one such factor is similarity to existing words (e.g., \citealt{Bailey2001}, \citealt[][151, fn. 27]{LSLT}, \citealt{Greenberg1964}, \citealt{Ohala1986b}, \citealt{Schutze2005}, \citealt{Sendlmeier1987}, \citealt{Vitz1973}). However, the mere fact that subjects use intermediate ratings does not show that they do so with any systematicity, or that, e.g., the contrast between \emph{447} and \emph{7} with respect could be given any satisfying explanation. \citeauthor{Schutze2011} further suggests that gradient responses are the product of the task itself:
%
%\begin{quote}
%Putting it another way, when asked for gradient responses, participants will find some way to oblige the experimenter; if doing so is incompatible with the experimenter's actual question, they apparently infer that she must have really intended to ask something slightly different. \citep[215]{Schutze2011}
%\end{quote}
%
%This introduces a troubling possibility, that subjects ``oblige'' experimenters by introducing random noise to their categorical judgements when presented with a Likert task. Were this the case, the observed intermediate judgements are unlikely to yield information about speakers' knowledge of possible and impossible words. The alternative considered here is that there exists some model in which the gradience in ratings is systematic. 
%
%The evaluation makes exclusive use of previously published English wordlikeness data. For inclusion in the evaluation, a study must conform to all three of the following conditions. First, the stimuli must consist only of monosyllabic words presented auditorily. Secondly, a significant portion of the stimuli must contain gross phonotactic violations (e.g., \emph{bnick}). Finally, the ratings, averaged across subjects, must be publicly available. The large-scale studies by \citet{Bailey2001} and \citet{Shademan2006,Shademan2007} are ineligible because the former lack stimuli with gross phonotactic violations and the latter data are not available to the public in any form. All data are reproduced in Appendix \ref{appendixA}.
%
%\subsubsection{\citealt{Scholes1966}}
%
%\citet{Scholes1966} conducted a number of English wordlikeness judgement with middle school children. The data used here come from his ``experiment 5'', in which 63 monosyllabic items were presented to 33 seventh-grade students. For each stimulus, the subjects produced forced choice ``yes''/``no'' answers to the question of whether the item ``is likely to be usable as a word of English''. \citet{Hayes2008a} and \citet{Albright2009a} analyze this data as gradient by performing an item averaging using the fraction of ``yes'' answers for each stimulus \citep[see also][]{Pierrehumbert1994,Coleman1997,Frisch2000}. For instance, 22 of the 33 students answered ``yes'' for \emph{shlerk} [ʃlɚk], so it is assigned a score of $0.666$.\footnote{This procedure conflates intraspeaker variation, which in some cases is considerable \citep{Shademan2007} with speaker-internal gradience, and its adoption here should not be construed as an endorsement. Rather, it is provided solely for comparison with prior studies using the \citeauthor{Scholes1966} data.} These stimuli are all ostensibly non-words, but include \emph{clung} [klʌŋ], the preterite and past participle of the verb \emph{cling}, and \emph{brung} [brʌŋ], a dialectical past participle for \emph{bring}. These two words were excluded and the remaining 61 stimuli were submitted to analysis. 
%
%\subsubsection{\citealt{Albright2003b}}
%
%\citet{Albright2003b} gathered wordlikeness judgements to serve as norms for a \emph{wug}-test. 87 items were presented to 24 undergraduate subjects, who rated each word on a seven-point Likert scale. The lowest point on the scale was labeled ``completely bizarre, impossible as an English word'', and that the highest point was labeled ``completely normal, would make a fine English word''.
%
%\begin{table}
%\centering
%\begin{tabular}{l r r r}
%\toprule
%                        & subjects & items & trials \\
%\midrule
%\citealt{Albright2003a} & 24       & 87    & 2,088  \\
%\citealt{Albright2007}  & 68       & 40    & 2,720  \\
%\citealt{HayesInPress}  & 29       & 80    & 2,320  \\
%\citealt{Scholes1966}   & 33       & 63    & 2,178  \\
%\midrule
%\textsc{Total}          & 154      & 270   & 9,306  \\
%\bottomrule
%\end{tabular}
%\caption{Subject and item counts}
%\label{counts}
%\end{table}
%
%\subsection{Model comparison}
%
%The four models used here consist of a binary baseline and three computationally implemented gradient models. These three models are chosen because prior studies have shown they are correlated with wordlikeness judgements. Non-parametric rank correlation statistics are used evaluate the correlation between model predictions and wordlikeness ratings. Rank correlation are used rather than the parametric Pearson correlation statistic that is used by \citet{Hayes2008a}, for instance, because they make none of the potentially troublesome assumptions of the latter method \citep[see][23, fn. 12]{Albright2009a}.
%
%The Spearman $\rho$ is most widely known rank correlation statistic, but it is difficult to give a natural interpretation to this quantity. On the other hand, the Kendall $\tau_b$ and Goodman-Kruskal $\gamma$ can be interpreted as fractions of the number of \emph{concordant} and \emph{discordant} pairs \citep{Noether1981}. Consider the case here, in which model scores are compared with wordlikeness ratings. If a model rates \emph{dresp} [dɹɛsp] more wordlike than *\emph{srest} [sɹɛst] and if speakers further rate \emph{dresp} more English-like than \emph{srest}, than \emph{dresp}/\emph{srest} are a concordant pair. If, however, a model and speakers disagree on the relative ranking of \emph{dresp} and \emph{srest}, the pair is discordant. The $\tau_b$ and $\gamma$ differ only in their treatment of ``ties'' (i.e., if e.g., *\emph{dresp} and *\emph{srest} are scored the same, or rated the same); $\tau_b$ includes a correction for ties, whereas $\gamma$ ignores tied pairs. Much like the familiar Pearson correlation, $\rho$, $\tau_b$ and $\gamma$ are all in the range [$-1$, $1$]. Correlations of the four models are given in Table \ref{cor}.
%
%\subsubsection{Binary baseline}
%
%The binary baseline used here is a crude implementation of the null hypothesis that there are no gradient effects in wordlikeness judgements. To create such a baseline, it is necessary to distinguish between nonce words which contain a gross phonotactic violation and those which do not. As all stimuli here are monosyllables, this task can be further simplified by separately considering the two major subcomponents of the syllable, the onset and rime. Speakers are particularly adept as separating onsets from rimes \citep{Treiman1986,Treiman1995,Fowler1993}, and a large portion of phonotactic generalizations and violations can be localized to either onset or rime \citep[e.g.,][]{Borowsky1989,Fudge1969,Treiman1995,Kessler1997,Treiman2000}.
%
%The baseline considers a nonce syllable to be well-formed if it consists of both an attested onset and an attested rime; the free combination of these two components is the only mechanism by which this model can generalize beyond attested words. It is surely the case that this is an ultimately insufficient model: \citet{Albright2009a} observes, for instance, that [ɛsp] is a well-formed rime, even though it is found in no word of English. This problem is more acute in languages with more permissive syllable structures than those of English. For instance, \citet{Fischer-Jorgensen1952} and \citet{Vogt1954} assert that there are many accidental gaps (i.e., possible but unattested structures) in the inventory of consonant clusters in Georgian, a language which admits as many as five adjacent consonants.\footnote{
%Chapter \ref{clusters} 
%%\citet{Gorman2012c}
%reports a similar result for English syllable contact clusters.} Despite these defects, the baseline outperforms the gradient models in most contexts.
%
%The rating densities from the three studies, linearly transformed to fit within the interval [0, 1] and split according to this binary baseline, are shown in Figure \ref{dsn}. In all three studies, it is possible to discern a relatively sharp separation between valid and invalid clusters, but also the presence of intermediate values.
%
%\begin{figure} 
%\centering
%\includegraphics{density.pdf}
%\caption{Average ratings of individual nonce words, linearly transformed to fit the interval [0, 1], tend to clump into two groups with little overlap; words which consist of  attested onsets and rimes receive ratings near ceiling, whereas ratings of phonotactically invalid words are spread across the lower half of the spectrum.}
%\label{dsn}
%\end{figure}
%
%\begin{table} 
%\centering
%\begin{tabular}{l l l l l}
%\toprule
%Spearman $\rho$          & baseline         & maxent  & bigram           & density          \\
%\midrule
%\citealt{Greenberg1964}  & {0.845}          & {0.765} & {\textbf{0.863}} & {0.648}          \\
%\citealt{Scholes1966}    & {0.791}          & {0.762} & {\textbf{0.827}} & {\textbf{0.827}} \\
%\citealt{Albright2003b}  & {0.725}          & {0.429} & {0.708}          & {\textbf{0.742}} \\
%\midrule
%Kendall $\tau_b$         & baseline         & maxent  & bigram           & density          \\
%\midrule
%\citealt{Greenberg1964}  & {\textbf{0.716}} & {0.585} & {0.692}          & {0.462}          \\
%\citealt{Scholes1966}    & {\textbf{0.664}} & {0.597} & {0.652}          & {0.565}          \\
%\citealt{Albright2003b}  & {\textbf{0.599}} & {0.343} & {0.506}          & {0.556}          \\
%\midrule
%Goodman-Kruskal $\gamma$ & baseline         & maxent  & bigram           & density          \\
%\midrule
%\citealt{Greenberg1964}  & {\textbf{1.000}} & {0.684} & {0.692}          & {0.462}          \\
%\citealt{Scholes1966}    & {\textbf{0.995}} & {0.634} & {0.667}          & {0.614}          \\
%\citealt{Albright2003b}  & {\textbf{0.953}} & {0.656} & {0.509}          & {0.575}          \\
%\bottomrule
%\end{tabular}
%\caption{Rank correlations between wordlikeness ratings and phonotactic models surprisingly reveal that the binary baseline meeds or exceeds the coverage of three state-of-the-art phonotactic models. All correlations are significant at $p = 0.05$.}
%\label{cor}
%\end{table}
%
%
%\subsubsection{Maximum entropy phonotactics}
%
%\citeauthor{Hayes2008a} (\citeyear{Hayes2008a}; henceforth H\&W) develop a sophisticated model of phonotactic grammaticality which estimates a probability distribution over phoneme sequences by weighing constraints according to the principle of maximum entropy, following \citet{Goldwater2003} and \citet{Jager2007}. H\&W report that the predictions of their model are closely correlated with the \citet{Scholes1966} wordlikeness ratings. A direct replication of their predictions was attempted by using the software, model parameters, and training data as described in that study. Since the training of the maximum entropy model is inherently stochastic, producing slightly different outcomes on each run, the lowest scoring of ten runs is reported (H\&W:396), though in general there is not a great deal of variation between individual runs. One limitation of this model is that it is not feasible to score whole words, as the number of constraints which must be inspected grows exponentially as the span of possible constraints increases. Following H\&W and of \citet{Albright2009a}, who also applies the maximum entropy model to the \citet{Albright2003b} norms, the model is trained and scored only on stimulus onsets. However, as a consequence, the maximum entropy model performs particularly poorly on this data set, as many stimuli contain phonotactic violations in rime positions.
%
%\subsubsection{Segment bigram probability} 
%\label{bigram}
%
%The bigram probability of a sequence $ijk$ is the product of the probability of an sequence-initial $i$, the probability that $j$ follows $i$, and the probability that $k$ follows $j$, and the product of sequence-final \emph{k}.
%
%\begin{unlabeledexample}
%$\displaystyle \hat{p}(ijk) = p(i|\textrm{start}) \cdot p(j|i) \cdot p(k|j) \cdot p(\textrm{stop}|k)$
%\end{unlabeledexample}
%
%\noindent \citet{Albright2009a} employs bigram probability to model wordlikeness judgements. While the focus of \citeauthor{Albright2009a}'s study is on developing a model which uses bigrams over phonological features rather than segments themselves, \citeauthor{Albright2009a}'s evaluation, which includes both the \citeauthor{Scholes1966} and \citeauthor{Albright2003b} data sets, finds an advantage for segmental bigrams. \citeauthor{Albright2009a} does not provide an implementation of the featural bigram model, nor does his study describe it in sufficient detail to allow for a new implementation, but segmental bigram scores for the \citeauthor{Albright2003a} data are reported in the appendix. As reported by \citeauthor{Albright2009a}, segmental bigrams outperform featural bigrams (see Table \ref{albrightimproved}).
%
%\citeauthor{Albright2009a} estimates bigram probabilities using the method of maximum likelihood over types in the lexicon. The variant of segmental bigrams used here computes probabilities with a simple type of smoothing in which the count of all possible bigrams (including those never observed) are incremented by one. This technique is known as Laplace, or ``add one'' smoothing. This has the desirable effect that no nonce word is ever assigned a zero probability, and produces a small increase in the correlation between the \citeauthor{Albright2003b} wordlikeness norms compared with the maximum likelihood estimate (Table \ref{albrightimproved}). For all three data sets, this model also consistenly outperforms positional probability models defined by \citet{Vitevitch2004} and \citet{Vaden2009}; given that these model scores are highly correlatd with bigram probability \citep[][54]{Vitevitch1997}, they are not considered further. The bigram model consistently performs well in all the evaluations, and has the highest Spearman correlation with the \citeauthor{Greenberg1964} and \citeauthor{Scholes1966} data, and is frequently second place model to the binary baseline elsewhere.
%
%\begin{table} 
%\centering
%\begin{tabular}{l r r r r}
%\toprule
%                         & \multicolumn{1}{c}{featural bigrams} & \multicolumn{2}{c}{segmental bigrams}  \\
%                         & maximum likelihood                   & maximum likelihood & Laplace smoothing \\
%\midrule
%%Pearson $\rho
%Spearman $\rho$          & 0.638                                & 0.660              & \textbf{0.708}    \\
%Kendall $\tau_b$         & 0.457                                & 0.467              & \textbf{0.506}    \\
%Goodman-Kruskal $\gamma$ & 0.462                                & 0.473              & \textbf{0.509}    \\
%\bottomrule
%\end{tabular}
%\caption{Segmental bigrams outperform ``featural bigrams'', and Laplace smoothing increases the correlation between the segmental bigram model proposed by \citet{Albright2009a}, which uses maximum likelihood estimation, and the \citet{Albright2003b} wordlikeness norms. All correlations are significant at $p = 0.05$.}
%\label{albrightimproved}
%\end{table}
%
%\subsubsection{Neighborhood density} 
%\label{density}
%
%There are now many methods for computing similarity between nonce words and existing words, long thought to be reflected in wordlikeness judgements. 
%For this study, a number of such methods were evaluated, including the Generalized Neighborhood Model \citep{Bailey2001}, PLD20 \citep{Suarez2011}, and a number of variations on neighborhood density \citep{Coltheart1977} provided by \citet{Vaden2009}. The best performance was obtained with the simplest version of neighborhood density, which is defined as the number of real monomorphemic words which can be changed into the target nonce word by a single insertion, deletion, or substitution of a phone.\footnote{\citet{Greenberg1964} use a variant in which only substitutions are counted.} For instance, the neighbors of \emph{blick} include \emph{blink} (insertion), \emph{lick} (deletion) and \emph{black} (substitution). While many studies \citep[e.g.,][]{Bailey2001} report robust lexical similarity effects, it may be that the relatively weak performance of neighborhood density is the result of the presence of gross phonotactic violations.
%
%\subsection{Modeling residual gradience}
%
%The primary result is that no gradient model reliably exceeds the accuracy of the binary baseline. Despite this, there are relatively strong correlations between the binary baseline and these gradient models (see Table \ref{bcor}). From the strong performance of the categorical model one can infer that the gradient models do not reliably predict intermediate ratings, or contrasts in ratings between words which are grouped together. To quantify this, the following method was used to estimate the residual contribution of the three gradient models once gross phonotactic violations are taken into account. Instead of calculating rank correlations directly on the model scores as in Table \ref{cor}, the model scores are mapped to ranks with the additional constraint that all ``valid'' stimuli be ranked above all ``invalid'' stimuli. The resulting ranks are used to compute new correlation statistics. Finally, the binary baseline correlation is subtracted from this number, so that the resulting value is the amount of improvement derived from augmenting the binary model with gradience. These difference numbers are shown in Table \ref{controlled}. In most cases, including the gradient models on top of the binary baseline produces a worse correlation than is obtained with the binary baseline alone.
%
%For $\tau_b$ and $\gamma$, the interpretation of this result is clear. The gradient models assign rankings to the sets of phonotactically valid and invalid clusters, respectively. For instance, the bigram model favors \emph{troog} [tɹuːɡ] over \emph{swach} [swætʃ], though neither contains any gross phonotactic violation. Similarly, the bigram model favors \emph{chwoop} [tʃwuːp] over \emph{zhrick} [ʒɹɪk], even though both contain ill-formed onsets. However, the majority of such predicted contrasts are not reflected in speakers' judgements; for instance, \emph{troog} is rated less English-like than \emph{swach} \citep{Greenberg1964}, contrary to the model predictions. This shows quite starkly that these models fail to reliably predict intermediate ratings.
%
%\begin{table} 
%\centering
%\begin{tabular}{l r r r}
%\toprule
%Kendall $\tau_b$          & maxent         & bigram         & density  \\
%\midrule
%\citealt{Greenberg1964}   & 0.670          & \textbf{0.680} & 0.501 \\
%\citealt{Scholes1966}     & \textbf{0.685} & 0.632          & 0.639 \\
%\citealt{Albright2003b}   & 0.542          & 0.603          & \textbf{0.623} \\
%\bottomrule
%\end{tabular}
%\caption{The binary baseline is strongly correlated with the three gradient model scores; all correlations are significant at $p = 0.05$.}
%\label{bcor}
%\end{table}
%
%\begin{table} 
%\centering
%\begin{tabular}{l r r r}
%\toprule
%$\Delta$ Spearman $\rho$          & maxent            & bigram            & density  \\
%\midrule
%\citealt{Greenberg1964}  & $-0.060$          &  \textbf{0.038} & $-0.017$ \\
%\citealt{Scholes1966}    & $-0.029$          &  \textbf{0.047} & $-0.035$ \\
%\citealt{Albright2003b}  & $-0.008$          & $-0.015$          & \textbf{0.018} \\
%\midrule
%$\Delta$ Kendall $\tau_b$         & maxent            & bigram            & density  \\
%\midrule
%\citealt{Greenberg1964}  & $-0.114$          & \textbf{$-$0.007} & $-0.084$ \\
%\citealt{Scholes1966}    & $-0.067$          & \textbf{0.003}  & $-0.061$ \\
%\citealt{Albright2003b}  & \textbf{$-$0.038} & $-0.092$          & $-0.049$ \\
%\midrule
%$\Delta$ Goodman-Kruskal $\gamma$ & maxent            & bigram            & density  \\
%\midrule
%\citealt{Greenberg1964}  & $-0.268$          & \textbf{$-$0.260} & $-0.337$ \\
%\citealt{Scholes1966}    & $-0.361$          & \textbf{$-$0.313} & $-0.345$ \\
%\citealt{Albright2003b}  & \textbf{$-$0.137} & $-0.443$          & $-0.386$ \\
%\bottomrule
%\end{tabular}
%\caption{The change in rank correlation generated by augmenting the purely binary model with gradient predictions is small and in most cases it is negative.}
%\label{controlled}
%\end{table}
%
%\subsection{The gradience hypothesis}
%
%This chapter has evaluated the axiom of gradience as a falsifiable alternative hypothesis. The surprising result is that virtually all of the apparent coverage of state-of-the-art gradient phonotactic models is simply a reflection of their ability to distinguish between the possible and the totally impossible; beyond this, they are unreliable. A trivial baseline, endowed with few abilities to project beyond the observed data, generally outperforms the state of the art. The projections made by the state-of-the-art gradient models are not like those made by speakers. It remains to be seen is whether any model can be put forth which accurately predicts these intermediate ratings.
%
%These result provide support for recent findings that speakers asked to perform gradient syntactic judgements produce responses closely corresponding to a widely recognized categorical grammatical/ungrammatical distinction \citep{Sprouse2007}.
%
%\subsection{Extensions to the binary baseline}
%
%The strong performance of the binary baseline should not be taken as evidence either that wordlikenesss judgements are binary, or that the binary baseline is a plausible model. The most serious limitation of this evaluation is the primitive nature of the binary baseline. The inability to generalization within onsets and rimes is a serious flaw, as is the assumption of independence of onset and rime. Regarding the rime, \citet{Borowsky1989} proposes a theory of possible rimes in English, which does make the correct prediction regarding the unattested but well-formed [ɛsp]. On the other hand, a cognitively plausible version of this model might need to entertain phonotactic generalizations that are larger than these units, since syllable-sized phonotactic generalizations have been proposed for English \citep[e.g.,][]{Berkley1994a,Berkley1994b,Coetzee2008b,Fudge1969}. 
%
%A possible further extension to the binary baseline would be the introduction of additional levels of wellformedness. While the evaluation has shown that current gradient models do not reliably identify intermediate wellformedness, it does seem possible to identify at least three levels of grammaticality: for instance, one might encode the intuition that \emph{zhlick} [ʒlɪk] is more English-like than \emph{bnick}, though both have unattested onsets. There are precedents for labeling certain attested words as phonotactically ``peripheral'' (see, e.g., the appendices in \citealt{Myers1987} and \citealt{Borowsky1989}); such words are regarded as lexical exceptions to language-general principles of syllabification. If this extends to nonce words, then an intermediate level of grammaticality could be assigned to ``possible'' but formally marked words. Another likely source of additional levels of grammaticality is the cumulative effect of multiple phonotactic violations. While, as \citet{Coleman1997} note, classical Optimality Theory predicts that a nonce word is as ill-formed as its worst deviation from syllable structure, it is possible to imagine that multiple phonotactic violations would result in greater degrees of ill-formedness. The bigram and maxent models make this prediction, as do many others \citep[e.g.,][]{Legendre1990,Levelt2000,Albright2008,Anttila2008a,Pater2009b} but despite this, there is still little data demonstrating cumulative effects in wordlikeness tasks.
%
%\subsection{Language acquisition}
%
%The weak empirical status of gradient phonotactic knowledge as reflected in adults has rammifications for language acquisition under the hypothesis that that infants acquiring language deploy the same representations as adults \citep[e.g.,][]{Macnamara1982,Pinker1984,Crain1991,Carey1995,deVilliers2001,Legate2007}. Gradient wordlikeness judgements in adults would provide support for claims that infants recognize statistical(inherently gradient) dependencies between segments \citep{Jusczyk1994} and use these to segment words \citep{Saffran1996}. An emerging consensus suggests, however, that infants attend to transitional probabilities primarily in the absence of grammatical cues \citep{Gambell2005,Hohne1994,Johnson2001,Jusczyk1999c,Lignos2012b,Mattys2001a,Shukla2007,Lew-Williams2012}. The vacuous nature of current evidence for gradient phonotactic knowledge in adults further weakens any hypothesis that would link statistical learning in infants to adults' behaviors.
%
%\section{Conclusion}
%
%%For instance, infants as young as 4.5 months seem to be aware that English nasal codas agree in place with following obstruents \citep{Mattys2001b,Jusczyk2002,Davidson2004}. 
%%It might be the case that syllable co-occurrence statistics might be little more than a reflection of infants' learning of categorical ``lexical viability'' constraints \citep{Johnson2003} of the sort also seen in adult speech processing \citep[e.g.,][]{Norris1997}. 
%
%% other ratings, but por qué?:
%% 
%% Hayes2000
%% 
%% Albright2003a
%% Albright2003b
%% Prasada1993 
%% 
%% Bard1996
%% 
%% Koo2009
%% 
%% Treiman2000
%% 
%% Warker2006
%% 
%% Massaro1983
%% 
%% Rusaw2009
%do not explicitly state why this data is relevant to the construction of models of wordlikeness. Presumably, these authors believe that these patterns of judgements demonstrate that wordlikeness, as an internal state, is gradient simply because subjects make use of intermediate degrees of wordlikeness in judgement tasks. This proposition, generalized below, is ``naïve'' not because it lacks sophistication, but because it is rooted in a belief in naïve realism, a philosophy which holds that perception provides a relatively direct picture of the nature of the world, an influential view in the cognitive sciences in general (see \citealt{Fodor1981a} for a critique).
%
%\citeauthor{Chomsky1965} were not the first to consider the notion of possible and impossible words. Their primary contribution is that their mentalist perspective: they recognize that naïve speakers effortlessly acquire language-specific generalizations about possible and impossible words and can report them without any explicit training.
%
%However, not all early literature is concerned with gradience. \citet[31]{Vogt1954}, for instance, recognizes that the taxonomic phoneme is insufficient to account for many wordlikeness contrasts. \citeauthor{Vogt1954} observes that allophony may account for the absence of certain phone sequences, but it does not provide a suitable explanation for the absence of initial [bn] in English, nor does it make correct predictions about the surface realization of an underlying initial /bn/. \citeauthor{Vogt1954} concludes that additional grammatical machinery will be needed to account for possible and impossible words. 
%
%Most relevant to the question at hand, \citet{Frisch2000} and \citet{Vitevitch1997} find that speakers' wordlikeness ratings of multisyllabic words are correlated wtih the positional probailities of the constituent syllables. Unfortunately, none of these researchers make any effort to eliminate the possibility that the low positional probability stimuli are ``impossible'' words of English. In fact, 
%Chapter \ref{clusters} argues
%%the author has argued elsewhere \citep{Gorman2012c}
%that many of the stimuli used by \citeauthor{Frisch2000} and \citeauthor{Vitevitch1997} contain illicit word-medial consonant clusters. While \citeauthor{Vitevitch1997} neither control nor manipulate the well-formedness of medial clusters, in a post-hoc test they consider a probabilistic measure of cluster well-formedness, which reveals that cluster well-formedness is correlated with syllable-internal positional probabilities and wordlikeness judgements, but \citeauthor{Vitevitch1997} ultimately conclude this cannot explain all the variation in wordlikeness. 
%
%Using the head-term preference paradigm, \citet{Jusczyk1993b} and \citet{Friederici1993} find that typically-developing children as young as 9 months of age distinguish between nonce words which are and are not phonotactically valid in their target language. \citet{Jusczyk1994} report that 9-month-old children acquiring English also show preferences for nonce words with high positional probability over those with low positional probability. Faciliatory effects of positional probability (i.e., shorter latencies) are reported for other nonce word tasks conducted with adults, including single-word shadowing \citep{Vitevitch1997,Vitevitch1998}, same/different judgements \citep{Vitevitch1999a,Luce2001,Lipinski2005,Vitevitch2005}, and lexical decision \citep{Pylkkanen2002a}.
%
%The aforementioned studies all conclude that the gradient measure of positional probability correlates with behavioral results. As the flaws of the \citet{Vitevitch1997} study demonstrate, the aforementioned studies do little to tease apart the gradient and categorical aspects of phonotactics. More generally, they do little to distinguish between positional probability and closely correlated measures like bigram probability (see \S\ref{bigram} below) or neighborhood density, since these studies carefully select stimuli which either have high or low values for all of positional probability, bigram probablility, and neighborhood density. This is particularly troublesome given that no justification has ever been given for the positional probability measure in the first place; it appears to have been created \emph{ex nihilo}; in contrast, the effects of neighborhood density in various psycholinguistic tasks are emergent properties of many models of speech production \citep[e.g.,][]{Luce1998,Luce2000} and perception \citep{Marslen-Wilson1984,Marslen-Wilson1987,McClelland1986,Norris1994,Norris2000}. 
%\subsubsection{\citealt{Greenberg1964}}
%
%\citet{Greenberg1964} investigated wordlikeness using the technique of free magnitude estimation, a mechanism which has become increasingly popular among syntacticians \citep[e.g.,][]{Bard1996}. At the beginning of the experiment, the subject heard a recording of the word \emph{stick}. In subsequent trials, the subjects heard a nonce word and were asked to report ``how far would you say that is from English?'', with \emph{stick} at ``1''; subjects are told that a word that is ``twice as far from English'' as \emph{stick} should be scored ``2''. The data used here are from \citeauthor{Greenberg1964}'s Experiment B, in which 17 undergraduates were presented 17 stimuli in all. In addition to \emph{stick}, the stimuli include three other English words; these four items were excluded from further analyses, leaving 13 stimuli. As is standard practice in psychophysics \citep[e.g.,][]{Butler1987}, magnitudes were log-transformed prior to analysis.
%

\label{clusters}
    \section{Domain} %% 2.1: Domain

This study is 

The next section describes the procedure used to syllabify surface representations, separating medial clusters into coda and onset. The following section motivates the inventory of onsets used, which allows medial onsets to be distinguished from nuclei.

        \subsection{Data source} %\subsection{Data sources}

The English CELEX database \citep{CELEX},

\citet{Aronoff1976}

\citet{Harley2009}

\citet{Taft1975} ?
\citet{Taft1981} ?
\citet{Taft2004a} ?

\citet{Halle1962}

        \subsection{Syllabification} % 2.1: Domain

This study is 

The next section describes the procedure used to syllabify surface representations, separating medial clusters into coda and onset. The following section motivates the inventory of onsets used, which allows medial onsets to be distinguished from nuclei.

        \subsection{Inventory considerations} % 2.1.3: Inventory considerations

The syllables derived by the above procedure must further be divided into onset, nucleus, and coda, and mapped to underlying representation. In most cases, this is trivial: the nucleus is a pure vowel and can be easily separated from the consonants, which surface faithfully. However, the status of several surface phones in the syllable and the relationship of surface segments to underlying forms is not always obvious. The treatment of those segments is described and motivated below. 

\subsubsection{The velar nasal}
\label{velarnasal}

There is a long-standing debate concering whether English [ŋ] is a phoneme in its own right, demanded by Kiparsky's Alternation Condition \citep{Kiparsky1968} or Lexicon Optimization \citep[][53]{OT}, or simply the allophone of /n/ found before /k, g/ \citep[][65]{Borowsky1986}. The strongest piece of evidence for the pure allophonic analysis (and an attentant process of /g/-\textsc{Deletion}) is the general absence of onset [ŋ], a position where it can never be followed by a dorsal consonant needed to derive the velar allophone. \citet{Pierrehumbert1994} assumes the pure allophonic analysis in her study of syllable contact clusters, and it is adopted here and formalized below in \S\ref{cnpasection}.
%\citealt[][62]{Halle1985a}), 

\subsubsection{[j] onglides}

I assume that the front onglide is not part of the onset except in the case where the onset would ortherwise be null (e.g., \emph{jun}[j]\emph{or}). %\footnote{English glides are transcribed here as full segments, not as ``subsegments'', as this distinction does not appear to be meaningful for English, or empirically motivated for other languages \citep{Rubach2002}.}
When [j] is a simplex onset, it may be followed by any vowel \citep[][276]{Borowsky1986}. However, when [j] is immediately preceded by an onset consonant (e.g., [bj]\emph{ugle}), the following vowel is always [u\lm]. This defective distribution of vowels following [j] follows if the onglide in this context is the first component of a phonological diphthong, and thus part of the nucleus (\citealp[][232]{Hayes1980}, \citealp[][61f.]{Harris1994}). External support for this account comes from the fact that in words like \emph{spew}, [ju\lm] may act as a ``unit'' in language games \citep{Nevins2003,Idsardi2005}, to the exclusion of the rest of preceding consonants. Finally, \citet[][42]{Clements1983} note 
that /m, v/ do not appear in onset clusters except in words like \emph{muse} or \emph{view}, in which I assume the glide is nuclear. 

%One potential problem with this account is noted by \citet{Kaye1996}, who obseres while [ju\lm] may follow any single tautosyllabic consonant, it never follows branching onsets unless they consist of [s] and a single consonant. 
%This is the only sign that [ju\lm] shows an affiliation for the onset. 

\subsubsection{[w] onglides}

The selective properties of the back onglide [w] contrast sharply with those of the front onglide, and I assume that it is assigned to the onset. Whereas the front onglide shows only limited selectivity for preceding tautosylalbic consonants \citep{Kaye1996}, the back onglide [w] is rarely preceded by tautosyllabic consonants other than [k] (e.g., \emph{tran}[kw]\emph{il}). Unlike the front glide, syllable-initial [kw] may be followed by any vowel.
%Whereas [Cju\lm] syllables attracts stress, [kwV] syllables do not \citep[][162f.]{Davis1995}.

\subsubsection{Post-vocalic \emph{r}}

Like this study, \citet{Pierrehumbert1994} uses a dictionary that transcribes RP, in which word-medial post-vocalic \emph{r} has been lost. There are independent reasons to believe that \emph{r}-full dialects of English assign post-vocalic \emph{r} to the nucleus and is not part of syllable contact clusters in such dialects. First, \citet[][255]{Harris1994} notes that the number of vowel contrasts is greatly reduced before \emph{r} when compared to other coronal consonants. For instance, the majority of North America has lost the historical contrast between \emph{Mary}, \emph{marry}, and \emph{merry}, and one holdout, Philadelphia, is losing the the contrast between \emph{merry} and \emph{Murray} \citep[14f.]{ANAE}. Another source of evidence for the nuclear status of post-vocalic \emph{r} comes from its failure to trigger two processes in English variable phonology. \citeauthor{Harris1994} reports that a variable process of /t/-\textsc{Glottalization} in many dialects of British English is blocked when /t/ is preceded by any consonant except post-vocalic \emph{r}.

\ex /t/-\textsc{Glottalization} in British English \citep[after][195, 258]{Harris1994}: \\
\begin{tabular}{l l l@{} l}
a. & fis[t]   & * & fis[ʔ]   \\
   & mis[t]er & * & mis[ʔ]er \\
b. & par[t]   &   & par[ʔ]   \\
   & car[t]on &   & car[ʔ]on \\
\end{tabular}
\xe

\noindent
Similarly, while /t, d/ delete in word-final position when immediately preceded by a consonant, including sonorants /n, l/, as in (\nextx a), deletion of /t, d/ after post-vocalic \emph{r} in American English is ``rare or nonexistent'' \citep[][8]{Guy1980}.

\ex /t, d/-\textsc{Deletion} in American English: \\
\begin{tabular}{l l l@{} l}
a. & be[lt]  &   & be[l]  \\
%   & we[ld]  &   & we[l]  \\
%   & cha[nt] &   & cha[n] \\
   & me[nd]  &   & me[n]  \\
%b. & fl[ɜ˞]  & * & fl[ɜ˞]  \\
%b. & fl[ɝt]  & * & fl[ɝ]  \\
b. & sh[ɚt]  & * & sh[ɚ] \\
%   & w[ɚd]   & * & w[ɚ]  \\
   & c[ɚd]   & * & c[ɚ]  \\
\end{tabular}
\xe

%Finally, \citet[][251]{Fromkin1973} also presents evidence that post-vocalic \emph{r} may behave as if nuclear in speech errors. 

        \subsection{Corpus statistics} % 2.1.4: Corpus statistics

Applying the above syllabification and mapping techniques to the 6,876 simplex words in CELEX produces 21 unique codas and 40 unique onsets. Of the 840 possible syllable contact clusters that could be produced by free combination of these codas and onsets, 158 are attested, producing an 18.8\% saturation rate. 

    \section{Evaluation} %% 2.2: Evaluation

\subsection{Static and derived constraints}                     \subsection{Data sources}

While these are by no means the only wordlikeness s studies in Englihs, these studies all have the following characteristics: 
the item-averaged ratings are reported in the document,
the stimuli are presented auditorily,
and at least some stimuli are not phonotactically valid in English.

(some caveats about subjects

\citep{Shademan2007}

cost of prior averaging
\citep{Baayen2004}

\subsubsection{\citealt{Greenberg1964}}

\citet{Greenberg1964} investigate English wordlikeness using the technique of free magnitude estimation, a mechanism which has become increasingly popular in syntax \citep[e.g.,][]{Bard1996,Sprouse2011}. At the beginning of each run of the experiment, a subject heard a recording of the word \emph{stick}. In each subsequent trial, the subject heard a new item and was asked to assign a value for ``How far would you say that is from English?'', with \emph{stick} representing ``1''. They were explicitly instructed that a word which was ``twice as far from English'' as \emph{stick} should be given ``a number that is twice as large'' (op. cit., 160). The data used here are from their ``experiment B'', in which 17 undergraduates were presented 13 items. This is the only study used here in which actual English words (\emph{stick}, \emph{grass}, \emph{spell}, and \emph{truck}) were intentionalyl included among the stimuli.

\subsubsection{\citealt{Scholes1966}}

\citet{Scholes1966} conducts a number of English wordlikeness judgement with middle school children. The data used here come from his ``experiment 5'', in which 66 items were presented to 33 seventh-grade students. For each stimulus, the subjects produced forced choice ``yes''/``no'' answers to the question of whether the item ``is likely to be usable as a word of English''. \citet{Hayes2008a} and \citet{Albright2009a} analyse this data, which is binary at the trial level, as gradient by performing an item averaging using the fraction of ``yes'' answers for each stimulus.

%mrupation problem

\subsubsection{\citealt{Albright2003b}}

\citet{Albright2003b} gather wordlikeness data to provide norms for the \emph{wug}-task that is the focus of their study. 87 items were presented to 20 undergraduate subjects, who rated each word on a seven-point Likert scale; though \citeauthor{Albright2003b} do not provide the precise instructions the subjects were given, they do report that ``1'', the lowest point on the scale, was labeled
``completely bizarre, impossible as an English word'', and that the highest point, ``7'', was named ``complete normal, would make a fine English word''. 

\subsubsection{\citet{Albright2007}}

Wordlikeness is the focus of an unpublished study by \citet{Albright2007}. \citeauthor{Albright2007} presetns an unknown number of subjects with 40 nonce words in addition to 170 fillers, which are rated on the same seven-point Likert scale used by \citet{Albright2003b}.

\subsection{Computational models}                               \subsection{Assessing models}

\subsubsection{Basic correlations}
\subsubsection{Intramodal correlations}
\subsubsection{Residual correlations}

\subsection{Statistical properties of English syllable contact} \subsection{Statistical properties of syllable contact}

Above, we have seen that constraints on URs which are derive from phonological alternations restrict the contents of the lexicon. Applying these derived constraints to the lexicon increases saturation from 18.8\% to 28.8\% and incurs only a handful of exceptions. Yet, despite the fact that ``attestation'' is the minority pattern, attempts to identify static lexical constraints, whether by hand or computational model, lend little additional predictive power. There is no evidence that the English syllable contact inventory is subject to any static constraint at all. I am forced to conclude that most, if not all, of 71.2\% of possible clusters which are unattested are accidental gaps. Below, I show that this state of affairs follows directly from the sparse distribution of codas and onsets. 

\subsubsection{The $Z_r$ transform}

In a sparse distribution, many types occur at the same low frequencies. \citet{Good1953} notes that this produces a type of quantization, an artificially long and flat right tail in the frequency spectrum. \citet[][29]{Church1991} propose a method to eliminate this quantization, the $Z_r$ transform, defined as follows.
Let $r, n$ be defined such that $n_i$ is the number of types which occur at frequency $r_i$ (i.e., $n$ contains frequencies of type frequencies). Further, let $r$ be sorted in increasing order. $Z$ is simply each element of $n$ scaled by the nearest points to the left and right. 

\ex $\displaystyle Z_i = \frac{2 n_i}{r_{i + 1} - r_{i - 1}}$ \xe 

\noindent
\citeauthor{Church1991} do not define this transform for the lowest and highest points (i.e., when $i = 1$ or $i = N$), and so I extend the definition by scaling these points using only one inward-facing point.

\ex $\displaystyle Z_1 = \frac{n_1}{r_2 - r_1}$ \xe
\ex $\displaystyle Z_N = \frac{n_N}{r_N - r_{N - 1}}$ \xe

%\indent
%The effect of this transform can be seen in Figure \ref{zr}. 

%\begin{figure}
%\centering
%\includegraphics{zr.pdf}
%\caption{The SUBTLEX word frequency norms \citep{Brysbaert2009} show a long right tail in the log-frequency/log-rank space (left panel), and the $Z_r$ transform (right panel) smooths out this tail.}
%\label{zr}
%\end{figure}

\subsubsection{Coda, onset, and cluster sparsity}

A number of previous studies \citep[e.g.,][]{Sigurd1968,Good1969,Borodovsky1989,Witten1990,Martindale1996,Tambovtsev2007} have observed that phonemes and graphemes exhibit sparse type (i.e., lexical) and token frequency distributions. The lexical frequencies of the medial codas ond onsets which make up syllable contact clusters show a similar distribution. In Figure \ref{codaonset}, $Z_r$-transformed lexical frequencies of codas and onsets are plot against rank, in log-log spaces. The near-linear relationship that obtains indicates that the data can be modeled by a generalization of Zipf's law \citep{Zipf1949}.

\begin{figure}
\centering
\includegraphics{coda.pdf} 
\includegraphics{onset.pdf}
\caption{Medial coda and medial onset lexical frequency exhibit the Zipfian log-log-linear relationship between rank and frequency.}
\label{codaonset}
\end{figure}

\ex $\displaystyle f(r; C, \alpha) = \frac{C}{r^\alpha}$ \xe

\noindent
The parameters $C$, a constant sensitive to sample size, and $\alpha$, the slope of the rank/frequency relationship in log-log space, are easily computed using the method of least squares in a linear regression, where $\epsilon$ represents the error term.

\ex $\displaystyle \textrm{log}~Z_r \sim C + \alpha~\textrm{log}~r + \epsilon$ \xe

\noindent
Codas ($\alpha = -0.720$, $R^2 = 0.749$) and onsets ($\alpha = -1.056$, $R^2 = 0.828$) are well-fit by this distribution, though there are some outliers. Similarly, clusters, shown in Figure \ref{clus}, are also a good fit to the generalized Zipf's law ($\alpha = -0.588$, $R^2 = 0.924$). 

\begin{figure}
\centering
\includegraphics{cluster.pdf}
\caption{Syllable contact clusters, themselves composed of Zipfian medial codas and medial onsets, also exhibit a Zipfian distribution.}
\label{clus}
\end{figure}

That clusters (and the codas and onsets that make them up conforms to Zipf's law is not itself a deep observation. Zipfian distributions are characteristic of both syntactic rules \citep{Yang2009}, but also word \citep{Baroni2009} and phoneme \citep{Daland2011a} n-grams, and tokens in non-linguistic symbol systems \citep{Chomsky1958,Sproat2010} or randomly-generated texts \citep{Miller1957,Li1992}. The import of this statistical property for the study of cluster phonotactics is that it makes it difficult, on statistical grounds alone, to determine which unobserved events are excluded, and which are accidentally missing. 

\citet{Good1953} puts forth a simple theory of just how frequent unobserved events might be. He proposes to estimate $\hat{p}_0$, the probability of all unseen events, as the ratio of events which occur just once $n_1$, to the number of observations, $\sum N$. This quantity is known as the Good-Turing estimate, since it was originally developed by Alan Turing for codebreaking during the second World War.

\ex $\displaystyle \hat{p}_0 = \frac{n_1}{\sum N}$ \xe

\noindent
In the CELEX data, $65$ clusters occur only once, and there are 873 clusters in all, so $\hat{p}_0 = 0.074$. One way to interpret this quantity is with respect to a ``replication''. If it were possible to generate a new corpus of syllable contact clusters, approximately 7\% of cluster tokens will be ones that were not attested in the previous sample, i.e., they accidentally failed to be sampled. This is a not-inconsiderable quantity. 

Unfortunately, the Good-Turing estimate does not provide a way to determine which unattested clusters might be found in a replication. But I have already proposed that the null hypothesis of generative phonology, combined with the principle of Stampean occultation, provides a way out of this conundrum. Unattested clusters like *[m.kl] are structurally excluded (since such a cluster is occulted by \textsc{Coda Nasal Place Assimilation}), whereas *[s.l] is probably an accidental gap.

\subsubsection{Sampling simulation}

As a final investigation into this data, I consider the statistical properties of a randomly generated novel ``lexicon''. The simulated lexicon is generated by  generating a distribution of clusters, by repeatedly applying the following procedure.

\ex {Simulation procedure}: \\
    \begin{tabular}{l l}
    a. & Sample a medial coda according to the observed probabilities  \\
    b. & Sample a medial onset according to the observed probabilities \\
    c. & Apply the SPE rules to the resulting cluster                  \\
    \end{tabular}
\xe

\noindent
This is a simulation version of the model proposed by \citet{Pierrehumbert1994} and \citet{Coleman1997}, but also takes occulting phonology into account. 

While this procedure can be repeated indefinitely (using the CELEX data, listed in Appendix \ref{A}), consideration of a single characteristic run of the simulation should suffice. The saturation rate of the simulated sample (18.2\%) is very close to the real sample (18.8\%). The slope of the log-rank/log-frequency relationship for the simulated data ($\alpha = -0.557$) is also very close to the observed slope for the CELEX data ($\alpha = -0.588$). And, as shown in Figure \ref{sim}, the two distributions are nearly indistinguishable. 

\begin{figure}
\centering
\includegraphics{sim.pdf}
\caption{A comparison of the observed lexical frequencies of English syllable contact clusters to an ``English lexicon'' simulated by combining.}
\label{sim}
\end{figure}

It is important to note that the results of this simulation should not be construed as an endorsement of a revised version of the \citet{Pierrehumbert1994} and \citet{Coleman1997} model which also incorporates occultation. In fact, as was shown above, the independent frequencies of coda and onset are very poor predictors of what clusters will be attested even when occultation is taken into account, and the set of attested clusters only partially overlaps those produced by the simulation. I propose that this is because the lexicon is not only finite, but far smaller than the combinatoric possibilities provided by phoneme sequences, making accidental gaps unavoidable. That these gaps will be arbitary from the perspective of phonology is nothing more than a corrolary of the principle of \emph{l'arbitraire du signe}. 

It is quite possible that phonology (and, perhaps, static phonotactics) acts as a filter on what URs are learned, but a cluster might be overrepresented or underrepresented for reasons that are quite external to phonology, e.g., language contact or historical change. Similarly, the factors that cause a /n/ to be a common word-medial coda in English may be independent of the propensity of that coda to combine with medial onset, assuming the resulting cluster is phonologically licit. The import of this simulation is somewhat more subtle; it simply shows free combination of codas and onsets according to their independent probabilities, constrained only by occultation, would produce the same sparse distribution of clusters that was used to argue for static phonotactics.

%\citet{Evert2004}
%\includegraphics{zipf.pdf}


        \subsection{Static and derived constraints} \subsection{Data sources}

While these are by no means the only wordlikeness s studies in Englihs, these studies all have the following characteristics: 
the item-averaged ratings are reported in the document,
the stimuli are presented auditorily,
and at least some stimuli are not phonotactically valid in English.

(some caveats about subjects

\citep{Shademan2007}

cost of prior averaging
\citep{Baayen2004}

\subsubsection{\citealt{Greenberg1964}}

\citet{Greenberg1964} investigate English wordlikeness using the technique of free magnitude estimation, a mechanism which has become increasingly popular in syntax \citep[e.g.,][]{Bard1996,Sprouse2011}. At the beginning of each run of the experiment, a subject heard a recording of the word \emph{stick}. In each subsequent trial, the subject heard a new item and was asked to assign a value for ``How far would you say that is from English?'', with \emph{stick} representing ``1''. They were explicitly instructed that a word which was ``twice as far from English'' as \emph{stick} should be given ``a number that is twice as large'' (op. cit., 160). The data used here are from their ``experiment B'', in which 17 undergraduates were presented 13 items. This is the only study used here in which actual English words (\emph{stick}, \emph{grass}, \emph{spell}, and \emph{truck}) were intentionalyl included among the stimuli.

\subsubsection{\citealt{Scholes1966}}

\citet{Scholes1966} conducts a number of English wordlikeness judgement with middle school children. The data used here come from his ``experiment 5'', in which 66 items were presented to 33 seventh-grade students. For each stimulus, the subjects produced forced choice ``yes''/``no'' answers to the question of whether the item ``is likely to be usable as a word of English''. \citet{Hayes2008a} and \citet{Albright2009a} analyse this data, which is binary at the trial level, as gradient by performing an item averaging using the fraction of ``yes'' answers for each stimulus.

%mrupation problem

\subsubsection{\citealt{Albright2003b}}

\citet{Albright2003b} gather wordlikeness data to provide norms for the \emph{wug}-task that is the focus of their study. 87 items were presented to 20 undergraduate subjects, who rated each word on a seven-point Likert scale; though \citeauthor{Albright2003b} do not provide the precise instructions the subjects were given, they do report that ``1'', the lowest point on the scale, was labeled
``completely bizarre, impossible as an English word'', and that the highest point, ``7'', was named ``complete normal, would make a fine English word''. 

\subsubsection{\citet{Albright2007}}

Wordlikeness is the focus of an unpublished study by \citet{Albright2007}. \citeauthor{Albright2007} presetns an unknown number of subjects with 40 nonce words in addition to 170 fillers, which are rated on the same seven-point Likert scale used by \citet{Albright2003b}.

        \subsection{Computational models} \subsection{Assessing models}

\subsubsection{Basic correlations}
\subsubsection{Intramodal correlations}
\subsubsection{Residual correlations}

        \subsection{Statistical properties of syllable contact} \subsection{Statistical properties of syllable contact}

Above, we have seen that constraints on URs which are derive from phonological alternations restrict the contents of the lexicon. Applying these derived constraints to the lexicon increases saturation from 18.8\% to 28.8\% and incurs only a handful of exceptions. Yet, despite the fact that ``attestation'' is the minority pattern, attempts to identify static lexical constraints, whether by hand or computational model, lend little additional predictive power. There is no evidence that the English syllable contact inventory is subject to any static constraint at all. I am forced to conclude that most, if not all, of 71.2\% of possible clusters which are unattested are accidental gaps. Below, I show that this state of affairs follows directly from the sparse distribution of codas and onsets. 

\subsubsection{The $Z_r$ transform}

In a sparse distribution, many types occur at the same low frequencies. \citet{Good1953} notes that this produces a type of quantization, an artificially long and flat right tail in the frequency spectrum. \citet[][29]{Church1991} propose a method to eliminate this quantization, the $Z_r$ transform, defined as follows.
Let $r, n$ be defined such that $n_i$ is the number of types which occur at frequency $r_i$ (i.e., $n$ contains frequencies of type frequencies). Further, let $r$ be sorted in increasing order. $Z$ is simply each element of $n$ scaled by the nearest points to the left and right. 

\ex $\displaystyle Z_i = \frac{2 n_i}{r_{i + 1} - r_{i - 1}}$ \xe 

\noindent
\citeauthor{Church1991} do not define this transform for the lowest and highest points (i.e., when $i = 1$ or $i = N$), and so I extend the definition by scaling these points using only one inward-facing point.

\ex $\displaystyle Z_1 = \frac{n_1}{r_2 - r_1}$ \xe
\ex $\displaystyle Z_N = \frac{n_N}{r_N - r_{N - 1}}$ \xe

%\indent
%The effect of this transform can be seen in Figure \ref{zr}. 

%\begin{figure}
%\centering
%\includegraphics{zr.pdf}
%\caption{The SUBTLEX word frequency norms \citep{Brysbaert2009} show a long right tail in the log-frequency/log-rank space (left panel), and the $Z_r$ transform (right panel) smooths out this tail.}
%\label{zr}
%\end{figure}

\subsubsection{Coda, onset, and cluster sparsity}

A number of previous studies \citep[e.g.,][]{Sigurd1968,Good1969,Borodovsky1989,Witten1990,Martindale1996,Tambovtsev2007} have observed that phonemes and graphemes exhibit sparse type (i.e., lexical) and token frequency distributions. The lexical frequencies of the medial codas ond onsets which make up syllable contact clusters show a similar distribution. In Figure \ref{codaonset}, $Z_r$-transformed lexical frequencies of codas and onsets are plot against rank, in log-log spaces. The near-linear relationship that obtains indicates that the data can be modeled by a generalization of Zipf's law \citep{Zipf1949}.

\begin{figure}
\centering
\includegraphics{coda.pdf} 
\includegraphics{onset.pdf}
\caption{Medial coda and medial onset lexical frequency exhibit the Zipfian log-log-linear relationship between rank and frequency.}
\label{codaonset}
\end{figure}

\ex $\displaystyle f(r; C, \alpha) = \frac{C}{r^\alpha}$ \xe

\noindent
The parameters $C$, a constant sensitive to sample size, and $\alpha$, the slope of the rank/frequency relationship in log-log space, are easily computed using the method of least squares in a linear regression, where $\epsilon$ represents the error term.

\ex $\displaystyle \textrm{log}~Z_r \sim C + \alpha~\textrm{log}~r + \epsilon$ \xe

\noindent
Codas ($\alpha = -0.720$, $R^2 = 0.749$) and onsets ($\alpha = -1.056$, $R^2 = 0.828$) are well-fit by this distribution, though there are some outliers. Similarly, clusters, shown in Figure \ref{clus}, are also a good fit to the generalized Zipf's law ($\alpha = -0.588$, $R^2 = 0.924$). 

\begin{figure}
\centering
\includegraphics{cluster.pdf}
\caption{Syllable contact clusters, themselves composed of Zipfian medial codas and medial onsets, also exhibit a Zipfian distribution.}
\label{clus}
\end{figure}

That clusters (and the codas and onsets that make them up conforms to Zipf's law is not itself a deep observation. Zipfian distributions are characteristic of both syntactic rules \citep{Yang2009}, but also word \citep{Baroni2009} and phoneme \citep{Daland2011a} n-grams, and tokens in non-linguistic symbol systems \citep{Chomsky1958,Sproat2010} or randomly-generated texts \citep{Miller1957,Li1992}. The import of this statistical property for the study of cluster phonotactics is that it makes it difficult, on statistical grounds alone, to determine which unobserved events are excluded, and which are accidentally missing. 

\citet{Good1953} puts forth a simple theory of just how frequent unobserved events might be. He proposes to estimate $\hat{p}_0$, the probability of all unseen events, as the ratio of events which occur just once $n_1$, to the number of observations, $\sum N$. This quantity is known as the Good-Turing estimate, since it was originally developed by Alan Turing for codebreaking during the second World War.

\ex $\displaystyle \hat{p}_0 = \frac{n_1}{\sum N}$ \xe

\noindent
In the CELEX data, $65$ clusters occur only once, and there are 873 clusters in all, so $\hat{p}_0 = 0.074$. One way to interpret this quantity is with respect to a ``replication''. If it were possible to generate a new corpus of syllable contact clusters, approximately 7\% of cluster tokens will be ones that were not attested in the previous sample, i.e., they accidentally failed to be sampled. This is a not-inconsiderable quantity. 

Unfortunately, the Good-Turing estimate does not provide a way to determine which unattested clusters might be found in a replication. But I have already proposed that the null hypothesis of generative phonology, combined with the principle of Stampean occultation, provides a way out of this conundrum. Unattested clusters like *[m.kl] are structurally excluded (since such a cluster is occulted by \textsc{Coda Nasal Place Assimilation}), whereas *[s.l] is probably an accidental gap.

\subsubsection{Sampling simulation}

As a final investigation into this data, I consider the statistical properties of a randomly generated novel ``lexicon''. The simulated lexicon is generated by  generating a distribution of clusters, by repeatedly applying the following procedure.

\ex {Simulation procedure}: \\
    \begin{tabular}{l l}
    a. & Sample a medial coda according to the observed probabilities  \\
    b. & Sample a medial onset according to the observed probabilities \\
    c. & Apply the SPE rules to the resulting cluster                  \\
    \end{tabular}
\xe

\noindent
This is a simulation version of the model proposed by \citet{Pierrehumbert1994} and \citet{Coleman1997}, but also takes occulting phonology into account. 

While this procedure can be repeated indefinitely (using the CELEX data, listed in Appendix \ref{A}), consideration of a single characteristic run of the simulation should suffice. The saturation rate of the simulated sample (18.2\%) is very close to the real sample (18.8\%). The slope of the log-rank/log-frequency relationship for the simulated data ($\alpha = -0.557$) is also very close to the observed slope for the CELEX data ($\alpha = -0.588$). And, as shown in Figure \ref{sim}, the two distributions are nearly indistinguishable. 

\begin{figure}
\centering
\includegraphics{sim.pdf}
\caption{A comparison of the observed lexical frequencies of English syllable contact clusters to an ``English lexicon'' simulated by combining.}
\label{sim}
\end{figure}

It is important to note that the results of this simulation should not be construed as an endorsement of a revised version of the \citet{Pierrehumbert1994} and \citet{Coleman1997} model which also incorporates occultation. In fact, as was shown above, the independent frequencies of coda and onset are very poor predictors of what clusters will be attested even when occultation is taken into account, and the set of attested clusters only partially overlaps those produced by the simulation. I propose that this is because the lexicon is not only finite, but far smaller than the combinatoric possibilities provided by phoneme sequences, making accidental gaps unavoidable. That these gaps will be arbitary from the perspective of phonology is nothing more than a corrolary of the principle of \emph{l'arbitraire du signe}. 

It is quite possible that phonology (and, perhaps, static phonotactics) acts as a filter on what URs are learned, but a cluster might be overrepresented or underrepresented for reasons that are quite external to phonology, e.g., language contact or historical change. Similarly, the factors that cause a /n/ to be a common word-medial coda in English may be independent of the propensity of that coda to combine with medial onset, assuming the resulting cluster is phonologically licit. The import of this simulation is somewhat more subtle; it simply shows free combination of codas and onsets according to their independent probabilities, constrained only by occultation, would produce the same sparse distribution of clusters that was used to argue for static phonotactics.

%\citet{Evert2004}
%\includegraphics{zipf.pdf}

    \section{Conclusion} %\section{Conclusions}

It is not the case that this evidence is inconsistent with the hypothesis that there are additional degrees of wordlikeness: it is just a demonstration that these data to not require it. 

\subsection{Future directions}

Imagine a run of a wordlikeness experiment in which a subject rates the nonce word \emph{dresp} higher than the nonce word \emph{srest}. This observation should indicates the need for many more.

Did the subject faithfully perceive the unattested [sr] onset of \emph{srest}, or was it perceived as, e.g., [ʃr]? Does the subject have intuitions about how to ``correct'' \emph{srest} as to make it more English-like?\footnote{This appears to be a prediction of constraint-based theories of phonotactic grammaticality. Thanks to Colin Wilson (p.c.) for some discussion on this point.} Does the subject rate the relative wordlikeness of \emph{dresp} and \emph{srest} the same a day, a week, a month, a year later? 
%Are relative judgements ``commutative'' in the sense of \citep{Sprouse2011}, so that if another word is \emph
Do other subjects agree with this relative judgement? And, if not, does this contrast correlate with other subject-specific properties such as age, level of education, exposure to foreign languages, and so on? 

%\chapter{Static and derived constraints on Turkish vowel harmony} %\chapter{Static and derived phonotactic preferences in Turkish}
\label{turkish}

There is a long tradition of distinguishing between structural phonotactic gaps and those which are simply accidental \citep[e.g.,][]{Fischer-Jorgensen1952,Saporta1955,Saporta1958,Vogt1954}. In purely statistical terms, an accidental gap is one which is likely to arise without any antecedent cause, i.e., by chance. The contribution of early generative discussions of this contrast \citep[e.g.,][]{Halle1962,Chomsky1965,SPE} is to ground the definitions of \emph{structural} and \emph{accidental} gaps not in statistical expectations, but rather in mentalism. For \citeauthor{SPE}, an accidental gap is one which is not learned, one which lacks a mental antecedent.

Much of the latter-day literature on phonotactic knowledge, while adopting the representational machinery of generative phonology, takes less interest in the question of what generalizations speakers internalize. In a seminal paper, \citet{McCarthy1988} uses a simple statistical technique to validate a generalization about Arabic root consonant co-occurrence facts originally due to \citet{Greenberg1950}. The assumption appears to be that static co-occurrence restrictions which have a statistically significant trace in the lexicon will also be reflected in psycholinguistic tasks such as wordlikeness judgements. While the Arabic restriction discussed by \citeauthor{McCarthy1988} was ultimately investigated experimentally \citep{Frisch2004}, this is the case for only a small minority of the studies inspired by \citeauthor{McCarthy1988}'s study; in most cases, phonotactic generalizations are inferred directly from lexical statistics \citep[e.g.,][]{Anttila2008a,Berkley1994b,Berkley1994a,Berkley2000,Brown2010,Buckley1997,Coetzee2008a,Elmedlaoui1995,Graff2011,MacEachern1999,Kinney2005,Kawahara2006,Martin2007,Martin2011,Mester1988,Miller-Ockhuizen2003,Padgett1992,Padgett1995,Pozdniakov2007,Yip1989}. In a study of consonant co-occurrence restrictions in Gitksan, \citet{Brown2010} claims that \emph{any} statistically significant pattern in the lexicon is one that is internalized by speakers:

\begin{quote}
\ldots{} the patterns outlined above are statistically significant. Given this, it stands that these sound patterns should be explained by some linguistic mechanism. \citep[][48]{Brown2010}
\end{quote}

There are several arguments against this inference. First, negative results in formal learnability theory (see \citealt{Yang2012} for a recent review) show that there are whole classes of rules and constraints which cannot be learned under standard assumptions about acquisition (e.g., that the data is finite, can appear in any order, and that speakers do not have to request grammatical judgements from adults to achieve adult-like competence). 

Secondly, there are a number of results which statistically reliable patterns are not learned by speakers \citep[e.g.,][]{Becker2011,Hayes2006,Hayes2009,HayesInPress}. This argues against a direct link between statistical criteria and the synchronic reality of phonotactic generalizations. The danger is that grammatical models which closely mimic lexical statistics but which lack psycholinguistic support (e.g., Gitksan: \citealt{Brown2010}; Muna: \citealt{Anttila2008a}, \citealt{Coetzee2008a}; Shona: \citealt[][385]{Hayes2008a}) may overstate the phonotactic generalizations internalized by speakers. 

Another type of non-learning occurs when generalizations are statistically reliable but demonstrably incorrect. \citet{Legate2012}, for instance, note that the overwhelming majority of English words have initial primary stress. While this suggests a quantity-insensitive stress system, English stress is in fact quantity-sensitive (see \citealt{Halle1998c} and references therein). 

Finally, there are many cases of statistical generalizations with a known diachronic cause but lacking any synchronic reality. One of the earliest arguments of this type was put forth by Saussure (\citeyear[202f.]{CLG}) in a discussion of rhotacism, a diachronic process which changed Old Latin intervocalic \emph{s} to \emph{r}. As a result, Classical Latin intervocalic \emph{s} is rare. However, Saussure denies that this generalization has any status in Classical Latin.\footnote{See \citet{Gorman2012e} and citations therein for a full explication of Saussure's position on rhotacism, and a restatement of the phonological argument in generative terms.} 

Similarly, [ʃ] in Modern English derives from Old English [sk] (e.g., \emph{fisc} `fish'), and since long vowels before complex codas are not found in Old English, [ʃ] is rarely preceded by long vowels in word-final syllables to this day according to \citet{Iverson2005}. The type frequencies of short and long vowels before final [ʃ] and a similar segment, [s] are shown in Table \ref{ssh}. As can be seen, long vowels before final [s] are twice as common before final [ʃ]; according to the appropriate statistical test (the Fisher exact test, described below), this is significant. Yet \citet{Iverson2005} label this constraint ``phonologically accidental'' since a millennium of speakers have produced coinages (e.g., \emph{posh}, \emph{swoosh}) and  borrowings (e.g., \emph{douche}, \emph{capiche}) that disregard this generalization.\footnote{In the loanword adaptation literature, there are many reports that statistically reliable generalizations over the native lexicon are ignored in nativization processes \citep[e.g.,][]{Ito1995a,Ito1995b,Ussishkin2003}. On the other hand, \citet{Frisch2001} suggest that novel borrowings in Egyptian Arabic rarely violate co-occurrence restrictions in that language; however, the generalization in question was exceptionless from Proto-Semitic \citep{Ehret1989} through Classical Arabic \citep{Greenberg1950}, so any violations of this generalization are unexpected.}

\begin{table}
\centering
\begin{tabular}{l r r r r}
\toprule
          & \{ɪ, ɛ, æ, ʌ, ʊ\}\gap{}\# & \{i, e, ɑ, ɔ, u\}\gap{}\# & \% long & $p$-value \\
\midrule
\gap{}ʃ\# & 78                & 9                 & 8\%      & \multirow{2}{*}{0.026} \\
\gap{}s\# & 410               & 107               & 16\%     & \\
\bottomrule
\end{tabular}
\caption{Type frequencies of short and long vowels before word-final [ʃ] and [s] in a subset of the CMU pronunciation dictionary containing all words with token frequency greater than 1 per million words in the SUBTLEX-US frequency norms; the 9 words containing ending in a long vowel followed by ʃ are \emph{douche}, \emph{leash}, \emph{gosh}, \emph{josh}, \emph{posh}, \emph{squash}, \emph{unleash}, \emph{wash}, and \emph{woosh}.}
\label{ssh}
\end{table}

This chapter is a comprehensive investigation of three phonotactic generalizations in Turkish, comparing lexical statistics and the results of a wordlikeness task performed by \citet{Zimmer1969}. Both the lexical statistics and \citeauthor{Zimmer1969}'s experimental results merit reconsideration, because prior discussions do not relate the lexical statistics to competing formalizations of the generalizations involved, and there is widespread disagreement concerning the experimental findings.

One of the three generalizations, \textsc{Labial Attraction}, is shown to   state a statistically reliable generalization about the Turkish lexicon, but has no effect on speakers' wordlikeness judgements. This further undermines the widespread use of statistical criteria to determine which of the vast number of statistically reliable phonotactic generalizations are internalized by speakers and which are not. The simplest hypothesis is that \textsc{Labial Attraction} is inactive in wordlikeness tasks because it is static, corresponding to no phonological process in the language.

\section{Turkish vowel sequence structure constraints}

\citet{Lees1966b,Lees1966a} proposes three constraints on Turkish vowel sequences; these constraints are the focus of many subsequent studies. In this section, these constraints are formalized, and where possible, related to phonological alternations and to behavioral evidence bearing on speakers' knowlege of the restrictions. The following feature specification for the eight vowels of Turkish is assumed throughout.

\begin{example}[Turkish vowel features]
\begin{tabular}{c c c c c}
                       & \multicolumn{2}{c}{[$-$\textsc{Back}]} & \multicolumn{2}{c}{[$+$\textsc{Back}]} \\
                       & [$-$\textsc{Rnd}] & [$+$\textsc{Rnd}] & [$-$\textsc{Rnd}] & [$+$\textsc{Rnd}] \\ 
\cmidrule{2-5}
\buf[$+$\textsc{High}] & {i} & {ü} [y] & {ı} [ɯ] & {u} \\
\buf[$-$\textsc{High}] & {e} & {ö} [ø] & {a} [ɑ] & {o} \\
\end{tabular}
\end{example}

Two less familiar notations are used in this chapter. First, directional application conditions are assumed, so as to derive the left-to-right, iterative properties of the harmony rules. Directional application provides a panacea for the considerable body of evidence against simultaneous rule application adduced in the 1970s (e.g., \citealt[209f.]{Anderson1974}, \citealt{Howard1972}, \citeauthor{Kenstowicz1973} \citeyear[14f.]{Kenstowicz1973}, \citeyear[189f.]{Kenstowicz1977}, \citeyear[318f.]{Kenstowicz1979}, \citealt{Piggott1975}). Further, \citet{Johnson1972} and \citet{Kaplan1994} prove directional application and simultaneous application are equivalent in terms of formal learnability.

Secondly, rather than the use of an unbounded number of Greek-letter variables ($\alpha$, $\beta$, etc.) over feature values $\{+, -\}$, only a single variable, denoted by `$=$', is used \citep{McCawley1973}. A structural description [$=$F]\ldots{}[$=$F] matches a string S$_i$\ldots{}S$_j$ if and only if S$_i$ and S$_j$ are both [$+$F] or both [$-$F]. This is more restrictive than Greek-letter notation, in that it prevents the value of one feature being applied to another. \citet{Odden2012} argues that the two counterexamples against this restriction adduced in \emph{SPE} (352-353) are not probative.

\subsection{Backness harmony}

\citeauthor{Lees1966b} (\citeyear[35]{Lees1966b}, \citeyear[284]{Lees1966a}) models the Turkish vowel harmony system with three rules. The most general of these rules spreads the specification [\textsc{Back}] rightward.

\begin{example}[\textsc{Backness Harmony} (condition: rightward application)]
$\begin{bmatrix} -\textsc{Cons} \end{bmatrix}~\goesto~\begin{bmatrix} =\textsc{Back} \end{bmatrix}~\big /~\begin{bmatrix} =\textsc{Back} \end{bmatrix}~\textrm{C}_0~\gap$
\end{example}

\noindent
A vowel becomes [$+$\textsc{Back}] after a [$+$\textsc{Back}] vowel, and [$-$\textsc{Back}] after a [$-$\textsc{Back}] vowel, ignoring any intervening consonants. The application of this rule proceeds from left to right; no vowel may be skipped.

If permitted to apply in non-derived environments, this rule accounts for the tendency of polysyllabic roots to contain only [$+$\textsc{Back}] or [$-$\textsc{Back}] vowels, a tendency which will be quantified below. \textsc{Backness Harmony} also triggers alternations in inflectional suffix vowels. For instance, the nominative plural (nom.pl.) suffix is \emph{-ler} when the final root vowel is [$-$\textsc{Back}], and \emph{-lar} when it is [$+$\textsc{Back}].

\begin{example}[The Turkish nominative]
\label{turknom}
\begin{tabular}{lllll}
   & \emph{nom.sg.} & \emph{nom.pl.} \\
a. & {ip}           & {ipler}    & `rope'         & \citep[][216]{Clements1982} \\
   & {köy}          & {köyler}   & `village'      \\
   & {yüz}          & {yüzler}   & `face'         \\
   & {kız}          & {kızlar}   & `girl'         \\
   & {pul}          & {pullar}   & `stamp'        \\
b. & {neden}        & {nedenler} & `reason'       & \citep{TELL} \\
   & {kiler}        & {kilerler} & `pantry'       \\
   & {pelür}        & {pelürler} & `onionskin'    \\
   & {boğaz}        & {boğazlar} & `throat'       \\
   & {sapık}        & {sapıklar} & `pervert'      \\
\end{tabular}
\end{example}

A few complications arise, however. First, as shown in (\ref{turkexcept}a), not all polysyllabic roots conform to \textsc{Backness Harmony}. In this case, suffix vowels generally exhibit harmony with the final root vowel. There is also a very small class of nouns, shown in (\ref{turkexcept}b), which take \emph{-ler}, although their final root vowel is [$+$\textsc{Back}]. Interestingly, they may themselves be harmonic.

\begin{example}[Exceptional Turkish nominatives] 
\label{turkexcept}
\begin{tabular}{lllll}
   & \emph{nom.sg.} & \emph{nom.pl.}&                    \\
a. & {mezar}        & {mezarlar}    & `grave'       & \citep{TELL}       \\
   & {model}        & {modeller}    & `model'                            \\
   & {silah}        & {silahlar}    & `weapon'                           \\
   & {memur}        & {memurlar}    & `official'                         \\
   & {sabun}        & {sabunlar}    & `soap'                             \\
b. & {etol}         & {etoller}     & `fur stole'   & \citep{Goksel2005} \\
   & {saat}         & {saatler}     & `hour, clock' 	              \\
   & {kahabat}      & {kahabatler}  & `fault'       \\
\end{tabular}
\end{example}

\citet[212]{Anderson1974} and \citet{Iverson1978} argue that suffix harmony in disharmonic roots found in (\ref{turkexcept}a) requires the rule governing suffix harmony alternations to be distinguished from a sequence structure constraint governing root harmony. Since the rule and sequence structure constraint are otherwise identical, this constitutes an undesirable ``duplication'' in the sense of \citealt{Kisseberth1970b}. However, the theory of exceptionality proposed in \emph{SPE} can account for these facts without duplication \citep[197f.]{Zonneveld1978}.\footnote{\citet[29f.]{Kiparsky1968} discusses a parallel case in Finnish, and proposes a similar separation between the exception-filled sequence structure constraint and an exceptionless suffix harmony process. \citet[171f.]{Howard1972} correctly observes that this duplication is unnecessary.}

\citeauthor{SPE} assume that the specification of the target (i.e., the segment or segments to be changed) of a rule \emph{R} must be marked [$+$\emph{R}] by convention. A root or affix which fails to undergo \emph{R} despite otherwise matching the structural description is simply said to be marked [$-$\emph{R}]. In other words, no form is never truly an ``exception''; rather, some underlying representations have non-default features which do not match the extended structural description of $R$ which requires that the target be [$+R$]. If disharmonic roots are marked [$-$\textsc{Backness Harmony}], then the final vowel of disharmonic roots will still trigger \textsc{Backness Harmony}, since the [$-$\textsc{Backness Harmony}] root is no longer the target but rather the trigger, which is not subject to the [$+$\textsc{Backness Harmony}] requirement.

Under this account, root and suffix harmony are derived from \textsc{Backness Harmony}, but they do not have the same ontology: suffix harmony is direct result of phonological rule application, whereas any ``effects'' of the  generalization concerning harmonic roots arises from a dispreference for lexical exceptionality.

\citeauthor{Anderson1974} also notes that roots which fail to undergo suffix harmony, like (\ref{turkexcept}b), may themselves be harmonic. He takes to be evidence for the necessity of duplication:

\begin{quote}
\ldots{}there are words which are exceptions to harmony across boundaries (e.g., \emph{kabahat} `fault', \emph{kabahattı} `his fault') but which are perfectly regular internally. Since the morpheme structure condition and the phonological rule in this case have distinct classes of exceptions, it is clear that they cannot be identified. \citep[289]{Anderson1974}
\end{quote}

\noindent
Under the assumptions so far, there is no reason to think that \emph{kabahat} is [$+$\textsc{Backness Harmony}], however: even roots with all back vowels need not undergo this rule. Whatever the ultimate analysis of those few words which fail to show suffix harmony, nothing depends on whether or not they are harmonic.

%\begin{example}[Turkish progressives]
%\begin{tabular}{l l l l l}
%   & \emph{1sg. present}  & \emph{1sg. progressive} & \\
%a. & gelirim              & geliyorum               & `come'  \\
%b. & görürüm              & görüyorum               & `see'   \\
%c. & atarım               & atıyorum                & `throw' \\
%d. & bulurum              & buluyorum               & `find'  \\
%\end{tabular}
%\end{example}

It is necessary to dispense with an alternative analysis proposed by \citet{Clements1982} and \citet{Inkelas1997}. This analysis has desirable properties, but a subtlety of it reintroduces the duplication of rule and sequence structure constraint. Root vowels exhibit a robust contrast for backness (e.g., \emph{kül} `ash' vs. \emph{kul} `servant', \emph{kepek} `bran' vs. \emph{kapak} `lid'), whereas backness of vowels in non-initial syllables is predictable in harmonic roots; there are no prefixes in Turkish. \citeauthor{Clements1982} propose that these vowels, as well as harmonizing suffix vowels, are underspecified for backness, whereas the non-initial vowels of disharmonic roots and of certain exceptional suffixes are fully specified. This is schematized below.

\begin{example}[Autosegmental underspecification in harmonic roots (after \citealp{Clements1982})] 
\label{spec}
\xymatrix@R=24pt@C=24pt{
\txt{a.} & \txt{harmonic root:} & \txt{C} & \txt{V} & \txt{C} & \txt{V} & \txt{C} \\
         &                      &         & \txt{[$-$\textsc{Back}]}\ar@{-}[u]\ar@{--}[urr] \\
\txt{b.} & \txt{disharmonic root:} & \txt{C} & \txt{V} & \txt{C} & \txt{V} & \txt{C} \\
         &                      &         & \txt{[$-$\textsc{Back}]}\ar@{-}[u] & & \txt{[$+$\textsc{Back}]}\ar@{-}[u]
}
\end{example}

One crucial detail is missing from this analysis: \textsc{Backness Harmony} needs to be prevented from overwriting the [$+$\textsc{Back}] specification of disharmonic roots, either by non-derived environment blocking \citep{Mascaro1976} or structure preservation condition \citep{Kiparsky1985}. However, any condition which prevents \textsc{Backness Harmony} from overwriting underlying backness specifications will reintroduce the duplication of sequence structure and phonological generalizations; under this analysis, disharmonic roots are no longer exceptional, despite considerable evidence (reviewed below) that they are marked in Turkish.\footnote{On the other hand, it is possible to interpret the presence of a single backness specification per root as a sort of default. A precedent for this is the surface-oriented interpretation of the tonal Obligatory Contour Principle proposed by \citet[134]{Goldsmith1976} and \citet{Odden1986}, under which adjacent identical tones are automatically attributed to a single underlying tone. However, this is merely a notational variant of the rule exceptionality account in which [$+$\textsc{Backness Harmony}] is the default.} On these grounds, the underspecification analysis is rejected here in favor of the rule exceptionality account.  

While harmony in non-derived environments can be inferred from the aforementioned suffix alternations, no evidence has yet been presented to show that Turkish speakers internalize the tendency for roots to conform to backness harmony. If Turkish speakers do not attend to this generalization, there is no need for the grammar to account for it. Several other ``external'' facts suggest that this is not the case. The discussion here is not intended to imply uncritical acceptance of evidence from loanword adaptation, language games or psycholinguistic tasks as evidence for phonological grammar, but rather to illustrate additional evidence that is pertinent if the linking hypothesis is correct.

The production of non-native word-initial onset clusters, discussed by \citet{Clements1982} and \citet{Kaun1999}, suggests that loanword adaptation respects \textsc{Backness Harmony}.\footnote{Thanks to Kie Zuraw for bringing this data to my attention.} Some speakers are said to be capable of pronouncing these non-native clusters, but in fast speech the cluster is split by anaptyxis. In most cases, this vowel matches the following root vowel for backness.

\begin{example}[Variable non-native cluster adaptation (\citealp{Clements1982}:247)] 
\begin{tabular}{lllll}
a. & {spiker}  & \alt{} & {sipiker}  & `announcer' \\
   & {fren}    & \alt{} & {firen}    & `brake'     \\
b. & {trablus} & \alt{} & {tırablus} & `Tripoli'   \\
   & {kral}    & \alt{} & {kıral}    & `king'      \\
c. & {brom}    & \alt{} & {burom}    & `bromide'   \\
   & {prusya}  & \alt{} & {purusya}  & `Prussia'   \\
\end{tabular}
\label{spiker}
\end{example}

\noindent
It is unclear whether the cluster-splitting vowel is deleted in the non-native variant or epenthesized in the fast speech variant. Under either analysis, there is no ready explanation for the tendency of the cluster-splitting vowel to have the backness features of following consonants; if anything, one might expect it to determine the backness features of following consonants. All that can be said with certainty is that the adaptation of non-native onset clusters appears to proceed in such a fashion so that the lexical items in question are [$-$\textsc{Backness Harmony}].

Similar evidence comes from a language game discussed by \citet{Harrison2001}.\footnote{Thanks to Bert Vaux for bringing this study to my attention.} 
The game is native to the related language Tuvan, where it is used to convey a sense of ``vagueness or jocularity''; it is not indigenous to Turkish, but can be taught quickly to children or adults. In this game, the base is reduplicated and the first vowel of the reduplicant replaced with a [$+$\textsc{Back}] vowel. In (\ref{redupgame}a), the second [$-$\textsc{Back}] vowel of the base is, in the reduplicant, ``reharmonized'' with the inserted [$+$\textsc{Back}] vowel. The disharmonic roots of (\ref{redupgame}b) do not reharmonize.\footnote{A similar contrast between harmonic and disharmonic roots is found in Tuvan \citep{Harrison2001} and in an unrelated Finnish language game \citep{Campbell1986}.}

\begin{example}[Turkish reduplication game (\citealp{Harrison2001}:231)] 
\label{redupgame}
\begin{tabular}{llll}
a. & {kibrit} & {kibrit}-\{{kabrıt}\} & `match'    \\
   & {bütün}  & {bütün}-\{{batın}\}   & `whole'    \\
b. & {mali}   & {mali}-\{{muli}\}     & `Mali'     \\
   & {butik}  & {butik}-\{{batik}\}   & `boutique' \\
\end{tabular}
\end{example}

\noindent 
In the full specification analysis adopted here, reharmonization is the result of \textsc{Backness Harmony} applying within the reduplicant. On the other hand, the lack of reharmonization in the reduplicants of disharmonic roots suggests that the [$-$\textsc{Backness Harmony}] exception feature is copied under reduplication.\footnote{There is reason to believe that reduplicants bear lexical diacritics. Kinande reduplication, documented by \citet{Mutaka1990} and further discussed by \citet{Downing2000}, provides such a case. Verb reduplication requires a bisyllabic reduplicant. This clearly has synchronic force in the gramar, since reduplicated monosyllabic roots actually have three copies of the root, and since reduplicated forms of many trisyllable verbs are ineffable. However, a few trisyllabic verbs exceptionally show full reduplication. Exceptionality appears to be a lexical property (since it is specific to certain lexical entries), but it surfaces only in the reduplicant.} \citet{Silverman2000} notes that static phonotactic generalizations are not preserved by operations like reduplication and truncation: under this definition, root harmony is not static.

A number of studies have investigated the role of harmony in word-spotting tasks, though to mimic auditory word recognition and segmentation in natural settings. Many of these studies have been carried out in Finnish, which has a vowel harmony system similar to that of Turkish. \citet{Suomi1997} and \citet{Vroomen1998} task Finnish speakers with identifying harmonic disyllables in an auditory stream. When the syllable preceding the disyllabic target has a different backness specification than the target, recognition of the target is facilitated. Presumably, disharmony facilitates the recognition of word boundaries. \citet{Kabak2010} find that Turkish \textsc{Backness Harmony} has a similar effect: Turkish speakers are quicker and more accurate at the task of spotting the nonce target word \emph{pavo} when preceded by a disharmonic juncture (e.g., \emph{gölü-PAVO}) than when preceded by a harmonic juncture (e.g., \emph{golu-PAVO}). \citet{Kabak2010} find that this effect does not obtain for speakers of French, a language which lacks vowel harmony. This implies that Turkish speakers have internalized the tendency of harmonic sequences to be root-internal and of disharmonic transitions to cross word boundaries. 

The Turkish word-spotting experiment is adapted for infants by \citet{Kampen2008}. In this study, 9-month-old infants are familiarized with an auditory stream containing harmonic disyllabic nonce words. At test time, the infants listen to the disyllabic nonce words in isolation using the head turn preference paradigm. Infants acquiring Turkish listen longer to nonce words preceded by a disharmonic juncture during familiarization (e.g., \emph{lo-NETIS}), whereas infants acquiring German, a language which lacks vowel harmony, do not exhibit this preference. Similarly, \citeauthor{Kampen2008} report that Turkish 6-month-old infants prefer to listen to harmonic nonce words such as \emph{paroz} over disharmonic nonce words like \emph{nelok}, but German 6-month-old infants show no such preference.

It is not clear, however, that the segmentation effects are evidence for grammatical generalizations; for instance, while the effect is defined in terms of harmony, it operates over a domain much larger than the prosodic word over which harmony is computed.

\subsection{Roundness harmony}

This rule is quite similar to \textsc{Backness Harmony}, but imposes an additional restriction, that targets be [$+$\textsc{High}]. 

\begin{example}[\textsc{Roundness Harmony} (condition: rightward application)]
$\begin{bmatrix} -\textsc{Cons} \\ +\textsc{High} \end{bmatrix}~\goesto~\begin{bmatrix} =\textsc{Rnd} \end{bmatrix}~/~\begin{bmatrix}~=\textsc{Rnd}~\end{bmatrix}~\textrm{C}_0~\gap$
\end{example}

\noindent
A [$+$\textsc{High}] vowel becomes [$+$\textsc{Rnd}] after a [$+$\textsc{Rnd}] vowel, and [$-$\textsc{Rnd}] after a [$-$\textsc{Rnd}] vowel, ignoring any intervening consonants, and applying from left to right. 

If permitted to apply in non-derived environments, this rule accounts for the tendency of polysyllabic roots to contain only [$+$\textsc{Rnd}] or [$-$\textsc{Rnd}] if the non-initial vowels are [$+$\textsc{High}]. In concert with \textsc{Backness Harmony}, \textsc{Roundness Harmony} also triggers alternations which account for the shape of the dative singular (dat.sg.) and genitive singular (gen.sg.) among other suffixes. As was the case for \textsc{Backness Harmony}, disharmonic roots are found, and the final vowel of disharmonic roots triggers suffix harmony.

\begin{example}[Turkish nominal suffix allomorphy]
\begin{tabular}{lllllll}
   & \emph{nom.sg.} & \emph{dat.sg.} & \emph{gen.sg.}  \\
a. & {ip}           & {ipi}          & {ipin}         & `rope' & \citep[][216]{Clements1982} \\
   & {kız}          & {kızı}         & {kızın}        & `girl'    \\
   & {sap}          & {sapı}         & {sapın}        & `stalk'  \\
   & {köy}          & {köyü}         & {köyün}        & `village' \\
   & {son}          & {sonu}         & {sonun}        & `end'     \\
b. & {boğaz}        & {boğazı}       & {boğazın}      & `throat'  & \citep{TELL} \\
   & {pelür}        & {pelürü}       & {pelürün}      & `onionskin' \\
   & {döviz}        & {dövizi}       & {dövizin}      & `currency'  \\
   & {yamuk}        & {yamuğu}       & {yamuğun}      & `trapezoid' \\
   & {ümit}         & {ümiti}        & {ümitin}       & `hope'      \\
\end{tabular}
\end{example}

Few studies have directly investigated whether speakers are aware of the tendency for roots to conform to \textsc{Roundness Harmony}. However, two pieces of the external evidence on \textsc{Backness Harmony} bear on this question. First, the cluster-splitting vowel found in non-native onset clusters
, discussed above, tends to agree in roundness with following high vowels (e.g., \emph{prusya}-\emph{purusya} `Prussia'). Secondly, \textsc{Roundness Harmony} participates in reharmonization in the language game described by \citeauthor{Harrison2001}: the second \emph{ü} in \emph{bütün} `whole' reharmonizes to the [$-$\textsc{Rnd}] vowel \emph{ı} in reduplicated \emph{bütün-batın}.

\subsection{Labial attraction}

\citet{Lees1966b} describes \textsc{Labial Attraction} as a phonological process by which ``a high, short harmonic vowel is rounded in the second syllable of a disyllabic word whose first vowel is /a/, and whose medial consonant cluster contains a labial /p, b, m, v/, and then it is de-harmonified'' (36).

\begin{example}[\textsc{Labial Attraction}]
$\begin{bmatrix} -\textsc{Cons} \\ +\textsc{Back} \\ +\textsc{High} \end{bmatrix}~\goesto~\begin{bmatrix} +\textsc{Rnd} \end{bmatrix}~/~\textrm{ɑ}~\textrm{C}_0~\begin{bmatrix} +\textsc{Cons} \\ +\textsc{Labial} \end{bmatrix}~\textrm{C}_0~\gap{}$
\end{example}

This rule is significantly more complex than the harmony rules, and this may obscure the fact that it produces exceptions to \textsc{Roundness Harmony}, producing \emph{a}C$_0$\emph{u} (e.g., \emph{çapul} `raid', \emph{sabur} `patient', \emph{şaful} `wooden honey tub', \emph{avuç} `palm of hand', \emph{samur} `sable'; \citealp[285]{Lees1966a}) rather than the expected \emph{a}C$_0$\emph{ı}. However, \textsc{Labial Attraction} does not apply in derived environments: the gen.sg. of \emph{sap} `stalk is \emph{sapın} rather than *\emph{sapun} that would be predicted if \textsc{Labial Attraction} triggered alternations. \citet[286]{Lees1966a} and \citet[311]{Zimmer1969} note the existence of root-internal exceptions to this rule (e.g., \emph{tavır} `mode') but they agree that they are surprisingly rare.

\section{Evaluation}

Others dispute that \textsc{Labial Attraction} is a reliable generalization about Turkish roots. \citet{Clements1982} point to additional exceptions to the rule, and \citet{Inkelas2001} provide lexical counts from the Turkish Electronic Living Lexicon \citep[TELL;][]{TELL}, a database of words known to two native Turkish speakers. However, no prior study reports any inferential statistics.

\citet[311]{Zimmer1969} administers two paired wordlikeness tasks designed to evaluate native speakers' knowledge of \textsc{Backness Harmony}, \textsc{Roundness Harmony}, and \textsc{Labial Attraction} in roots. Speakers are presented with two nonce words and simply have to indicate the nonce word that is more Turkish-like.\footnote{Compared to the unpaired ratings tasks  commonly used in wordlikeness research, paired rating tasks have considerably more statistical power \citep{Gigerenzer2004}. When items under comparison differ only in whether or not they conform to the phonotactic generalizations, it is less likely that any contrast between the members of the pairs are caused by some omitted variable. Also, there is no unnecessary deception regarding the purpose of the experiment, which is known to reduce spurious response variability. Consequently, paired tasks are particularly sensitive to small phonotactic contrasts and are ideal for collecting wordlikeness judgements.} \citeauthor{Zimmer1969} concludes that the former two rules are reflected in wordlikeness judgements, whereas \textsc{Labial Attraction} is not. 

Below, both lexical statistics and \citeauthor{Zimmer1969}'s wordlikeness results are analyzed statistically; \textsc{Labial Attraction} is shown to be a statistically robust generalization over the Turkish lexicon, but no variant of \textsc{Labial Attraction} is reflected in \citeauthor{Zimmer1969}'s wordlikeness study. In contrast, the two harmony processes have robust effects both on the lexicon and on wordlikeness.  This dissociation between lexical generalizations and wordlikeness results provides further evidence against the common practice of inferring phonotactic knowledge directly from lexical statistics.

\subsection{Lexical statistics}

Counts were computed by regular expression matching on a 9,601-root subset of the TELL database which consists of words which show no surface variation between the two informants in any inflected form.

To test for associations between the process (more specifically, the constraint that it imposes on roots) and type frequency in this database, each root was sorted into a $2 \times 2$ contingency table; the contents of each cell are specific to the process in question. The counts in this table are not expected to add up to 9,601, since many roots neither could exemplify nor violate the process in question; for instance, monosyllabic words are irrelevant to root harmony. The Fisher exact test is used to compute a $p$-value representing the probability of the observed data arising under the null hypothesis that there is no association between the process and type frequency.

\subsubsection{Backness harmony}

\textsc{Backness Harmony} is exemplified in the lexicon insofar as there is an association between the backness of vowels in all adjacent syllables. Any disagreement on the backness specification of vowels in adjacent syllables are counted solely as exceptions, even if other vowel transitions in the root are harmonic. 
%This follows from \emph{SPE} theory of lexical exceptionality (see also \citealt{Gouskova2012}), in which phonological exceptionality is a property of underlying representations, not individual segments. 
To construct the contingency table, roots are binned according to the backness specification of the first nucleus, and that of following nuclei. For example, the first two syllables of \emph{adalet} `justice' are harmonic, but it is coded as disharmonic because there is a \emph{a\ldots{}e} transition later in the word. The resulting counts are shown in Table \ref{bhs}. 61\% of roots conform to \textsc{Backness Harmony}, and the interaction between the backness of the first and of the subsequent vowels
%predicted by root-internal \textsc{Backness Harmony} 
is significant.

\begin{table}
\centering
\begin{tabular}{lrrr}
\toprule
                             & [$+$\textsc{Back}]$_1$ & [$-$\textsc{Back}]$_1$ & $p$-value                     \\
\midrule
\buf{}[$+$\textsc{Back}]$_{2\ldots{}n}$ & 3,089                     & 1,704              & \multirow{2}{*}{$1.19$\e{-89}} \\
\buf{}[$-$\textsc{Back}]$_{2\ldots{}n}$ & 1,698                     & 2,250                                               \\
\bottomrule
\end{tabular}
\caption{TELL roots sorted according to \textsc{Backness Harmony}}
\label{bhs}
\end{table}

\subsubsection{Roundness harmony}

\textsc{Roundness Harmony} predicts correlation between the roundness of a vowel and the roundness of high vowels in the next syllable. Any root for which a vowel does not agree in roundness with a high vowel in the following syllable (e.g., \emph{ümit}) is considered to be an exception. Roots are binned according to the roundness of non-initial high vowels, and according to the roundness of the preceding vowel. 
%Roots lacking non-initial high vowels do not bear on the status of root \textsc{Roundness Harmony}. 
The resulting counts are shown in Table \ref{rhs}. 83\% of the roots conform to \textsc{Roundness Harmony}, and the interaction between the roundness of the $i$th vowel and the roundness of the $(i +1)$th high vowel is significant.

\begin{table}
\centering
\begin{tabular}{lrrr}
\toprule
                                              & [$+$\textsc{Rnd}]$_i$ & [$-$\textsc{Rnd}]$_i$ & $p$-value                      \\
\midrule
\buf{}[$+$\textsc{High}, $+$\textsc{Rnd}]$_{i+1}$ & 613                   &   261                 & \multirow{2}{*}{$1.02$\e{-36}} \\
\buf{}[$+$\textsc{High}, $-$\textsc{Rnd}]$_{i+1}$ & 581                   & 2,841                                                  \\
\bottomrule
\end{tabular}
\caption{TELL roots sorted according to \textsc{Roundness Harmony}}
\label{rhs}
\end{table}

\noindent
The counts in the bottom row of Table \ref{rhs} contain a number of roots which are apparent exceptions to \textsc{Roundness Harmony} but conform to \textsc{Labial Attraction} (bottom left), and which conform to \textsc{Roundness Harmony} at the expense of \textsc{Labial Attraction} (bottom right). Excluding these types of roots would have the effect of slightly increasing the overall rate of \textsc{Roundness Harmony}, since the former is more common.

\subsubsection{Labial attraction}

Reviewing the results of wordlikeness experiments, \citet{Zimmer1969} proposes a third variant of \textsc{Labial Attraction} which is insensitive to intervening consonant place. This is formalized below. 

\begin{example}[\textsc{Labial Attraction} (revised by \citealt{Zimmer1969})]
$\begin{bmatrix} -\textsc{Cons} \\ +\textsc{Back} \\ +\textsc{High} \end{bmatrix}~\goesto~\begin{bmatrix} +\textsc{Rnd} \end{bmatrix}~/~\textrm{ɑ}~\textrm{C}_0~\gap{}$
\end{example}

\noindent
\citet{Clements1982} also  to \textsc{Labial Attraction}, arguing that it is not ``systematic''.

\begin{quote}
Even more decisive evidence against a rule of Labial Attraction is the existence of a further, much larger set of roots containing /\ldots{}aCu\ldots/ sequences in which the intervening consonant or consonant cluster does not contain a labial\ldots{}We conclude that there is no systematic restriction on the set of consonants that may occur medially in roots of the form /\ldots{}aCu\ldots/. \citep[225]{Clements1982}
\end{quote}

\noindent 
This claim can be evaluated using the Fisher exact test. Let P denote a sequence of one or more consonants, one of which is labial, and let T denote a sequence of one or more consonants none of which is labial. The null hypothesis is that \emph{a}P\emph{u} sequences, which conform to \textsc{Labial Attraction}, are no more likely than would be expected from other \emph{a}T\emph{u} sequences violating \textsc{Roundness Harmony}. The resulting counts are shown in Table \ref{las}. Whereas the sequence \emph{a}P\emph{u} is more than twice as likely as \emph{a}P\emph{ı}, the sequence \emph{a}T\emph{u} is 5 times less likely than \emph{a}T\emph{ı}. This interaction is significant, as predicted by \textsc{Labial Attraction}, but contradicting a statistical interpretation of the \citeauthor{Clements1982}'s claim. In fact, \emph{a\ldots{}u} sequences are less, not more, common than \emph{a\ldots{}ı}.

\begin{table}
\centering
\begin{tabular}{lrrr}
\toprule
       & a\ldots{}u & a\ldots{}ı & $p$-value                      \\
\midrule
aP\ldots{} & 124    & 57     & \multirow{2}{*}{$1.02$\e{-36}} \\
aT\ldots{} & 136    & 590    &                                \\
\bottomrule
\end{tabular}
\caption{TELL roots sorted according to \textsc{Labial Attraction}}
\label{las}
\end{table}

\noindent
\citet[196]{Inkelas2001} propose instead that the trigger must contain a labial consonant but does not require a preceding \emph{a} vowel.

\begin{quote}
Vowel labialization following labials is not a synchronic alternation in Turkish, yet it (unlike \textsc{Labial Attraction} per se) \emph{is} a statistically supported tendency worthy of further research. \citep[][196]{Inkelas2001}
\end{quote}

\noindent
This is formalized below.

\begin{example}[\textsc{Labial Attraction} (revised by \citealt{Inkelas2001})]
$\begin{bmatrix} -\textsc{Cons} \\ +\textsc{Back} \\ +\textsc{High} \end{bmatrix}~\goesto~\begin{bmatrix} +\textsc{Rnd} \end{bmatrix}~/~\begin{bmatrix} +\textsc{Cons} \\ +\textsc{Labial} \end{bmatrix}~\gap{}$
\end{example}

Counts from TELL for this formulation of the process are shown in Table \ref{lasi}. This triples the number of roots which exemplify \textsc{Labial Attraction} (top left cell) but also slightly increases the number of exceptions. The interaction remains significant.

\begin{table}
\centering
\begin{tabular}{lrrr}
\toprule
       & \ldots{}u  & \ldots{}ı & $p$-value                      \\
\midrule
P\ldots{}  & 371    & 71        & \multirow{2}{*}{$6.98$\e{-49}} \\
T\ldots{}  & 811    & 922       &                                \\
\bottomrule
\end{tabular}
\caption{TELL roots sorted according to the reformulation of \textsc{Labial Attraction} proposed by \citet{Inkelas2001}}
\label{lasi}
\end{table}

\subsection{Wordlikeness ratings}

\citet{Zimmer1969} administers two variants of the paired nonce word rating task. The first used 23 native adult speakers who were permitted to select either nonce word as more like Turkish, or to indicate `no preference'. For the purposes below, `no preference' results are discarded. The second experiment used 32 native adults, none of whom appeared in the preceding study, and used a forced binary choice with some minor changes to the stimulus set. 

Each response is coded as \emph{concordant} if the nonce word conforming to the process is preferred, and \emph{discordant} if the disharmonic word is selected. To test for an association between the constraints, a non-parametric statistic, the Goodman-Kruskal (\citeyear{Goodman1954}) $\gamma$ is computed from the count of concordant ($c$) and discordant ($d$) pairs:

\begin{unlabeledexample}
$\displaystyle \gamma = \frac{c - d}{c + d}$
\end{unlabeledexample}

\noindent
The $\gamma$ statistic ranges between -1, in the case that all paired choices are discordant, and 1, if all paired choices are concordant.\footnote{It also is possible to perform statistical tests aggregating over items, but for this small number of items, such tests have very low statistical power.}

\subsubsection{Backness harmony}

Both of the \citet{Zimmer1969} experiments include 5 pairs which differ in whether or not the nonce words conform to, or violate, \textsc{Backness Harmony}. As can be seen from Table \ref{bhw}, harmonic pairs are preferred approximately 6-to-1, and aggregating over speakers, no disharmonic member of a pair is favored. Speakers have a highly reliable preference for nonce words which exhibit \textsc{Backness Harmony} ($\gamma = 0.694$, $p = 1.7$\e{-59}). It is interesting to note that the disharmonic nonce word which has the highest rating is found in the pair \emph{terüz}-\emph{teruz}, both of which violate \textsc{Roundness Harmony}. While this is little more than an anecdote, this may be  indicative of a link between the two processes, and their exceptions, in the minds of native speakers. 

\begin{table}
\centering
\begin{tabular}{lrlr|lrlr}
\toprule
\multicolumn{4}{c|}{Experiment 1} & \multicolumn{4}{c}{Experiment 2} \\
\multicolumn{2}{c}{\textsc{harmonic}} & \multicolumn{2}{c|}{\textsc{disharmonic}} & \multicolumn{2}{c}{\textsc{harmonic}} & \multicolumn{2}{c}{\textsc{disharmonic}} \\
\midrule
{temez} & 19            & {temaz} & 3 & {pemez} & 30            & {pemaz} & 2 \\
{teriz} & 23            & {terız} & 0 & {teriz} & 28            & {terız} & 3 \\
{tokaz} & 21            & {tokez} & 1 & {tokaz} & 26            & {tokez} & 6 \\
{tipez} & 21            & {tipaz} & 1 & {tipez} & 24            & {tipaz} & 8 \\
{terüz} & 20            & {teruz} & 1 & {terüz} & 19            & {teruz} & 13 \\
\bottomrule
\end{tabular}
\caption{Effects of \textsc{Backness Harmony} on wordlikeness \citep[from][]{Zimmer1969}}
\label{bhw}
\end{table}
% harmonic responses are favored approximately 6 to 1.
%$\gamma = 0.717$, $p = 7.5$\e{-69}

\subsubsection{Roundness harmony}

Both experiments include 5 pairs which differ in the presence or absence of \textsc{Roundness Harmony}. As shown in Table \ref{rhw}, there is an approximately 5-to-1 preference for harmonic nonce words, and as was the case above, no disharmonic member of any pair is overall preferred. Turkish speakers have a reliable preference for nonce words to conform to \textsc{Roundness Harmony} ($\gamma = 0.680$, $p = 1.1$\e{-47}).

\begin{table}
\center
\begin{tabular}{lrlr|lrlr}
\toprule
\multicolumn{4}{c|}{Experiment 1} & \multicolumn{4}{c}{Experiment 2} \\
\multicolumn{2}{c}{\textsc{harmonic}} & \multicolumn{2}{c|}{\textsc{disharmonic}} & \multicolumn{2}{c}{\textsc{harmonic}} & \multicolumn{2}{c}{\textsc{disharmonic}} \\
\midrule
{törüz} & 19 & {töriz} & 1 & {pörüz} & 32 & {pöriz} & 0  \\
{tüpüz} & 22 & {tüpiz} & 0 & {tüpüz} & 31 & {tüpiz} & 1  \\
{takız} & 15 & {takuz} & 3 & {takız} & 22 & {takuz} & 10 \\
{tatız} & 12 & {tatuz} & 6 & {tatız} & 20 & {tatuz} & 12 \\
\bottomrule
\end{tabular}
\caption{Effects of \textsc{Roundness Harmony} on wordlikeness \citep[from][]{Zimmer1969}}
\label{rhw}
\end{table}

\subsubsection{Labial attraction}

Both experiments include 5 pairs which either conform to \textsc{Labial Attraction} and violate \textsc{Roundness Harmony}, or vice versa; the preferences are shown in Table \ref{law}. There is a small preference against \textsc{Labial Attraction}, though this is non-significant ($\gamma = -0.043$, $p = 0.305$). Speakers do not have the preferences predicted by \textsc{Labial Attraction}. This result also extends for the variants of \textsc{Labial Attraction} proposed by \citet{Zimmer1969} and \citet{Inkelas2001}, since these variants have structural descriptions targeting a superset of the original formulation by \citet{Lees1966a}. It is interesting to note, however, that speakers do not have the clear opposite preference, as is predicted by \textsc{Roundness Harmony}. 

\begin{table}
\centering
\begin{tabular}{lrlr|lrlr}
\toprule
\multicolumn{4}{c|}{Experiment 1} & \multicolumn{4}{c}{Experiment 2} \\
\multicolumn{2}{c}{aPu} & \multicolumn{2}{c|}{\textsc{disharmonic}} & \multicolumn{2}{c}{\textsc{harmonic}} & \multicolumn{2}{c}{\textsc{disharmonic}} \\
\midrule
{tamuz} & 3 & {tamız} & 16 & {pamuz} & 15 & {pamız} & 17 \\
{tafuz} & 3 & {tafız} & 17 & {tafuz} & 21 & {tafız} & 11 \\
{tavuz} & 9 & {tavız} & 4  & {mavuz} & 16 & {mavız} & 16 \\
{tapuz} & 7 & {tapız} & 9  & {tapuz} & 17 & {tapız} & 15 \\
{tabuz} & 5 & {tabız} & 12 & {tabuz} & 16 & {tabız} & 16 \\
\bottomrule
\end{tabular}
\caption{Effects of \textsc{Labial Attraction} on wordlikeness \citep[from][]{Zimmer1969}}
\label{law}
\end{table}

\subsection{Discussion}

It has been shown that while \textsc{Labial Attraction} is a highly reliable generalization about Turkish roots, it is not reflected in wordlikeness judgements. In contrast, harmony processes have a similar statistical profile, but have large effects on wordlikeness. The most plausible explanation for this is that  \textsc{Labial Attraction} does not trigger alternations; indeed, it is counter-exemplified by the effects of \textsc{Roundness Harmony} in suffixes. As is noted by \citet[412f.]{Inkelas1997}, the lexicon of Turkish will, under the assumptions here, remain as it is whether or not \textsc{Labial Attraction} has a synchronic reality. A principle of simplicity suggests that it does not exist at all.

\citet{NiChiosain1993} and \citet{Ito1995b} argue that \textsc{Labial Attraction} is true only of native vocabulary. This would be a potential confound for the experimental results, since it is not implausible that the subject's in \citeauthor{Zimmer1969}'s study treated nonce words as if they were loanwords. 
%\footnote{Indeed, many wordlikeness studies include instructions to the participants to treat the stimuli much as if they were loanwords \citep[e.g.,][]{Hay2004a}. This may influence the choice of task model used by participants.}
This was not the opinion of \citet[266]{Lees1966a}, though, and it is conclusively refuted by  \citet{Inkelas2001}. They observe that in the subset of the TELL database with etymological coding, a full 75\% of foreign vocabulary conform to \textsc{Labial Attraction}. Among native lexical items, only 52\% do, a significant trend but in an unexpected direction (Fisher exact test, $p = 0.042$). One possible explanation is that many of the languages in contact with Turkish lack the \emph{ı} phoneme, and therefore cannot contribute exceptions to \textsc{Labial Attraction}.

\citet{Becker2011} attribute the inactivity of \textsc{Labial Attraction} in wordlikeness to a claim that it is phonologically unnatural, and therefore goes unlearned. This claim is difficult to evaluate insofar as phonotactic naturalness has continually eluded formalization and \citeauthor{Becker2011} neither propose nor refer to any theory thereof. \citeauthor{Becker2011} use the lack of phonetic precursors as a diagnostic of unnaturalness; however, there is extensive formal and experimental evidence for the learning of ``unnatural'' generalizations \citep[e.g.,][]{Anderson1981,Bach1972,Blevins2003,Buckley2000a,Hayes2009,Pierrehumbert2006c,Seidl2005}. Furthermore, other functionally-oriented theorists have considered \textsc{Labial Attraction} sufficiently natural \citep[e.g.,][]{NiChiosain1993,Ito1993,Ito1995b}. 

It is not even obvious that it is desirable to exclude \textsc{Labial Attraction} as a possible rule. \citet[394, fn. 2]{Inkelas1997} suggest that \textsc{Labial Attraction} may have even induced alternations at one point in the history of Turkish. Beyond Turkish, the rounding of a high back vowel after a labial consonant, as in the alternative formulation of \textsc{Labial Attraction} proposed by \citet{Inkelas2001}, is widely attested \citep[e.g.,][]{Vaux1993}. It is also phonetically natural, as both labial consonants and high round vowels are distinguished by a very low first and second formant. As shown in Table \ref{lasi}, it is even a statistically reliable generalization about the Turkish lexicon. However, it is not reflected in alternations or in wordlikeness judgements.

%Regarding the non-local condition holding between the \emph{a} portion of the trigger and the \emph{u} target, Classical Arabic has many adjectives which form corresponding inchoative verbs by overwriting the vocalic melody with /a\ldots{}u/ (the final /-a/ is an inflectional suffix).

%\begin{example}[Arabic derived inchoatives]
%\begin{tabular}{l l l l}
%\buf{}[kabiːr]  & `big'      & [kabura]  & `become big'       \\
%\buf{}[ħasan]   & `handsome' & [ħasuna]  & `become beautiful' \\
%\buf{}[dʒadiːb] & `barren'   & [dʒaduba] & `become dry'       \\
%\end{tabular}
%\end{example}

%\noindent
%In an Optimality Theory framework, for instance, constraints responsible for rounding of high back vowels before labials and for /a\ldots{}u/ overwriting can be conjoined to approximate the structural description of \textsc{Labial Attraction}.

\section{Conclusions}

It has been shown that statistical reliability is not a sufficient condition for a sequence structure generalization to be reflected in psycholinguistic tasks. However, in the case of constraints on Turkish vowels, it is a necessary condition that the constraint in question be derived from an alternation. Even alternations that have large numbers of lexical exceptions have robust effects on a constrained wordlikeness task. Purely statistical approaches to phonotactic knowledge fail to draw this important distinction.

%Generative phonological theory at the time of \emph{SPE} lacked the representational vocabulary to directly distinguish between word-initial /bn/, which is disallowed, and /bn/ as a licit syllable contact cluster (e.g., \emph{o}[b.n]\emph{oxious}), a point first made by \citep{Hooper1973}. It is not the introduction of prosodic primitives into generative phonology that explains the contrast between \emph{blick} and \emph{bnick}, though, but rather the principle that surface forms must be syllabified \citep[e.g.,][63f.]{Kiparsky1982b} which \emph{bnick} fails. If, as is standardly assumed, phonology repairs unsyllabifiable outputs \citep[e.g.][]{Ito1989a,Noske1992}, then \emph{bnick} is an impossible output, doomed to be modified (perhaps to \emph{nick}; \citealt[][19f.]{Wolf2009}), and /bnɪk/ is an impossible UR by Stampean occultation. 
%A further widespread assumption is that static co-occurrence restrictions validated statistically will be reflected by wordlikeness judgements. This assumption is so deeply entrenched that a number of recent studies of phonotactic learning forgo wordlikeness judgements altogether and instead model the lexical statistics themselves (e.g., \citealt{Coetzee2008a} and \citealt{Anttila2008} on Muna and Arabic, \citealt[][385]{Hayes2008a} on Shona and Wargamay, \citealt{Brown2010} on Gitksan). 
%Since the null hypothesis does not countenance statistical significance as evidence for the synchronic reality of a static lexical pattern, such patterns do not adjudicate between the null hypothesis and the alternative, and another source of evidence is needed. In the remainder of Part I, I turn to data from wordlikeness judgement tasks for these purposes. 
%In a wordlikeness task, a speaker is presented with nonce words and asked to report such intuitions about the ``possibility'' of a word. 
%The alternative hypothesis, then, is that a statistically reliable co-occurrence restriction will be reflected in native speakers' wordlikeness judgements. 
%The data below, drawn from Turkish, shows that 
%This chapter takes the argument one step farther by showing a contrast between static and derived constraints on URs in Turkish. Whereas the derived constraint has a robust effect on native speakers' wordlikeness judgements, the static constraint, which is equally statistically valid, has no effect on these judgements. Anticipating the conclusion, the simplest explanation for the non-effect of this static constraint on wordlikeness judgements is that speakers do not internalize static co-occurrence restrictions at all.
%As a quick validation of this claim, consider the results reported by \citet{Albright2007} and discussed in the previous chapter. English speakers asked to rate non-words on a scale from 1-7, where 7 indicates ``most like English'', favor non-words [blʌs, blɑd] (average scores 4.67 and 5.13) over [bnʌs, bnɑd] (average scores 2.06, 2.00).
%This hypothesis is not ahdered to by early research on sequence structure constraints \citep[e.g.,][]{SPR,Chomsky1965,SPE,Stanley1967}. Work in this vein sought to use sequence structure constraints to compress the lexicon as much as possible. This work has a second interest, namely in ``compressing'' the lexicon as much as possible,
%motivations for the development of a theory of static co-occurrence restrictions as developed by \citet{SPR}, \citet{Stanley1967} and others is to account for speakers' abilities to distinguish between accidental and structural gaps. \citet{Chomsky1965} propose that a theory of phonology must account for speakers' abilities to distinguish between accidental and structural gaps in their language, citing a contrast between \emph{blick}, which is judged to be a possible but unattested word of English, and \emph{bnick}, which is judged to be impossible. \emph{SPE} (p.~3810f.) places this contrast in a system of a system of sequence structure rules that also are used to eliminate lexical redundancy. In the phonological theory of the era, there is no process that could generate the \emph{blick} $\sim$ \emph{bnick} contrast or numerous other lexical regularities
%Generative phonological theory at the time of \emph{SPE} lacked the representational vocabulary to directly distinguish between word-initial /bn/, which is disallowed, and /bn/ as a licit syllable contact cluster (e.g., \emph{o}[b.n]\emph{oxious}), a point first made by \citep{Hooper1973}. It is not the introduction of prosodic primitives into generative phonology that explains the contrast between \emph{blick} and \emph{bnick}, though, but rather the principle that surface forms must be syllabified \citep[e.g.,][63f.]{Kiparsky1982b} which \emph{bnick} fails. If, as is standardly assumed, phonology repairs unsyllabifiable outputs \citep[e.g.][]{Ito1989a,Noske1992}, then \emph{bnick} is an impossible output, doomed to be modified (perhaps to \emph{nick}; \citealt[][19f.]{Wolf2009}), and /bnɪk/ is an impossible UR by Stampean occultation. 
%Further, \citet[][528f.]{Halle1975} rejects his own principle that lexical entries should be free of all redundancies (see \citealt[][201]{Reiss2003a} and \citealt{Vaux2003} for further discussion). This rejected principle of a redundancy-free lexicon is one of the two principles that motivates Morpheme Structure Constraints; the other is the subject of this and the following chapter.
%\citet{Berent2007a}
%\citet{Berent2007b}
%\citet{Berent2008a}
% 598: Highly marked inputs are repaired in perception to abide by the grammatical restrictions of the language
% Exps 1-2: syllable count task (misperception due to linguistic experience; "disyllables were more likely to be perceived as monosyllables if their monosyllabic counterparts had a less marked cluster" in both Russian and English, not what OT predicts)
% Exps 3-4: AB task
% Exps 5-6: priming effect (non-epenthetic and epenethetic things represented the same)
% Berent and Lennertz 2007: (640: If the high rate of monosyllabic errors to unmarked onsets is only due to phonetic failures to encode the input, then it is puzzling why the same onsets also yield the highest rate of accurate response.)
% Berent et al. 2008a:
%\citet{Peperkamp2007}
% Peperkamp 2007: during one stage of processing phonology is undone with help from the lexicon...it only takes licit inputs and is irrelevant (p. 633)
%\citet{Dupoux1999}
% Dupoux et al. 1999: Japanese speakers still can't distinguish between ebzo and ebuzo in an ABX task using the SAME a or b.
%Once additional source of evidence on root (dis)harmony is inconclusive. There is a small class of bisyllabic words in which the second vowel, always [$+$\textsc{High}, $-$\textsc{Back}], alternates with zero. 
%\ex High-vowel/zero alternations \citep[][243]{Clements1982}: \\
%\begin{tabular}{l l l l}
%   & nom.sg. & gen.sg. \\
%a. & fikir   & fikri  & `idea' \\
%   & hüküm   & hükmün & `judgement' \\
%%  & filim   & filmi & `film' & \citep[][178]{Inkelas2001} \\
%b. & vakit   & vaktin & `time' \\
%   & rahim   & rahmin & `womb' \\
%\end{tabular} \xe
%It is possible that \textsc{Backness Harmony} might produce a fluctuating \emph{ı} after root \emph{a}, but this does not obtain (\lastx b). However, this might simply indicate that the fluctuating vowel is epenthetic and that harmony applies before epenthesis (see \citealt{Clements1982} for both sides of this argument), making it less than a counterexample. 
%duplication problem: \citep{Kisseberth1970b} (see also \citealp[][401]{Stanley1967} and \emph{SPE}:382)
%(e.g., \emph{deve} `camel' vs. \emph{deva} `medicine', \emph{sene} `year' vs. \emph{sena} `praise').
%This is not to imply uncritical acceptance of \citeauthor{Becker2011}'s account. \citeauthor{Becker2011} place this generalization ``in the grammar'' to account for their claim that ``naturalness'', which they equate with Universal Grammar, constrains the types of generalizations speakers extract in this task.
%It is difficult to evaluate their claim in the absense of a theory of phonotactic naturalness, something which has consisteny eluded formalization for decades. Even the informal definition used by \citeauthor{Becker2011} is flawed. \citeauthor{Becker2011} label generalizations unnatural if they lack phonetic precursors, or if they are not unattested---no description of the methodology used to determine attestedness is given. These two varieties of evidence are not fully independent, however: extragrammatical \emph{channel bias} effects are thought to be a first-order predictor of attestation \citep{Blevins2004,Moreton2008}. In light of extensive evidence for generalizations which are phonetically unnatural \citep[e.g.,]{Anderson1981,Bach1972,Buckley2000a}, even as revealed in nonce word studies \citep[e.g.,][]{Hayes2009}, the generalizations extracted in the \citet{Becker2011} study are phonetically natural is either accident or conundrum. It is certainly not something that can be explained by the tacit theory of possible gneeralizations the authors adhere to.
%Whereas the classic analysis of this phenomenon, by \citet{Inkelas1997}, does not make use of lexical exceptionality, \citeauthor{Becker2011}'s analysis of final aspirated stops and affricates as the underlying form requires that _every_ lexical entry with a final stop or affricate be treated as exceptional according to some phonological constraint. This projection of a lexical contrast into the grammar has another potential flaw: the constraints used to produce the voiced variants in, e.g., the dative, are not surface true. Consider the constraint *VpV used to induce alternations like \emph{kap}-\emph{kabı} `coat'. Ignoring the case of non-alternating root-final \emph{p} (e.g., \emph{ip}-\emph{ipi} `rope'), this is still not a surface-true generalization: \emph{p} occurs freely in intervocalic position: \emph{ahtapot} `octopus', \emph{köpük} `bubble', \emph{öpücük} `kiss'. More serious is the presence of root-internal intervocalic \emph{p} in ooots which simultaneously exhibit the root-final \emph{p}-\emph{b} alternation: \emph{supap}-\emph{supabı} `valve', \emph{hipermetrop}-\emph{hipermetrobu} `far-sightedness'. One prediction shared by virtually all theories of lexical exceptionality is that a lexical entry cannot simultaneously be a target for, and exception to, a phonological generalization, but this precisely what is needed to account for the behavior of \emph{supap}. 
% \citet{inkelas1997}
%küp-kübü `cube'
%kasap-kasabı `butcher'
%Further, the empirical status of the core instances of NDEB have recently been quite stridently disputed \citep{InkelasInPress}. There is some reason to suspect that NDEB is a symptom of interface restrictions on phonological application, rather than a diagnosis itself.
%\begin{example}[\textsc{Backness Harmony} lexical statistics]
%\begin{example}[Wordlikeness comparisons, backness harmony]
%\begin{example}[Wordlikeness comparisons, roundness harmony]
%\begin{example}[\textsc{Labial harmony} and etymology in TELL (counts from \citealp{Inkelas2001}:187)]
%\begin{example}[Wordlikeness comparisons, labial attraction]
%\begin{table}
%\centering
%\begin{tabular}{lrrrr}
%\toprule
%        & {a}P{u} & {a}P{ı} & \% aPu & $p$-value \\
%\midrule
%native  & 12      & 11      & 52     & \multirow{2}{*}{0.042} \\
%foreign & 84      & 28      & 75                              \\
%\bottomrule
%\end{tabular}
%\caption{?}
%\label{lae}
%\end{table}
%The loanword adaptation literature also contains many reports that statistically reliable phonotactic generalizations about the native lexicon are absent in processes of nativization \citep[e.g.,][]{Ito1995a,Ito1995b,Ussishkin2003}. It is not immediately clear that this is evidence that the generalizations in question are external to the synchronic grammar, however.

%  & {harf}         & {harfler}  & `(alphabetic) letter' \\ 
%  & {el}           & {eller}    & `hand'         \\
%  & {sap}          & {saplar}   & `stalk'        \\
%  & {son}          & {sonlar}   & `end'          \\
%b. & grip    & \alt{} & gırip    & `grippe'    \\ % unexpectedly back
%   & kredi   & \alt{} & kıredi   & `credit'    \\
%  & {el}           & {eli}          & {elin}         & `hand'    \\
%  & {yüz}          & {yüzü}         & {yüzün}        & `face'    \\
%  & {pul}          & {pulu}         & {pulun}        & `stamp'   \\

%\label{vowels}
%    \section{Wordlikeness predictions} %\section{Turkish vowel sequence structure constraints}

\citet{Lees1966a,Lees1966b} proposes three sequence structure constraints on Turkish vowels, and these constraints have been discussed extensively in the subsequent literature. Below, these constraints are formalized and related to Turkish alternation phonology and various sources of external evidence. A traditional binary feature specification for the Turkish vowel system, given below, is assumed without discussion.

\begin{example}
Turkish vowel features: 

\vspace{0.5\baselineskip}
\begin{tabular}{c | c c c c}
                       & \multicolumn{2}{c}{[$-$\textsc{Back}]} & \multicolumn{2}{c}{[$+$\textsc{Back}]} \\
                       & [$-$\textsc{Round}] & [$+$\textsc{Round}] & [$-$\textsc{Round}] & [$+$\textsc{Round}] \\ \midrule
\buf[$+$\textsc{High}] & i & ü [y] & ı [ɯ] & u \\
\buf[$-$\textsc{High}] & e & ö [ø] & a [ɑ] & o \\
\end{tabular}
\end{example}

\subsection{Backness harmony}

One of the most salient properties of Turkish is its harmony system, which \citeauthor{Lees1966b} (\citeyear[][35]{Lees1966b}, \citeyear[][284]{Lees1966a}) derives by means of feature spreading rules. The least constrained of these processes is the rule of backness harmony described in this section. 

\subsubsection{Phonological description}

Backness harmony spreads the value of \textsc{Back} rightward over any intervening consonants onto the next vowel. 

%\citep[after][229]{Clements1982}: \\
\begin{example}
\textsc{Backness Harmony}: 

\xymatrix@R=24pt@C=24pt{
\txt{V}                                        & \txt{C$_0$} & \txt{V} \\
\txt{[α \textsc{Back}]}\ar@{-}[u]\ar@{..}[urr] \\
}
\end{example}

Along with the rule of \textsc{Roundness Harmony} discussed in the next section, this rule accounts for a great deal of the suffix allomorphy found in Turkish. The nominative plural (nom.pl.) is formed by adding the suffixes \emph{-ler} or \emph{-lar} to the nominative singular (nom.sg.). 

\begin{example}
The Turkish nominative: 

\vspace{0.5\baselineskip}
\begin{tabular}{l l l l l}
   & \emph{nom.sg.} & \emph{nom.pl.} \\ 
a. & ip             & ipler          & `rope' & \citep[][216]{Clements1982} \\
   & el             & eller          & `hand'    \\
   & köy            & köyler         & `village' \\
   & yüz            & yüzler         & `face'    \\
   & kız            & kizlar         & `girl'    \\
   & sap            & saplar         & `stalk'   \\
   & son            & sonlar         & `end'     \\
   & pul            & pullar         & `stamp'   \\
b. & neden          & nedenler       & `reason'  & (TELL) \\
   & boğaz          & boğazlar       & `throat'  \\
   & kiler          & kilerler       & `pantry'  \\
   & pelür          & pelürler       & `tissue paper' \\
   & sapık          & sapıklar       & `pervert' \\
\end{tabular}
\end{example}

\noindent
\textsc{Backness Harmony} correctly predicts the choice of suffix. There are a few complications, however. First, while most polysyllabic roots conform to \textsc{Backness Harmony} (\lastx b), not all do. In this case, suffix vowels generally harmonize with the final root vowel (\nextx a). Yet there is also a small class of nouns which exhibit [$-$\textsc{Back}] suffix vowels despite the fact that their final root vowel is [$+$\textsc{Back}] (\nextx b).

\begin{example}
Exceptional Turkish nominatives: 

\vspace{0.5\baselineskip}
\begin{tabular}{l l l l l}
   & \emph{nom.sg.} & \emph{nom.pl.} \\
a. & mezar          & mezarlar       & `grave' & \citep{TELL} \\
   & model          & modeller       & `model' \\
   & silah          & silahlar       & `weapon'     \\
   & memur          & memurlar       & `bureaucrat' \\
   & sabun          & sabunlar       & `soap'       \\
b. & saat           & saatler        & `hour, clock' \\
   & harf           & harfler        & `(alphabetic) letter' \\ %& \citep{Goksel2005}
   & etol           & etoller        & `fur stole' \\
\end{tabular}
\end{example}

While it is uncontroversial that disharmonic suffixes (\lastx b) are no more than sporadic exceptions to \textsc{Backness Harmony}, the status of root disharmony (\lastx a) has been the subject of much debate. \citet[][212, 289]{Anderson1974} views the fact that disharmonic roots still trigger suffix harmony as evidence that suffix harmony is distinct from root harmony, the latter being a sequence structure constraint. 

%Once additional source of evidence on root (dis)harmony is inconclusive. There is a small class of bisyllabic words in which the second vowel, always [$+$\textsc{High}, $-$\textsc{Back}], alternates with zero. 

%\ex High-vowel/zero alternations \citep[][243]{Clements1982}: \\
%\begin{tabular}{l l l l}
%   & nom.sg. & gen.sg. \\
%a. & fikir   & fikri  & `idea' \\
%   & hüküm   & hükmün & `judgement' \\
%%  & filim   & filmi & `film' & \citep[][178]{Inkelas2001} \\
%b. & vakit   & vaktin & `time' \\
%   & rahim   & rahmin & `womb' \\
%\end{tabular} \xe
%
%\noindent
%It is possible that \textsc{Backness Harmony} might produce a fluctuating \emph{ı} after root \emph{a}, but this does not obtain (\lastx b). However, this might simply indicate that the fluctuating vowel is epenthetic and that harmony applies before epenthesis (see \citealt{Clements1982} for both sides of this argument), making it less than a counterexample. 

Underspecification provides an alternative under which the exceptionality of disharmonic roots is only apparent. While it is clear that the first vowel of Turkish roots are contrastively specified for backness (e.g., \emph{kül} `ash' vs.  \emph{kul} `servant', \emph{kepek} `bran' vs. \emph{kapak} `lid'), there are also some minimal pairs with regards to backness (dis)harmony (e.g., \emph{deve} `camel' vs. \emph{deva} `medicine', \emph{sene} `year' vs. \emph{sena} `praise'). This supports the possibility, first suggested by \citet{Clements1982}, that harmonic roots may be underspecified and disharmonic roots fully specified, exemplified below.

%There is some further evidence that individual vowels may differ in specification for this feature even within individual roots or affixes. For instance, the present continuous suffix has harmony-determined allomorphs \emph{-iyor}, \emph{-üyor}, \emph{-ıyor}, \emph{-uyor}, but the \emph{o} of the suffix is invariant. A similar situation might obtain in Turkish roots. 

\begin{example}
Underlying specification of \textsc{Back} for (dis)harmonic roots: 

\xymatrix@R=24pt@C=24pt{
\txt{a.} & \txt{s} & \txt{V} & \txt{n} & \txt{V}  & \txt{\emph{sene} `year'} \\
&        & \txt{[$-$\textsc{Back}]}\ar@{-}[u]     \\
\txt{b.} & \txt{s} & \txt{V} & \txt{n} & \txt{V} & \txt{\emph{sena} `praise'} \\
         &         & \txt{[$-$\textsc{Back}]}\ar@{-}[u] & & \txt{[$+$\textsc{Back}]}\ar@{-}[u]
}
\end{example}

\noindent
Harmonizing suffix vowels will also be underspecified for \textsc{Back}. Of course, \textsc{Backness Harmony} needs to be prevented from overwriting the [$+$\textsc{Back}] specification of disharmonic roots, one option being the use of a a \textsc{Structure Preservation} condition \citep{Kiparsky1985}. Unfortunately, this detail reintroduces the duplication alluded to above. Any condition on the rule of \textsc{Backness Harmony} which prevents overwriting entails that it has no control over the distribution of harmonic and disharmonic roots, and thus fails to account for predominance of harmonic roots.

However, the theory of exceptionality presented in \emph{SPE} (p.~374f.) provides a direct account of suffix harmony in disharmonic roots without duplication. In \emph{SPE}, the specification of the target (i.e., the segment to be changed) of every rule $R$ must be [$+$rule $R$] by convention.  A root or affix which fails to undergo $R$ despite otherwise matching the structural description is simply said to be marked [$-$rule $R$], and thus failing to meet the full structural description. If disharmonic roots are [$-$\textsc{Backness Harmony}], then suffix vowels are correctly predicted to undergo \textsc{Backness Harmony}, since the [$-$\textsc{Backness Harmony}] root is no longer the target but rather the environment, which is not required to be [$+$rule $R$]. 




\subsubsection{Psycholinguistic evidence}

The evidence from alternations leaves open the question of whether speakers internalize a generalization regarding the tendency of roots to conform to \textsc{Backness Harmony}. One piece of evidence that speaks in favor of root-internal harmony comes from a language game discussed by \citet{Harrison2001}.\footnote{Thanks to Bert Vaux for bringing this study to my attention.} This game is not indigenous to Turkish, but it corresponds to a morphological process native to the related language Tuvan, in which it conveys a sense of ``vagueness or jocularity'', and \citeauthor{Harrison2001} report that it can be quickly taught to even young Turkish speakers. The game consists of reduplication of the base and replacement of the first vowel in the reduplicant with \emph{a} or \emph{u}. 

In both Tuvan and Turkish, the unchanged reduplicant vowel is also affected. Reduplication interacts with root harmony in both Tuvan and Turkish. When the base is harmonically [$-$\textsc{Back}], the insertion of a [$+$\textsc{Back}] results in what \citeauthor{Harrison2001} call ``reharmonization'' (\nextx a). 

\begin{example}
\label{redupgame}
Turkish reduplication game \citep[][231]{Harrison2001}: 

\vspace{0.5\baselineskip}
\begin{tabular}{l l l l}
a. & kibrit & kibrit-kabrıt & `match'    \\
   & bütün  & bütün-batın   & `whole'    \\
b. & mali   & mali-muli     & `Mali'     \\
   & butik  & butik-batik   & `boutique' \\
\end{tabular}
\end{example}

\noindent
Under the underspecification hypothesis, non-initial harmonic vowels lack an underlying \textsc{Back} feature, so it comes as no surprise that changing the \textsc{Back} specification of the root-initial vowel results in reharmonization. And the full specification of disharmonic roots correctly predicts that they will be exempt from reharmonization (\lastx b), which is also borne out in Tuvan and in an unrelated Finnish language game \citep{Campbell1986}.

A number of studies have found that speakers of Finnish use disharmony in speech processing experiments.\footnote{Thanks to Charles Yang for pointing out the relevance of these studies to me.} \citet{Suomi1997} and \citet{Vroomen1998} generate nonce trisyllabic words by adding a monosyllabic pseudo-prefix to real and nonce disyllabic words, all of which are harmonic for the feature \textsc{Back}. These stimuli are auditorily presented to subjects who are asked to press a button when the nonce trisyllable ends with a target nonce disyllable, or a real disyllabic word. Speakers are quicker to press the button when the prefix and disyllabic word disagree for \textsc{Back}. These results suggest that speakers are attuned to the fact that disharmonic transisitions are good predictors of word boundaries. If speakers have also internalized the converse generalization, that harmonic transistions are more likely to be root-internal, then there is additional evidence that harmony is active not just in Finnish affix alternations but also in roots. 

\citet{Kabak2010} report that Turkish \textsc{Backness Harmony} has the same effect on word-spotting as it does in Finnish: speakers are quicker and more accurate at the task of spotting the nonce target word \emph{pavo} when preceded by the pseudo-prefix \emph{gölü-}, a disharmonic transistion, than when it is preceded by the pseudo-prefix \emph{golu-}, a harmonic transition. \citeauthor{Kabak2010} find that effect of harmony does not obtain for speakers of French, a language which lacks vowel harmony. As in Finnish, the results imply speakers have internalized the predominance of root-internal harmony.

It seems that the root-internal harmony bias is in fact learned by Turkish speakers very early. The pseudoword spotting experiment has been adapted for 9-month-old Turkish infants by \citet{Kampen2008}. Infants are familiarized with harmonic disyllabic pseudowords bearing a pseudo-prefix, which may be harmonic or disharmonic. At test time, the infants are played the disyllabic pseudowords in isolation using the head turn preference paradigm. Infants show a preference to listen to those pseudowords which were familiarized with a disharmonic pseudo-prefix over those which were familiarized with a harmonic pseudo-prefix. This preference is not observed in 9-month-old infants learning German, which also lacks vowel harmony. Similarly, \citeauthor{Kampen2008} report that Turkish 6-month-old infants prefer to listen to harmonic pseudowords such as \emph{paroz} over disharmonic pseudowords like \emph{nelok}, but German 6-month-old infants show no such preference.

\subsection{Roundness Harmony}

The rule of roundness harmony and the data that motivates it overlaps with the preceding evidence for \textsc{Backness Harmony}. 

\subsubsection{Phonological description}

The environment for roundness harmony differs from \textsc{Backness Harmony} in that it requiers the target to be [$+$\textsc{High}]. 

\begin{example}
\textsc{Roundness Harmony}:

\xymatrix@R=24pt@C=24pt{
\txt{[α \textsc{Round}]}\ar@{-}[d]\ar@{..}[drr] &             & \txt{[$+$\textsc{High}]} \\
\txt{V}                                         & \txt{C$_0$} & \txt{V}\ar@{-}[u] \\
}
\end{example}

This rule, in concert with \textsc{Backness Harmony}, accounts for the forms of the dative singular (dat.sg.) and genitive singular (gen.sg.), among other suffixes.

\begin{example}
Turkish nominal suffix allomorphy: 

\vspace{0.5\baselineskip}
\begin{tabular}{l l l l l l l}
   & \emph{nom.sg.} & \emph{dat.sg.} & \emph{gen.sg.}  \\
a. & ip             & ipi            & ipin           & `rope' & (CS:216) \\
   & el             & eli            & elin           & `hand'    \\
   & kız            & kızı           & kızın          & `girl'    \\
   & sap            & sapı           & sapın          & `stalk'   \\
   & yüz            & yüzü           & yüzün          & `face'    \\
   & köy            & köyü           & köyün          & `village' \\
   & pul            & pulu           & pulun          & `stamp'   \\
   & son            & sonu           & sonun          & `end'     \\
b. & boğaz          & boğazı         & boğazın        & `throat'  & (TELL) \\
   & pelür          & pelürü         & pelürün        & `tissue paper' \\
   & döviz          & dövizi         & dövizin        & `currency' \\
   & yamuk          & yamuğu         & yamuğun        & `trapezoid' \\
   & ümit           & ümiti          & ümitin         & `hope'     \\
\end{tabular}
\end{example}

\noindent
Much as was the case for \textsc{Backness Harmony}, there are disharmonic roots which participate in suffix harmony. Once again, underspecification of harmonic roots and full specification of disharmonic roots allows for the use of the \emph{SPE} exception convention. 

\subsubsection{Psycholinguistic evidence}

The only psycholinguistic evidence for \textsc{Roundness Harmony} comes from \citeauthor{Harrison2001}'s language game (\ref{redupgame}). As shown above, the second vowel in the reduplicated form of \emph{bütün} `whole' is \emph{bütün-batın}. The second vowel in the base, \emph{ü}, is reharmonized as [$+$\textsc{Back}, $-$\textsc{Round}], indicating that \textsc{Roundness Harmony} also participates in reharmonization.

\subsection{Labial attraction}

%\begin{quote}
%\ldots{}these generalizations\ldots{}have no observable consequences in the course of the normal use of the language. \citep[][320]{Zimmer1969}
%\end{quote}

\subsubsection{Phonological description}

\citet[][36]{Lees1966a} describes \textsc{Labial Attraction} as a process by which ``a high, short harmonic vowel is rounded in the second syllable of a disyllabic word whose first vowel is /a/, and whose medial consonant cluster contains a labial /p, b, m, v/, and then it is de-harmonified''. This description is transalted into an autosegmental rule below.

\begin{example}
\textsc{Labial Attraction} (after \citealt[][286]{Lees1966b}, \citealt[][171]{Inkelas2001}): 

\xymatrix@R=24pt@C=8pt{
\txt{[$-$\textsc{Round}]} &                                         & \txt{[$-$\textsc{High}]} & \txt{[$+$\textsc{Labial}]}    & \txt{[$+$\textsc{High}]}\ar@{-}[dr] &         & \txt{[$+$\textsc{Round}]}\ar@{--}[dl] \\
                         & \txt{V}\ar@{-}[ul]\ar@{-}[ur]\ar@{-}[d] &                           & \txt{C$_0$ C C$_0$}\ar@{-}[u] &                                      & \txt{V} & \\
                         & \txt{[$+$\textsc{Back}]}\ar@{-}[urrrr] 
}
\end{example}

The formalization of this rule is naturally complex, and perhaps obscures the fact that \textsc{Labial Attraction} generates exceptions to \textsc{Roundness Harmony}; it produces \emph{aCu} sequences (e.g., \emph{çapul} `raid', \emph{sabur} `patient', \emph{şaful} `wooden honey tub', \emph{avuç} `palm of hand', and \emph{samur} `sable' \citep[][285]{Lees1966b}) instead of the otherwise expected \emph{aCı}. \citeauthor{Lees1966b} notes the existence of exceptions (e.g., \emph{tavır} `mode') but writes that they are they are ``surprisingly rare'' (ibid., 286); \citet[][225]{Clements1982} note additional exceptions.  

There is reason to believe that \textsc{Labial Attraction} is at best a sequence structure constraint as it never applies in derived environments. If, contrary to fact, there was a \textsc{Labial Attraction} alternation, then one would expect, for example, that the gen.sg. of \emph{sap} `stalk' would be *\emph{sapun} instead of the observed \emph{sapın}.

\citet{Lees1966a}
\citet{Zimmer1969}

\begin{quote}
\ldots decisive evidence against a rule of Labial Attraction is the existence of a further, much larger set of roots containing /\ldots~aCu~\ldots/ sequences in which the intervening consonant or consonant cluster does not contain a labial\ldots We conclude that there is no systematic restriction on the set of consonants that may occur medially in rotos of the form /\ldots~aCu~\ldots/. \citep[][225]{Clements1982}
\end{quote}

\noindent
\citeauthor{Clements1982} appear to be suggesting that \textsc{Labial Attraction} implies that \emph{aTu}, where \emph{T} represents a non-labial consonant, should be infrequent, but this fact is not inconsistent with the formulation of the constraint by \citet{Lees1966a,Lees1966b} and \citet{Zimmer1969}; \emph{aTu} clusters do not meet \textsc{Labial Attraction}'s structural description. This is less than conclusive, since the fluctuating vowel alternation also fails to undergo harmony, and might just indicate that the flucutating vowel is epenthesized relatively late in the derivation; the frequency of \emph{aTu} provides only indirect information about \textsc{Labial Attraction} in that it provides a baseline estimate for just how frequent this disharmonic sequence of vowels is.

\citet{Inkelas1997}
\citet{Inkelas2001}

\begin{quote}
Lee's rule of \textsc{Labial Attraction}\ldots is not a real generalization about the Turkish lexicon. It is not true synchronically, either of native or nonnative items; nor, according to the historical and dialectical literature, does \textsc{Labial Attraction} appear to have been true at any stage going back as far as Old Turkic. \citep[][196]{Inkelas2001}
\end{quote}

zimmer: 319-320 
inkelas et al.: 196

\subsubsection{Psycholinguistic evidence}

To the author's knowledge, there are no psycholinguistic studies investigating \textsc{Labial Attraction} beyond the experimental results of \citet{Zimmer1969} reviewed in the next section. 


%        \subsection{Alternation vs. phonotactic} %\subsection{Backness harmony}

One of the most salient properties of Turkish is its harmony system, which \citeauthor{Lees1966b} (\citeyear[][35]{Lees1966b}, \citeyear[][284]{Lees1966a}) derives by means of feature spreading rules. The least constrained of these processes is the rule of backness harmony described in this section. 

\subsubsection{Phonological description}

Backness harmony spreads the value of \textsc{Back} rightward over any intervening consonants onto the next vowel. 

%\citep[after][229]{Clements1982}: \\
\begin{example}
\textsc{Backness Harmony}: 

\xymatrix@R=24pt@C=24pt{
\txt{V}                                        & \txt{C$_0$} & \txt{V} \\
\txt{[α \textsc{Back}]}\ar@{-}[u]\ar@{..}[urr] \\
}
\end{example}

Along with the rule of \textsc{Roundness Harmony} discussed in the next section, this rule accounts for a great deal of the suffix allomorphy found in Turkish. The nominative plural (nom.pl.) is formed by adding the suffixes \emph{-ler} or \emph{-lar} to the nominative singular (nom.sg.). 

\begin{example}
The Turkish nominative: 

\vspace{0.5\baselineskip}
\begin{tabular}{l l l l l}
   & \emph{nom.sg.} & \emph{nom.pl.} \\ 
a. & ip             & ipler          & `rope' & \citep[][216]{Clements1982} \\
   & el             & eller          & `hand'    \\
   & köy            & köyler         & `village' \\
   & yüz            & yüzler         & `face'    \\
   & kız            & kizlar         & `girl'    \\
   & sap            & saplar         & `stalk'   \\
   & son            & sonlar         & `end'     \\
   & pul            & pullar         & `stamp'   \\
b. & neden          & nedenler       & `reason'  & (TELL) \\
   & boğaz          & boğazlar       & `throat'  \\
   & kiler          & kilerler       & `pantry'  \\
   & pelür          & pelürler       & `tissue paper' \\
   & sapık          & sapıklar       & `pervert' \\
\end{tabular}
\end{example}

\noindent
\textsc{Backness Harmony} correctly predicts the choice of suffix. There are a few complications, however. First, while most polysyllabic roots conform to \textsc{Backness Harmony} (\lastx b), not all do. In this case, suffix vowels generally harmonize with the final root vowel (\nextx a). Yet there is also a small class of nouns which exhibit [$-$\textsc{Back}] suffix vowels despite the fact that their final root vowel is [$+$\textsc{Back}] (\nextx b).

\begin{example}
Exceptional Turkish nominatives: 

\vspace{0.5\baselineskip}
\begin{tabular}{l l l l l}
   & \emph{nom.sg.} & \emph{nom.pl.} \\
a. & mezar          & mezarlar       & `grave' & \citep{TELL} \\
   & model          & modeller       & `model' \\
   & silah          & silahlar       & `weapon'     \\
   & memur          & memurlar       & `bureaucrat' \\
   & sabun          & sabunlar       & `soap'       \\
b. & saat           & saatler        & `hour, clock' \\
   & harf           & harfler        & `(alphabetic) letter' \\ %& \citep{Goksel2005}
   & etol           & etoller        & `fur stole' \\
\end{tabular}
\end{example}

While it is uncontroversial that disharmonic suffixes (\lastx b) are no more than sporadic exceptions to \textsc{Backness Harmony}, the status of root disharmony (\lastx a) has been the subject of much debate. \citet[][212, 289]{Anderson1974} views the fact that disharmonic roots still trigger suffix harmony as evidence that suffix harmony is distinct from root harmony, the latter being a sequence structure constraint. 

%Once additional source of evidence on root (dis)harmony is inconclusive. There is a small class of bisyllabic words in which the second vowel, always [$+$\textsc{High}, $-$\textsc{Back}], alternates with zero. 

%\ex High-vowel/zero alternations \citep[][243]{Clements1982}: \\
%\begin{tabular}{l l l l}
%   & nom.sg. & gen.sg. \\
%a. & fikir   & fikri  & `idea' \\
%   & hüküm   & hükmün & `judgement' \\
%%  & filim   & filmi & `film' & \citep[][178]{Inkelas2001} \\
%b. & vakit   & vaktin & `time' \\
%   & rahim   & rahmin & `womb' \\
%\end{tabular} \xe
%
%\noindent
%It is possible that \textsc{Backness Harmony} might produce a fluctuating \emph{ı} after root \emph{a}, but this does not obtain (\lastx b). However, this might simply indicate that the fluctuating vowel is epenthetic and that harmony applies before epenthesis (see \citealt{Clements1982} for both sides of this argument), making it less than a counterexample. 

Underspecification provides an alternative under which the exceptionality of disharmonic roots is only apparent. While it is clear that the first vowel of Turkish roots are contrastively specified for backness (e.g., \emph{kül} `ash' vs.  \emph{kul} `servant', \emph{kepek} `bran' vs. \emph{kapak} `lid'), there are also some minimal pairs with regards to backness (dis)harmony (e.g., \emph{deve} `camel' vs. \emph{deva} `medicine', \emph{sene} `year' vs. \emph{sena} `praise'). This supports the possibility, first suggested by \citet{Clements1982}, that harmonic roots may be underspecified and disharmonic roots fully specified, exemplified below.

%There is some further evidence that individual vowels may differ in specification for this feature even within individual roots or affixes. For instance, the present continuous suffix has harmony-determined allomorphs \emph{-iyor}, \emph{-üyor}, \emph{-ıyor}, \emph{-uyor}, but the \emph{o} of the suffix is invariant. A similar situation might obtain in Turkish roots. 

\begin{example}
Underlying specification of \textsc{Back} for (dis)harmonic roots: 

\xymatrix@R=24pt@C=24pt{
\txt{a.} & \txt{s} & \txt{V} & \txt{n} & \txt{V}  & \txt{\emph{sene} `year'} \\
&        & \txt{[$-$\textsc{Back}]}\ar@{-}[u]     \\
\txt{b.} & \txt{s} & \txt{V} & \txt{n} & \txt{V} & \txt{\emph{sena} `praise'} \\
         &         & \txt{[$-$\textsc{Back}]}\ar@{-}[u] & & \txt{[$+$\textsc{Back}]}\ar@{-}[u]
}
\end{example}

\noindent
Harmonizing suffix vowels will also be underspecified for \textsc{Back}. Of course, \textsc{Backness Harmony} needs to be prevented from overwriting the [$+$\textsc{Back}] specification of disharmonic roots, one option being the use of a a \textsc{Structure Preservation} condition \citep{Kiparsky1985}. Unfortunately, this detail reintroduces the duplication alluded to above. Any condition on the rule of \textsc{Backness Harmony} which prevents overwriting entails that it has no control over the distribution of harmonic and disharmonic roots, and thus fails to account for predominance of harmonic roots.

However, the theory of exceptionality presented in \emph{SPE} (p.~374f.) provides a direct account of suffix harmony in disharmonic roots without duplication. In \emph{SPE}, the specification of the target (i.e., the segment to be changed) of every rule $R$ must be [$+$rule $R$] by convention.  A root or affix which fails to undergo $R$ despite otherwise matching the structural description is simply said to be marked [$-$rule $R$], and thus failing to meet the full structural description. If disharmonic roots are [$-$\textsc{Backness Harmony}], then suffix vowels are correctly predicted to undergo \textsc{Backness Harmony}, since the [$-$\textsc{Backness Harmony}] root is no longer the target but rather the environment, which is not required to be [$+$rule $R$]. 




\subsubsection{Psycholinguistic evidence}

The evidence from alternations leaves open the question of whether speakers internalize a generalization regarding the tendency of roots to conform to \textsc{Backness Harmony}. One piece of evidence that speaks in favor of root-internal harmony comes from a language game discussed by \citet{Harrison2001}.\footnote{Thanks to Bert Vaux for bringing this study to my attention.} This game is not indigenous to Turkish, but it corresponds to a morphological process native to the related language Tuvan, in which it conveys a sense of ``vagueness or jocularity'', and \citeauthor{Harrison2001} report that it can be quickly taught to even young Turkish speakers. The game consists of reduplication of the base and replacement of the first vowel in the reduplicant with \emph{a} or \emph{u}. 

In both Tuvan and Turkish, the unchanged reduplicant vowel is also affected. Reduplication interacts with root harmony in both Tuvan and Turkish. When the base is harmonically [$-$\textsc{Back}], the insertion of a [$+$\textsc{Back}] results in what \citeauthor{Harrison2001} call ``reharmonization'' (\nextx a). 

\begin{example}
\label{redupgame}
Turkish reduplication game \citep[][231]{Harrison2001}: 

\vspace{0.5\baselineskip}
\begin{tabular}{l l l l}
a. & kibrit & kibrit-kabrıt & `match'    \\
   & bütün  & bütün-batın   & `whole'    \\
b. & mali   & mali-muli     & `Mali'     \\
   & butik  & butik-batik   & `boutique' \\
\end{tabular}
\end{example}

\noindent
Under the underspecification hypothesis, non-initial harmonic vowels lack an underlying \textsc{Back} feature, so it comes as no surprise that changing the \textsc{Back} specification of the root-initial vowel results in reharmonization. And the full specification of disharmonic roots correctly predicts that they will be exempt from reharmonization (\lastx b), which is also borne out in Tuvan and in an unrelated Finnish language game \citep{Campbell1986}.

A number of studies have found that speakers of Finnish use disharmony in speech processing experiments.\footnote{Thanks to Charles Yang for pointing out the relevance of these studies to me.} \citet{Suomi1997} and \citet{Vroomen1998} generate nonce trisyllabic words by adding a monosyllabic pseudo-prefix to real and nonce disyllabic words, all of which are harmonic for the feature \textsc{Back}. These stimuli are auditorily presented to subjects who are asked to press a button when the nonce trisyllable ends with a target nonce disyllable, or a real disyllabic word. Speakers are quicker to press the button when the prefix and disyllabic word disagree for \textsc{Back}. These results suggest that speakers are attuned to the fact that disharmonic transisitions are good predictors of word boundaries. If speakers have also internalized the converse generalization, that harmonic transistions are more likely to be root-internal, then there is additional evidence that harmony is active not just in Finnish affix alternations but also in roots. 

\citet{Kabak2010} report that Turkish \textsc{Backness Harmony} has the same effect on word-spotting as it does in Finnish: speakers are quicker and more accurate at the task of spotting the nonce target word \emph{pavo} when preceded by the pseudo-prefix \emph{gölü-}, a disharmonic transistion, than when it is preceded by the pseudo-prefix \emph{golu-}, a harmonic transition. \citeauthor{Kabak2010} find that effect of harmony does not obtain for speakers of French, a language which lacks vowel harmony. As in Finnish, the results imply speakers have internalized the predominance of root-internal harmony.

It seems that the root-internal harmony bias is in fact learned by Turkish speakers very early. The pseudoword spotting experiment has been adapted for 9-month-old Turkish infants by \citet{Kampen2008}. Infants are familiarized with harmonic disyllabic pseudowords bearing a pseudo-prefix, which may be harmonic or disharmonic. At test time, the infants are played the disyllabic pseudowords in isolation using the head turn preference paradigm. Infants show a preference to listen to those pseudowords which were familiarized with a disharmonic pseudo-prefix over those which were familiarized with a harmonic pseudo-prefix. This preference is not observed in 9-month-old infants learning German, which also lacks vowel harmony. Similarly, \citeauthor{Kampen2008} report that Turkish 6-month-old infants prefer to listen to harmonic pseudowords such as \emph{paroz} over disharmonic pseudowords like \emph{nelok}, but German 6-month-old infants show no such preference.

%        \subsection{Lexical statistics} %\subsection{Roundness Harmony}

The rule of roundness harmony and the data that motivates it overlaps with the preceding evidence for \textsc{Backness Harmony}. 

\subsubsection{Phonological description}

The environment for roundness harmony differs from \textsc{Backness Harmony} in that it requiers the target to be [$+$\textsc{High}]. 

\begin{example}
\textsc{Roundness Harmony}:

\xymatrix@R=24pt@C=24pt{
\txt{[α \textsc{Round}]}\ar@{-}[d]\ar@{..}[drr] &             & \txt{[$+$\textsc{High}]} \\
\txt{V}                                         & \txt{C$_0$} & \txt{V}\ar@{-}[u] \\
}
\end{example}

This rule, in concert with \textsc{Backness Harmony}, accounts for the forms of the dative singular (dat.sg.) and genitive singular (gen.sg.), among other suffixes.

\begin{example}
Turkish nominal suffix allomorphy: 

\vspace{0.5\baselineskip}
\begin{tabular}{l l l l l l l}
   & \emph{nom.sg.} & \emph{dat.sg.} & \emph{gen.sg.}  \\
a. & ip             & ipi            & ipin           & `rope' & (CS:216) \\
   & el             & eli            & elin           & `hand'    \\
   & kız            & kızı           & kızın          & `girl'    \\
   & sap            & sapı           & sapın          & `stalk'   \\
   & yüz            & yüzü           & yüzün          & `face'    \\
   & köy            & köyü           & köyün          & `village' \\
   & pul            & pulu           & pulun          & `stamp'   \\
   & son            & sonu           & sonun          & `end'     \\
b. & boğaz          & boğazı         & boğazın        & `throat'  & (TELL) \\
   & pelür          & pelürü         & pelürün        & `tissue paper' \\
   & döviz          & dövizi         & dövizin        & `currency' \\
   & yamuk          & yamuğu         & yamuğun        & `trapezoid' \\
   & ümit           & ümiti          & ümitin         & `hope'     \\
\end{tabular}
\end{example}

\noindent
Much as was the case for \textsc{Backness Harmony}, there are disharmonic roots which participate in suffix harmony. Once again, underspecification of harmonic roots and full specification of disharmonic roots allows for the use of the \emph{SPE} exception convention. 

\subsubsection{Psycholinguistic evidence}

The only psycholinguistic evidence for \textsc{Roundness Harmony} comes from \citeauthor{Harrison2001}'s language game (\ref{redupgame}). As shown above, the second vowel in the reduplicated form of \emph{bütün} `whole' is \emph{bütün-batın}. The second vowel in the base, \emph{ü}, is reharmonized as [$+$\textsc{Back}, $-$\textsc{Round}], indicating that \textsc{Roundness Harmony} also participates in reharmonization.

%    \section{Evaluation} %\section{Evaluation}

The results reveal a dissociation between the lexical statistics and wordlikeness judgements, a dissociation that finds a natural explanation from the principle of ...

There is some disagreement in the literature about just what \citet{Zimmer1969} finds regarding \textsc{Backness Harmony} and \textsc{Labial Attraction}.

\subsection{Lexical statistics}
\label{lexstats}

This choice of statistical test is justified in Section \ref{stattech}.

\subsubsection{Backness harmony}

\subsubsection{Roundness harmony}

%\ex Lexical effects of \textsc{Roundness Harmony} \citep{TELL}: \vspace{6pt} \\
%\begin{tabular}{l r r r r r}
%\toprule              & 
%Corpus                & $p$-value \\
%\midrule
%Full TELL             &  
%Elicited TELL         & 
%TELL with etymologies & 
%\end{tabular}
%\xe

%\citet{Harrison2004} 73\% 
%of lexical types are harmonic (both back and round)

\subsubsection{Labial attraction}

\begin{quotation}
Vowel labialization following labials is not a synchronic alternation in Turkish, yet it (unlike \textsc{Labial Attraction} per se) \emph{is} a statistically supported tendency worthy of further research. \citep[][196, emphasis in original]{Inkelas2001} 
\end{quotation}

%\ex Lexical effects of \textsc{Labial Attraction} \citep[][186]{Inkelas2001}: \vspace{6pt} \\
%\begin{tabular}{l r r r r r}
%\toprule
%Corpus                & aPu & aPı & aTu & aTı   & $p$-value   \\ % & corpus size
%\midrule
%Full TELL             & 378 & 248 & 446 & 1,140 & 2.83\e{-44} \\ % & 31,236 \\
%Elicited TELL         & 152 & 265 & 101 & 1,839 & 9.84\e{-60} \\ % & 16,541 \\
%TELL with etymologies & 128 & 109 &  79 &   470 & 6.56\e{-32} \\ 
%\bottomrule
%\end{tabular}
%\xe

%Etymological subset of TELL
%Native    & Foreign   \\
%aBu & aBI & aBu & aBI \\
%12  & 11  & 84  & 28  \\
%p = 0.0417

% 3.2.2: Wordlikeness results

\citet{Goodman1954}

\ex $\displaystyle \gamma = \frac{C - D}{C + D}$ \xe

\subsubsection{Roundness harmony}

($\gamma = 0.597$, $p = 5.2$\e{-29}) 

Roundness harmony wordlikeness forced choices \citep[314]{Zimmer1969}: \vspace{6pt} \\ 
\begin{tabular}{l r l r}
\toprule
\multicolumn{2}{l}{harmonic} & \multicolumn{2}{l}{disharmonic} \\
\midrule
pemez & 30                   & pemaz & 2 \\
teriz & 28                   & terız & 3 \\
tokaz & 26                   & tokez & 6 \\
tipez & 24                   & tipaz & 8 \\
terüz & 19                   & teruz & 13 \\
\bottomrule
\end{tabular}

\subsubsection{Labial harmony}

Labial attraction wordlikeness forced choices \citep[314]{Zimmer1969}: \vspace{6pt} \\ 
\begin{tabular}{l r l r}
\toprule
\multicolumn{2}{l}{aPu} & \multicolumn{2}{l}{aPı} \\
\midrule
tafuz & 21              & tafız & 11 \\
pamuz & 17              & pamız & 15 \\
tapuz & 17              & tapız & 15 \\
mavuz & 16              & mavız & 16 \\
tabuz & 16              & tabız & 16 \\
\bottomrule
\end{tabular}

($\gamma = 0.087$, $p = 0.101$)


%        \subsection{Lexical statistics} %\subsection{Lexical statistics}
\label{lexstats}

This choice of statistical test is justified in Section \ref{stattech}.

\subsubsection{Backness harmony}

\subsubsection{Roundness harmony}

%\ex Lexical effects of \textsc{Roundness Harmony} \citep{TELL}: \vspace{6pt} \\
%\begin{tabular}{l r r r r r}
%\toprule              & 
%Corpus                & $p$-value \\
%\midrule
%Full TELL             &  
%Elicited TELL         & 
%TELL with etymologies & 
%\end{tabular}
%\xe

%\citet{Harrison2004} 73\% 
%of lexical types are harmonic (both back and round)

\subsubsection{Labial attraction}

\begin{quotation}
Vowel labialization following labials is not a synchronic alternation in Turkish, yet it (unlike \textsc{Labial Attraction} per se) \emph{is} a statistically supported tendency worthy of further research. \citep[][196, emphasis in original]{Inkelas2001} 
\end{quotation}

%\ex Lexical effects of \textsc{Labial Attraction} \citep[][186]{Inkelas2001}: \vspace{6pt} \\
%\begin{tabular}{l r r r r r}
%\toprule
%Corpus                & aPu & aPı & aTu & aTı   & $p$-value   \\ % & corpus size
%\midrule
%Full TELL             & 378 & 248 & 446 & 1,140 & 2.83\e{-44} \\ % & 31,236 \\
%Elicited TELL         & 152 & 265 & 101 & 1,839 & 9.84\e{-60} \\ % & 16,541 \\
%TELL with etymologies & 128 & 109 &  79 &   470 & 6.56\e{-32} \\ 
%\bottomrule
%\end{tabular}
%\xe

%Etymological subset of TELL
%Native    & Foreign   \\
%aBu & aBI & aBu & aBI \\
%12  & 11  & 84  & 28  \\
%p = 0.0417

%        \subsection{Wordlikeness results} %% 3.2.2: Wordlikeness results

\citet{Goodman1954}

\ex $\displaystyle \gamma = \frac{C - D}{C + D}$ \xe

\subsubsection{Roundness harmony}

($\gamma = 0.597$, $p = 5.2$\e{-29}) 

Roundness harmony wordlikeness forced choices \citep[314]{Zimmer1969}: \vspace{6pt} \\ 
\begin{tabular}{l r l r}
\toprule
\multicolumn{2}{l}{harmonic} & \multicolumn{2}{l}{disharmonic} \\
\midrule
pemez & 30                   & pemaz & 2 \\
teriz & 28                   & terız & 3 \\
tokaz & 26                   & tokez & 6 \\
tipez & 24                   & tipaz & 8 \\
terüz & 19                   & teruz & 13 \\
\bottomrule
\end{tabular}

\subsubsection{Labial harmony}

Labial attraction wordlikeness forced choices \citep[314]{Zimmer1969}: \vspace{6pt} \\ 
\begin{tabular}{l r l r}
\toprule
\multicolumn{2}{l}{aPu} & \multicolumn{2}{l}{aPı} \\
\midrule
tafuz & 21              & tafız & 11 \\
pamuz & 17              & pamız & 15 \\
tapuz & 17              & tapız & 15 \\
mavuz & 16              & mavız & 16 \\
tabuz & 16              & tabız & 16 \\
\bottomrule
\end{tabular}

($\gamma = 0.087$, $p = 0.101$)

%    \section{Conclusion} %\section{Conclusions}

\citet{Becker2011}

This has an ad hoc nature to it; 
There are two senses in which this objection is ad hoc. 
First, \citeauthor{Becker2011} appeal to no particular theory of the naturalness of processes or static constraints which excludes \textsc{Labial Attraction}. 
Secondly, this appears to be a minority view: \textsc{Labial Attraction} was considered a true generalization by early specialists
\citep[e.g.,][]{Lees1966a}, and despite \citeauthor{Zimmer1969}'s suggestive psycholinguistic results, reviewed above, it also been treated as a plausible constraint by later theorists \citep[e.g.,][]{NiChiosain1993,Ito1993,Ito1995a,Zuraw2000}.
Further, there is a real danger that if the label ``unnatural'' 
%Labial Attraction} 
describes an impossible structural change or structural description, that one will fail to account for earlier forms of Turkish or sound changes therein.

%Classical Arabic adjectives often have stative verbs in which the root is imposed onto the template CaCuCa:

%\ex Arabic verbs of coming into being: \\
%\begin{tabular}{r l l l}
%a. & kabura & `become big'       & (cf. \emph{kabiːr} `big') \\
%b. & saʁiir & `become small'     & (cf. \emph{saʁiːr} `small') \\
%c. & ħasuna & `become beautiful' & (cf. \emph{ħasan} `handsome') \\
%\end{tabular}
%\xe 


% FORMER 3.3.2: Diachronic factors 
 
% etymological issues
\citet{Inkelas2001}

\begin{example}
\textsc{Labial harmony} and etymology in TELL \citep[][187]{Inkelas2001}: 

\vspace{0.5\baselineskip}
\begin{tabular}{l r r r}
        & aPu & aPı & $p$-value \\
Native  & 12  & 11  & \multirow{2}{*}{0.042} \\
Foreign & 84  & 28  \\
\end{tabular}
\end{example}

\citet{NiChiosain1993} and \citet{Ito1995b} 



``sonority projection''
Ther

\citet{Daland2011b} finds that this can be learned easily from positive data but a number of different psych

%        \subsection{Naturalness} %% 3.3.1: Naturalness

\citet{Becker2011}

This has an ad hoc nature to it; 
There are two senses in which this objection is ad hoc. 
First, \citeauthor{Becker2011} appeal to no particular theory of the naturalness of processes or static constraints which excludes \textsc{Labial Attraction}. 
Secondly, this appears to be a minority view: \textsc{Labial Attraction} was considered a true generalization by early specialists
\citep[e.g.,][]{Lees1966a}, and despite \citeauthor{Zimmer1969}'s suggestive psycholinguistic results, reviewed above, it also been treated as a plausible constraint by later theorists \citep[e.g.,][]{NiChiosain1993,Ito1993,Ito1995a,Zuraw2000}.
Further, there is a real danger that if the label ``unnatural'' Labial Attraction} describes an impossible structural change or structural description, that one will fail to account for earlier forms of Turkish or sound changes therein.

%        \subsection{Diachronic factors} %% 3.3.2: Diachronic factors 
 
% etymological issues
\citet{Inkelas2001}

\ex \textsc{Labial harmony} and etymology in TELL \citep[][187]{Inkelas2001}: \vspace{6pt} \\
\begin{tabular}{l r r r}
        & aPu & aPı & $p$-value \\
Native  & 12  & 11  & \multirow{2}{*}{0.042} \\
Foreign & 84  & 28  \\
\end{tabular}
\xe 

\citet{NiChiosain1993} and \citet{Ito1995b} 

%\chapter{Gradient and categorical aspects of wordlikeness judgements} %\chapter{Structural and accidental gaps} 
\label{clusters}

Early work in generative phonology recognized that neutralizing, surface-true phonological alternations impose constraints on surface forms and underlying representations (\citealp[283]{Anderson1974}, \citealp[382]{SPE}, \citealp[205f.]{Dell1973}, \citealp[22f.]{SPR}, \citealp{Kenstowicz1977}: chapter 3, \citealp[28f.]{Stampe1973}, \citealp[410f.]{Stanley1967}). Despite isolated claims to the contrary (e.g., \citealp[297]{Hale1965}; \citealp[212f.]{Postal1968}), however, the \emph{communis opinio} (including many of the authors cited above) holds that not all phonotactics---that is, language-specific sound sequence co-occurrence restrictions---can be derived in this fashion, and phonotactic knowledge is independent of the phonology.

\begin{quote}
It would be a result of the greatest interest if it were to turn out that every intramorphemic regularity was necessarily reflected in an alternation, and thus in a rule; but this position cannot seriously be maintained. \citep[283]{Anderson1974}
\end{quote}

English word-medial consonant clusters have been used, by \citet{Pierrehumbert1994} and others, to argue for the necessity of an independent phonotactics module and illustrate proposals for its architecture. This chapter argues, however, that the only structural constraints on the concatenation of codas and onsets to form word-medial clusters are derivative of well-documented phonological alternations in English.

\section{Background}

English admits a variety of word-medial consonant clusters spanning syllable boundaries, known as \emph{syllable contact clusters} or \emph{interludes}. These clusters can be as short as two consonants (e.g., \emph{a}[n.t]\emph{ics}; singleton intervocalic consonants are ignored here) or as long as four (e.g., \emph{mi}[n.str]\emph{el}). \citeauthor{Pierrehumbert1994} presents the following as a null hypothesis for which clusters are admissible in any given language \citep[cf.][]{Haugen1956}.

\begin{quote}
\ldots{}in the absence of additional provisos, any concatenation of a well-formed coda and a well-formed onset is predicted to be possible medially in a word. \citep[][168]{Pierrehumbert1994}
\end{quote}

This chapter is concerned with the nature of these ``provisos'' as reflected by which of the ``possible'' clusters, so defined, are attested. While many phonotactic generalizations can be thought of as inventory restrictions imposed by prosodic constituents, additional filters on the set of possible clusters are essentially combinatoric in nature and thus distinct, there is no prosodic constituent consisting of the coda of one consonant-final syllable and the onset of a following consonant-initial syllable \citep[though see][]{Steriade1999}.

\section{Evaluation}

\citeauthor{Pierrehumbert1994} argues that the inventory of clusters in English shows the effect of a small number of static phonotactic constraints. The remainder of this chapter provides an exhaustive quantitative evaluation of this hypothesis. The most important constraints on medial clusters are those derived from phonological rules which apply across syllable boundaries; there is no remaining role for static constraints. The sparse nature of the lexicon plays a secondary role, giving rise to apparent phonotactic gaps even in the absence of static phonotactic constraints.

\subsection{Corpus construction}

A list of medial clusters is gathered by applying a number of filters and transformations to an English lexical database. The resulting list of clusters is reproduced in Appendix \ref{appendixB} and in electronic form at the author's website.

\subsubsection{Simplex words}

An analysis of lexical constraints requires a corpus of lexical representations. So as to remain agnostic about particulars of morphological theory, the goal here is to identify words which bear no overt inflection, and which do not admit any plausible further decomposition, henceforth ``simplex words''. For the structuralists, a procedure for parsing words into morphs, just the thing for locating simplex words, held the status of ``philosopher's stone'', a sort of near-mythical objective of the endeavor, but actual attempts to develop such a procedure \citep[e.g.,][]{Harris1955,Nida1948} are now recognized as anything but theory-neutral. Yet, there is no alternative to adopting some computational procedure or human-coded database as the ``gold standard'' for a study of this type. \citet{Pierrehumbert1994} analyzes a list of entries in the Collins English dictionary by hand, using her intuitions to filter out ``complex'' words as well as those which are not ``reasonably familiar''. Unfortunately, the resulting list was neither published nor circulated, and the possibility of replicating \citeauthor{Pierrehumbert1994}'s' sensations of morphological complexity is remote. Following prior studies of syllable contact clusters by \citet[ chapter 3]{Hammond1999a} and \citet[ chapter 8]{Duanmu2009}, a list of simplex words was derived from the ``lemmas'' (that is, uninflected forms) in the English portion of CELEX \citep{CELEX}, a database of morphological, phonological, and syntactic annotations based on the COBUILD corpus, a resource that was also used to construct the Collins English dictionary employed by \citeauthor{Pierrehumbert1994}.

Two further filters were applied to this list. First, any lemma coded as an unassimilated loanword is excluded. Secondly, all lemmas that have a ``morphological status'' other than ``monomorpheme'' are excluded. This is crucial because CELEX makes no guarantee that lemmas are simplex; indeed, many lemmas are products of derivation (e.g., \emph{abusive}, from \emph{abuse}). These stringent criteria have a profound effect on the makeup of the data. For instance, nearly all exceptions to \textsc{Obstruent Voice Assimilation} (see \S\ref{ova} below) cited by \citet[74]{Hammond1999a} are excluded as unassimilated loanwords (e.g., \emph{vodka}, \emph{smorgasbord}) or as morphologically complex (e.g., \emph{jurisdiction}, \emph{madcap}, \emph{tadpole}, \emph{scapegoat}, \emph{magpie}). 

``Neo-classical compounds'', words that appear to consist of a Latinate prefix and a bound stem, like \emph{inspect} or \emph{excrete}, have an ambiguous and long-debated morphological status. \citeauthor{Pierrehumbert1994} attempts to distinguish between words like \emph{complete}, \emph{extreme}, \emph{inspect}, \emph{obtuse}, which she assumes to be simplex, and \emph{excrete} and \emph{transparent}, which she considers complex. However, CELEX codes these words as complex and they are thus excluded. This is consistent with converging evidence that speakers decompose these Latinate forms. The baroque treatment of Latinate forms in \emph{The sound pattern of English} (\citealp{SPE}, henceforth \emph{SPE}) assumes a prefix/bound stem decomposition to simplify various morphophonological generalizations.  In a similar vein, \citet[11--13]{Aronoff1976} observes that Latinate forms which share the same bound stem also share irregular allomorphs of that stem under derivation, which is presented as evidence of prefix/bound stem decomposition.

%\footnote{Similar behaviors are found in Russian verbs which share roots but which take different prefixes or the reflexive \citep{Pesetsky1977}.}

\begin{example}[Bound stem-specific allomorphy]
\begin{tabular}{l l l l l l}
a. & {adhere}   & {adhesion}   \\
   & {cohere}   & {cohesion}   \\
b. & {conceive} & {conception} \\
   & {perceive} & {perception} \\
\end{tabular}
\end{example}

There is also a syntactic interaction between Latinate prefixes on verbs and the makeup of the verbal complement. \citet{Harley2009} observes that Latinate verbs do not generally participate in ditransitive, verb participle, or adjectival resultative constructions, all acceptable with similar Anglo-Saxon verbs \citep[see also][]{Gropen1989,Coppock2008}. \citeauthor{Harley2009} proposes that the Latinate prefix is introduced in a low small clause, the same position as the theme, participle, and result adjectival in the Anglo-Saxon examples.

\begin{example}[Latinate verbs and small clauses] \label{harley}
\begin{tabular}{ll@{}ll@{}l}
a. & * & {exhibit him the painting} & ~ & {show him the painting} \\
b. & * & {imbibe himself stupid}    & ~ & {drink himself stupid}  \\
c. & * & {exhibit it off}           & ~ & {show it off}           \\
\end{tabular}
\end{example}

Decomposition is also supported by lexical decision results. Among nonce words, those that can be exhaustively decomposed into prefix and bound stem, like *\emph{de}-\emph{juvenate}, are processed slower that contain a licit morph but cannot be exhaustively decomposed, like *\emph{de-pertoire}, and these in turn are processed slower than those which admit no such decomposition \citep{Taft1975}.\footnote{The same contrasts are present in many other domains where decomposition into root and affix(es) is less controversial, for instance, inflected verbs in Italian \citep{Caramazza1988}.} \citet{Emmorey1989} reports facilitative auditory priming between pairs of words which share a bound stem, like \emph{permit}-\emph{submit} \citep[though see][]{Marslen-Wilson1994}.\footnote{It may be the case that full decomposition is only one of the two methods for the processing of complex words, the other being whole word looking, either running in serial \citep{Caramazza1988} or in parallel \citep{Baayen1997b}. Whatever evidence that is amassed in support of a whole word lookup procedure, however, does not undermine the considerable evidence for decomposition.}

\subsubsection{Syllabification and phonologization}

CELEX includes broad, syllabified Received Pronunciation transcriptions of individual wordforms. However, a casual inspection of the data reveals the unsystematic nature of these syllabifications. Any two words sharing a nucleus-medial cluster-nucleus sequence should be given the same parse, since syllabification is universally non-contrastive. Yet many putative syllabification contrasts can be found in CELEX; for instance, compare the parses given to the clusters in \emph{chemistry} [ˈkɛ.mɪ.stɹɪ] and \emph{ministry} [ˈmɪ.nɪs.tɹɪ].\footnote{Note that word-final \emph{y} is generally lax [ɪ] in Received Pronunciation \citep[][II.294]{AOE}.}

A systematic syllabification requires an automated procedure for delimiting medial clusters and parsing them into coda and onset. Since this procedure is applied only to medial clusters, this study remains agnostic on several contentious issues concerning syllabification of English; for instance, there is no need to address the status of so-called ``ambisyllabic'' consonants, since this does not occur with medial clusters, or morphological effects on syllabification, since no complex words are included. The technique used here is a variation on the theme of onset maximization (e.g., \citealp[42f.]{Kahn1976}, \citealp{Kurylowicz1948}, \citealp[75]{Pulgram1970}, \citealp[][358f.]{Selkirk1982b}) which favors parses of word-medial clusters in which as much of the cluster as possible is assigned to the onset. Initial onsets are used to define what is a ``possible'' medial onset \citep[though cf.][36]{Fischer-Jorgensen1952}. Medial clusters in words like \emph{neu}[.tɹ]\emph{on} or \emph{bi}[.stɹ]\emph{o} are found in word-initial position (e.g., [tɹ]\emph{ansit}, [stɹ]\emph{ike}), so onset maximization assigns the entire medial cluster to the onset, leaving the medial coda empty. In contrast, the cluster in \emph{mi}[n.stɹ]\emph{el} is not found word-initially; the maximal onset here is [stɹ], and the residual [n] is assigned to the coda.

%\citet[208]{Wetzels2001}, cite English as an example of a language without ``general devoicing or assimilatory effects''
%\citet[?]{Lombardi1999} says that English lacks devoicing in Level II.

\paragraph{Stressed lax vowels} When a medial consonant cluster is preceded by a stressed lax vowel, as in words like \emph{wh}[ɪs.p]\emph{er}, \emph{v}[ɛs.t]\emph{ige}, or \emph{m}[ʌs.k]\emph{et}, the first consonant of the cluster checks the lax vowel \citep[e.g.,][3]{Hammond1997}. \citet[][55]{Harris1994} notes that onset maximization produces an incorrect result when, in addition, the medial cluster is a valid onset: in \emph{whisper}, \emph{vestige}, and \emph{musket}, onset maximization incorrectly assigns [sp, st, sk] to the medial onset, leaving the lax vowel unchecked. Consequently, the first consonant of a medial cluster is assigned to the coda of a preceding syllable before a stressed lax vowel \citep[cf.][48]{Pulgram1970}, a minimal modification of the onset maximization parse.

\paragraph{Affricates} If the ffricates [tʃ, dʒ] are treated as stop-fricative sequences \citep[e.g.,][]{Hualde1988,Lombardi1990}, the previous principle will incorrectly assign the two segments of the affricate to separate syllables before stressed lax vowels (e.g., *\emph{ra}[t.ʃ]\emph{et} or \emph{a}[d.ʒ]\emph{ile}) unless some ad hoc constraint \citep[e.g.,][]{Wells1990} is in place. A more traditional analysis of affricates, in which they are simple segments distinguished from pure stops by stridency \citep[24]{Jakobson1961} or delayed release (\emph{SPE}:321f.) is assumed here, following \citeauthor{Pierrehumbert1994}. This assumption can be motivated by the tendency of affricates to pattern with individual segments in phonotactic generalizations. For instance, Classical Nahua allows the affricate series [ts, tʃ, tɬ] in onsets, but bans true onset clusters \citep[9]{Launey2011}.
%\citet[86]{Butskhrikidze2002} notes a similar generalization in the consonant clusters of Georgian.

\paragraph{The velar nasal} According to some \citep[e.g.,][]{Sapir1925}, [ŋ] in English is a phoneme in its own right, but others (e.g., \emph{SPE}:85, \citealp{Borowsky1986}:65f.) hold that it is simply the allophone of /n/ before underlying /k, ɡ/ (with later deletion of /ɡ/ in some contexts). The total absence of onset [ŋ], where it cannot be followed by a dorsal consonant which conditions the velar allophone, is strong evidence that [ŋ] is a pure allophone of /n/, the analysis assumed here.

\paragraph{[j] onglides} While some have argued that the front onglide in words such as \emph{val}[j]\emph{ue} is inserted by rule (\emph{SPE}:196, \citealp[][89]{Halle1985a}, \citealp[][217]{McMahon1990}), the very presence or absence of the glide is contrastive (e.g., \emph{coot} \alt{} \emph{cute}, \emph{booty} \alt{} \emph{beauty}) indicating that it is present in underlying representation (\citealp{Anderson1988b}, \citealp[278]{Borowsky1986}). The front onglide is further assumed to be assigned to the nucleus, except when the onset would otherwise be null (e.g., \emph{ju}[n.j]\emph{or}). There is considerable evidence for this assumption. When [j] is a simplex onset, it may be followed by any vowel \citep[][276]{Borowsky1986}, but when [j] is immediately preceded by an onset consonant (e.g., [bj]\emph{ugle}), the following vowel is always [u], suggesting that the glide is nuclear (\citealp{Davis1995}, \citealp[][61f.]{Harris1994}, \citealp[][232]{Hayes1980}). \citet[][42]{Clements1983} note that /m, v/ do not appear in onset clusters, though they may be followed by [ju] in words like [mj]\emph{use} or [vj]\emph{iew}, also suggesting the nuclear affiliation of the glide. There is also external support for this analysis: the [ju] in words like \emph{spew} may pattern together in Pig Latin \citep{Davis1995,Idsardi2005}, \emph{shm}-reduplication \citep{Nevins2003}, and speech errors (e.g., [kju]\emph{mor} [h]\emph{omponent}, intended [hju]\emph{mor component}; \citealp[130]{Shattuck-Hufnagel1986}).

\paragraph{[w] onglides} The phonotactic properties of the back onglide [w] are the opposite of the front onglide. Whereas the front onglide shows only limited selectivity for preceding tautosyllabic consonants \citep{Davis1995,Kaye1996}, the back onglide [w] is rarely preceded by tautosyllabic consonants other than [k] (e.g., \emph{tran}[kw]\emph{il}). Unlike the front glide, syllable-initial [kw] may be followed by nearly any vowel \citep[161]{Davis1995}. Unlike [juː], onglide [w] followed by a vowel does not pattern together in Pig Latin \citep[166]{Davis1995}. Taken together, these facts indicate that the back onglide is assigned to the onset.

\paragraph{Syllabic \emph{r}} Word-medial syllabic \emph{r} has been lost in Received Pronunciation, the accent used for the CELEX transcriptions. Even in \emph{r}-ful dialects, though, there is reason to believe that syllabic \emph{r} is fact nuclear, and thus irrelevant to syllable contact clusters. Many vowel contrasts are suspended before \emph{r} (e.g., \citealt[269f.]{Fudge1969}, \citealt[][255]{Harris1994}): compare American English \emph{fern}, \emph{fir}, \emph{fur} to \emph{pet}, \emph{pit}, \emph{putt}. Syllabic \emph{r} patterns with vowels and not other consonants for several phonological processes: it is the only consonant which does not block variable glottalization of following /t/ in British dialects \citep[258]{Harris1994}. 

\begin{example}[/t/-\textsc{Glottalization} in British English]
\begin{tabular}{l l l l@{} l l l}
a. & {des}[ɚt]    & \alt{} &   & {des}[ɚʔ]    \\
   & {c}[ɚt]{ain} & \alt{} &   & {c}[ɚʔ]{ain} \\
b. & {fi}[st]     & \alt{} & * & {fi}[sʔ]     \\
   & {mi}[st]{er} & \alt{} & * & {mi}[sʔ]{er} \\
\end{tabular}
\end{example}

\noindent Syllabic \emph{r} is the only consonant which does not trigger variable deletion of a following /t, d/ in American dialects \citep[8]{Guy1980}.

\begin{example}[/t, d/-\textsc{Deletion} in American English]
\begin{tabular}{l l l l@{} l l l}
a. & {be}[lt] & \alt{} &   & {be}[l] \\
   & {me}[nd] & \alt{} &   & {me}[n] \\
b. & {sh}[ɚt] & \alt{} & * & {sh}[ɚ] \\
   & {c}[ɚd]  & \alt{} & * & {c}[ɚ]  \\
\end{tabular}
\end{example}

\subsubsection{Lexical statistics}

CELEX contains 6,876 simplex words of English, comprising 21 unique medial codas and 40 unique medial onsets. Of the 840 ($= 21 \times 40$) possible syllable contact clusters that could be produced by free combinations of the attested medials codas and onsets, 158 are attested.

\subsection{Evaluating derived constraints}

\emph{SPE} describes three phonological alternations which target syllable contact clusters. The corresponding constraints on simplex syllable contact clusters are considered below. The first such derived constraint, obstruent voice assimilation, is presented as a tutorial introduction. The second set of constraints consists of three static phonotactic constraints proposed by \citet{Pierrehumbert1994}.

\subsubsection{Obstruent voice assimilation} \label{ova}

The [t \alt{} d] allomorphs of the regular past (e.g. \emph{nap}[t] $\sim$ \emph{nab}[d]), and the [s \alt{} z] allomorphs of the regular noun plural (e.g., \emph{lap}[s] $\sim$ \emph{lab}[z]), Saxon genitive, and third person singular active indicative verb agreement, exemplify \textsc{Obstruent Voice Assimilation} (\emph{SPE}:178).\footnote{Underlying /-d, -z/ are argued for by \citet[282]{Hockett1958}, \citet[210]{SPE}, \citet{Basboll1972}, \citet{Shibatani1972}, \citet{Anderson1973a}, \citet[102]{Pinker1988}, and \citet[284f.]{Bakovic2005b}; alternatives are suggested by \citet[210f.]{LANGUAGE}, \citet[426]{Nida1948}, \citet{Luelsdorff1969}, \citet{Lightner1970}, \citet{Hoard1971}, \citet{Miner1975}, \citet{Zwicky1975}, \citet{Kiparsky1985}, and \citet[135]{Borowsky1986}.}

\begin{example}[Inflectional affix voice assimilation]
\begin{tabular}{l l l l l l}
a. & {nap} & {nap}[t] & & {nab} & {nab}[d] \\
b. & {lap} & {lap}[s] & & {lab} & {lab}[z] \\
\end{tabular}
\end{example}

\noindent The devoicing of the two voiced obstruent suffixes can be formalized as a 
rule spreading the voicing specification of an obstruent rightward.

\begin{example}[\textsc{Obstruent Voice Assimilation}]
%\xymatrix@R=24pt@C=24pt{
%\txt{[α \textsc{Voice}]}\ar@{-}[d]\ar@{--}[dr] &                               \\
%\txt{C}\ar@{-}[d]                              & \txt{C}\ar@{-}[d]             \\
%\txt{[$-$\textsc{Sonorant}]}                  & \txt{[$-$\textsc{Sonorant}]} \\
%}
$\begin{bmatrix} -\textsc{Son} \end{bmatrix}~\goesto~\begin{bmatrix} =\textsc{Voi} \end{bmatrix}~/~\gap~\begin{bmatrix} =\textsc{Voi} \\ -\textsc{Son} \end{bmatrix}$
\end{example}

It is quite apparent that root-internal hetero-voiced obstruent clusters like [s.b] or [z.t] are also rare compared to clusters which are uniformly voiced or uniformly voiceless, as predicted the assimilation rule. Yet, \citet{Pierrehumbert1994} does not mention voice assimilation in her study of English syllable contact clusters. \citet[][74f.]{Hammond1999a} cites words like \emph{a}[b.s]\emph{inth} and \emph{a}[s.b]\emph{estos}, which contain hetero-voiced obstruent clusters, as evidence that the process does not apply root-internally, though few of his examples are in fact simplex according to CELEX.\footnote{The \textsc{Revised Alternation Condition} (RAC) proposed by \citeauthor{Kiparsky1973a} (\citeyear{Kiparsky1973a}:163, \citeyear{Kiparsky1982a}:152) blocks the application of obligatory neutralization processes like \textsc{Obstruent Voice Assimilation} in root-internal (i.e., non-derived) environments. Simply because there are exceptions in non-derived environments, \textsc{Obstruent Voice Assimilation} is always consistent with the RAC whether or not it actually applies in non-derived environments. This is due to the ``obligatory'' condition of the RAC. If the rule applies in non-derived environments, then it has lexical exceptions (e.g., \emph{a}[s.b]\emph{estos}) and is not obligatory and thus free from the RAC. On the other hand, if the rule is subject to the RAC, it only applies in derived environments, in which it is obligatory.}

The question is: does \textsc{Obstruent Voice Assimilation} contribute a meaningful characterization the English lexicon? More specifically, are hetero-voiced clusters underrepresented in a way that is unlikely under the hypothesis of free combination? To answer this question, the 720 clusters in which both the final coda consonant and initial onset consonant are obstruent are sorted into four bins, according to whether they are attested or not, and whether or not the conform to the derived constraint---both obstruents are either voiced or voiceless, like in \emph{hu}[z.b]\emph{and} or \emph{rha}[p.s]\emph{osdy}, respectively---or violate it, like the aforementioned \emph{a}[b.s]\emph{inth} and \emph{a}[s.b]\emph{estos}. 

This table, shown in (\ref{ovatable}), is then submitted to the Fisher Exact Test. This statistical test is so named because it is isomorphic to the chi-square test 
for contingency tables, but computes an exact $p$-value, whereas the chi-square test $p$-value depends on an approximation which is inappropriate for small samples. The small $p$-value produced by the Fisher test indicates that the rarity of the disagreeing clusters is unlikely to be due to chance. The simplest interpretation is that \textsc{Obstruent Voice Assimilation} applies in non-derived environments, preventing learners from positing underlying representations in which adjacent obstruents do not agree in voice.

\begin{example}[Lexical effects of \textsc{Obstruent Voice Assimilation}] \label{ovatable}
\begin{tabular}{l r r r r}
\toprule
           & attested & unattested & \% saturation & $p$-value                   \\
\midrule
conforming & 80       & 370        & 17.8          & \multirow{2}{*}{1.1\e{-11}} \\
violating  &  6       & 264        & 2.2           \\
\bottomrule
\end{tabular}
\end{example}

There are two possible analyses of the small number of roots that violate the derived generalization. Some hetero-voiced obstruent clusters, like the one in \emph{ja}[k.d]\emph{aw}, for instance, might be analyzed as compounds by native speakers, and this analysis might itself be driven by the rarity of morph-internal adjacent hetero-voiced obstruents (cf. \citealp[546]{Rice2009d} on compounds in Slave). In fact, \citet{Mattys2001b} find that 9-month-old infants are sensitive to the contrast between [f.t] and exceptional (and unattested) [v.t], treating the latter as an indication of a word boundary and this heuristic is also available to adults \citep{Brown1956,McQueen1998b}. Less explanatorily, exceptional roots can be marked [$-$\textsc{Obstruent Voice Assimilation}].

\subsubsection{Nasal place assimilation} 
\label{npa}

In addition to [n $\sim$ ŋ] allophony, \textsc{Nasal Place Assimilation} (\emph{SPE}:85) is responsible for the allomorphs of the \emph{im-}/\emph{in-} prefix, determined by the place of the following consonant.

\begin{example}[\emph{im-}/\emph{in-} allomorphy] \label{nparule}
\begin{tabular}{l l l l l l l}
a. & {polite}   & {i}[m.p]{olite}   & & {balance} & {i}[m.b]{alance} \\
b. & {tangible} & {i}[n.t]{angible} & & {decent}  & {i}[n.d]{ecent}  \\
\end{tabular}
\end{example}

% mention Jensen here

\begin{example}[\textsc{Nasal Place Assimilation}]
%\xymatrix@R=24pt@C=24pt{
%                          & \txt{\textsc{Place}}\ar@{-}[d]\ar@{--}[dl] \\
%\txt{C}\ar@{-}[d]         & \txt{C}\ar@{-}[d]                          \\
%\txt{[$+$\textsc{Nasal}]} & \txt{[$-$\textsc{Sonorant}]}              \\
%}
$\begin{bmatrix} $+$\textsc{Nas} \end{bmatrix}~\goesto~\begin{bmatrix} =\textsc{Lab} \\ =\textsc{Cor} \\ =\textsc{Dor} \end{bmatrix}~/~\gap{}~\begin{bmatrix} =\textsc{Lab} \\ =\textsc{Cor} \\ =\textsc{Dor} \\ -\textsc{Son} \end{bmatrix}$
\end{example}

The shape of the prefix before the velar stops /k, g/ is somewhat more complex. \citet[][62]{Halle1985a} and \citet[][90]{Borowsky1986} report that coda nasals assimilate to [ŋ] before dorsal consonants, but that assimilation of \emph{im-}/\emph{in-} is blocked by stress on the following syllable, citing contrasts like the one between \emph{í}[ŋ.k]\emph{ubate} and \emph{i}[n.k]\emph{lúde}. However, the stress condition does not hold for simplex words (e.g., \emph{a}[ŋ.ɡ]\emph{óra}), and is ignored here.

There is evidence that even young infants acquiring English are aware of this generalization. \citet{Davidson2004}, \citet{Mattys1999}, and \citet{Jusczyk2002} find that infants at 4.5 months, 9 moths, and 10 months of age, respectively, prefer to listen to nonce words which contain the homorganic nasal-obstruent clusters (e.g., \emph{u}[m.b]\emph{o}) over those which contain heterorganic clusters (e.g., \emph{u}[n.b]\emph{o}). Nasal-obstruent clusters arising in speech errors also undergo \textsc{Nasal Place Assimilation} (e.g. \emph{ra}[nd] \emph{orker}, intended \emph{ra}[ŋk] \emph{order}; \citealt[228]{Myers1993}).

\citet[175]{Pierrehumbert1994} observes that ``nasal-stop sequences agree in labiality'', but makes no reference to \textsc{Nasal Place Assimilation}. Exceptions like \emph{pli}[m.s]\emph{oll}, \emph{da}[m.z]\emph{el}, \emph{scri}[m.ʃ]\emph{aw}, and \emph{ra}[m.k]\emph{in} do occur, but they are much rarer than homorganic nasal-obstruent clusters such as \emph{pi}[m.p]\emph{le}, \emph{sta}[n.z]\emph{a}, and \emph{mo}[ŋ.k]\emph{ey}.

\begin{example}[Lexical effects of \textsc{Nasal Place Assimilation}]
\begin{tabular}{l r r r r}
\toprule
           & attested & unattested & \% saturation & $p$-value                   \\
\midrule
conforming & 41       & 11         & 78.8          & \multirow{2}{*}{2.6\e{-05}} \\
violating  & 8        & 20         & 28.6                                    \\
\bottomrule
\end{tabular}
\end{example}

\subsubsection{Degemination} \label{deg}

A final alternation targeting English syllable contact clusters is the simplification of geminates derived by certain affixes. For instance, the deadjectival adverbial suffix \emph{-ly} rarely produces a geminate when it attaches to /l/-final roots (e.g., \emph{fu}[l]\emph{y}, cf. \emph{free}[l]\emph{y}). Degemination is also found in /d/-final roots selecting the irregular /-t/ past tense suffixes (e.g., \emph{bend}/\emph{ben}[t], \emph{build}/\emph{buil}[t]). More complicated cases of \textsc{Degemination} can be found in Latinate prefix allomorphy (see \emph{SPE}:148 and \citealt[102]{Borowsky1986}). Identical adjacent consonants do not occur in the corpus, nor do obstruent clusters which differ only in voice (e.g., *[p.b]), which would be made identical by the operation of \textsc{Obstruent Voice Assimilation}. As shown by the results of the Fisher test, this absence is unlikely to be due to chance.

\begin{example}[\textsc{Degemination}]
$\begin{bmatrix} =\textsc{Lab} \\ =\textsc{Cor} \\ =\textsc{Dor} \\ =\textsc{Son} \\ =\textsc{Nas} \\ =\textsc{Cont} \end{bmatrix}~\goesto~\zero~/~\gap~\begin{bmatrix} =\textsc{Lab} \\ =\textsc{Cor} \\ =\textsc{Dor} \\ =\textsc{Son} \\ =\textsc{Nas} \\ =\textsc{Cont} \end{bmatrix}$
\end{example}

% consider what coda-final consonants there are

\begin{example}[Lexical effects of \textsc{Degemination}]
\begin{tabular}{l r r r r}
\toprule
           & attested & unattested & \% saturation & $p$-value                   \\
\midrule
conforming & 157      & 596        & 20.8      & \multirow{2}{*}{7.2\e{-09}} \\
violating  & 0        &  87        & 0.00                                    \\
\bottomrule
\end{tabular}
\end{example}

\subsubsection{Local summary}

The three \emph{SPE} rules affecting syllable consonant clusters are reliably reflected in the lexicon. Possible clusters that are surface-exceptions to these three rules are much more likely to be unattested than those that conform to them.

\subsection{Evaluating static constraints}

The second set of constraints, proposed by \citet{Pierrehumbert1994}, are ``static'' in that they have no analogue in English morphophonemics. Clusters of all length are considered here, not only the triconsonantal clusters considered by \citeauthor{Pierrehumbert1994}, as this restriction appears to be arbitrary.

\subsubsection{Dorsal/non-coronal consonants}

\citet[173]{Pierrehumbert1994} writes that ``velar obstruents occurred only before coronals in the clusters studied, never before labials or other velars.'' Two-consonant clusters like \emph{a}[k.m]\emph{e}, \emph{ru}[ɡ.b]\emph{y}, or \emph{pi}[ɡ.m]\emph{ent} are found, however. Despite this, dorsal/non-coronal clusters are no less likely to occur than non-dorsal codas with non-coronal onsets (e.g. \emph{oi}[nt.m]\emph{ent}) or dorsal codas with coronal onsets (e.g., \emph{ve}[k.t]\emph{or}).

\begin{example}[Lexical effects of \textsc{*[$+$Dorsal][$-$Coronal]}]
\begin{tabular}{l r r r r}
\toprule
           & attested & unattested & \% saturation & $p$-value \\
\midrule
conforming & 76 & 359 & 17.5 & \multirow{2}{*}{0.145} \\
violating  &  6 &  57 & 9.5 \\
\bottomrule
\end{tabular}
\end{example}

\subsubsection{Coda coronal obstruents}

\citet[175]{Pierrehumbert1994} proposes that ``clusters with a coronal obstruent in the coda do not occur'', while noting exceptions like \emph{a}[nt.l]\emph{er}, \emph{ke}[s.tr]\emph{el} and \emph{oi}[nt.m]\emph{ent}, as well as many clusters such as \emph{a}[t.l]\emph{as} or \emph{e}[s.t]\emph{er}. Once again, coda coronal obstruent clusters are no less likely to occur than non-coronal obstruent clusters like \emph{re}[p.t]\emph{ile} or coda coronal sonorant clusters such as \emph{co}[n.d]\emph{or}.

\begin{example}[Lexical effects of \textsc{*Coda Coronal Obstruent}]
\begin{tabular}{l r r r r}
\toprule
           & attested & unattested & \% saturation & $p$-value              \\
\midrule
conforming & 48       & 312        & 13.3      & \multirow{2}{*}{0.301} \\
violating  & 38       & 322        & 10.6                               \\
\bottomrule
\end{tabular}
\end{example}

\subsubsection{ABA clusters}

Finally, \citet[][176]{Pierrehumbert1994} observes ``a lack of clusters with identical first and third elements''. Here, this is operationalized in the same fashion as \textsc{Degemination}, by definining ``identity'' as consonants which share the same place and manner features, but which do not necessarily agree on voicing. Despite the fact that the corpus contains no exceptions to this generalization, these \textsc{ABA} clusters are not significantly less common than other triconsonantal and quadraconsonantal clusters.

\begin{example}[Lexical effects of \textsc{*ABA}]
\begin{tabular}{l r r r r}
\toprule
           & attested & unattested & \% saturation & $p$-value \\
\midrule
conforming & 41       & 478        & 7.9      & \multirow{2}{*}{0.818} \\
violating  &  0       &  11        & 0.0                               \\
\bottomrule
\end{tabular}
\end{example}

\subsubsection{Local summary}

There is no statistically reliable evidence for the three static constraints proposed by \citeauthor{Pierrehumbert1994}. However, it is premature to conclude that no static constraint exists. It is possible that an explicit model of ``constraint discovery'', such as those proposed by \citet{Pierrehumbert1994} and \citet{Hayes2008a}, could identify reliable static constraints, and predict which of the possible clusters are attested and unattested beyond the derived constraints. 

\subsection{Evaluating computational models}

The computational models of \citeauthor{Pierrehumbert1994} and \citeauthor{Hayes2008a}, as well as two baselines, are scored for their ability to predict which of the possible medial clusters are attested and which are not. For any single cluster, there are four possible outcomes.

\begin{example}[Outcomes in cluster classification task]
\begin{tabular}{c l | l l}
                   & & \multicolumn{2}{c}{\textbf{actual outcome}}            \\
                   & & \emph{attested}   & \emph{unattested}            \\
\midrule
\textbf{predicted} & \emph{attested}   & true positive ($tp$)  & false positive ($fp$) \\
\textbf{outcome}   & \emph{unattested} & false negative ($fn$) & true negative ($tn$)  \\
\end{tabular}
\end{example}

\noindent Accuracy is the probability of a correct classification, whether a true positive or negative.

\begin{unlabeledexample}
$\displaystyle \textrm{Accuracy} = \frac{tp + tn}{tp + tn + fp + fn}$
\end{unlabeledexample}

\noindent Precision is the probability that a cluster predicted to occur is attested.

\begin{unlabeledexample}
$\displaystyle \textrm{Precision} = \frac{tp}{tp + fp}$
\end{unlabeledexample}

\noindent Recall (or sensitivity) is the probability an attested cluster is predicted to occur.

\begin{unlabeledexample}
$\displaystyle \textrm{Recall} = \frac{tp}{tp + fn}$
\end{unlabeledexample}

\noindent It is possible to maximize recall at the expense of precision, by predicting attestation for a greater number of clusters, or to increase precision at the expense of recall by predicting non-attestation for a greater number of clusters. $F_1$, the harmonic mean of these two measures, quantifies this trade-off.

\begin{unlabeledexample}
$\displaystyle F_1 = 2 \left( \frac{\textrm{Precision} \times \textrm{Recall}}{\textrm{Precision} + \textrm{Recall}}\right)$
\end{unlabeledexample}

A soft-margin support vector machine \citep{Cortes1995} with a linear kernel is used to convert from the per-cluster scores to attestation predictions, by finding a single cutoff which optimally separates attested and unattested clusters according to their model scores. This is in no way intended to be a cognitively plausible model for learning phonotactics; it simply represents an upper bound on the performance of these models. The results for all four models are summarized in Table \ref{cmresults}.

\begin{table} \centering
\begin{tabular}{l | r r r r}
\toprule
                          & accuracy & precision & recall & $F_1$ \\
\midrule
Null baseline             & 0.813    & n.a.      & n.a.   & n.a.  \\
Derived constraints       & 0.857    & 0.401     & 0.708  & 0.512 \\
Expected frequency        & 0.839    & 0.210     & 0.750  & 0.328 \\
\citet{Hayes2008a}        & 0.877    & 0.777     & 0.642  & 0.703 \\
\bottomrule
\end{tabular}
\caption{The \citeauthor{Hayes2008a} model makes the most accurate predictions about possible and impossible syllable contact clusters, though it is only slightly better than the derived constraint baseline.}
\label{cmresults}
\end{table}

\subsubsection{Null baseline}

18.8\% of the 840 possible clusters are attested. Consequently, a null baseline which predicts no clusters to be attested trivially achieves 81.2\% accuracy. The other metrics are undefined for this baseline.

\subsubsection{Derived constraints}

The three derived constraints define a structured baseline, in which attestation is the predicted outcome for all clusters which not violate the \emph{SPE}-derived generalizations, and non-attestation otherwise. This produces a significant improvement to the null baseline (accuracy = 0.857; sign test $p = 2.16$\e{-05}). The high recall (0.708) but low precision (0.401) of this model indicates that there are more false positives (i.e., unattested clusters which do not violate any constraint) than false negatives (i.e., exceptions to the three rules). It is not clear that low precision is a flaw, since some of these false positives might be accidental gaps---that is, clusters that are well-formed, but unattested.

\subsubsection{Expected frequency}

\citet{Pierrehumbert1994} proposes a simple model in which the wellformedness of syllable contact clusters is proportional to the joint probability, estimated from word-final coda and word-initial onset frequencies and assuming that coda and onset are statistically independent. There are two problems with this procedure, however. First, there exist languages, such as Finnish \citep[36]{Fischer-Jorgensen1952}, in which there are medial codas not found in final position; this alone suggests that ``outer'' frequencies are not the best model of what is possible medially. More to the point, a pilot study determined that using medial coda and onset frequencies produced a better fit, and this variant is adopted here. 
%\citeauthor{Pierrehumbert1994}'s procedure would assign clusters of this type the lowest score. 
While \citeauthor{Pierrehumbert1994} reports that this is the best predictor of which complex clusters occur, it imposes no constraints on sequences spanning the syllable boundary, and consequently has lower accuracy (0.839) and $F_1$ (0.328) than the derived constraint baseline, though it is a small but significant improvement over the null baseline (sign test $p = 1.23$\e{-04}).

\subsubsection{\citealt{Hayes2008a}}

\citet{Hayes2008a} present a model which uses the principle of maximum entropy to weigh a large number of competing phonotactic constraints. Since the model has many experimenter-determined parameters, a direct replication of the experiments reported by \citeauthor{Hayes2008a} is attempted; their software, feature specification, and model settings are all used. Since the training of this model is inherently stochastic, producing slightly different outcomes on each run, the results reported here from the lowest accuracy of ten independent runs, a practice also used by \citeauthor{Hayes2008a}. The model achieves the highest accuracy (0.877) and $F_1$ (0.703), but lower recall (0.642) than the alternation baseline, a significant improvement over the null baseline (sign test $p = 2.9$\e{-05}), but not significantly different from the performance of the alternation baseline (sign test $p = 0.070$). 

The generalizations extracted by the \citeauthor{Hayes2008a} model are in general quite similar to those derived from the \emph{SPE} alternations, but there are some notable differences. Unattested (and ill-formed) *[m.kl] is assigned the highest score, though it is ruled out by \textsc{Coda Nasal Place Assimilation}. On the other hand, very low scores are assigned to the clusters in \emph{hu}[z.b]\emph{and} and \emph{pla}[t.f]\emph{orm}, though they are both attested and consistent with all three alternation-derived constraints. 

\subsubsection{\citealt{McGowan2009}}

\citet{McGowan2009} claims that the type frequency of individual syllable contact clusters in English is predicted by the change of sonority in the clusters.\footnote{Thanks to Maria Gouskova for bringing this study to my attention.} Using a sonority scale proposed by \citet{Jespersen1904}, \citeauthor{McGowan2009} reports that clusters like [m.p], with sharply falling sonority, are slightly more frequent than those like [t.j], with rising sonority. However, \citeauthor{McGowan2009} finds that sonority of clusters accounts for a very small of the variance in cluster frequency ($R^2 = 0.077$). As a predictor of cluster attestation, sonority distance provided no improvement over the null baseline; consequently, it is not included in Table \ref{cmresults}.

\subsubsection{Local summary}

Compared to phonological alternations, current computational models of phonotactic learning are not significantly better predictors of which syllable contact clusters a

\section{Conclusions}

None of the foregoing results have revealed any evidence for structural constraints on the English syllable contact cluster, whether ``discovered'' by linguist or computer, except those derived from phonological alternations. What then is to be said about the countless clusters which do not violate any phonologically derived constraint, but yet are unattested, like [b.z] or [z.n]? Since these gaps appear to be quite arbitrary from a phonological perspective, all that can be said at this juncture is that these clusters must be regarded \emph{accidental gaps}: they are well-formed clusters, but simply missing from the English lexicon due to the sparsity thereof, a possibility that has long been recognized.

\begin{quote}
\ldots{}the fact that some [clusters--KG] are not found must be due to accidental gaps in the inventory of signs, and cannot be explained by structural laws of the language. \citep[][16]{Fischer-Jorgensen1952}

The material will never be complete. It will always contain accidental gaps \ldots partly because some clusters by pure chance do not occur in the vocabulary. \citep[][30]{Vogt1954}
\end{quote}

\noindent Below, attempts are made to quantify the extent of accidental gaps in this domain below.

\subsection{Zipfian distribution}

A number of previous studies \citep[e.g.,][]{Weiss1961,Sigurd1968,Good1969,Borodovsky1989,Witten1990,Martindale1996,Tambovtsev2007} observe that phoneme or letter frequencies exhibit sparse distributions, in which a few types account for a large amount of the probability mass, and the remaining is divided up among a large number of rare types. The same is true for the type frequencies of the medial codas and onsets which make up syllable contact clusters, as well as cluster frequencies themselves. In Figure \ref{clus}, cluster frequencies are plotted against rank in the log-log space, and a near-linear relationship obtains ($R^2 = 0.924$), indicating that the data conform to a generalization of Zipf's (\citeyear{Zipf1949}) Law. See Appendix \ref{zr} for estimation details.

\begin{figure} \centering
\includegraphics{cluster.pdf}
\caption{Syllable contact clusters exhibit a log-log linear relationship between frequency and rank that is consistent with Zipf's Law.}
\label{clus}
\end{figure}

Sparse distributions characterize the frequencies of many linguistic objects, from syntactic rules \citep{Yang2011b} to word \citep{Baroni2009} and phoneme \citep{Belevitch1956,Daland2011a} $n$-grams, but are also found in non-linguistic symbol systems \citep{Mandelbrot1954,Miller1957,Chomsky1958,Sproat2010} and randomly generated texts \citep{Miller1957,Li1992}. The importance of sparsity in this context is that it entails a long right tail, making it difficult to determine on statistical grounds alone which unobserved events are impossible and which are accidental gaps.

\subsection{Good-Turing estimate}

\citet{Good1953} proposes an estimate for the probability of all accidental gaps, $\hat{p}_0$, as the number of events that occur just once, $n_1$ divided by the number of observations.

\begin{unlabeledexample}
$\displaystyle \hat{p}_0 = \frac{n_1}{\displaystyle\sum N}$
\end{unlabeledexample}

\noindent In the CELEX data, $65$ clusters occur only once, and there are 873 cluster tokens, and thus $\hat{p}_0 = 0.074$. There is thus a non-trivial probability that many of the ``false positive'' clusters are in fact accidental gaps.

\subsection{Sampling simulation}

It is finally possible to ask what the lexicon might look like if it was stochastically generated from independent coda and onset frequencies, \emph{a la} \citeauthor{Pierrehumbert1994}, but also constrained by phonological neutralizations. The following method was repeated to generate a simulated lexicon.

\begin{example}[Simulation procedure]
\begin{tabular}{l l}
a. & Sample a medial coda according to the observed probabilities  \\
b. & Sample a medial onset according to the observed probabilities \\
c. & Apply all matching \emph{SPE} rules to cluster formed by their concatenation \\
\end{tabular}
\end{example}

The cluster frequencies for a single characteristic simulation are plotted in Figure \ref{sim}, along with observed frequencies; the two distributions are nearly indistinguishable. The sparse cluster inventory, previously taken as evidence for static constraints on syllable contact, is approximately what one would expect even if the only constraints on possible clusters are phonologically derived.

\begin{figure}
\centering
\includegraphics{sim.pdf}
\caption{The observed cluster frequencies are closely matched by a simulated cluster inventory generated by random sampling, assuming independence and applying phonological neutralizations. The $Z_r$ transform \citep[][29]{Church1991} has been used to eliminate quantization in lower frequency bands.}
\label{sim}
\end{figure}

%\label{wordlikeness}
%    \section{Aspects in the theory of wordlikeness} %\section{Domain}

% WHY I'm DOING THIS STUDY
% describes makeup of the corpus and how p

% 4.1.1: Gradience in judgements

at least as far back as \emph{LSLT} \citep{LSLT}

%and has become a recent feature of the discussion
is that non-gradient models are totally incompatible with psycholinguistic results.

\subsubsection{Task model}

\citet[][16]{Schutze1996}

\citet{Schutze2005}
\citet{Schutze2011}

%\citet{Coleman1996}

\subsubsection{Falsifiability}

\citet{Armstrong1983}

% wordlikeness critique
\citet{Yang2008a}

a mix of ``declarative'' knowledge,
corresponding to rote facts and thus the lexicon,
and ``procedural'' knowledge, or knowledge of generative procedures 
(in the sense of J. \citealt{Anderson1993}). 

% math cog
\citet{Logan1988}
\citet{Sfard1991}
\citet{Delazer2005}



\subsection{Syllabification}

For this task, it is necessary to separate medial consonant clusters into codas and onsets. While CELEX provides syllabified transcriptions, no description of the syllabification procedure is given in the documentation, and inspection of these syllabifications suggests the procedure used is not fully systematic. For instance, \emph{chemistry} is syllabified as [ˈkɛ.mɪ.strɪ] but \emph{ministry} as [ˈmɪ.nɪs.trɪ].\footnote{Note that word-final \emph{y} is generally lax [ɪ] in Received Pronunciation \citep[][294]{AOE2}.} This putative [ɪ.strɪ $\sim$ ɪs.trɪ] contrast, and many other such contrasts in the CELEX data, are dubious simply because there is no clear evidence that contrastive syllabification occurs in any language. Apparent counterexamples appear to be effects of vowel or consonant length \citep[e.g.,][]{Elfner2006}, word stress, or morphological structure, none of which explain the \emph{chemistry}/\emph{ministry} syllabification contrast.

An automated syllabification system was constructed and applied to medial clusters in the CELEX data. This system need only deal with simplex English words that contain the medial consonant clusters which are the focus of this study.  There is, for instance, no need to address the status of ``ambisyllabic'' consonants, i.e., singleton medial consonants preceded by a stressed lax vowel \citep[][219f.]{Rubach1996}, or to address morphological effects on syllabification. The technique used is a variation on the theme of onset maximization \citep[42f.]{Kahn1976}, which parses word-medial clusters so that the onset is as large as possible. Medial clusters in words like \emph{neu}[.tr]\emph{on} or \emph{bi}[.str]\emph{o} are found in word-initial position (e.g., [tr]\emph{ansit}, [str]\emph{ike}), and onset maximization leaves the coda of the first syllable empty. In contrast, the [nstr] cluster in \emph{mi}[n.str]\emph{el} does not occur word-initially; here the maximal onset is [str], and the [n] is assigned to the coda.

There is one case where unchecked maximization of medial onsets produces incorrect syllabifications. When a medial consonant cluster is preceded by a stressed lax vowel, as in words like \emph{propaga}[n.d]\emph{a}, \emph{whi}[s.p]\emph{er}, \emph{vi}[s.t]\emph{a}, or \emph{bi}[s.k]\emph{uit}, the first consonant of the cluster checks the lax vowel \citep[e.g.,][3]{Hammond1997}. \citet[][55]{Harris1994} notes that onset maximization makes the wrong predictions when the medial cluster is a valid onset: in \emph{whisper}, \emph{vista}, and \emph{biscuit}, it incorrectly assigns [sp, st, sk] to the medial onset, leaving the coda empty. The approach adopted here is to first apply onset maximization, then to reassign the first consonant of a medial onset to the preceding coda if it is immediately preceded by a stressed lax vowel.

If the English affricates [tʃ, dʒ] are treated as sequences and not individual segments, this addition to the standard onset maximization technique would itself make the wrong prediction for medial affricates preceded by lax stressed vowels in words like \emph{ra}[.tʃ]\emph{et} or \emph{a}[.dʒ]\emph{ile}, incorrectly assigning the stop and fricative portions to separate syllables. For this reason, an additional constraint is assumed which prevents the two components of the affricates from being split by syllabification. This is motivated by the tendency of affricates to pattern with single segments in many languages. For instance, the only complex onsets in Classical Nahua are the affricates [ts, tʃ, tɬ] \citep[][9]{Launey2011}.

%In fact, [t.ʃ, d.ʒ] clusters are not found in simplex English words, despite the fact that that affricates occur as medial onsets in clusters (e.g., \emph{tru}[n.tʃ]\emph{eon}, \emph{so}[l.dʒ]\emph{er}).

\subsection{Inventory considerations}

To identify the inventory of syllable contact clusters, it is also necessary to parse syllables into onset, nucleus, and coda. In most cases this is trivial, but a few distributional heuristics for difficult cases are described below.

\subsubsection{The velar nasal}
\label{velarnasal}

There is a long-standing debate regarding whether English [ŋ] is a phoneme in its own right, as demanded by Kiparsky's Alternation Condition \citep{Kiparsky1968} or Lexicon Optimization \citep[][53]{OT}, or simply the allophone of /n/ found before /k, g/ \citep[][65]{Borowsky1986}. The strongest piece of evidence for treating the velar nasal as a pure allophone is the general absence of ?
nset [ŋ], a position where it can never be followed by a dorsal consonant needed to derive the velar allophone. \citet{Pierrehumbert1994} assumes the pure allophonic analysis in her study of syllable contact clusters, and it is assumed here. English nasal place allophony is formalized in Section \ref{cnpasection} below.
%\citealt[][62]{Halle1985a}),

\subsubsection{[j] onglides}

The [j] onglide in words such as in words such as \emph{ass}[j]\emph{ume} is assumed to be present in underlying representations (e.g., \citealt[][278]{Borowsky1986}, J. \citealt{Anderson1988b}, pace \emph{SPE}:196, \citealt[][89]{Halle1985a}, \citealt[][217]{McMahon1990}), since the presence or absence of the glide is at least marginally contrastive (e.g., \emph{coo}/\emph{coup} \alt{} \emph{queue}, \emph{booty} \alt{} \emph{beauty}). The front onglide is further assumed to be assigned to the nucleus, except when the onset would otherwise be null (e.g., \emph{jun}[j]\emph{or}). 
%\footnote{English glides are transcribed here as full segments, not as ``subsegments'', as this distinction does not appear to be meaningful for English, or empirically motivated for other languages \citep{Rubach2002}.}
There is extensive motivation for this principle. When [j] is a simplex onset, it may be followed by any vowel \citep[][276]{Borowsky1986}, but when [j] is immediately preceded by an onset consonant (e.g., [bj]\emph{ugle}), the following vowel is always [uː]. In fact, speakers judge words in which the front onglide is preceded by an onset consonant and followed by a vowel other than [uː] to be anomalous \citep{Moreland2009}. This defective distribution of vowels following [j] suggests that the onglide in this context is the first component of a phonological diphthong (\citealp[][232]{Hayes1980}, \citealp[][61f.]{Harris1994}, \citealp{Davis1995}). \citet[][42]{Clements1983} note that /m, v/ do not appear in onset clusters except in words like \emph{muse} or \emph{view}; the exceptionality of [juː] also suggests the glide is nuclear. There is also strong external support for this principle. The [juː] in words like \emph{spew} may pattern together in Pig Latin \citep{Davis1995,Idsardi2005} and \emph{shm}-reduplication \citep{Nevins2003} to the exclusion of the rest of preceding tautosyllabic consonants. The same fusion of [juː] at the expense of the onset is also found in speech errors, e.g., [kjuː]\emph{mor} [h]\emph{omponent} for [hjuː]\emph{mor component} \citep[][130]{Shattuck-Hufnagel1986}. Finally, \citet{Buchwald2005} considers [j] onglides in the speech of VBR, an aphasic patient who has difficulties producing complex onsets. 

\begin{example}[VBR's complex onsets (\citealp{Buchwald2005}:79--80, his transcriptions)]
%\protect\citep[79--80, his transcriptions]{Buchwald2005}]

\begin{tabular}{l l l}
a. & kəræb  & `crab'  \\
   & bəlid  & `bleed' \\
b. & kəwin  & `queen' \\
   & kəwoʊt & `quote' \\
c. & kut    & `cute'  \\
   & musɪk  & `music  \\ 
\end{tabular}
\label{VBR}
\end{example}

\noindent
VBR breaks up complex onsets with epenthesis, including those that consist of a consonant and a back onglide cluster (\ref{VBR}b). However, no epenthesis occurs between a consonant and a front onglide; rather, the glide is absent (\ref{VBR}c). The failure of the front onglide to pattern with other consonant clusters suggests once again that the glide is part of the nucleus. 

%One potential problem with this account is noted by \citet{Kaye1996}, who obseres while [juː] may follow any single tautosyllabic consonant, it never follows branching onsets unless they consist of [s] and a single consonant. 
%This is the only sign that [juː] shows an affiliation for the onset. 

\subsubsection{[w] onglides}

The selective properties of the back onglide [w] contrast sharply with those of the front onglide, and I assume that it is assigned to the onset. Whereas the front onglide shows only limited selectivity for preceding tautosyllabic consonants \citep{Davis1995,Kaye1996}, the back onglide [w] is rarely preceded by tautosyllabic consonants other than [k] (e.g., \emph{tran}[kw]\emph{il}). Unlike the front glide, syllable-initial [kw] may be followed by nearly any vowel \citep[][161]{Davis1995}. Unlike [juː], onglide [w] followed by a vowel does not pattern together in Pig Latin \citep[][166]{Davis1995}. And onsets followed by [w] pattern with other complex onsets in undergoing epenthesis in the patholical speech of VBR, discussed immediately above.

%Whereas [Cjuː] syllables attracts stress, [kwV] syllables do not \citep[][162f.]{Davis1995}.

\subsubsection{Post-vocalic \emph{r}}

Both CELEX and the dictionary used by \citet{Pierrehumbert1994} in her study transcribe Received Pronunciation, in which word-medial post-vocalic \emph{r} has been lost. In \emph{r}-full dialects, there is reason to believe that post-vocalic \emph{r} is in fact nuclear, and therefore its presence or absence is irrelevant to the constraints on syllable contact clusters. Many vowel contrasts are suspended before \emph{r} \citep[][255]{Harris1994}; for example, \emph{fern}, \emph{fir}, and \emph{fur} lack the contrasts found in \emph{pet}, \emph{pit}, and \emph{putt}. Further evidence for the nuclear status of post-vocalic \emph{r} comes from variable phonological processes in which post-vocalic \emph{r} patterns differently than internal coda consonants. \citeauthor{Harris1994} reports that a variable process of /t/-\textsc{Glottalization} in many dialects of British English is blocked when /t/ is preceded by any consonant except post-vocalic \emph{r}.

\begin{example}
/t/-\textsc{Glottalization} in British English \citep[after][195, 258]{Harris1994}: 

\vspace{0.5\baselineskip}
\begin{tabular}{l l l@{} l}
a. & fis[t]   & * & fis[ʔ]   \\
   & mis[t]er & * & mis[ʔ]er \\
b. & par[t]   &   & par[ʔ]   \\
   & car[t]on &   & car[ʔ]on \\
\end{tabular}
\end{example}

\noindent
Similarly, while /t, d/ delete in word-final position when immediately preceded by a consonant, including sonorants /n, l/, as in (\ref{td}a), deletion of /t, d/ after post-vocalic \emph{r} in American English is ``rare or nonexistent'' \citep[][8]{Guy1980}. 

\begin{example}
\label{td}
/t, d/-\textsc{Deletion} in American English: 

\vspace{0.5\baselineskip}
\begin{tabular}{l l l@{} l}
a. & be[lt]  &   & be[l]  \\
%   & we[ld]  &   & we[l]  \\
%   & cha[nt] &   & cha[n] \\
   & me[nd]  &   & me[n]  \\
%b. & fl[ɜ˞]  & * & fl[ɜ˞]  \\
%b. & fl[ɝt]  & * & fl[ɝ]  \\
b. & sh[ɚt]  & * & sh[ɚ] \\
%   & w[ɚd]   & * & w[ɚ]  \\
   & c[ɚd]   & * & c[ɚ]  \\
\end{tabular}
\end{example}

%Finally, \citet[][251]{Fromkin1973} also presents evidence that post-vocalic \emph{r} may behave as if nuclear in speech errors. 


%        \subsection{Gradience in judgements} %% 4.1.1: Gradience in judgements

at least as far back as \emph{LSLT} \citep{LSLT}

%and has become a recent feature of the discussion
is that non-gradient models are totally incompatible with psycholinguistic results.

\subsubsection{Task model}

\citet[][16]{Schutze1996}

\citet{Schutze2005}
\citet{Schutze2011}

%\citet{Coleman1996}

\subsubsection{Falsifiability}

\citet{Armstrong1983}

% wordlikeness critique
\citet{Yang2008a}

a mix of ``declarative'' knowledge,
corresponding to rote facts and thus the lexicon,
and ``procedural'' knowledge, or knowledge of generative procedures 
(in the sense of J. \citealt{Anderson1993}). 

% math cog
\citet{Logan1988}
\citet{Sfard1991}
\citet{Delazer2005}



%        \subsection{The ontology of wordlikeness} %\subsection{Syllabification}

For this task, it is necessary to separate medial consonant clusters into codas and onsets. While CELEX provides syllabified transcriptions, no description of the syllabification procedure is given in the documentation, and inspection of these syllabifications suggests the procedure used is not fully systematic. For instance, \emph{chemistry} is syllabified as [ˈkɛ.mɪ.strɪ] but \emph{ministry} as [ˈmɪ.nɪs.trɪ].\footnote{Note that word-final \emph{y} is generally lax [ɪ] in Received Pronunciation \citep[][294]{AOE2}.} This putative [ɪ.strɪ $\sim$ ɪs.trɪ] contrast, and many other such contrasts in the CELEX data, are dubious simply because there is no clear evidence that contrastive syllabification occurs in any language. Apparent counterexamples appear to be effects of vowel or consonant length \citep[e.g.,][]{Elfner2006}, word stress, or morphological structure, none of which explain the \emph{chemistry}/\emph{ministry} syllabification contrast.

An automated syllabification system was constructed and applied to medial clusters in the CELEX data. This system need only deal with simplex English words that contain the medial consonant clusters which are the focus of this study.  There is, for instance, no need to address the status of ``ambisyllabic'' consonants, i.e., singleton medial consonants preceded by a stressed lax vowel \citep[][219f.]{Rubach1996}, or to address morphological effects on syllabification. The technique used is a variation on the theme of onset maximization \citep[42f.]{Kahn1976}, which parses word-medial clusters so that the onset is as large as possible. Medial clusters in words like \emph{neu}[.tr]\emph{on} or \emph{bi}[.str]\emph{o} are found in word-initial position (e.g., [tr]\emph{ansit}, [str]\emph{ike}), and onset maximization leaves the coda of the first syllable empty. In contrast, the [nstr] cluster in \emph{mi}[n.str]\emph{el} does not occur word-initially; here the maximal onset is [str], and the [n] is assigned to the coda.

There is one case where unchecked maximization of medial onsets produces incorrect syllabifications. When a medial consonant cluster is preceded by a stressed lax vowel, as in words like \emph{propaga}[n.d]\emph{a}, \emph{whi}[s.p]\emph{er}, \emph{vi}[s.t]\emph{a}, or \emph{bi}[s.k]\emph{uit}, the first consonant of the cluster checks the lax vowel \citep[e.g.,][3]{Hammond1997}. \citet[][55]{Harris1994} notes that onset maximization makes the wrong predictions when the medial cluster is a valid onset: in \emph{whisper}, \emph{vista}, and \emph{biscuit}, it incorrectly assigns [sp, st, sk] to the medial onset, leaving the coda empty. The approach adopted here is to first apply onset maximization, then to reassign the first consonant of a medial onset to the preceding coda if it is immediately preceded by a stressed lax vowel.

If the English affricates [tʃ, dʒ] are treated as sequences and not individual segments, this addition to the standard onset maximization technique would itself make the wrong prediction for medial affricates preceded by lax stressed vowels in words like \emph{ra}[.tʃ]\emph{et} or \emph{a}[.dʒ]\emph{ile}, incorrectly assigning the stop and fricative portions to separate syllables. For this reason, an additional constraint is assumed which prevents the two components of the affricates from being split by syllabification. This is motivated by the tendency of affricates to pattern with single segments in many languages. For instance, the only complex onsets in Classical Nahua are the affricates [ts, tʃ, tɬ] \citep[][9]{Launey2011}.

%In fact, [t.ʃ, d.ʒ] clusters are not found in simplex English words, despite the fact that that affricates occur as medial onsets in clusters (e.g., \emph{tru}[n.tʃ]\emph{eon}, \emph{so}[l.dʒ]\emph{er}).

%    \section{Evaluation} %% 4.2: Evaluation

The remainder of this chapter is dedicated to evaluating two claim

%        \subsection{Assessing gradience} %% 4.2.1: Assessing gradience

\subsubsection{Response distributions}

This is shown in Figure \ref{density}.

\begin{figure}
\centering
\includegraphics{density.pdf}
\caption{Density of responses}
\label{density}
\end{figure}

\ex EM test results: \vspace{6pt} \\
\begin{tabular}{l | r r | r r r r r r | r r}
\toprule
     & \multicolumn{2}{c|}{one $\mathcal{N}$} & \multicolumn{6}{c|}{mixture of two $\mathcal{N}$s} & \multicolumn{2}{c}{log $\mathcal{L}$ test} \\
     & $\mu$ & $\sigma$ & $\mu_1$ & $\sigma_1$ & mix$_1$ & $\mu_2$ & $\sigma_2$ & mix$_2$ & $\Lambda$ & $p$-value  \\
\midrule
G\&J & 0.378 & 0.295    & 0.150   & 0.090      & 0.436     & 0.554   & 0.278    & 0.564     & 5.190    & 0.158   \\
S    & 0.596 & 0.361    & 0.236   & 0.232      & 0.437     & 0.876   & 0.105    & 0.563 & 48.965    & 1.3\e{-10} \\
A\&H & 0.431 & 0.266    & 0.301   & 0.200      & 0.559     & 0.595   & 0.246    & 0.411 & 16.186    & 0.001      \\
\bottomrule
\end{tabular}
\xe

%\ex Residualized correlations
%\begin{tabular}
%\toprule
%ASDF
%\midrule
%G\&J & 
%S    & 
%A\&H & 
%\bottomrule
%\xe

%\subsubsection{Prediction distributions}

\citet{Coltheart1977}

\citet{Hayes2008a}
\citet{Albright2009a}

% stats thereof
%\citet{Schilling2002}
%\citet{Helguerro1904}
%\citet{Cohen1956}

% EM
\citet{EM}

\footnote{In performing this evaluation, I have benefitted greatly from course notes by Mark Liberman and Stephen Isard.}
%available at \url{http://www.ling.upenn.edu/courses/cogs501/K-meansHW.html}

% -2 log like ratio test

where the difference in degrees o freedom equals $D$

%\ex $\displaystyle \textrm{Reject } H_0 \textrm{ iff } 
%\ex $\displaystyle \Lambda = -2 \textrm{ ln } L_0 + 2 \textrm{ ln } L_1}$ \xe
% > χ^2_{d.f.=D}$ \xe


%            \subsubsection{Response multimodality} %% 4.2.1: Assessing gradience

\subsubsection{Response distributions}

This is shown in Figure \ref{density}.

\begin{figure}
\centering
\includegraphics{density.pdf}
\caption{Density of responses}
\label{density}
\end{figure}

\ex EM test results: \vspace{6pt} \\
\begin{tabular}{l | r r | r r r r r r | r r}
\toprule
     & \multicolumn{2}{c|}{one $\mathcal{N}$} & \multicolumn{6}{c|}{mixture of two $\mathcal{N}$s} & \multicolumn{2}{c}{log $\mathcal{L}$ test} \\
     & $\mu$ & $\sigma$ & $\mu_1$ & $\sigma_1$ & mix$_1$ & $\mu_2$ & $\sigma_2$ & mix$_2$ & $\Lambda$ & $p$-value  \\
\midrule
G\&J & 0.378 & 0.295    & 0.150   & 0.090      & 0.436     & 0.554   & 0.278    & 0.564     & 5.190    & 0.158   \\
S    & 0.596 & 0.361    & 0.236   & 0.232      & 0.437     & 0.876   & 0.105    & 0.563 & 48.965    & 1.3\e{-10} \\
A\&H & 0.431 & 0.266    & 0.301   & 0.200      & 0.559     & 0.595   & 0.246    & 0.411 & 16.186    & 0.001      \\
\bottomrule
\end{tabular}
\xe

%\ex Residualized correlations
%\begin{tabular}
%\toprule
%ASDF
%\midrule
%G\&J & 
%S    & 
%A\&H & 
%\bottomrule
%\xe

%\subsubsection{Prediction distributions}

\citet{Coltheart1977}

\citet{Hayes2008a}
\citet{Albright2009a}

% stats thereof
%\citet{Schilling2002}
%\citet{Helguerro1904}
%\citet{Cohen1956}

% EM
\citet{EM}

\footnote{In performing this evaluation, I have benefitted greatly from course notes by Mark Liberman and Stephen Isard.}
%available at \url{http://www.ling.upenn.edu/courses/cogs501/K-meansHW.html}

% -2 log like ratio test

where the difference in degrees o freedom equals $D$

%\ex $\displaystyle \textrm{Reject } H_0 \textrm{ iff } 
%\ex $\displaystyle \Lambda = -2 \textrm{ ln } L_0 + 2 \textrm{ ln } L_1}$ \xe
% > χ^2_{d.f.=D}$ \xe

%            \subsubsection{Prediction multimodality} %input{4.2.1.2}

%        \subsection{Assessing models} %\subsection{Computational models}


Applying these derived constraints to the lexicon increases saturation from 18.8\% to 28.8\% and incurs only a handful of exceptions. Yet, despite the fact that ``attestation'' is the minority pattern, attempts to identify static lexical constraints, whether by hand or computational model, lend little additional predictive power. There is no evidence that the English syllable contact inventory is subject to any static constraint at all. I am forced to conclude that most, if not all, of 71.2\% of possible clusters which are unattested are accidental gaps. Below, I show that this state of affairs follows directly from the sparse distribution of codas and onsets. 

The three alternations targeting English syllable contact clusters have been shown to have the expected robust effects on the English lexicon. 


Whereas the three alternations targeting English syllable contact clusters have robust lexical reflexes as measured by the Fisher exact test, such tests reveal no evidence for the static dispreferences identified by \citeauthor{Pierrehumbert1994}. There is now a growing literature on computational cognitive models of phonotactic learning. By applying these models to the syllable contact data, it is possible to exhaustively search for statistically motivated constraints, elimianting the possibility of human error. 

To evaluate these statistical models, they are first trained on the set of attested clusters, then put to the task of classifying all possible clusters into those which are and are not attested. Four metrics are used to evaluate these computational models. Classification accuracy corresponds to the probability of a classification being either a true positive ($t.p.$), correctly predicting a cluster to occur, or a true negative ($t.n.$), correctly predicting a cluster to be absent. Incorrect outcomes are either false positives ($f.p.$), incorrectly predicting a cluster to occur, and false negatives ($f.n.$), incorrectly predicting a cluster to be absent. 

%\ex Binary classification contingency table: \\
%\begin{tabular}{c | l l} \toprule
%                             & \multicolumn{2}{c}{prediction} \\
%\midrule
%\multirow{2}{*}{observation} & true positive ($tp$)  & false positive ($fp$) \\
%                             & false negative ($fn$) & true negative ($tn$) \\
%\end{tabular}
%\begin{tabular}{|l l|}
%\toprule
%true positive ($tp$)  & false positive ($fp$) \\
%false negative ($fn$) & true negative ($tn$) \\
%\bottomrule
%\end{tabular}
%\xe 

\begin{example}
$\displaystyle \textrm{Accuracy} = \frac{t.p. + t.n.}{t.p. + t.n. + f.p. + f.n.}$
\end{example}

\noindent
A few other metrics can be used to assess classification performance. Precision corresponds to the probability that a cluster predicted to occur is observed in the data. 
\begin{example}
$\displaystyle \textrm{Precision} = \frac{t.p.}{t.p. + f.p.}$ 
\end{example}

\noindent
Recall (also known as sensitivity) is the probability a observed cluster is predicted to occur. 

\begin{example}
$\displaystyle \textrm{Recall} = \frac{t.p.}{t.p. + f.n.}$
\end{example}

It is possible to maximize recall at the expense of precision, by predicting more clusters to be attested, or to increase precision at the expense of recall by predicting fewer clusters to be attested. A common way to balance these concerns is the $F_1$ score, which is the harmonic mean of precision and recall.

\begin{example}
$\displaystyle F_1 = 2 \left( \frac{\textrm{Precision} \times \textrm{Recall}}{\textrm{Precision} + \textrm{Recall}}\right)$ 
\end{example}

\subsubsection{Null baseline}

The 840 possible clusters have an 18.7\% of saturdaion rate. As a consequence of this sparsity, a null model which classifies all clusters as unattested achieves 81.3\% accuracy without the need to posit any constraints either derived or static. Since this model does not have any positive classifications, either true or false, precision, recall, and $F_1$ are undefined.

\subsubsection{Alternations}

As was shown above, \textsc{Obstruent Voice Assimiation}, \textsc{Coda Nasal Place Assimilation}, and \textsc{Degemination} are reliable predictors of what clusters are and are not attested and have few lexical exceptions. I generate another baseline by predicting only those clusters which do not violate any of these generalizations to be attested. This provides a significant improvement to the null baseline (accuracy = 0.857). As precision (0.401) is smaller than recall (0.708), there are more false positives (i.e., unattested clusters that do not violate any of these three alternation-derived generalizations) than there are false negatives (i.e., attested exceptions to these derived generalizations). 

\subsubsection{\citet{Pierrehumbert1994}}

\citet{Pierrehumbert1994} first proposes to measure the well-formedness of syllable contact clusters by the independent probabilities of codas and onsets, and \citet{Coleman1997} extend this model to whole words. \citet{Pierrehumbert1994} uses both medial and initial/final codas and onsets to compute these probabilities, but I obtain a better fit to the data by only using medial coda and onset frequencies. Medial coda and onset probabilities are multipled to obtain a well-formedness score. This produces a continuous value between 0 and 1. To derive a binary classification from these values, I use a soft-margin support vector machine \citep{Cortes1995} with a linear kernel. This technique derives a ``decision stump'' providing the best single split into attested and an unattested clusters. This is not put forth as a cognitively plausible model for relating this expected probability to attestation: it is simply the upper bound that such a model could reach. 

Unfortunately, this score only provides a small improvement to the null baseline
 (accuracy = 0.839), and has a lower accuracy and $F_1$ (0.328) than the alternation baseline. 

\subsubsection{\citet{Hayes2008a}}

\citeauthor{Hayes2008a} develop a sophisticated method of learning phonotactic distributions from positive data. They develop software for estimating probability distributions over sequences of (sets of) phonological features using the principle of maximum entropy and making few assumptions about the independence of features. As above, a support vector machine with a linear kernel is used to turn these probability values into binary predictions. The software provided by these authors exposes many ``metaparameters'', operational decisions to be made in conducting an experiment with this system. To avoid any potential bias induced by the choice of these metaparameters, I use the same settings as used by \citet{Hayes2008a} in their similar study of English onset consonants. This includes the ``accuracy schedule'' [0.001, 0.01, 0.1, 0.2, 0.3] and no constraints on the number of constraints learned, as well English consonantal features used by \citeauthor{Hayes2008a}. Since this model is inherently stochastic, producing slightly different outcomes on each run, I follow the practice of \citeauthor{Hayes2008a} (p.~396) and report the lowest-accuracy of ten runs, though in general there is not a great deal of variation between individual runs.  

This model provides a small improvement in accuracy (0.877) and $F_1$ (0.703) compared to the alternation baseline, but it has lower recall (0.642) than this baseline, indicating that the \citeauthor{Hayes2008a} model introduces many false negatives. 
%; it assigns the lowest probability value to 122 unattested clusters.
More specifically, it suggests that this model prefers narrower phonotactic generalizations than those derived from alternations. For instance, unattested *[m.kl] is assigned the highest probability, whereas it is ruled out by \textsc{Coda Nasal Place Assimilation}. On the other hand, the \citeauthor{Hayes2008a} model assigns a low score to clusters like those found in \emph{hu}[z.b]\emph{and} and \emph{pla}[t.f]\emph{orm}, though they are both attested and consistent with all three alternation-derived constraints. 

\citet{Berent2012} identify an important defect with the \citeauthor{Hayes2008a} model used here: it does not make use of variables over features and thus fails to unify the constraint against geminate clusters. While the model may miss this unifying generalization, it correctly predicts all clusters containing geminates to be unattested. 

The results for the four models are summarized below.

\begin{example}
Summary of computational model accuracy: 

\vspace{0.5\baselineskip} 
\begin{tabular}{l | r r r r}
\toprule
                          & accuracy & precision & recall & $F_1$ \\ 
\midrule
Null baseline             & 0.813    & n.a.      & n.a.   & n.a.  \\
Alternations              & 0.857    & 0.401     & 0.708  & 0.512 \\
\citet{Pierrehumbert1994} & 0.839    & 0.210     & 0.750  & 0.328 \\
\citet{Hayes2008a}        & 0.877    & 0.777     & 0.642  & 0.703 \\
\bottomrule
\end{tabular}
\end{example}

One additional model was evaluated for this study. \citet{McGowan2011} claims that the type frequency of individual syllable contact clusters in English is predicted by the change of sonority in the clusters.\footnote{Thanks to Maria Gouskova for bringing this study to my attention.} Using a sonority scale proposed by \citet{Jespersen1904}, \citeauthor{McGowan2011} reports that clusters like [m.p], with sharply falling sonority, are slightly more frequent than those like [t.j], with rising sonority. However, \citeauthor{McGowan2011} finds that sonority of clusters accounts for only a tiny amount of the variance in cluster frequency ($R^2 = 0.077$). In this study, inclusion of sonority distance provided no improvement over the null baseline, and therefore was not considered further.


%            \subsubsection{Unimodal fit} %\subsection{Computational models}


Applying these derived constraints to the lexicon increases saturation from 18.8\% to 28.8\% and incurs only a handful of exceptions. Yet, despite the fact that ``attestation'' is the minority pattern, attempts to identify static lexical constraints, whether by hand or computational model, lend little additional predictive power. There is no evidence that the English syllable contact inventory is subject to any static constraint at all. I am forced to conclude that most, if not all, of 71.2\% of possible clusters which are unattested are accidental gaps. Below, I show that this state of affairs follows directly from the sparse distribution of codas and onsets. 

The three alternations targeting English syllable contact clusters have been shown to have the expected robust effects on the English lexicon. 


Whereas the three alternations targeting English syllable contact clusters have robust lexical reflexes as measured by the Fisher exact test, such tests reveal no evidence for the static dispreferences identified by \citeauthor{Pierrehumbert1994}. There is now a growing literature on computational cognitive models of phonotactic learning. By applying these models to the syllable contact data, it is possible to exhaustively search for statistically motivated constraints, elimianting the possibility of human error. 

To evaluate these statistical models, they are first trained on the set of attested clusters, then put to the task of classifying all possible clusters into those which are and are not attested. Four metrics are used to evaluate these computational models. Classification accuracy corresponds to the probability of a classification being either a true positive ($t.p.$), correctly predicting a cluster to occur, or a true negative ($t.n.$), correctly predicting a cluster to be absent. Incorrect outcomes are either false positives ($f.p.$), incorrectly predicting a cluster to occur, and false negatives ($f.n.$), incorrectly predicting a cluster to be absent. 

%\ex Binary classification contingency table: \\
%\begin{tabular}{c | l l} \toprule
%                             & \multicolumn{2}{c}{prediction} \\
%\midrule
%\multirow{2}{*}{observation} & true positive ($tp$)  & false positive ($fp$) \\
%                             & false negative ($fn$) & true negative ($tn$) \\
%\end{tabular}
%\begin{tabular}{|l l|}
%\toprule
%true positive ($tp$)  & false positive ($fp$) \\
%false negative ($fn$) & true negative ($tn$) \\
%\bottomrule
%\end{tabular}
%\xe 

\begin{example}
$\displaystyle \textrm{Accuracy} = \frac{t.p. + t.n.}{t.p. + t.n. + f.p. + f.n.}$
\end{example}

\noindent
A few other metrics can be used to assess classification performance. Precision corresponds to the probability that a cluster predicted to occur is observed in the data. 
\begin{example}
$\displaystyle \textrm{Precision} = \frac{t.p.}{t.p. + f.p.}$ 
\end{example}

\noindent
Recall (also known as sensitivity) is the probability a observed cluster is predicted to occur. 

\begin{example}
$\displaystyle \textrm{Recall} = \frac{t.p.}{t.p. + f.n.}$
\end{example}

It is possible to maximize recall at the expense of precision, by predicting more clusters to be attested, or to increase precision at the expense of recall by predicting fewer clusters to be attested. A common way to balance these concerns is the $F_1$ score, which is the harmonic mean of precision and recall.

\begin{example}
$\displaystyle F_1 = 2 \left( \frac{\textrm{Precision} \times \textrm{Recall}}{\textrm{Precision} + \textrm{Recall}}\right)$ 
\end{example}

\subsubsection{Null baseline}

The 840 possible clusters have an 18.7\% of saturdaion rate. As a consequence of this sparsity, a null model which classifies all clusters as unattested achieves 81.3\% accuracy without the need to posit any constraints either derived or static. Since this model does not have any positive classifications, either true or false, precision, recall, and $F_1$ are undefined.

\subsubsection{Alternations}

As was shown above, \textsc{Obstruent Voice Assimiation}, \textsc{Coda Nasal Place Assimilation}, and \textsc{Degemination} are reliable predictors of what clusters are and are not attested and have few lexical exceptions. I generate another baseline by predicting only those clusters which do not violate any of these generalizations to be attested. This provides a significant improvement to the null baseline (accuracy = 0.857). As precision (0.401) is smaller than recall (0.708), there are more false positives (i.e., unattested clusters that do not violate any of these three alternation-derived generalizations) than there are false negatives (i.e., attested exceptions to these derived generalizations). 

\subsubsection{\citet{Pierrehumbert1994}}

\citet{Pierrehumbert1994} first proposes to measure the well-formedness of syllable contact clusters by the independent probabilities of codas and onsets, and \citet{Coleman1997} extend this model to whole words. \citet{Pierrehumbert1994} uses both medial and initial/final codas and onsets to compute these probabilities, but I obtain a better fit to the data by only using medial coda and onset frequencies. Medial coda and onset probabilities are multipled to obtain a well-formedness score. This produces a continuous value between 0 and 1. To derive a binary classification from these values, I use a soft-margin support vector machine \citep{Cortes1995} with a linear kernel. This technique derives a ``decision stump'' providing the best single split into attested and an unattested clusters. This is not put forth as a cognitively plausible model for relating this expected probability to attestation: it is simply the upper bound that such a model could reach. 

Unfortunately, this score only provides a small improvement to the null baseline
 (accuracy = 0.839), and has a lower accuracy and $F_1$ (0.328) than the alternation baseline. 

\subsubsection{\citet{Hayes2008a}}

\citeauthor{Hayes2008a} develop a sophisticated method of learning phonotactic distributions from positive data. They develop software for estimating probability distributions over sequences of (sets of) phonological features using the principle of maximum entropy and making few assumptions about the independence of features. As above, a support vector machine with a linear kernel is used to turn these probability values into binary predictions. The software provided by these authors exposes many ``metaparameters'', operational decisions to be made in conducting an experiment with this system. To avoid any potential bias induced by the choice of these metaparameters, I use the same settings as used by \citet{Hayes2008a} in their similar study of English onset consonants. This includes the ``accuracy schedule'' [0.001, 0.01, 0.1, 0.2, 0.3] and no constraints on the number of constraints learned, as well English consonantal features used by \citeauthor{Hayes2008a}. Since this model is inherently stochastic, producing slightly different outcomes on each run, I follow the practice of \citeauthor{Hayes2008a} (p.~396) and report the lowest-accuracy of ten runs, though in general there is not a great deal of variation between individual runs.  

This model provides a small improvement in accuracy (0.877) and $F_1$ (0.703) compared to the alternation baseline, but it has lower recall (0.642) than this baseline, indicating that the \citeauthor{Hayes2008a} model introduces many false negatives. 
%; it assigns the lowest probability value to 122 unattested clusters.
More specifically, it suggests that this model prefers narrower phonotactic generalizations than those derived from alternations. For instance, unattested *[m.kl] is assigned the highest probability, whereas it is ruled out by \textsc{Coda Nasal Place Assimilation}. On the other hand, the \citeauthor{Hayes2008a} model assigns a low score to clusters like those found in \emph{hu}[z.b]\emph{and} and \emph{pla}[t.f]\emph{orm}, though they are both attested and consistent with all three alternation-derived constraints. 

\citet{Berent2012} identify an important defect with the \citeauthor{Hayes2008a} model used here: it does not make use of variables over features and thus fails to unify the constraint against geminate clusters. While the model may miss this unifying generalization, it correctly predicts all clusters containing geminates to be unattested. 

The results for the four models are summarized below.

\begin{example}
Summary of computational model accuracy: 

\vspace{0.5\baselineskip} 
\begin{tabular}{l | r r r r}
\toprule
                          & accuracy & precision & recall & $F_1$ \\ 
\midrule
Null baseline             & 0.813    & n.a.      & n.a.   & n.a.  \\
Alternations              & 0.857    & 0.401     & 0.708  & 0.512 \\
\citet{Pierrehumbert1994} & 0.839    & 0.210     & 0.750  & 0.328 \\
\citet{Hayes2008a}        & 0.877    & 0.777     & 0.642  & 0.703 \\
\bottomrule
\end{tabular}
\end{example}

One additional model was evaluated for this study. \citet{McGowan2011} claims that the type frequency of individual syllable contact clusters in English is predicted by the change of sonority in the clusters.\footnote{Thanks to Maria Gouskova for bringing this study to my attention.} Using a sonority scale proposed by \citet{Jespersen1904}, \citeauthor{McGowan2011} reports that clusters like [m.p], with sharply falling sonority, are slightly more frequent than those like [t.j], with rising sonority. However, \citeauthor{McGowan2011} finds that sonority of clusters accounts for only a tiny amount of the variance in cluster frequency ($R^2 = 0.077$). In this study, inclusion of sonority distance provided no improvement over the null baseline, and therefore was not considered further.

%            \subsubsection{Bimodal fit} %\subsection{Computational models}


Applying these derived constraints to the lexicon increases saturation from 18.8\% to 28.8\% and incurs only a handful of exceptions. Yet, despite the fact that ``attestation'' is the minority pattern, attempts to identify static lexical constraints, whether by hand or computational model, lend little additional predictive power. There is no evidence that the English syllable contact inventory is subject to any static constraint at all. I am forced to conclude that most, if not all, of 71.2\% of possible clusters which are unattested are accidental gaps. Below, I show that this state of affairs follows directly from the sparse distribution of codas and onsets. 

The three alternations targeting English syllable contact clusters have been shown to have the expected robust effects on the English lexicon. 


Whereas the three alternations targeting English syllable contact clusters have robust lexical reflexes as measured by the Fisher exact test, such tests reveal no evidence for the static dispreferences identified by \citeauthor{Pierrehumbert1994}. There is now a growing literature on computational cognitive models of phonotactic learning. By applying these models to the syllable contact data, it is possible to exhaustively search for statistically motivated constraints, elimianting the possibility of human error. 

To evaluate these statistical models, they are first trained on the set of attested clusters, then put to the task of classifying all possible clusters into those which are and are not attested. Four metrics are used to evaluate these computational models. Classification accuracy corresponds to the probability of a classification being either a true positive ($t.p.$), correctly predicting a cluster to occur, or a true negative ($t.n.$), correctly predicting a cluster to be absent. Incorrect outcomes are either false positives ($f.p.$), incorrectly predicting a cluster to occur, and false negatives ($f.n.$), incorrectly predicting a cluster to be absent. 

%\ex Binary classification contingency table: \\
%\begin{tabular}{c | l l} \toprule
%                             & \multicolumn{2}{c}{prediction} \\
%\midrule
%\multirow{2}{*}{observation} & true positive ($tp$)  & false positive ($fp$) \\
%                             & false negative ($fn$) & true negative ($tn$) \\
%\end{tabular}
%\begin{tabular}{|l l|}
%\toprule
%true positive ($tp$)  & false positive ($fp$) \\
%false negative ($fn$) & true negative ($tn$) \\
%\bottomrule
%\end{tabular}
%\xe 

\begin{example}
$\displaystyle \textrm{Accuracy} = \frac{t.p. + t.n.}{t.p. + t.n. + f.p. + f.n.}$
\end{example}

\noindent
A few other metrics can be used to assess classification performance. Precision corresponds to the probability that a cluster predicted to occur is observed in the data. 
\begin{example}
$\displaystyle \textrm{Precision} = \frac{t.p.}{t.p. + f.p.}$ 
\end{example}

\noindent
Recall (also known as sensitivity) is the probability a observed cluster is predicted to occur. 

\begin{example}
$\displaystyle \textrm{Recall} = \frac{t.p.}{t.p. + f.n.}$
\end{example}

It is possible to maximize recall at the expense of precision, by predicting more clusters to be attested, or to increase precision at the expense of recall by predicting fewer clusters to be attested. A common way to balance these concerns is the $F_1$ score, which is the harmonic mean of precision and recall.

\begin{example}
$\displaystyle F_1 = 2 \left( \frac{\textrm{Precision} \times \textrm{Recall}}{\textrm{Precision} + \textrm{Recall}}\right)$ 
\end{example}

\subsubsection{Null baseline}

The 840 possible clusters have an 18.7\% of saturdaion rate. As a consequence of this sparsity, a null model which classifies all clusters as unattested achieves 81.3\% accuracy without the need to posit any constraints either derived or static. Since this model does not have any positive classifications, either true or false, precision, recall, and $F_1$ are undefined.

\subsubsection{Alternations}

As was shown above, \textsc{Obstruent Voice Assimiation}, \textsc{Coda Nasal Place Assimilation}, and \textsc{Degemination} are reliable predictors of what clusters are and are not attested and have few lexical exceptions. I generate another baseline by predicting only those clusters which do not violate any of these generalizations to be attested. This provides a significant improvement to the null baseline (accuracy = 0.857). As precision (0.401) is smaller than recall (0.708), there are more false positives (i.e., unattested clusters that do not violate any of these three alternation-derived generalizations) than there are false negatives (i.e., attested exceptions to these derived generalizations). 

\subsubsection{\citet{Pierrehumbert1994}}

\citet{Pierrehumbert1994} first proposes to measure the well-formedness of syllable contact clusters by the independent probabilities of codas and onsets, and \citet{Coleman1997} extend this model to whole words. \citet{Pierrehumbert1994} uses both medial and initial/final codas and onsets to compute these probabilities, but I obtain a better fit to the data by only using medial coda and onset frequencies. Medial coda and onset probabilities are multipled to obtain a well-formedness score. This produces a continuous value between 0 and 1. To derive a binary classification from these values, I use a soft-margin support vector machine \citep{Cortes1995} with a linear kernel. This technique derives a ``decision stump'' providing the best single split into attested and an unattested clusters. This is not put forth as a cognitively plausible model for relating this expected probability to attestation: it is simply the upper bound that such a model could reach. 

Unfortunately, this score only provides a small improvement to the null baseline
 (accuracy = 0.839), and has a lower accuracy and $F_1$ (0.328) than the alternation baseline. 

\subsubsection{\citet{Hayes2008a}}

\citeauthor{Hayes2008a} develop a sophisticated method of learning phonotactic distributions from positive data. They develop software for estimating probability distributions over sequences of (sets of) phonological features using the principle of maximum entropy and making few assumptions about the independence of features. As above, a support vector machine with a linear kernel is used to turn these probability values into binary predictions. The software provided by these authors exposes many ``metaparameters'', operational decisions to be made in conducting an experiment with this system. To avoid any potential bias induced by the choice of these metaparameters, I use the same settings as used by \citet{Hayes2008a} in their similar study of English onset consonants. This includes the ``accuracy schedule'' [0.001, 0.01, 0.1, 0.2, 0.3] and no constraints on the number of constraints learned, as well English consonantal features used by \citeauthor{Hayes2008a}. Since this model is inherently stochastic, producing slightly different outcomes on each run, I follow the practice of \citeauthor{Hayes2008a} (p.~396) and report the lowest-accuracy of ten runs, though in general there is not a great deal of variation between individual runs.  

This model provides a small improvement in accuracy (0.877) and $F_1$ (0.703) compared to the alternation baseline, but it has lower recall (0.642) than this baseline, indicating that the \citeauthor{Hayes2008a} model introduces many false negatives. 
%; it assigns the lowest probability value to 122 unattested clusters.
More specifically, it suggests that this model prefers narrower phonotactic generalizations than those derived from alternations. For instance, unattested *[m.kl] is assigned the highest probability, whereas it is ruled out by \textsc{Coda Nasal Place Assimilation}. On the other hand, the \citeauthor{Hayes2008a} model assigns a low score to clusters like those found in \emph{hu}[z.b]\emph{and} and \emph{pla}[t.f]\emph{orm}, though they are both attested and consistent with all three alternation-derived constraints. 

\citet{Berent2012} identify an important defect with the \citeauthor{Hayes2008a} model used here: it does not make use of variables over features and thus fails to unify the constraint against geminate clusters. While the model may miss this unifying generalization, it correctly predicts all clusters containing geminates to be unattested. 

The results for the four models are summarized below.

\begin{example}
Summary of computational model accuracy: 

\vspace{0.5\baselineskip} 
\begin{tabular}{l | r r r r}
\toprule
                          & accuracy & precision & recall & $F_1$ \\ 
\midrule
Null baseline             & 0.813    & n.a.      & n.a.   & n.a.  \\
Alternations              & 0.857    & 0.401     & 0.708  & 0.512 \\
\citet{Pierrehumbert1994} & 0.839    & 0.210     & 0.750  & 0.328 \\
\citet{Hayes2008a}        & 0.877    & 0.777     & 0.642  & 0.703 \\
\bottomrule
\end{tabular}
\end{example}

One additional model was evaluated for this study. \citet{McGowan2011} claims that the type frequency of individual syllable contact clusters in English is predicted by the change of sonority in the clusters.\footnote{Thanks to Maria Gouskova for bringing this study to my attention.} Using a sonority scale proposed by \citet{Jespersen1904}, \citeauthor{McGowan2011} reports that clusters like [m.p], with sharply falling sonority, are slightly more frequent than those like [t.j], with rising sonority. However, \citeauthor{McGowan2011} finds that sonority of clusters accounts for only a tiny amount of the variance in cluster frequency ($R^2 = 0.077$). In this study, inclusion of sonority distance provided no improvement over the null baseline, and therefore was not considered further.

%            \subsubsection{Intramodal fit} %\subsection{Computational models}


Applying these derived constraints to the lexicon increases saturation from 18.8\% to 28.8\% and incurs only a handful of exceptions. Yet, despite the fact that ``attestation'' is the minority pattern, attempts to identify static lexical constraints, whether by hand or computational model, lend little additional predictive power. There is no evidence that the English syllable contact inventory is subject to any static constraint at all. I am forced to conclude that most, if not all, of 71.2\% of possible clusters which are unattested are accidental gaps. Below, I show that this state of affairs follows directly from the sparse distribution of codas and onsets. 

The three alternations targeting English syllable contact clusters have been shown to have the expected robust effects on the English lexicon. 


Whereas the three alternations targeting English syllable contact clusters have robust lexical reflexes as measured by the Fisher exact test, such tests reveal no evidence for the static dispreferences identified by \citeauthor{Pierrehumbert1994}. There is now a growing literature on computational cognitive models of phonotactic learning. By applying these models to the syllable contact data, it is possible to exhaustively search for statistically motivated constraints, elimianting the possibility of human error. 

To evaluate these statistical models, they are first trained on the set of attested clusters, then put to the task of classifying all possible clusters into those which are and are not attested. Four metrics are used to evaluate these computational models. Classification accuracy corresponds to the probability of a classification being either a true positive ($t.p.$), correctly predicting a cluster to occur, or a true negative ($t.n.$), correctly predicting a cluster to be absent. Incorrect outcomes are either false positives ($f.p.$), incorrectly predicting a cluster to occur, and false negatives ($f.n.$), incorrectly predicting a cluster to be absent. 

%\ex Binary classification contingency table: \\
%\begin{tabular}{c | l l} \toprule
%                             & \multicolumn{2}{c}{prediction} \\
%\midrule
%\multirow{2}{*}{observation} & true positive ($tp$)  & false positive ($fp$) \\
%                             & false negative ($fn$) & true negative ($tn$) \\
%\end{tabular}
%\begin{tabular}{|l l|}
%\toprule
%true positive ($tp$)  & false positive ($fp$) \\
%false negative ($fn$) & true negative ($tn$) \\
%\bottomrule
%\end{tabular}
%\xe 

\begin{example}
$\displaystyle \textrm{Accuracy} = \frac{t.p. + t.n.}{t.p. + t.n. + f.p. + f.n.}$
\end{example}

\noindent
A few other metrics can be used to assess classification performance. Precision corresponds to the probability that a cluster predicted to occur is observed in the data. 
\begin{example}
$\displaystyle \textrm{Precision} = \frac{t.p.}{t.p. + f.p.}$ 
\end{example}

\noindent
Recall (also known as sensitivity) is the probability a observed cluster is predicted to occur. 

\begin{example}
$\displaystyle \textrm{Recall} = \frac{t.p.}{t.p. + f.n.}$
\end{example}

It is possible to maximize recall at the expense of precision, by predicting more clusters to be attested, or to increase precision at the expense of recall by predicting fewer clusters to be attested. A common way to balance these concerns is the $F_1$ score, which is the harmonic mean of precision and recall.

\begin{example}
$\displaystyle F_1 = 2 \left( \frac{\textrm{Precision} \times \textrm{Recall}}{\textrm{Precision} + \textrm{Recall}}\right)$ 
\end{example}

\subsubsection{Null baseline}

The 840 possible clusters have an 18.7\% of saturdaion rate. As a consequence of this sparsity, a null model which classifies all clusters as unattested achieves 81.3\% accuracy without the need to posit any constraints either derived or static. Since this model does not have any positive classifications, either true or false, precision, recall, and $F_1$ are undefined.

\subsubsection{Alternations}

As was shown above, \textsc{Obstruent Voice Assimiation}, \textsc{Coda Nasal Place Assimilation}, and \textsc{Degemination} are reliable predictors of what clusters are and are not attested and have few lexical exceptions. I generate another baseline by predicting only those clusters which do not violate any of these generalizations to be attested. This provides a significant improvement to the null baseline (accuracy = 0.857). As precision (0.401) is smaller than recall (0.708), there are more false positives (i.e., unattested clusters that do not violate any of these three alternation-derived generalizations) than there are false negatives (i.e., attested exceptions to these derived generalizations). 

\subsubsection{\citet{Pierrehumbert1994}}

\citet{Pierrehumbert1994} first proposes to measure the well-formedness of syllable contact clusters by the independent probabilities of codas and onsets, and \citet{Coleman1997} extend this model to whole words. \citet{Pierrehumbert1994} uses both medial and initial/final codas and onsets to compute these probabilities, but I obtain a better fit to the data by only using medial coda and onset frequencies. Medial coda and onset probabilities are multipled to obtain a well-formedness score. This produces a continuous value between 0 and 1. To derive a binary classification from these values, I use a soft-margin support vector machine \citep{Cortes1995} with a linear kernel. This technique derives a ``decision stump'' providing the best single split into attested and an unattested clusters. This is not put forth as a cognitively plausible model for relating this expected probability to attestation: it is simply the upper bound that such a model could reach. 

Unfortunately, this score only provides a small improvement to the null baseline
 (accuracy = 0.839), and has a lower accuracy and $F_1$ (0.328) than the alternation baseline. 

\subsubsection{\citet{Hayes2008a}}

\citeauthor{Hayes2008a} develop a sophisticated method of learning phonotactic distributions from positive data. They develop software for estimating probability distributions over sequences of (sets of) phonological features using the principle of maximum entropy and making few assumptions about the independence of features. As above, a support vector machine with a linear kernel is used to turn these probability values into binary predictions. The software provided by these authors exposes many ``metaparameters'', operational decisions to be made in conducting an experiment with this system. To avoid any potential bias induced by the choice of these metaparameters, I use the same settings as used by \citet{Hayes2008a} in their similar study of English onset consonants. This includes the ``accuracy schedule'' [0.001, 0.01, 0.1, 0.2, 0.3] and no constraints on the number of constraints learned, as well English consonantal features used by \citeauthor{Hayes2008a}. Since this model is inherently stochastic, producing slightly different outcomes on each run, I follow the practice of \citeauthor{Hayes2008a} (p.~396) and report the lowest-accuracy of ten runs, though in general there is not a great deal of variation between individual runs.  

This model provides a small improvement in accuracy (0.877) and $F_1$ (0.703) compared to the alternation baseline, but it has lower recall (0.642) than this baseline, indicating that the \citeauthor{Hayes2008a} model introduces many false negatives. 
%; it assigns the lowest probability value to 122 unattested clusters.
More specifically, it suggests that this model prefers narrower phonotactic generalizations than those derived from alternations. For instance, unattested *[m.kl] is assigned the highest probability, whereas it is ruled out by \textsc{Coda Nasal Place Assimilation}. On the other hand, the \citeauthor{Hayes2008a} model assigns a low score to clusters like those found in \emph{hu}[z.b]\emph{and} and \emph{pla}[t.f]\emph{orm}, though they are both attested and consistent with all three alternation-derived constraints. 

\citet{Berent2012} identify an important defect with the \citeauthor{Hayes2008a} model used here: it does not make use of variables over features and thus fails to unify the constraint against geminate clusters. While the model may miss this unifying generalization, it correctly predicts all clusters containing geminates to be unattested. 

The results for the four models are summarized below.

\begin{example}
Summary of computational model accuracy: 

\vspace{0.5\baselineskip} 
\begin{tabular}{l | r r r r}
\toprule
                          & accuracy & precision & recall & $F_1$ \\ 
\midrule
Null baseline             & 0.813    & n.a.      & n.a.   & n.a.  \\
Alternations              & 0.857    & 0.401     & 0.708  & 0.512 \\
\citet{Pierrehumbert1994} & 0.839    & 0.210     & 0.750  & 0.328 \\
\citet{Hayes2008a}        & 0.877    & 0.777     & 0.642  & 0.703 \\
\bottomrule
\end{tabular}
\end{example}

One additional model was evaluated for this study. \citet{McGowan2011} claims that the type frequency of individual syllable contact clusters in English is predicted by the change of sonority in the clusters.\footnote{Thanks to Maria Gouskova for bringing this study to my attention.} Using a sonority scale proposed by \citet{Jespersen1904}, \citeauthor{McGowan2011} reports that clusters like [m.p], with sharply falling sonority, are slightly more frequent than those like [t.j], with rising sonority. However, \citeauthor{McGowan2011} finds that sonority of clusters accounts for only a tiny amount of the variance in cluster frequency ($R^2 = 0.077$). In this study, inclusion of sonority distance provided no improvement over the null baseline, and therefore was not considered further.


%    \section{Conclusion} %\section{Conclusions}

\citet{Borowsky1989} on peripherality.

productive \citet{Duanmu2008}

The above is a stark reminder.


A reasonable objection to the arguments presented in this chapter is to view the results as little more than an indictment of the results of \citealt{Pierrehumbert1994}.

However, the fact that state-of-the-art models are not capable of providing large improvements to the predictive accuracy indicates


 do not have a statistically significant effect on the shape of the English lexicon, but that experiments might turn up evidence that speakers have internalized 


shown to be aware of static constraints if they reached statistical significance. 



 statistically reliable static constraints could be identified 

The of



Regarding the historical developments,

\citet{Martin2007}

On the other hand, patterns created by sound change are not guaranteed to persist over time. 
One example of non-persistence is discussed by \citet{Iverson2005}.  
Around 1100 CE, Old English \emph{sk} became [ʃ]. 
This sound change introduced no alternations.
Since long vowels were not found before tautosyllabic syllable clusters at this time, there were no \emph{V\lm sk\#} words when the
 change was actuated, and \emph{V\lm sh\#} continues to be rare in Modern English. 
What \citeauthor{Iverson2005} observe, however, is that there is nothing apparently peripheral about words like \emph{leash} or \emph{whoosh}, and loanwords and coinages have readily filled the gap.

A third pattern is that a historically inherited pattern

\citet[][140]{Frisch2004} suggest that the strong tendency for the first and second consonants of the Arabic root to be non-identic
al is the ``a diachronic result of a processing constraint that disfavors repetition.'' 
Unfortunately, there is no evidence that this pattern is diachronic other than in the sense that it appears to be inherited from the proto-language: there is simply no Proto-Semitic verb roots with identical first and second consonants \citep[][178]{Greenberg1950}. 
In other Semitic languages, the inherited patern has experienced considerable erosion. 

%\begin{example}
%Tigrinya roots with identical first and second consonants \citep{Buckley1990a}:
%\begin{tabular}{l l l l}
%a. & lʌlʌw     & `scorch'                   & (< Ge'ez \emph{lʌwlʌw} `inflame')     \\
%   & mʌmʌy     & `winnow'                   & (< Ge'ez \emph{mʌymʌy} `distinguish') \\
%   & mʌmʌt & `pick out loot' & (< 
%b. & s’ʌs’ʌw   & `finish off a drink'       & (cf. \emph{s’ʌws’ʌw} `gulp down')           \\
%   & t’ʌt’ʌf   & `prune tree'               & (cf. \emph{t’ʌft’ʌf} `smear wall with mud') \\
%c. & kʷakʷkʷʌr & `waste away, be emaciated' & (cf. \emph{kʷarkʷʌr} `interrogate')         \\
%   & kakʷkʷɨʕ  & `clean wax from ears'      & (cf. \emph{kaʕkʷɨʕ} `start to form pods')   \\
%\end{tabular}
%\end{example}

Similar exceptions are found in 
%Amharic (\citealp[][?]{Broselow1984}, \citealp[][?]{McCarthy1985}) and 
Hebrew \citep[][29]{Bat-El2005}.

The next two chapters return to the question of synchrony, addressing the relationship between statistical patterns in the lexicon and speakers' behaviors when presented with underrepresented sequences.

accidental gaps
\citet[][419f.]{Hayes2008a}

%\chapter{Inflectional gaps in generative grammar} %\chapter{Inflectional gaps in generative grammar}

\section{A brief history of inflectional gaps} % 5.1: A brief history of inflectional gaps

\section{The null hypothesis}                  % 5.2: The null hypothesis

\section{Outline of Part II}                   % 5.3: Outline of Part II


%    \section{A brief history of inflectional gaps} %% 5.1: A brief history of inflectional gaps

%    \section{The null hypothesis} %% 5.2: The null hypothesis

%    \section{Outline of Part II} %% 5.3: Outline of Part II

%\chapter{A new theory of inflectional gaps} %% 6: A new theory of inflectional gaps
\chapter{A new theory of inflectional gaps}

\citet{Halle1973}

\section{Proposal}                  \section{Proposal}

\subsection{Unproducitivity and inflectional gaps}

\citet{GormanInPressa%}

\citet{Prasada1993}
\citet{Albright2003b}

\subsection{The Tolerance Principle}

\citet{Yang2005a}
\citet{Legate2011}





for irregular priming
\citet{Emmorey1989}
\citet{Allen2002}
\citet{Stockall2006}

against
\citet{Stanners1979}
\citet{Marslen-Wilson1993}

\citet{ELP}

SHOW IRREGULARS FASTER
\footnote{Thanks to Constantine Lignos}

SHOW IRREGULAR FREQUENCY EFFECTS

for regular non-freq effects
\citet{LignosSubmitted}

against
\citet{Alegre1999}
\citet{Gordon1999}
\citet{Baayen2008}

\subsubsection{Regularity and processing}

\citet{Yang2005a}

dual-root models \citet{Baayen1997b}

frequency effects differing

reduced priming (storage?)
\citet{Stanners1979}
\citet{Marslen-Wilson1993}

\citet{ODonnell2011a}
\citet{ODonnell2011b}

but everything shows that effect?

\citet{Alegre1999}
\citet{Gordon1999}




\section{Case studies in Tolerance} \section{Case studies in Tolerance}

\subsection{English past tense verbs}

dual-root models \citet{Baayen1997b}

frequency effects differing

reduced priming (storage?)
\citet{Stanners1979}
\citet{Marslen-Wilson1993}

\citet{ODonnell2011a}
\citet{ODonnell2011b}

but everything shows that effect?

\citet{Alegre1999}
\citet{Gordon1999}

\subsubsection{Acquisition evidence}

\citet{Gorman2011f}

\citet{Prasada1993}
\citet{Albright2003b}

\subsubsection{Processing evidence}

for irregular priming
\citet{Emmorey1989}
\citet{Allen2002}
\citet{Stockall2006}

against
\citet{Stanners1979}
\citet{Marslen-Wilson1993}

\citet{ELP}

SHOW IRREGULARS FASTER
\footnote{Thanks to Constantine Lignos}

SHOW IRREGULAR FREQUENCY EFFECTS

for regular non-freq effects
\citet{Lignos2011}

against
\citet{Alegre1999}
\citet{Gordon1999}
\citet{Baayen2008b}

\subsection{German noun plurals}

% umlaut general
\citet{Wright1907}
\citet{Twaddell1938}
\citet{Bach1970,Wiese1987}

% umlaut morpholexical
\citet{Hieble1957}
\citet{Zwicky1967}
\citet{Lieber1980}
\citet{Wurzel1970,Wurzel1981}
\citet{LessenKloeke1982}
\citet{Voyles1992}
\citet{Janda1998}

\emph{fordern} `to demand' (< OHG \emph{fordoran}
\emph{fördern} `to further' (< OHG \emph{furdiren})

For monomorphemic feminine nouns that appear at least once per million in the Mannheim corpus, 709 take \emph{-(e)n}, while 61 take some other suffix, well below the Tolerance threshold, 116. 

\subsubsection{Acquisition evidence}

multiple defaults
\citet{Wunderlich1999}
\citet{Bech1963}

% effect of gender
\citet{Kopcke1988,Dressler1999,Indefrey1999,Wiese1999,Spreng2004}

% acquisition of gender at 2 years of age
\citet{Mills1986}

% published data
\citet{Park1978} % two children
\citet{Kopcke1998} % Clahsen's lab data
\citet{Szagun2001} % cohort of 22
\citet{Clahsen1992} % one child, Simone

% new production data
\citet{Kauschke2011} % have
\citet{Bittner2000} % requested
\citet{Gawlitzek-Maiwald1994} % have

% elictation
\citet{Laaha2006} % have
\citet{Schaner-Wolles1988} % have
\citet{Mugdan1977} % have
\citet{MacWhinney1978} % have
\citet{Spreng2004} % have
\citet{Niedeggen-Bartke1999} % have
\citet{Ewers1999} % have

% waiting on 5 sources of data

% CHILDES
\citet{Miller1979}
\citet{Wagner1985}
\citet{Weissenborn1986}

% umlaut as phonology
\citet{Bach1970,Wiese1996a}
\citet{Janda1998}
and \citet{Hall1997}, in a review of \citet{Wiese1996b}

\subsubsection{Processing evidence}

\citet{Sonnenstuhl2002}

\citet{Weyerts1997}


\section{Evaluation}                \section{Evaluation}

\subsection{English past tense verbs}

``\emph{*Stridden} hovers in the mists of memory, tainting \emph{strided} without stepping onto the stage itself.'' \citep[][125}{Pinker1999}


\subsection{Spanish diphthongization and velar insertion}

\subsection{Modern Greek gen.sg. nouns}

\emph{Lexiko tis koinis neoellinikis} \citep{LKN}, henceforth LKN.

\citep{Sims2006}

\citet{Idsardi1992}


\section{Conclusions}               \section{Conclusions}

\citet{Halle1973}


%\label{theory}
%    \section{Proposal} %\section{Proposal}

\subsection{Unproducitivity and inflectional gaps}

\citet{GormanInPressa%}

\citet{Prasada1993}
\citet{Albright2003b}

\subsection{The Tolerance Principle}

\citet{Yang2005a}
\citet{Legate2011}





for irregular priming
\citet{Emmorey1989}
\citet{Allen2002}
\citet{Stockall2006}

against
\citet{Stanners1979}
\citet{Marslen-Wilson1993}

\citet{ELP}

SHOW IRREGULARS FASTER
\footnote{Thanks to Constantine Lignos}

SHOW IRREGULAR FREQUENCY EFFECTS

for regular non-freq effects
\citet{LignosSubmitted}

against
\citet{Alegre1999}
\citet{Gordon1999}
\citet{Baayen2008}

\subsubsection{Regularity and processing}

\citet{Yang2005a}

dual-root models \citet{Baayen1997b}

frequency effects differing

reduced priming (storage?)
\citet{Stanners1979}
\citet{Marslen-Wilson1993}

\citet{ODonnell2011a}
\citet{ODonnell2011b}

but everything shows that effect?

\citet{Alegre1999}
\citet{Gordon1999}




%        \subsection{Unproductivity and inflectional gaps} %\subsection{Unproducitivity and inflectional gaps}

\citet{GormanInPressa%}

\citet{Prasada1993}
\citet{Albright2003b}

%        \subsection{The Tolerance Principle} %\subsection{The Tolerance Principle}

\citet{Yang2005a}
\citet{Legate2011}





for irregular priming
\citet{Emmorey1989}
\citet{Allen2002}
\citet{Stockall2006}

against
\citet{Stanners1979}
\citet{Marslen-Wilson1993}

\citet{ELP}

SHOW IRREGULARS FASTER
\footnote{Thanks to Constantine Lignos}

SHOW IRREGULAR FREQUENCY EFFECTS

for regular non-freq effects
\citet{LignosSubmitted}

against
\citet{Alegre1999}
\citet{Gordon1999}
\citet{Baayen2008}

\subsubsection{Regularity and processing}

\citet{Yang2005a}

dual-root models \citet{Baayen1997b}

frequency effects differing

reduced priming (storage?)
\citet{Stanners1979}
\citet{Marslen-Wilson1993}

\citet{ODonnell2011a}
\citet{ODonnell2011b}

but everything shows that effect?

\citet{Alegre1999}
\citet{Gordon1999}



%    \section{Case studies in Tolerance} %\section{Case studies in Tolerance}

\subsection{English past tense verbs}

dual-root models \citet{Baayen1997b}

frequency effects differing

reduced priming (storage?)
\citet{Stanners1979}
\citet{Marslen-Wilson1993}

\citet{ODonnell2011a}
\citet{ODonnell2011b}

but everything shows that effect?

\citet{Alegre1999}
\citet{Gordon1999}

\subsubsection{Acquisition evidence}

\citet{Gorman2011f}

\citet{Prasada1993}
\citet{Albright2003b}

\subsubsection{Processing evidence}

for irregular priming
\citet{Emmorey1989}
\citet{Allen2002}
\citet{Stockall2006}

against
\citet{Stanners1979}
\citet{Marslen-Wilson1993}

\citet{ELP}

SHOW IRREGULARS FASTER
\footnote{Thanks to Constantine Lignos}

SHOW IRREGULAR FREQUENCY EFFECTS

for regular non-freq effects
\citet{Lignos2011}

against
\citet{Alegre1999}
\citet{Gordon1999}
\citet{Baayen2008b}

\subsection{German noun plurals}

% umlaut general
\citet{Wright1907}
\citet{Twaddell1938}
\citet{Bach1970,Wiese1987}

% umlaut morpholexical
\citet{Hieble1957}
\citet{Zwicky1967}
\citet{Lieber1980}
\citet{Wurzel1970,Wurzel1981}
\citet{LessenKloeke1982}
\citet{Voyles1992}
\citet{Janda1998}

\emph{fordern} `to demand' (< OHG \emph{fordoran}
\emph{fördern} `to further' (< OHG \emph{furdiren})

For monomorphemic feminine nouns that appear at least once per million in the Mannheim corpus, 709 take \emph{-(e)n}, while 61 take some other suffix, well below the Tolerance threshold, 116. 

\subsubsection{Acquisition evidence}

multiple defaults
\citet{Wunderlich1999}
\citet{Bech1963}

% effect of gender
\citet{Kopcke1988,Dressler1999,Indefrey1999,Wiese1999,Spreng2004}

% acquisition of gender at 2 years of age
\citet{Mills1986}

% published data
\citet{Park1978} % two children
\citet{Kopcke1998} % Clahsen's lab data
\citet{Szagun2001} % cohort of 22
\citet{Clahsen1992} % one child, Simone

% new production data
\citet{Kauschke2011} % have
\citet{Bittner2000} % requested
\citet{Gawlitzek-Maiwald1994} % have

% elictation
\citet{Laaha2006} % have
\citet{Schaner-Wolles1988} % have
\citet{Mugdan1977} % have
\citet{MacWhinney1978} % have
\citet{Spreng2004} % have
\citet{Niedeggen-Bartke1999} % have
\citet{Ewers1999} % have

% waiting on 5 sources of data

% CHILDES
\citet{Miller1979}
\citet{Wagner1985}
\citet{Weissenborn1986}

% umlaut as phonology
\citet{Bach1970,Wiese1996a}
\citet{Janda1998}
and \citet{Hall1997}, in a review of \citet{Wiese1996b}

\subsubsection{Processing evidence}

\citet{Sonnenstuhl2002}

\citet{Weyerts1997}


%        \subsection{English past tense} %\subsection{English past tense verbs}

dual-root models \citet{Baayen1997b}

frequency effects differing

reduced priming (storage?)
\citet{Stanners1979}
\citet{Marslen-Wilson1993}

\citet{ODonnell2011a}
\citet{ODonnell2011b}

but everything shows that effect?

\citet{Alegre1999}
\citet{Gordon1999}

\subsubsection{Acquisition evidence}

\citet{Gorman2011f}

\citet{Prasada1993}
\citet{Albright2003b}

\subsubsection{Processing evidence}

for irregular priming
\citet{Emmorey1989}
\citet{Allen2002}
\citet{Stockall2006}

against
\citet{Stanners1979}
\citet{Marslen-Wilson1993}

\citet{ELP}

SHOW IRREGULARS FASTER
\footnote{Thanks to Constantine Lignos}

SHOW IRREGULAR FREQUENCY EFFECTS

for regular non-freq effects
\citet{Lignos2011}

against
\citet{Alegre1999}
\citet{Gordon1999}
\citet{Baayen2008b}

%            %\subsubsection{Acquisition evidence} %\subsection{English past tense verbs}

dual-root models \citet{Baayen1997b}

frequency effects differing

reduced priming (storage?)
\citet{Stanners1979}
\citet{Marslen-Wilson1993}

\citet{ODonnell2011a}
\citet{ODonnell2011b}

but everything shows that effect?

\citet{Alegre1999}
\citet{Gordon1999}

\subsubsection{Acquisition evidence}

\citet{Gorman2011f}

\citet{Prasada1993}
\citet{Albright2003b}

\subsubsection{Processing evidence}

for irregular priming
\citet{Emmorey1989}
\citet{Allen2002}
\citet{Stockall2006}

against
\citet{Stanners1979}
\citet{Marslen-Wilson1993}

\citet{ELP}

SHOW IRREGULARS FASTER
\footnote{Thanks to Constantine Lignos}

SHOW IRREGULAR FREQUENCY EFFECTS

for regular non-freq effects
\citet{Lignos2011}

against
\citet{Alegre1999}
\citet{Gordon1999}
\citet{Baayen2008b}

%            %\subsubsection{Processing evidence} %\subsection{English past tense verbs}

dual-root models \citet{Baayen1997b}

frequency effects differing

reduced priming (storage?)
\citet{Stanners1979}
\citet{Marslen-Wilson1993}

\citet{ODonnell2011a}
\citet{ODonnell2011b}

but everything shows that effect?

\citet{Alegre1999}
\citet{Gordon1999}

\subsubsection{Acquisition evidence}

\citet{Gorman2011f}

\citet{Prasada1993}
\citet{Albright2003b}

\subsubsection{Processing evidence}

for irregular priming
\citet{Emmorey1989}
\citet{Allen2002}
\citet{Stockall2006}

against
\citet{Stanners1979}
\citet{Marslen-Wilson1993}

\citet{ELP}

SHOW IRREGULARS FASTER
\footnote{Thanks to Constantine Lignos}

SHOW IRREGULAR FREQUENCY EFFECTS

for regular non-freq effects
\citet{Lignos2011}

against
\citet{Alegre1999}
\citet{Gordon1999}
\citet{Baayen2008b}

%        \subsection{German noun plurals} %\subsection{German noun plurals}

% umlaut general
\citet{Wright1907}
\citet{Twaddell1938}
\citet{Bach1970,Wiese1987}

% umlaut morpholexical
\citet{Hieble1957}
\citet{Zwicky1967}
\citet{Lieber1980}
\citet{Wurzel1970,Wurzel1981}
\citet{LessenKloeke1982}
\citet{Voyles1992}
\citet{Janda1998}

\emph{fordern} `to demand' (< OHG \emph{fordoran}
\emph{fördern} `to further' (< OHG \emph{furdiren})

For monomorphemic feminine nouns that appear at least once per million in the Mannheim corpus, 709 take \emph{-(e)n}, while 61 take some other suffix, well below the Tolerance threshold, 116. 

\subsubsection{Acquisition evidence}

multiple defaults
\citet{Wunderlich1999}
\citet{Bech1963}

% effect of gender
\citet{Kopcke1988,Dressler1999,Indefrey1999,Wiese1999,Spreng2004}

% acquisition of gender at 2 years of age
\citet{Mills1986}

% published data
\citet{Park1978} % two children
\citet{Kopcke1998} % Clahsen's lab data
\citet{Szagun2001} % cohort of 22
\citet{Clahsen1992} % one child, Simone

% new production data
\citet{Kauschke2011} % have
\citet{Bittner2000} % requested
\citet{Gawlitzek-Maiwald1994} % have

% elictation
\citet{Laaha2006} % have
\citet{Schaner-Wolles1988} % have
\citet{Mugdan1977} % have
\citet{MacWhinney1978} % have
\citet{Spreng2004} % have
\citet{Niedeggen-Bartke1999} % have
\citet{Ewers1999} % have

% waiting on 5 sources of data

% CHILDES
\citet{Miller1979}
\citet{Wagner1985}
\citet{Weissenborn1986}

% umlaut as phonology
\citet{Bach1970,Wiese1996a}
\citet{Janda1998}
and \citet{Hall1997}, in a review of \citet{Wiese1996b}

\subsubsection{Processing evidence}

\citet{Sonnenstuhl2002}

\citet{Weyerts1997}

%            %\subsubsection{Acquisition evidence} %\subsection{German noun plurals}

% umlaut general
\citet{Wright1907}
\citet{Twaddell1938}
\citet{Bach1970,Wiese1987}

% umlaut morpholexical
\citet{Hieble1957}
\citet{Zwicky1967}
\citet{Lieber1980}
\citet{Wurzel1970,Wurzel1981}
\citet{LessenKloeke1982}
\citet{Voyles1992}
\citet{Janda1998}

\emph{fordern} `to demand' (< OHG \emph{fordoran}
\emph{fördern} `to further' (< OHG \emph{furdiren})

For monomorphemic feminine nouns that appear at least once per million in the Mannheim corpus, 709 take \emph{-(e)n}, while 61 take some other suffix, well below the Tolerance threshold, 116. 

\subsubsection{Acquisition evidence}

multiple defaults
\citet{Wunderlich1999}
\citet{Bech1963}

% effect of gender
\citet{Kopcke1988,Dressler1999,Indefrey1999,Wiese1999,Spreng2004}

% acquisition of gender at 2 years of age
\citet{Mills1986}

% published data
\citet{Park1978} % two children
\citet{Kopcke1998} % Clahsen's lab data
\citet{Szagun2001} % cohort of 22
\citet{Clahsen1992} % one child, Simone

% new production data
\citet{Kauschke2011} % have
\citet{Bittner2000} % requested
\citet{Gawlitzek-Maiwald1994} % have

% elictation
\citet{Laaha2006} % have
\citet{Schaner-Wolles1988} % have
\citet{Mugdan1977} % have
\citet{MacWhinney1978} % have
\citet{Spreng2004} % have
\citet{Niedeggen-Bartke1999} % have
\citet{Ewers1999} % have

% waiting on 5 sources of data

% CHILDES
\citet{Miller1979}
\citet{Wagner1985}
\citet{Weissenborn1986}

% umlaut as phonology
\citet{Bach1970,Wiese1996a}
\citet{Janda1998}
and \citet{Hall1997}, in a review of \citet{Wiese1996b}

\subsubsection{Processing evidence}

\citet{Sonnenstuhl2002}

\citet{Weyerts1997}

%            %\subsubsection{Processing evidence} %\subsection{German noun plurals}

% umlaut general
\citet{Wright1907}
\citet{Twaddell1938}
\citet{Bach1970,Wiese1987}

% umlaut morpholexical
\citet{Hieble1957}
\citet{Zwicky1967}
\citet{Lieber1980}
\citet{Wurzel1970,Wurzel1981}
\citet{LessenKloeke1982}
\citet{Voyles1992}
\citet{Janda1998}

\emph{fordern} `to demand' (< OHG \emph{fordoran}
\emph{fördern} `to further' (< OHG \emph{furdiren})

For monomorphemic feminine nouns that appear at least once per million in the Mannheim corpus, 709 take \emph{-(e)n}, while 61 take some other suffix, well below the Tolerance threshold, 116. 

\subsubsection{Acquisition evidence}

multiple defaults
\citet{Wunderlich1999}
\citet{Bech1963}

% effect of gender
\citet{Kopcke1988,Dressler1999,Indefrey1999,Wiese1999,Spreng2004}

% acquisition of gender at 2 years of age
\citet{Mills1986}

% published data
\citet{Park1978} % two children
\citet{Kopcke1998} % Clahsen's lab data
\citet{Szagun2001} % cohort of 22
\citet{Clahsen1992} % one child, Simone

% new production data
\citet{Kauschke2011} % have
\citet{Bittner2000} % requested
\citet{Gawlitzek-Maiwald1994} % have

% elictation
\citet{Laaha2006} % have
\citet{Schaner-Wolles1988} % have
\citet{Mugdan1977} % have
\citet{MacWhinney1978} % have
\citet{Spreng2004} % have
\citet{Niedeggen-Bartke1999} % have
\citet{Ewers1999} % have

% waiting on 5 sources of data

% CHILDES
\citet{Miller1979}
\citet{Wagner1985}
\citet{Weissenborn1986}

% umlaut as phonology
\citet{Bach1970,Wiese1996a}
\citet{Janda1998}
and \citet{Hall1997}, in a review of \citet{Wiese1996b}

\subsubsection{Processing evidence}

\citet{Sonnenstuhl2002}

\citet{Weyerts1997}

%    \section{Evaluation} %\section{Evaluation}

\subsection{English past tense verbs}

``\emph{*Stridden} hovers in the mists of memory, tainting \emph{strided} without stepping onto the stage itself.'' \citep[][125}{Pinker1999}


\subsection{Spanish diphthongization and velar insertion}

\subsection{Modern Greek gen.sg. nouns}

\emph{Lexiko tis koinis neoellinikis} \citep{LKN}, henceforth LKN.

\citep{Sims2006}

\citet{Idsardi1992}


%       \subsection{English pasts} %\subsection{English past tense verbs}

``\emph{*Stridden} hovers in the mists of memory, tainting \emph{strided} without stepping onto the stage itself.'' \citep[][125}{Pinker1999}


%            %?
%       \subsection{Spanish diphthongization and velar insertion} %\subsection{Spanish diphthongization and velar insertion}

%            %?
%       \subsection{Greek stress shift} %\subsection{Modern Greek gen.sg. nouns}

\emph{Lexiko tis koinis neoellinikis} \citep{LKN}, henceforth LKN.

\citep{Sims2006}

\citet{Idsardi1992}

%            %?
%       \subsection{Russian 1sg. imperfectives} %% 6.3.4: Russian 1sg. imperfectives 

``\emph{*Stridden} hovers in the mists of memory, tainting \emph{strided} without stepping onto the stage itself.'' \citep[][125}{Pinker1999}

\citet[][213]{Pinker1999}

% discussions
\citet{Halle1973}
\citet{Sims2006}

% data
\citet{Zaliznjak1977}
\citet{Ilola1989}



\citet{Baerman2008}

\citet{Baerman2010b} % chapter



\citet{Gorman2011e}

%            %?
%    \section{Conclusion} %\section{Conclusions}

\citet{Halle1973}

%\chapter{Other theories of inflectional gaps} %\label{igaps}

\section{Inviolable constraints}

\subsection{Proposal}

\citet{OT}
\citet{Orgun1999}

\subsubsection{Inviolable constraints in a rule-based phonology}

\citet{Vaux2008}
(28 ``violability'')

such as developed by Andrea \citet{Calabrese1995,Calabrese2005}. 

citing the OCP as an example...it's not clear wtf he's talking about
\citet{Goldsmith1976}

\citet{Nevins2003} % shm-gaps
\citet{Rice2007} % is being produced but avoided?

\citet[][9f.]{Baerman2010b}
\citet[][437]{Heath2005}

\citet{OT}
\citet{Orgun1999}

\subsubsection{MParse}

\citet{OT}
\citet{Raffelsiefen1999}
\citet{Raffelsiefen2004}

\citet{Wolf2009}

\subsubsection{Control}

\citet{Orgun1999}
\citet{Orgun2009}

\subsection{Evaluation}


% 7.1.3.1: Turkish and Turkish'

\citet{Orgun1999}
\citet{Orgun2009}

\citet{Ito1989b}

\subsubsection{Kinande reduplication}

``hurriedness or repetitive action \citep[][85; henceforth, MH]{Mutaka1990}.

\citet{Orgun1999} provide an analysis of a similar gap in Tiene, but it appears that the judgements they report are inferred, as \citet{Ellington1977}, the source for the generalization they report, provides no relevant judgements or examples, and his discussion suggests the gap is morphological in nature.
Since I have been unable to find any other published material on Tiene reduplication, Kinande, which is in comparison well documented, is presented here in its place.
But I predict that any close study of Tiene reduplication will reveal similar lexical exceptions.

\footnote{Thanks to Toni Cook for bringing this example to my attention.}

The material making up the so-called ``pre-stem'' will not be a major object of concern here: it consists of the nominal augment vowel (\textsc{Aug}), a verbal prefix (\textsc{Pre}), and where appropriate, the object marker (\textsc{OM}). Valence-chagning suffixes, known as ``extensions'', may appear to the right (in the below example, the applicative \textsc{Appl} and the reciprocal \textsc{Recp}), and the final \textsc{FV}, which indicates aspect. 

\ex Linear structure of the Kinande infinitive: \\
\begin{tabular}{l l l l l l l}
e- & -rí- & -bí & tum & -ir & -an & -a \\
\{\textsc{Aug} & \textsc{Pre} & \textsc{OM}\} & $\surd$\textsc{send} & \textsc{Appl} & \textsc{FV}\} \\
\multicolumn{7}{l}{``to send them to each other''}
\end{tabular}
\xe

bisyllabic
\citet{Downing2000}

\citet{Mutaka2008}

\subsubsection{Swedish indefinite adjectives}

\citet{Lowenadler2010a}
\citet{Lowenadler2010b}

\citet[][ chapter 5]{Lofstedt2010}

\citet[][7]{Buchanan2007}

\citet{Linell1973}
\citet{Sunden1869}
\citet{Hylen1920}
\citet{Cederschiold1912}
\citet{Brunnmark1826}
\citet{Eliasson1973}
\citet{Sigurd1965}
\citet{Teleman1999}
\citet{Kiefer1970}
\citet{Hellberg1972}
\citet{Iverson1981}
\citet{Pettersson1990}

\subsubsection{Tagalog infixation}

\citet{OT}

\citet{Orgun1999}
\citet{Orgun2009}

\citet{Ross1995}

\citet{Baerman2010b}


\section{Morphophonological ``confidence''}

\citet{Albright2003a}

\citet{Albright2001}

\subsection{Proposal}

\subsubsection{Confidence}

\subsubsection{Familiarity}
 % proposal
\subsection{Evaluation}

\citet{Boye2010}


\subsubsection{English past tense verbs}

\subsubsection{Spanish diphthongization}
 % evaluation

\section{Neutralization}

\subsection{Proposal}

\citet{Downing2011}

\subsubsection{Syntactic and semantic neutralization}

\citet{Legendre2009}

wh-example in English

% german verb

\footnote{Thanks to Tony Kroch for bringing this literature this to my attention.}

immobile complex verbs
bracketing paradox
compositionality
verb-second and back

\citet{Wunderlich1983}
\citet{Koopman1995}
\citet{McIntyre2002}
\citet{Dikken2003}
\citet{Muller2003}
\citet{Vikner2005}


\subsubsection{Morphological neutralization}

\citet{Anderson2010b}
% downing review

\subsection{Evaluation}

\subsubsection{Surmiran modals}

\citet{Anderson2010b}

\subsubsection{English *\emph{amn't}}

\citet{Bresnan2001}
\citet{Embick2010}


\section{Storage}

\citet{Boye2010}

\citet{Maiden2010}

\citet{Mithun2010}

\subsection{Proposal}

\subsubsection{Stem storage}

\citet{Boye2010}
\citet{Maiden2010}
\citet{Baronian2005}

see \citet{Embick2005} for discussion

\subsubsection{Whole word storage}

\citet{Mithun2010}
\citet{Gorman2011e}

\subsection{Evaluation}

\citet{Chan2008}

\subsubsection{English past tense verbs}

\citet{CELEX}

simulation

\subsubsection{Spanish diphthongization}


\label{igaps}

\section{Inviolable constraints}

\input{7.1.1}
\input{7.1.2}

\section{Morphophonological ``confidence''}

\citet{Albright2003a}

\citet{Albright2001}

\input{7.2.1} % proposal
\input{7.2.2} % evaluation

\section{Neutralization}

\input{7.3.1}
\input{7.3.2}

\section{Storage}

\citet{Boye2010}

\citet{Maiden2010}

\citet{Mithun2010}

\input{7.4.1}
\input{7.4.2}

\label{igaps}

\input{7.1}
\input{7.2}
\input{7.3}
\input{7.4}
\input{7.5}



%\label{insufficient}
%    \section{Inviolable constraints} %\section{Inviolable constraints}

\subsection{Proposal}

\citet{OT}
\citet{Orgun1999}

\subsubsection{Inviolable constraints in a rule-based phonology}

\citet{Vaux2008}
(28 ``violability'')

such as developed by Andrea \citet{Calabrese1995,Calabrese2005}. 

citing the OCP as an example...it's not clear wtf he's talking about
\citet{Goldsmith1976}

\citet{Nevins2003} % shm-gaps
\citet{Rice2007} % is being produced but avoided?

\citet[][9f.]{Baerman2010b}
\citet[][437]{Heath2005}

\citet{OT}
\citet{Orgun1999}

\subsubsection{MParse}

\citet{OT}
\citet{Raffelsiefen1999}
\citet{Raffelsiefen2004}

\citet{Wolf2009}

\subsubsection{Control}

\citet{Orgun1999}
\citet{Orgun2009}

\subsection{Evaluation}


% 7.1.3.1: Turkish and Turkish'

\citet{Orgun1999}
\citet{Orgun2009}

\citet{Ito1989b}

\subsubsection{Kinande reduplication}

``hurriedness or repetitive action \citep[][85; henceforth, MH]{Mutaka1990}.

\citet{Orgun1999} provide an analysis of a similar gap in Tiene, but it appears that the judgements they report are inferred, as \citet{Ellington1977}, the source for the generalization they report, provides no relevant judgements or examples, and his discussion suggests the gap is morphological in nature.
Since I have been unable to find any other published material on Tiene reduplication, Kinande, which is in comparison well documented, is presented here in its place.
But I predict that any close study of Tiene reduplication will reveal similar lexical exceptions.

\footnote{Thanks to Toni Cook for bringing this example to my attention.}

The material making up the so-called ``pre-stem'' will not be a major object of concern here: it consists of the nominal augment vowel (\textsc{Aug}), a verbal prefix (\textsc{Pre}), and where appropriate, the object marker (\textsc{OM}). Valence-chagning suffixes, known as ``extensions'', may appear to the right (in the below example, the applicative \textsc{Appl} and the reciprocal \textsc{Recp}), and the final \textsc{FV}, which indicates aspect. 

\ex Linear structure of the Kinande infinitive: \\
\begin{tabular}{l l l l l l l}
e- & -rí- & -bí & tum & -ir & -an & -a \\
\{\textsc{Aug} & \textsc{Pre} & \textsc{OM}\} & $\surd$\textsc{send} & \textsc{Appl} & \textsc{FV}\} \\
\multicolumn{7}{l}{``to send them to each other''}
\end{tabular}
\xe

bisyllabic
\citet{Downing2000}

\citet{Mutaka2008}

\subsubsection{Swedish indefinite adjectives}

\citet{Lowenadler2010a}
\citet{Lowenadler2010b}

\citet[][ chapter 5]{Lofstedt2010}

\citet[][7]{Buchanan2007}

\citet{Linell1973}
\citet{Sunden1869}
\citet{Hylen1920}
\citet{Cederschiold1912}
\citet{Brunnmark1826}
\citet{Eliasson1973}
\citet{Sigurd1965}
\citet{Teleman1999}
\citet{Kiefer1970}
\citet{Hellberg1972}
\citet{Iverson1981}
\citet{Pettersson1990}

\subsubsection{Tagalog infixation}

\citet{OT}

\citet{Orgun1999}
\citet{Orgun2009}

\citet{Ross1995}

\citet{Baerman2010b}


%        \subsection{Proposal} %\subsection{Proposal}

\citet{OT}
\citet{Orgun1999}

\subsubsection{Inviolable constraints in a rule-based phonology}

\citet{Vaux2008}
(28 ``violability'')

such as developed by Andrea \citet{Calabrese1995,Calabrese2005}. 

citing the OCP as an example...it's not clear wtf he's talking about
\citet{Goldsmith1976}

\citet{Nevins2003} % shm-gaps
\citet{Rice2007} % is being produced but avoided?

\citet[][9f.]{Baerman2010b}
\citet[][437]{Heath2005}

\citet{OT}
\citet{Orgun1999}

\subsubsection{MParse}

\citet{OT}
\citet{Raffelsiefen1999}
\citet{Raffelsiefen2004}

\citet{Wolf2009}

\subsubsection{Control}

\citet{Orgun1999}
\citet{Orgun2009}

%            \subsubsection{Inviolable constraints in a rule-based phonology} %\subsection{Proposal}

\citet{OT}
\citet{Orgun1999}

\subsubsection{Inviolable constraints in a rule-based phonology}

\citet{Vaux2008}
(28 ``violability'')

such as developed by Andrea \citet{Calabrese1995,Calabrese2005}. 

citing the OCP as an example...it's not clear wtf he's talking about
\citet{Goldsmith1976}

\citet{Nevins2003} % shm-gaps
\citet{Rice2007} % is being produced but avoided?

\citet[][9f.]{Baerman2010b}
\citet[][437]{Heath2005}

\citet{OT}
\citet{Orgun1999}

\subsubsection{MParse}

\citet{OT}
\citet{Raffelsiefen1999}
\citet{Raffelsiefen2004}

\citet{Wolf2009}

\subsubsection{Control}

\citet{Orgun1999}
\citet{Orgun2009}

%            \subsubsection{MParse} %\subsection{Proposal}

\citet{OT}
\citet{Orgun1999}

\subsubsection{Inviolable constraints in a rule-based phonology}

\citet{Vaux2008}
(28 ``violability'')

such as developed by Andrea \citet{Calabrese1995,Calabrese2005}. 

citing the OCP as an example...it's not clear wtf he's talking about
\citet{Goldsmith1976}

\citet{Nevins2003} % shm-gaps
\citet{Rice2007} % is being produced but avoided?

\citet[][9f.]{Baerman2010b}
\citet[][437]{Heath2005}

\citet{OT}
\citet{Orgun1999}

\subsubsection{MParse}

\citet{OT}
\citet{Raffelsiefen1999}
\citet{Raffelsiefen2004}

\citet{Wolf2009}

\subsubsection{Control}

\citet{Orgun1999}
\citet{Orgun2009}

%            \subsubsection{Control} %\subsection{Proposal}

\citet{OT}
\citet{Orgun1999}

\subsubsection{Inviolable constraints in a rule-based phonology}

\citet{Vaux2008}
(28 ``violability'')

such as developed by Andrea \citet{Calabrese1995,Calabrese2005}. 

citing the OCP as an example...it's not clear wtf he's talking about
\citet{Goldsmith1976}

\citet{Nevins2003} % shm-gaps
\citet{Rice2007} % is being produced but avoided?

\citet[][9f.]{Baerman2010b}
\citet[][437]{Heath2005}

\citet{OT}
\citet{Orgun1999}

\subsubsection{MParse}

\citet{OT}
\citet{Raffelsiefen1999}
\citet{Raffelsiefen2004}

\citet{Wolf2009}

\subsubsection{Control}

\citet{Orgun1999}
\citet{Orgun2009}

%       \subsection{Evaluation} %\section{Inviolable constraints}

\subsection{Proposal}

\citet{OT}
\citet{Orgun1999}

\subsubsection{Inviolable constraints in a rule-based phonology}

\citet{Vaux2008}
(28 ``violability'')

such as developed by Andrea \citet{Calabrese1995,Calabrese2005}. 

citing the OCP as an example...it's not clear wtf he's talking about
\citet{Goldsmith1976}

\citet{Nevins2003} % shm-gaps
\citet{Rice2007} % is being produced but avoided?

\citet[][9f.]{Baerman2010b}
\citet[][437]{Heath2005}

\citet{OT}
\citet{Orgun1999}

\subsubsection{MParse}

\citet{OT}
\citet{Raffelsiefen1999}
\citet{Raffelsiefen2004}

\citet{Wolf2009}

\subsubsection{Control}

\citet{Orgun1999}
\citet{Orgun2009}

\subsection{Evaluation}


% 7.1.3.1: Turkish and Turkish'

\citet{Orgun1999}
\citet{Orgun2009}

\citet{Ito1989b}

\subsubsection{Kinande reduplication}

``hurriedness or repetitive action \citep[][85; henceforth, MH]{Mutaka1990}.

\citet{Orgun1999} provide an analysis of a similar gap in Tiene, but it appears that the judgements they report are inferred, as \citet{Ellington1977}, the source for the generalization they report, provides no relevant judgements or examples, and his discussion suggests the gap is morphological in nature.
Since I have been unable to find any other published material on Tiene reduplication, Kinande, which is in comparison well documented, is presented here in its place.
But I predict that any close study of Tiene reduplication will reveal similar lexical exceptions.

\footnote{Thanks to Toni Cook for bringing this example to my attention.}

The material making up the so-called ``pre-stem'' will not be a major object of concern here: it consists of the nominal augment vowel (\textsc{Aug}), a verbal prefix (\textsc{Pre}), and where appropriate, the object marker (\textsc{OM}). Valence-chagning suffixes, known as ``extensions'', may appear to the right (in the below example, the applicative \textsc{Appl} and the reciprocal \textsc{Recp}), and the final \textsc{FV}, which indicates aspect. 

\ex Linear structure of the Kinande infinitive: \\
\begin{tabular}{l l l l l l l}
e- & -rí- & -bí & tum & -ir & -an & -a \\
\{\textsc{Aug} & \textsc{Pre} & \textsc{OM}\} & $\surd$\textsc{send} & \textsc{Appl} & \textsc{FV}\} \\
\multicolumn{7}{l}{``to send them to each other''}
\end{tabular}
\xe

bisyllabic
\citet{Downing2000}

\citet{Mutaka2008}

\subsubsection{Swedish indefinite adjectives}

\citet{Lowenadler2010a}
\citet{Lowenadler2010b}

\citet[][ chapter 5]{Lofstedt2010}

\citet[][7]{Buchanan2007}

\citet{Linell1973}
\citet{Sunden1869}
\citet{Hylen1920}
\citet{Cederschiold1912}
\citet{Brunnmark1826}
\citet{Eliasson1973}
\citet{Sigurd1965}
\citet{Teleman1999}
\citet{Kiefer1970}
\citet{Hellberg1972}
\citet{Iverson1981}
\citet{Pettersson1990}

\subsubsection{Tagalog infixation}

\citet{OT}

\citet{Orgun1999}
\citet{Orgun2009}

\citet{Ross1995}

\citet{Baerman2010b}


%            \subsubsection{Turkish and Turkish$'$} %\input{7.1.3.1}
%            \subsubsection{Kinande reduplication} %\input{7.1.3.2}
%            \subsubsection{Swedish indefinite adjectives} %\input{7.1.3.3}
%            \subsubsection{Tagalog infixation} %\input{7.1.3.4}
%    \section{Morphophonological confidence} %\section{Morphophonological ``confidence''}

\citet{Albright2003a}

\citet{Albright2001}

\subsection{Proposal}

\subsubsection{Confidence}

\subsubsection{Familiarity}
 % proposal
\subsection{Evaluation}

\citet{Boye2010}


\subsubsection{English past tense verbs}

\subsubsection{Spanish diphthongization}
 % evaluation

%        \subsection{Proposal} %\subsection{Proposal}

\subsubsection{Confidence}

\subsubsection{Familiarity}

%            \subsubsection{Confidence} %\subsection{Proposal}

\subsubsection{Confidence}

\subsubsection{Familiarity}

%            \subsubsection{Familiarity} %\subsection{Proposal}

\subsubsection{Confidence}

\subsubsection{Familiarity}

%        \subsection{Evaluation} %\subsection{Evaluation}

\citet{Boye2010}


\subsubsection{English past tense verbs}

\subsubsection{Spanish diphthongization}

%            \subsubsection{English past tense verbs} %\subsection{Evaluation}

\citet{Boye2010}


\subsubsection{English past tense verbs}

\subsubsection{Spanish diphthongization}

%            \subsubsection{Spanish diphthongization} %\subsection{Evaluation}

\citet{Boye2010}


\subsubsection{English past tense verbs}

\subsubsection{Spanish diphthongization}

%    \section{Neutralization} %\section{Neutralization}

\subsection{Proposal}

\citet{Downing2011}

\subsubsection{Syntactic and semantic neutralization}

\citet{Legendre2009}

wh-example in English

% german verb

\footnote{Thanks to Tony Kroch for bringing this literature this to my attention.}

immobile complex verbs
bracketing paradox
compositionality
verb-second and back

\citet{Wunderlich1983}
\citet{Koopman1995}
\citet{McIntyre2002}
\citet{Dikken2003}
\citet{Muller2003}
\citet{Vikner2005}


\subsubsection{Morphological neutralization}

\citet{Anderson2010b}
% downing review

\subsection{Evaluation}

\subsubsection{Surmiran modals}

\citet{Anderson2010b}

\subsubsection{English *\emph{amn't}}

\citet{Bresnan2001}
\citet{Embick2010}


%        \subsection{Proposal} %\subsection{Proposal}

\citet{Downing2011}

\subsubsection{Syntactic and semantic neutralization}

\citet{Legendre2009}

wh-example in English

% german verb

\footnote{Thanks to Tony Kroch for bringing this literature this to my attention.}

immobile complex verbs
bracketing paradox
compositionality
verb-second and back

\citet{Wunderlich1983}
\citet{Koopman1995}
\citet{McIntyre2002}
\citet{Dikken2003}
\citet{Muller2003}
\citet{Vikner2005}


\subsubsection{Morphological neutralization}

\citet{Anderson2010b}
% downing review

%            \subsubsection{Syntactic semantic neutralization} %\subsection{Proposal}

\citet{Downing2011}

\subsubsection{Syntactic and semantic neutralization}

\citet{Legendre2009}

wh-example in English

% german verb

\footnote{Thanks to Tony Kroch for bringing this literature this to my attention.}

immobile complex verbs
bracketing paradox
compositionality
verb-second and back

\citet{Wunderlich1983}
\citet{Koopman1995}
\citet{McIntyre2002}
\citet{Dikken2003}
\citet{Muller2003}
\citet{Vikner2005}


\subsubsection{Morphological neutralization}

\citet{Anderson2010b}
% downing review

%            \subsubsection{Morphological neutralization} %\subsection{Proposal}

\citet{Downing2011}

\subsubsection{Syntactic and semantic neutralization}

\citet{Legendre2009}

wh-example in English

% german verb

\footnote{Thanks to Tony Kroch for bringing this literature this to my attention.}

immobile complex verbs
bracketing paradox
compositionality
verb-second and back

\citet{Wunderlich1983}
\citet{Koopman1995}
\citet{McIntyre2002}
\citet{Dikken2003}
\citet{Muller2003}
\citet{Vikner2005}


\subsubsection{Morphological neutralization}

\citet{Anderson2010b}
% downing review

%        \subsection{Evaluation} %\subsection{Evaluation}

\subsubsection{Surmiran modals}

\citet{Anderson2010b}

\subsubsection{English *\emph{amn't}}

\citet{Bresnan2001}
\citet{Embick2010}

%            \subsubsection{Surmiran modals} %\subsection{Evaluation}

\subsubsection{Surmiran modals}

\citet{Anderson2010b}

\subsubsection{English *\emph{amn't}}

\citet{Bresnan2001}
\citet{Embick2010}

%            \subsubsection{English *\emph{amn't}} %\subsection{Evaluation}

\subsubsection{Surmiran modals}

\citet{Anderson2010b}

\subsubsection{English *\emph{amn't}}

\citet{Bresnan2001}
\citet{Embick2010}

%    \section{Storage} %\section{Storage}

\citet{Boye2010}

\citet{Maiden2010}

\citet{Mithun2010}

\subsection{Proposal}

\subsubsection{Stem storage}

\citet{Boye2010}
\citet{Maiden2010}
\citet{Baronian2005}

see \citet{Embick2005} for discussion

\subsubsection{Whole word storage}

\citet{Mithun2010}
\citet{Gorman2011e}

\subsection{Evaluation}

\citet{Chan2008}

\subsubsection{English past tense verbs}

\citet{CELEX}

simulation

\subsubsection{Spanish diphthongization}


%        \subsection{Proposal} %\subsection{Proposal}

\subsubsection{Stem storage}

\citet{Boye2010}
\citet{Maiden2010}
\citet{Baronian2005}

see \citet{Embick2005} for discussion

\subsubsection{Whole word storage}

\citet{Mithun2010}
\citet{Gorman2011e}

%            \subsubsection{Stem storage} %\subsection{Proposal}

\subsubsection{Stem storage}

\citet{Boye2010}
\citet{Maiden2010}
\citet{Baronian2005}

see \citet{Embick2005} for discussion

\subsubsection{Whole word storage}

\citet{Mithun2010}
\citet{Gorman2011e}

%            \subsubsection{Whole word storage} %\subsection{Proposal}

\subsubsection{Stem storage}

\citet{Boye2010}
\citet{Maiden2010}
\citet{Baronian2005}

see \citet{Embick2005} for discussion

\subsubsection{Whole word storage}

\citet{Mithun2010}
\citet{Gorman2011e}

%        \subsection{Evaluation} %\subsection{Evaluation}

\citet{Chan2008}

\subsubsection{English past tense verbs}

\citet{CELEX}

simulation

\subsubsection{Spanish diphthongization}

%            \subsubsection{English past tense verbs} %\subsection{Evaluation}

\citet{Chan2008}

\subsubsection{English past tense verbs}

\citet{CELEX}

simulation

\subsubsection{Spanish diphthongization}

%            \subsubsection{Spanish diphthongization} %\subsection{Evaluation}

\citet{Chan2008}

\subsubsection{English past tense verbs}

\citet{CELEX}

simulation

\subsubsection{Spanish diphthongization}

%    \section{Conclusion} %\label{igaps}

\section{Inviolable constraints}

\subsection{Proposal}

\citet{OT}
\citet{Orgun1999}

\subsubsection{Inviolable constraints in a rule-based phonology}

\citet{Vaux2008}
(28 ``violability'')

such as developed by Andrea \citet{Calabrese1995,Calabrese2005}. 

citing the OCP as an example...it's not clear wtf he's talking about
\citet{Goldsmith1976}

\citet{Nevins2003} % shm-gaps
\citet{Rice2007} % is being produced but avoided?

\citet[][9f.]{Baerman2010b}
\citet[][437]{Heath2005}

\citet{OT}
\citet{Orgun1999}

\subsubsection{MParse}

\citet{OT}
\citet{Raffelsiefen1999}
\citet{Raffelsiefen2004}

\citet{Wolf2009}

\subsubsection{Control}

\citet{Orgun1999}
\citet{Orgun2009}

\subsection{Evaluation}


% 7.1.3.1: Turkish and Turkish'

\citet{Orgun1999}
\citet{Orgun2009}

\citet{Ito1989b}

\subsubsection{Kinande reduplication}

``hurriedness or repetitive action \citep[][85; henceforth, MH]{Mutaka1990}.

\citet{Orgun1999} provide an analysis of a similar gap in Tiene, but it appears that the judgements they report are inferred, as \citet{Ellington1977}, the source for the generalization they report, provides no relevant judgements or examples, and his discussion suggests the gap is morphological in nature.
Since I have been unable to find any other published material on Tiene reduplication, Kinande, which is in comparison well documented, is presented here in its place.
But I predict that any close study of Tiene reduplication will reveal similar lexical exceptions.

\footnote{Thanks to Toni Cook for bringing this example to my attention.}

The material making up the so-called ``pre-stem'' will not be a major object of concern here: it consists of the nominal augment vowel (\textsc{Aug}), a verbal prefix (\textsc{Pre}), and where appropriate, the object marker (\textsc{OM}). Valence-chagning suffixes, known as ``extensions'', may appear to the right (in the below example, the applicative \textsc{Appl} and the reciprocal \textsc{Recp}), and the final \textsc{FV}, which indicates aspect. 

\ex Linear structure of the Kinande infinitive: \\
\begin{tabular}{l l l l l l l}
e- & -rí- & -bí & tum & -ir & -an & -a \\
\{\textsc{Aug} & \textsc{Pre} & \textsc{OM}\} & $\surd$\textsc{send} & \textsc{Appl} & \textsc{FV}\} \\
\multicolumn{7}{l}{``to send them to each other''}
\end{tabular}
\xe

bisyllabic
\citet{Downing2000}

\citet{Mutaka2008}

\subsubsection{Swedish indefinite adjectives}

\citet{Lowenadler2010a}
\citet{Lowenadler2010b}

\citet[][ chapter 5]{Lofstedt2010}

\citet[][7]{Buchanan2007}

\citet{Linell1973}
\citet{Sunden1869}
\citet{Hylen1920}
\citet{Cederschiold1912}
\citet{Brunnmark1826}
\citet{Eliasson1973}
\citet{Sigurd1965}
\citet{Teleman1999}
\citet{Kiefer1970}
\citet{Hellberg1972}
\citet{Iverson1981}
\citet{Pettersson1990}

\subsubsection{Tagalog infixation}

\citet{OT}

\citet{Orgun1999}
\citet{Orgun2009}

\citet{Ross1995}

\citet{Baerman2010b}


\section{Morphophonological ``confidence''}

\citet{Albright2003a}

\citet{Albright2001}

\subsection{Proposal}

\subsubsection{Confidence}

\subsubsection{Familiarity}
 % proposal
\subsection{Evaluation}

\citet{Boye2010}


\subsubsection{English past tense verbs}

\subsubsection{Spanish diphthongization}
 % evaluation

\section{Neutralization}

\subsection{Proposal}

\citet{Downing2011}

\subsubsection{Syntactic and semantic neutralization}

\citet{Legendre2009}

wh-example in English

% german verb

\footnote{Thanks to Tony Kroch for bringing this literature this to my attention.}

immobile complex verbs
bracketing paradox
compositionality
verb-second and back

\citet{Wunderlich1983}
\citet{Koopman1995}
\citet{McIntyre2002}
\citet{Dikken2003}
\citet{Muller2003}
\citet{Vikner2005}


\subsubsection{Morphological neutralization}

\citet{Anderson2010b}
% downing review

\subsection{Evaluation}

\subsubsection{Surmiran modals}

\citet{Anderson2010b}

\subsubsection{English *\emph{amn't}}

\citet{Bresnan2001}
\citet{Embick2010}


\section{Storage}

\citet{Boye2010}

\citet{Maiden2010}

\citet{Mithun2010}

\subsection{Proposal}

\subsubsection{Stem storage}

\citet{Boye2010}
\citet{Maiden2010}
\citet{Baronian2005}

see \citet{Embick2005} for discussion

\subsubsection{Whole word storage}

\citet{Mithun2010}
\citet{Gorman2011e}

\subsection{Evaluation}

\citet{Chan2008}

\subsubsection{English past tense verbs}

\citet{CELEX}

simulation

\subsubsection{Spanish diphthongization}


\label{igaps}

\section{Inviolable constraints}

\input{7.1.1}
\input{7.1.2}

\section{Morphophonological ``confidence''}

\citet{Albright2003a}

\citet{Albright2001}

\input{7.2.1} % proposal
\input{7.2.2} % evaluation

\section{Neutralization}

\input{7.3.1}
\input{7.3.2}

\section{Storage}

\citet{Boye2010}

\citet{Maiden2010}

\citet{Mithun2010}

\input{7.4.1}
\input{7.4.2}

\label{igaps}

\input{7.1}
\input{7.2}
\input{7.3}
\input{7.4}
\input{7.5}



\appendix
\renewcommand{\arraystretch}{0.25}
\chapter{Data from Chapter \ref{clusters}} \chapter{English wordlikeness ratings} 
\label{ratings}

\section{\citet{Albright2007}}

\begin{longtable}{l@{ }l@{ }l@{ }lrrrrrrr}
\toprule
   &   &    &    & lexical & $-$log $p$ & $-$log $p$ & gross  & rating    \\
   &   &    &    & density & (bigram)   & (MaxEnt)   & status & (7-point) \\
\midrule
hi \\
\bottomrule
\end{longtable}

\section{\citet{Albright2003b}, norming study} 

\begin{longtable}{l@{ }l@{ }l@{ }lrrrrrrr}
\toprule
   &    &    &    &    &   & lexical & $-$log $p$ & $-$log $p$ & gross  & rating    \\
   &    &    &    &    &   & density & (bigram)   & (MaxEnt)   & status & (7-point) \\
\midrule
hi \\
\bottomrule
\end{longtable}

\section{\citet{Scholes1966}, experiment 5}    

\begin{longtable}{l@{ }l@{ }l@{ }lrrrrrrr}
\toprule
  &   &    &     & lexical & $-$log $p$ & $-$log $p$ & gross  & rating   \\
  &   &    &     & density & (bigram)   & (MaxEnt)   & status & (binary) \\
\midrule 
hi \\
\bottomrule
\end{longtable}

 \label{A}
%\chapter{Data from Chapter \ref{vowels}} %% B: Data from Chapter 2

%\chapter{Data from Chapter \ref{wordlikeness}} 
%    \section{\citet{Greenberg1964} experiment B} \section{\citet{Greenberg1964} experiment B}  

\begin{longtable}{l@{ } l@{ } l@{ } l r r r r r r} 
\toprule
  &   &    &     & lexical & $-$log $p$ & $-$log $p$ & gross & rating \\
&&&& density & (bigram) & (MaxEnt) & phonotactics & (MagE) \\
\midrule
S  & W & IH & T  & 19 & 15.975 & 0.000  & valid   & $-$25.10 \\
K  & L & AE & B  & 12 & 15.641 & 0.000  & valid   & $-$28.15 \\
S  & L & AH & K  & 14 & 13.905 & 0.000  & valid   & $-$29.16 \\
S  & W & AE & CH &  3 & 20.051 & 0.000  & valid   & $-$29.25 \\
B  & R & AH & D  & 16 & 14.144 & 0.000  & valid   & $-$33.40 \\
K  & N & AE & P  &  5 & 20.884 & 10.877 & invalid & $-$33.90 \\
K  & L & EH & B  &  4 & 17.325 & 0,000  & valid   & $-$32.92 \\
T  & R & UW & G  & 10 & 17.383 & 0.000  & valid   & $-$41.16 \\
S  & R & AH & M  &  9 & 20.200 & 5.010  & invalid & $-$46.12 \\
TH & Y & AH & NG &  1 & 23.470 & 10.121 & invalid & $-$46.49 \\
TH & W & AE & ZH &  0 & 27.819 & 3.876  & invalid & $-$63.19 \\
ZH & R & IH & K  &  8 & 29.197 & 13.640 & invalid & $-$67.59 \\
CH & W & UW & P  &  2 & 29.071 & 7.467  & invalid & $-$87.97 \\
\bottomrule
\end{longtable}

%    \section{\citet{Scholes1966} experiment 5} \chapter{Data from Chapter \ref{wordlikeness}}

\section{\citet{Greenberg1964} experiment B}  

\begin{longtable}{l@{ } l@{ } l@{ } l r r r r r r} 
\toprule
  &   &    &     & lexical & $-$log $p$ & $-$log $p$ & gross & rating \\
&&&& density & (bigram) & (MaxEnt) & phonotactics & (MagE) \\
\midrule
S  & W & IH & T  & 19 & 15.975 & 0.000  & valid   & $-$25.10 \\
K  & L & AE & B  & 12 & 15.641 & 0.000  & valid   & $-$28.15 \\
S  & L & AH & K  & 14 & 13.905 & 0.000  & valid   & $-$29.16 \\
S  & W & AE & CH &  3 & 20.051 & 0.000  & valid   & $-$29.25 \\
B  & R & AH & D  & 16 & 14.144 & 0.000  & valid   & $-$33.40 \\
K  & N & AE & P  &  5 & 20.884 & 10.877 & invalid & $-$33.90 \\
K  & L & EH & B  &  4 & 17.325 & 0,000  & valid   & $-$32.92 \\
T  & R & UW & G  & 10 & 17.383 & 0.000  & valid   & $-$41.16 \\
S  & R & AH & M  &  9 & 20.200 & 5.010  & invalid & $-$46.12 \\
TH & Y & AH & NG &  1 & 23.470 & 10.121 & invalid & $-$46.49 \\
TH & W & AE & ZH &  0 & 27.819 & 3.876  & invalid & $-$63.19 \\
ZH & R & IH & K  &  8 & 29.197 & 13.640 & invalid & $-$67.59 \\
CH & W & UW & P  &  2 & 29.071 & 7.467  & invalid & $-$87.97 \\
\bottomrule
\end{longtable}

\chapter{Data from Chapter \ref{wordlikeness}}

\section{\citet{Greenberg1964} experiment B}  

\begin{longtable}{l@{ } l@{ } l@{ } l r r r r r r} 
\toprule
  &   &    &     & lexical & $-$log $p$ & $-$log $p$ & gross & rating \\
&&&& density & (bigram) & (MaxEnt) & phonotactics & (MagE) \\
\midrule
S  & W & IH & T  & 19 & 15.975 & 0.000  & valid   & $-$25.10 \\
K  & L & AE & B  & 12 & 15.641 & 0.000  & valid   & $-$28.15 \\
S  & L & AH & K  & 14 & 13.905 & 0.000  & valid   & $-$29.16 \\
S  & W & AE & CH &  3 & 20.051 & 0.000  & valid   & $-$29.25 \\
B  & R & AH & D  & 16 & 14.144 & 0.000  & valid   & $-$33.40 \\
K  & N & AE & P  &  5 & 20.884 & 10.877 & invalid & $-$33.90 \\
K  & L & EH & B  &  4 & 17.325 & 0,000  & valid   & $-$32.92 \\
T  & R & UW & G  & 10 & 17.383 & 0.000  & valid   & $-$41.16 \\
S  & R & AH & M  &  9 & 20.200 & 5.010  & invalid & $-$46.12 \\
TH & Y & AH & NG &  1 & 23.470 & 10.121 & invalid & $-$46.49 \\
TH & W & AE & ZH &  0 & 27.819 & 3.876  & invalid & $-$63.19 \\
ZH & R & IH & K  &  8 & 29.197 & 13.640 & invalid & $-$67.59 \\
CH & W & UW & P  &  2 & 29.071 & 7.467  & invalid & $-$87.97 \\
\bottomrule
\end{longtable}

\chapter{Data from Chapter \ref{wordlikeness}}

\input{C.1}
\input{C.2}
\input{C.3}

\begin{longtable}{l@{ } l@{ } l@{ } l@{ } l@{ } l r r r r r r} 
\toprule
\multicolumn{6}{l}{\multirow{2}{*}{word}} & lexical & $-$log $p$ & $-$log $p$ \\
&&&&&& density & (bigram) & (MaxEnt) & phonotactics & rating \\
\midrule
S  & L  & EY & M  &    &   & 10 & 16.570 & 0.000 & valid   & 5.84 \\
W  & IH & S  &    &    &   & 37 & 13.297 & 0.000 & valid   & 5.84 \\
P  & IH & N  & T  &    &   & 20 & 12.577 & 0.000 & valid   & 5.67 \\
P  & AE & NG & K  &    &   & 17 & 11.868 & 0.000 & valid   & 5.63 \\
S  & T  & IH & P  &    &   & 15 & 12.998 & 0.000 & valid   & 5.53 \\
M  & IH & P  &    &    &   & 34 & 12.581 & 0.000 & valid   & 5.47 \\
S  & T  & AY & R  &    &   &  9 & 15.698 & 0.000 & valid   & 5.47 \\
M  & ER & N  &    &    &   & 40 & 12.140 & 0.000 & valid   & 5.42 \\
P  & L  & EY & K  &    &   & 15 & 16.127 & 0.000 & valid   & 5.39 \\
S  & N  & EH & L  &    &   &  9 & 16.220 & 0.000 & valid   & 5.32 \\
S  & T  & IH & N  &    &   & 17 & 11.124 & 0.000 & valid   & 5.28 \\
R  & AE & S  & K  &    &   & 12 & 14.624 & 0.000 & valid   & 5.21 \\
T  & R  & IH & S  & K  &   &  3 & 16.760 & 0.000 & valid   & 5.21 \\
S  & P  & A  & E  & K  &   & 16 & 13.254 & 0.000 & valid   & 5.16 \\
D  & EY & P  &    &    &   & 19 & 14.047 & 0.000 & valid   & 5.11 \\
G  & EH & R  &    &    &   & 36 & 12.084 & 0.000 & valid   & 5.11 \\
G  & L  & IH & T  &    &   &  6 & 15.497 & 0.000 & valid   & 5.11 \\
S  & K  & EH & L  &    &   & 13 & 14.049 & 0.000 & valid   & 5.11 \\
SH & ER & N  &    &    &   & 26 & 14.441 & 0.000 & valid   & 5.11 \\
T  & AA & R  & K  &    &   & 14 & 16.148 & 0.000 & valid   & 5.11 \\
CH & EY & K  &    &    &   & 22 & 15.137 & 0.000 & valid   & 5.05 \\
G  & L  & IY & D  &    &   & 13 & 16.794 & 0.000 & valid   & 5.05 \\
G  & R  & AY & N  & T  &   &  3 & 17.164 & 0.000 & valid   & 5.00 \\
P  & R  & IY & K  &    &   & 14 & 14.722 & 0.000 & valid   & 5.00 \\
SH & IH & L  & K  &    &   &  3 & 17.983 & 0.000 & valid   & 4.89 \\
D  & AY & Z  &    &    &   & 38 & 13.332 & 0.000 & valid   & 4.84 \\
N  & EY & S  &    &    &   & 17 & 15.495 & 0.000 & valid   & 4.84 \\
T  & AH & NG & K  &    &   & 18 & 13.400 & 0.000 & valid   & 4.84 \\
S  & K  & W  & IH & L  &   &  5 & 17.684 & 0.000 & valid   & 4.83 \\
L  & AH & M  &    &    &   & 35 & 10.329 & 0.000 & valid   & 4.79 \\
P  & AH & M  &    &    &   & 28 & 10.006 & 0.000 & valid   & 4.79 \\
S  & P  & L  & IH & NG &   & 13 & 17.093 & 0.000 & valid   & 4.72 \\
G  & R  & EH & L  &    &   &  6 & 13.686 & 0.000 & valid   & 4.63 \\
T  & EH & SH &    &    &   &  8 & 15.421 & 0.000 & valid   & 4.63 \\
T  & IY & P  &    &    &   & 29 & 12.914 & 0.000 & valid   & 4.63 \\
B  & AY & Z  &    &    &   & 47 & 12.688 & 0.000 & valid   & 4.58 \\
G  & L  & IH & P  &    &   &  6 & 16.393 & 0.000 & valid   & 4.53 \\
CH & AY & N  & D  &    &   & 15 & 16.768 & 0.000 & valid   & 4.37 \\
P  & L  & IH & M  &    &   &  6 & 14.320 & 0.000 & valid   & 4.37 \\
G  & UW & D  &    &    &   & 21 & 14.519 & 0.000 & valid   & 4.32 \\
B  & L  & EY & F  &    &   &  6 & 17.997 & 0.000 & valid   & 4.21 \\
G  & EH & Z  &    &    &   &  9 & 16.619 & 0.000 & valid   & 4.21 \\
D  & R  & IH & T  &    &   &  7 & 14.369 & 0.000 & valid   & 4.16 \\
F  & L  & IY & P  &    &   &  9 & 15.224 & 0.000 & valid   & 4.16 \\
Z  & EY &    &    &    &   & 35 & 13.153 & 2.769 & valid   & 4.16 \\
S  & K  & R  & AY & D  &   &  4 & 17.976 & 0.000 & valid   & 4.11 \\
K  & IH & V  &    &    &   & 12 & 14.008 & 0.000 & valid   & 4.05 \\
F  & L  & EH & T  &    &   & 15 & 14.449 & 0.000 & valid   & 4.00 \\
N  & OW & L  & D  &    &   & 23 & 19.131 & 0.000 & valid   & 4.00 \\
S  & K  & IH & K  &    &   & 13 & 14.146 & 0.000 & valid   & 4.00 \\
B  & R  & EH & JH &    &   &  5 & 16.438 & 0.000 & valid   & 3.95 \\
K  & W  & IY & D  &    &   &  7 & 16.694 & 0.000 & valid   & 3.95 \\
S  & K  & OY & L  &    &   &  7 & 16.937 & 0.000 & valid   & 3.89 \\
D  & R  & AY & S  &    &   & 16 & 16.479 & 0.000 & valid   & 3.84 \\
F  & L  & IH & JH &    &   &  5 & 16.594 & 0.000 & valid   & 3.79 \\
B  & L  & IH & G  &    &   &  4 & 15.487 & 0.000 & valid   & 3.53 \\
Z  & EY & P  & S  &    &   &  5 & 24.147 & 2.769 & valid   & 3.47 \\
CH & UW & L  &    &    &   & 13 & 14.085 & 0.000 & valid   & 3.42 \\
SH & AY & N  & T  &    &   &  5 & 16.438 & 0.000 & valid   & 3.42 \\
SH & R  & UH & K  & S  &   &  4 & 22.453 & 2.127 & invalid & 3.32 \\
G  & W  & EH & N  & JH &   &  1 & 23.165 & 2.929 & invalid & 3.32 \\
N  & AH & NG &    &    &   & 15 & 14.633 & 0.000 & valid   & 3.28 \\
S  & K  & W  & AA & L  & K &  0 & 24.319 & 0.000 & invalid & 3.26 \\
T  & W  & UW &    &    &   &  8 & 16.779 & 0.000 & valid   & 3.17 \\
S  & M  & AH & M  &    &   &  6 & 14.638 & 0.000 & valid   & 3.05 \\
S  & N  & OY & K  & S  &   &  3 & 25.335 & 0.000 & invalid & 3.00 \\
S  & F  & UW & N  & D  &   &  0 & 23.906 & 6.703 & invalid & 2.94 \\
P  & W  & IH & P  &    &   &  2 & 24.135 & 4.818 & invalid & 2.89 \\
R  & AY & N  & T  &    &   &  8 & 14.815 & 0.000 & valid   & 2.89 \\
S  & K  & L  & UW & N  & D &  0 & 20.721 & 3.046 & invalid & 2.83 \\
S  & M  & IY & R  & G  &   &  0 & 27.414 & 0.000 & invalid & 2.79 \\
F  & R  & IH & L  & G  &   &  2 & 22.117 & 0.000 & invalid & 2.68 \\
SH & W  & UW & JH &    &   &  0 & 30.868 & 4.878 & invalid & 2.68 \\
TH & R  & OY & K  & S  &   &  0 & 25.623 & 1.907 & invalid & 2.68 \\
T  & R  & IH & L  & B  &   &  1 & 20.903 & 0.000 & invalid & 2.63 \\
K  & R  & IH & L  & G  &   &  0 & 21.254 & 0.000 & invalid & 2.58 \\
S  & M  & EH & R  & G  &   &  0 & 21.873 & 0.000 & invalid & 2.58 \\
TH & W  & IY & K  & S  &   &  2 & 25.867 & 3.879 & invalid & 2.53 \\
S  & M  & EH & R  & F  &   &  0 & 23.030 & 0.000 & invalid & 2.47 \\
S  & M  & IY & L  & TH &   &  0 & 25.712 & 0.000 & invalid & 2.47 \\
P  & L  & OW & M  & F  &   &  0 & 21.857 & 0.000 & invalid & 2.42 \\
D  & W  & OW & JH &    &   &  0 & 24.150 & 2.929 & invalid & 2.29 \\
P  & L  & OW & N  & TH &   &  0 & 20.893 & 0.000 & invalid & 2.26 \\
TH & EY & P  & T  &    &   &  3 & 23.473 & 1.907 & invalid & 2.26 \\
S  & M  & IY & N  & TH &   &  0 & 23.098 & 0.000 & invalid & 2.06 \\
S  & P  & R  & AA & R  & F &  0 & 23.675 & 0.000 & valid   & 2.05 \\
P  & W  & AH & JH &    &   &  0 & 25.463 & 4.818 & invalid & 1.74 \\
\bottomrule
\end{longtable}


\begin{longtable}{l@{ } l@{ } l@{ } l@{ } l@{ } l r r r r r r} 
\toprule
\multicolumn{6}{l}{\multirow{2}{*}{word}} & lexical & $-$log $p$ & $-$log $p$ \\
&&&&&& density & (bigram) & (MaxEnt) & phonotactics & rating \\
\midrule
S  & L  & EY & M  &    &   & 10 & 16.570 & 0.000 & valid   & 5.84 \\
W  & IH & S  &    &    &   & 37 & 13.297 & 0.000 & valid   & 5.84 \\
P  & IH & N  & T  &    &   & 20 & 12.577 & 0.000 & valid   & 5.67 \\
P  & AE & NG & K  &    &   & 17 & 11.868 & 0.000 & valid   & 5.63 \\
S  & T  & IH & P  &    &   & 15 & 12.998 & 0.000 & valid   & 5.53 \\
M  & IH & P  &    &    &   & 34 & 12.581 & 0.000 & valid   & 5.47 \\
S  & T  & AY & R  &    &   &  9 & 15.698 & 0.000 & valid   & 5.47 \\
M  & ER & N  &    &    &   & 40 & 12.140 & 0.000 & valid   & 5.42 \\
P  & L  & EY & K  &    &   & 15 & 16.127 & 0.000 & valid   & 5.39 \\
S  & N  & EH & L  &    &   &  9 & 16.220 & 0.000 & valid   & 5.32 \\
S  & T  & IH & N  &    &   & 17 & 11.124 & 0.000 & valid   & 5.28 \\
R  & AE & S  & K  &    &   & 12 & 14.624 & 0.000 & valid   & 5.21 \\
T  & R  & IH & S  & K  &   &  3 & 16.760 & 0.000 & valid   & 5.21 \\
S  & P  & A  & E  & K  &   & 16 & 13.254 & 0.000 & valid   & 5.16 \\
D  & EY & P  &    &    &   & 19 & 14.047 & 0.000 & valid   & 5.11 \\
G  & EH & R  &    &    &   & 36 & 12.084 & 0.000 & valid   & 5.11 \\
G  & L  & IH & T  &    &   &  6 & 15.497 & 0.000 & valid   & 5.11 \\
S  & K  & EH & L  &    &   & 13 & 14.049 & 0.000 & valid   & 5.11 \\
SH & ER & N  &    &    &   & 26 & 14.441 & 0.000 & valid   & 5.11 \\
T  & AA & R  & K  &    &   & 14 & 16.148 & 0.000 & valid   & 5.11 \\
CH & EY & K  &    &    &   & 22 & 15.137 & 0.000 & valid   & 5.05 \\
G  & L  & IY & D  &    &   & 13 & 16.794 & 0.000 & valid   & 5.05 \\
G  & R  & AY & N  & T  &   &  3 & 17.164 & 0.000 & valid   & 5.00 \\
P  & R  & IY & K  &    &   & 14 & 14.722 & 0.000 & valid   & 5.00 \\
SH & IH & L  & K  &    &   &  3 & 17.983 & 0.000 & valid   & 4.89 \\
D  & AY & Z  &    &    &   & 38 & 13.332 & 0.000 & valid   & 4.84 \\
N  & EY & S  &    &    &   & 17 & 15.495 & 0.000 & valid   & 4.84 \\
T  & AH & NG & K  &    &   & 18 & 13.400 & 0.000 & valid   & 4.84 \\
S  & K  & W  & IH & L  &   &  5 & 17.684 & 0.000 & valid   & 4.83 \\
L  & AH & M  &    &    &   & 35 & 10.329 & 0.000 & valid   & 4.79 \\
P  & AH & M  &    &    &   & 28 & 10.006 & 0.000 & valid   & 4.79 \\
S  & P  & L  & IH & NG &   & 13 & 17.093 & 0.000 & valid   & 4.72 \\
G  & R  & EH & L  &    &   &  6 & 13.686 & 0.000 & valid   & 4.63 \\
T  & EH & SH &    &    &   &  8 & 15.421 & 0.000 & valid   & 4.63 \\
T  & IY & P  &    &    &   & 29 & 12.914 & 0.000 & valid   & 4.63 \\
B  & AY & Z  &    &    &   & 47 & 12.688 & 0.000 & valid   & 4.58 \\
G  & L  & IH & P  &    &   &  6 & 16.393 & 0.000 & valid   & 4.53 \\
CH & AY & N  & D  &    &   & 15 & 16.768 & 0.000 & valid   & 4.37 \\
P  & L  & IH & M  &    &   &  6 & 14.320 & 0.000 & valid   & 4.37 \\
G  & UW & D  &    &    &   & 21 & 14.519 & 0.000 & valid   & 4.32 \\
B  & L  & EY & F  &    &   &  6 & 17.997 & 0.000 & valid   & 4.21 \\
G  & EH & Z  &    &    &   &  9 & 16.619 & 0.000 & valid   & 4.21 \\
D  & R  & IH & T  &    &   &  7 & 14.369 & 0.000 & valid   & 4.16 \\
F  & L  & IY & P  &    &   &  9 & 15.224 & 0.000 & valid   & 4.16 \\
Z  & EY &    &    &    &   & 35 & 13.153 & 2.769 & valid   & 4.16 \\
S  & K  & R  & AY & D  &   &  4 & 17.976 & 0.000 & valid   & 4.11 \\
K  & IH & V  &    &    &   & 12 & 14.008 & 0.000 & valid   & 4.05 \\
F  & L  & EH & T  &    &   & 15 & 14.449 & 0.000 & valid   & 4.00 \\
N  & OW & L  & D  &    &   & 23 & 19.131 & 0.000 & valid   & 4.00 \\
S  & K  & IH & K  &    &   & 13 & 14.146 & 0.000 & valid   & 4.00 \\
B  & R  & EH & JH &    &   &  5 & 16.438 & 0.000 & valid   & 3.95 \\
K  & W  & IY & D  &    &   &  7 & 16.694 & 0.000 & valid   & 3.95 \\
S  & K  & OY & L  &    &   &  7 & 16.937 & 0.000 & valid   & 3.89 \\
D  & R  & AY & S  &    &   & 16 & 16.479 & 0.000 & valid   & 3.84 \\
F  & L  & IH & JH &    &   &  5 & 16.594 & 0.000 & valid   & 3.79 \\
B  & L  & IH & G  &    &   &  4 & 15.487 & 0.000 & valid   & 3.53 \\
Z  & EY & P  & S  &    &   &  5 & 24.147 & 2.769 & valid   & 3.47 \\
CH & UW & L  &    &    &   & 13 & 14.085 & 0.000 & valid   & 3.42 \\
SH & AY & N  & T  &    &   &  5 & 16.438 & 0.000 & valid   & 3.42 \\
SH & R  & UH & K  & S  &   &  4 & 22.453 & 2.127 & invalid & 3.32 \\
G  & W  & EH & N  & JH &   &  1 & 23.165 & 2.929 & invalid & 3.32 \\
N  & AH & NG &    &    &   & 15 & 14.633 & 0.000 & valid   & 3.28 \\
S  & K  & W  & AA & L  & K &  0 & 24.319 & 0.000 & invalid & 3.26 \\
T  & W  & UW &    &    &   &  8 & 16.779 & 0.000 & valid   & 3.17 \\
S  & M  & AH & M  &    &   &  6 & 14.638 & 0.000 & valid   & 3.05 \\
S  & N  & OY & K  & S  &   &  3 & 25.335 & 0.000 & invalid & 3.00 \\
S  & F  & UW & N  & D  &   &  0 & 23.906 & 6.703 & invalid & 2.94 \\
P  & W  & IH & P  &    &   &  2 & 24.135 & 4.818 & invalid & 2.89 \\
R  & AY & N  & T  &    &   &  8 & 14.815 & 0.000 & valid   & 2.89 \\
S  & K  & L  & UW & N  & D &  0 & 20.721 & 3.046 & invalid & 2.83 \\
S  & M  & IY & R  & G  &   &  0 & 27.414 & 0.000 & invalid & 2.79 \\
F  & R  & IH & L  & G  &   &  2 & 22.117 & 0.000 & invalid & 2.68 \\
SH & W  & UW & JH &    &   &  0 & 30.868 & 4.878 & invalid & 2.68 \\
TH & R  & OY & K  & S  &   &  0 & 25.623 & 1.907 & invalid & 2.68 \\
T  & R  & IH & L  & B  &   &  1 & 20.903 & 0.000 & invalid & 2.63 \\
K  & R  & IH & L  & G  &   &  0 & 21.254 & 0.000 & invalid & 2.58 \\
S  & M  & EH & R  & G  &   &  0 & 21.873 & 0.000 & invalid & 2.58 \\
TH & W  & IY & K  & S  &   &  2 & 25.867 & 3.879 & invalid & 2.53 \\
S  & M  & EH & R  & F  &   &  0 & 23.030 & 0.000 & invalid & 2.47 \\
S  & M  & IY & L  & TH &   &  0 & 25.712 & 0.000 & invalid & 2.47 \\
P  & L  & OW & M  & F  &   &  0 & 21.857 & 0.000 & invalid & 2.42 \\
D  & W  & OW & JH &    &   &  0 & 24.150 & 2.929 & invalid & 2.29 \\
P  & L  & OW & N  & TH &   &  0 & 20.893 & 0.000 & invalid & 2.26 \\
TH & EY & P  & T  &    &   &  3 & 23.473 & 1.907 & invalid & 2.26 \\
S  & M  & IY & N  & TH &   &  0 & 23.098 & 0.000 & invalid & 2.06 \\
S  & P  & R  & AA & R  & F &  0 & 23.675 & 0.000 & valid   & 2.05 \\
P  & W  & AH & JH &    &   &  0 & 25.463 & 4.818 & invalid & 1.74 \\
\bottomrule
\end{longtable}


%    \section{\citet{Albright2003b} norming study} \begin{longtable}{l@{ } l@{ } l@{ } l@{ } l@{ } l r r r r r r} 
\toprule
\multicolumn{6}{l}{\multirow{2}{*}{word}} & lexical & $-$log $p$ & $-$log $p$ \\
&&&&&& density & (bigram) & (MaxEnt) & phonotactics & rating \\
\midrule
S  & L  & EY & M  &    &   & 10 & 16.570 & 0.000 & valid   & 5.84 \\
W  & IH & S  &    &    &   & 37 & 13.297 & 0.000 & valid   & 5.84 \\
P  & IH & N  & T  &    &   & 20 & 12.577 & 0.000 & valid   & 5.67 \\
P  & AE & NG & K  &    &   & 17 & 11.868 & 0.000 & valid   & 5.63 \\
S  & T  & IH & P  &    &   & 15 & 12.998 & 0.000 & valid   & 5.53 \\
M  & IH & P  &    &    &   & 34 & 12.581 & 0.000 & valid   & 5.47 \\
S  & T  & AY & R  &    &   &  9 & 15.698 & 0.000 & valid   & 5.47 \\
M  & ER & N  &    &    &   & 40 & 12.140 & 0.000 & valid   & 5.42 \\
P  & L  & EY & K  &    &   & 15 & 16.127 & 0.000 & valid   & 5.39 \\
S  & N  & EH & L  &    &   &  9 & 16.220 & 0.000 & valid   & 5.32 \\
S  & T  & IH & N  &    &   & 17 & 11.124 & 0.000 & valid   & 5.28 \\
R  & AE & S  & K  &    &   & 12 & 14.624 & 0.000 & valid   & 5.21 \\
T  & R  & IH & S  & K  &   &  3 & 16.760 & 0.000 & valid   & 5.21 \\
S  & P  & A  & E  & K  &   & 16 & 13.254 & 0.000 & valid   & 5.16 \\
D  & EY & P  &    &    &   & 19 & 14.047 & 0.000 & valid   & 5.11 \\
G  & EH & R  &    &    &   & 36 & 12.084 & 0.000 & valid   & 5.11 \\
G  & L  & IH & T  &    &   &  6 & 15.497 & 0.000 & valid   & 5.11 \\
S  & K  & EH & L  &    &   & 13 & 14.049 & 0.000 & valid   & 5.11 \\
SH & ER & N  &    &    &   & 26 & 14.441 & 0.000 & valid   & 5.11 \\
T  & AA & R  & K  &    &   & 14 & 16.148 & 0.000 & valid   & 5.11 \\
CH & EY & K  &    &    &   & 22 & 15.137 & 0.000 & valid   & 5.05 \\
G  & L  & IY & D  &    &   & 13 & 16.794 & 0.000 & valid   & 5.05 \\
G  & R  & AY & N  & T  &   &  3 & 17.164 & 0.000 & valid   & 5.00 \\
P  & R  & IY & K  &    &   & 14 & 14.722 & 0.000 & valid   & 5.00 \\
SH & IH & L  & K  &    &   &  3 & 17.983 & 0.000 & valid   & 4.89 \\
D  & AY & Z  &    &    &   & 38 & 13.332 & 0.000 & valid   & 4.84 \\
N  & EY & S  &    &    &   & 17 & 15.495 & 0.000 & valid   & 4.84 \\
T  & AH & NG & K  &    &   & 18 & 13.400 & 0.000 & valid   & 4.84 \\
S  & K  & W  & IH & L  &   &  5 & 17.684 & 0.000 & valid   & 4.83 \\
L  & AH & M  &    &    &   & 35 & 10.329 & 0.000 & valid   & 4.79 \\
P  & AH & M  &    &    &   & 28 & 10.006 & 0.000 & valid   & 4.79 \\
S  & P  & L  & IH & NG &   & 13 & 17.093 & 0.000 & valid   & 4.72 \\
G  & R  & EH & L  &    &   &  6 & 13.686 & 0.000 & valid   & 4.63 \\
T  & EH & SH &    &    &   &  8 & 15.421 & 0.000 & valid   & 4.63 \\
T  & IY & P  &    &    &   & 29 & 12.914 & 0.000 & valid   & 4.63 \\
B  & AY & Z  &    &    &   & 47 & 12.688 & 0.000 & valid   & 4.58 \\
G  & L  & IH & P  &    &   &  6 & 16.393 & 0.000 & valid   & 4.53 \\
CH & AY & N  & D  &    &   & 15 & 16.768 & 0.000 & valid   & 4.37 \\
P  & L  & IH & M  &    &   &  6 & 14.320 & 0.000 & valid   & 4.37 \\
G  & UW & D  &    &    &   & 21 & 14.519 & 0.000 & valid   & 4.32 \\
B  & L  & EY & F  &    &   &  6 & 17.997 & 0.000 & valid   & 4.21 \\
G  & EH & Z  &    &    &   &  9 & 16.619 & 0.000 & valid   & 4.21 \\
D  & R  & IH & T  &    &   &  7 & 14.369 & 0.000 & valid   & 4.16 \\
F  & L  & IY & P  &    &   &  9 & 15.224 & 0.000 & valid   & 4.16 \\
Z  & EY &    &    &    &   & 35 & 13.153 & 2.769 & valid   & 4.16 \\
S  & K  & R  & AY & D  &   &  4 & 17.976 & 0.000 & valid   & 4.11 \\
K  & IH & V  &    &    &   & 12 & 14.008 & 0.000 & valid   & 4.05 \\
F  & L  & EH & T  &    &   & 15 & 14.449 & 0.000 & valid   & 4.00 \\
N  & OW & L  & D  &    &   & 23 & 19.131 & 0.000 & valid   & 4.00 \\
S  & K  & IH & K  &    &   & 13 & 14.146 & 0.000 & valid   & 4.00 \\
B  & R  & EH & JH &    &   &  5 & 16.438 & 0.000 & valid   & 3.95 \\
K  & W  & IY & D  &    &   &  7 & 16.694 & 0.000 & valid   & 3.95 \\
S  & K  & OY & L  &    &   &  7 & 16.937 & 0.000 & valid   & 3.89 \\
D  & R  & AY & S  &    &   & 16 & 16.479 & 0.000 & valid   & 3.84 \\
F  & L  & IH & JH &    &   &  5 & 16.594 & 0.000 & valid   & 3.79 \\
B  & L  & IH & G  &    &   &  4 & 15.487 & 0.000 & valid   & 3.53 \\
Z  & EY & P  & S  &    &   &  5 & 24.147 & 2.769 & valid   & 3.47 \\
CH & UW & L  &    &    &   & 13 & 14.085 & 0.000 & valid   & 3.42 \\
SH & AY & N  & T  &    &   &  5 & 16.438 & 0.000 & valid   & 3.42 \\
SH & R  & UH & K  & S  &   &  4 & 22.453 & 2.127 & invalid & 3.32 \\
G  & W  & EH & N  & JH &   &  1 & 23.165 & 2.929 & invalid & 3.32 \\
N  & AH & NG &    &    &   & 15 & 14.633 & 0.000 & valid   & 3.28 \\
S  & K  & W  & AA & L  & K &  0 & 24.319 & 0.000 & invalid & 3.26 \\
T  & W  & UW &    &    &   &  8 & 16.779 & 0.000 & valid   & 3.17 \\
S  & M  & AH & M  &    &   &  6 & 14.638 & 0.000 & valid   & 3.05 \\
S  & N  & OY & K  & S  &   &  3 & 25.335 & 0.000 & invalid & 3.00 \\
S  & F  & UW & N  & D  &   &  0 & 23.906 & 6.703 & invalid & 2.94 \\
P  & W  & IH & P  &    &   &  2 & 24.135 & 4.818 & invalid & 2.89 \\
R  & AY & N  & T  &    &   &  8 & 14.815 & 0.000 & valid   & 2.89 \\
S  & K  & L  & UW & N  & D &  0 & 20.721 & 3.046 & invalid & 2.83 \\
S  & M  & IY & R  & G  &   &  0 & 27.414 & 0.000 & invalid & 2.79 \\
F  & R  & IH & L  & G  &   &  2 & 22.117 & 0.000 & invalid & 2.68 \\
SH & W  & UW & JH &    &   &  0 & 30.868 & 4.878 & invalid & 2.68 \\
TH & R  & OY & K  & S  &   &  0 & 25.623 & 1.907 & invalid & 2.68 \\
T  & R  & IH & L  & B  &   &  1 & 20.903 & 0.000 & invalid & 2.63 \\
K  & R  & IH & L  & G  &   &  0 & 21.254 & 0.000 & invalid & 2.58 \\
S  & M  & EH & R  & G  &   &  0 & 21.873 & 0.000 & invalid & 2.58 \\
TH & W  & IY & K  & S  &   &  2 & 25.867 & 3.879 & invalid & 2.53 \\
S  & M  & EH & R  & F  &   &  0 & 23.030 & 0.000 & invalid & 2.47 \\
S  & M  & IY & L  & TH &   &  0 & 25.712 & 0.000 & invalid & 2.47 \\
P  & L  & OW & M  & F  &   &  0 & 21.857 & 0.000 & invalid & 2.42 \\
D  & W  & OW & JH &    &   &  0 & 24.150 & 2.929 & invalid & 2.29 \\
P  & L  & OW & N  & TH &   &  0 & 20.893 & 0.000 & invalid & 2.26 \\
TH & EY & P  & T  &    &   &  3 & 23.473 & 1.907 & invalid & 2.26 \\
S  & M  & IY & N  & TH &   &  0 & 23.098 & 0.000 & invalid & 2.06 \\
S  & P  & R  & AA & R  & F &  0 & 23.675 & 0.000 & valid   & 2.05 \\
P  & W  & AH & JH &    &   &  0 & 25.463 & 4.818 & invalid & 1.74 \\
\bottomrule
\end{longtable}

%\chapter{Data from Chapter \ref{theory}} %\chapter{Estimating Zipf's Law parameters} \label{zr}

\citet{Zipf1949} notes that word frequencies show a sparse distribution. \citeauthor{Zipf1949} notes a linear relationship between log word frequency $r$ and log frequency $r$. A generalized form of this relationship is what is now known as Zipf's Law.

\begin{unlabeledexample} 
$\displaystyle f(C, \alpha) = \frac{C}{r^\alpha}$ 
\end{unlabeledexample} 

\noindent $C$ is a constant which is primarily sensitive to sample size. \citeauthor{Zipf1949}'s assumes that $\alpha = -1$, but both $C$ and $\alpha$ can be estimated using the method of least squares ($\epsilon$ represents the error term).
 
\begin{unlabeledexample} 
$\displaystyle \textrm{log}~f \sim C + \alpha~\textrm{log}~r + \epsilon$  
\end{unlabeledexample}

\citet{Good1953} notes that sparse empirically-estimated distributions exhibit quantization at low frequencies, producing an artificially long and flat right tail and biasing $\alpha$ upward. \citet[][29]{Church1991} describe a transform to eliminate this quantization. The vectors $r, n$ are defined so such that $n_i$ is the number of types which occur at frequency $r_i$ (that is, $n$ is a vector of frequencies of individual type frequencies). $Z$ is simply a vector in which each element of $n$ scaled by the nearest points to the left and right.

\begin{unlabeledexample}
$\displaystyle Z_i = \frac{2 n_i}{r_{i + 1} - r_{i - 1}}$
\end{unlabeledexample}

\noindent \citeauthor{Church1991} do not define this transform for the lowest and highest points (i.e., when $i = 1$ or $i = N$), but a natural extension of their definition is to scale the endpoints according to the next intermost point, as defined below.

\begin{unlabeledexample}
$\displaystyle Z_1 = \frac{n_1}{r_2 - r_1}$
\end{unlabeledexample}

\begin{unlabeledexample}
$\displaystyle Z_N = \frac{n_N}{r_N - r_{N - 1}}$
\end{unlabeledexample}

\noindent The effect of applying this transform to sparse frequency data is shown in Figure \ref{subtlex}.

\begin{figure}
\centering
\includegraphics{zr.pdf}
\caption{The left panel shows the empirical word frequencies in SUBTLEX-US \citep{Brysbaert2009}. The right panel shows the effect of the $Z_r$ transform.}
\label{subtlex}
\end{figure}

%\chapter{Data from Chapter \ref{insufficient}} %\input{E}

%\lstset{language=Python, basicstyle=\scriptsize\ttfamily, showstringspaces=false, upquote=false}
%\lstinputlisting{zipf_sim.py}

\bibliographystyle{pwpl}
\bibliography{kgorman.bib}
\end{document}
