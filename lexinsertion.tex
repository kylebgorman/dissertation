
\citep{Pater2000,Pater2005b,Pater2005a,Pater2006,Pater2009,Becker2009a}

``the smaller set of indexed morphemes could be built in'' \citep[][?]{Pater2009}

This fails in the case of the 



\footnote{This observation is due to Charles Yang (personal communication).}

There are more seriosu failures:

if as standardly assumed 
(CITATION)
the initial state of the child's grammar ranks all markedness constraints above faithfulness constraints, then 

it is hard to see how even regular inflection is applied to nonce words.

Consider the case of 
/bl\textsci k/

Consider separately a nonce word such 
\emph{vuwt} [v\textupsilon t]

/t-z/
a marked contrast in voicing, so clearly markedesss concerns favor [ts] is favored. As it happens, however, word-final [ts] is a marked structure in Latin,
where it is resolved by deleting the final consonant.

\ex \label{para} Latin plosive-final third declension nouns \citep[from][]{Gorman2011b}: 

\begin{tabular}{ r l l l l}
       & \textsc{UR}  & {nom.sg.} & {gen.sg.}         & {gloss} \\ \cmidrule{2-5}
    a. & /stirp-/   & \emph{stirps}   & \emph{stirpis}  & `scion'  \\
       & /urb-/     & \emph{urbs}\footnote{}   & \emph{urbis}    & `city'   \\
       & /fak-/     & \emph{fax}   & \emph{facis}    & `torch' \\
%       & /re\lm g-/ & \emph{r\emac x} & \emph{r\emac gis}  & `king' \\
%    b. & /ko\lm t-/ & \emph{c\omac s}    & \emph{c\omac tis}  & `flint' \\
%       & /ped-/     & \emph{p\emac s}    & \emph{pedis}    & `foot'
\end{tabular} \xe

\footnotetext[\value{footnote}]{This is also written \emph{urps} in Classical texts, and it is generally accepted that the surface form was [urps] %\citep[21]{Lindsay1895} 
\citep[][137f.]{Neue1866}.}




