\section{The reality of inflectional gaps}

Comparing the counts of the preterite and non-preterite forms of \emph{undergo} and \emph{forgo} in both corpora with the Fisher exact test demonstrates that the non-usage of a preterite of \emph{forgo} is unlikely to be accidental (BNC: $p = 3.4$\textsc{e}-31, COCA: $p = 1.7$\textsc{e}-186). This is even true of \emph{sightsee}, which occurs with low frequency as a verb (BNC: $p = 0.0227$).

Surveys of inflectional gaps can be fonud in 

as surveyed in \citealt{Fanselow2002}, \citealt{Rice2009b}, and \citealt{Baerman2010b}. 

A characteristic of the traditional split between derivational morphology and inflectional morphology is 

As \citet{Marantz1997b} notes, there are many cases of productive category-changing morphology. In English the suffix \emph{-ness} productively forms abstract adjectives from nouns, and the suffix \emph{-ōs-} plays a similar productive role in Latin:

\ex Classical Latin denominal adjectives:
\begin{tabular}{l l l l l}
   & nom.sg. noun  &            & nom.sg. masc. adjective & \\
a. & \emph{macula} & `spot'     & \emph{maculōsus}       & `spotted'     \\
b. & \emph{uentus} & `windy'    & \emph{uentōsus}        & `windy'       \\
%c. & \emph{mo:ns}  & `mountain' & \emph{monto:sus}        & `mountainous' \\ 
c. & \emph{lutum } & `mud'      & \emph{lutōsus}         & `muddy'       \\
d. & \emph{ulcus}  & `ulcer'    & \emph{ulcerōsus}      & `uclerous'     \\
\end{tabular} \xe


%\ex English simple past suffixes:
%\begin{tabular}{l l l l l l }
%T[\textsc{Past}] & $\rightleftarrow$ & -\zr & \gap & \{$\surd$\textsc{hit}, $\surd$\textsc{sing}, \ldots\}
%                 & $\rightleftarrow$ & -t   & \gap & \{$\surd$\textsc{leave}, $\surd$\textsc{bend}, \ldots\}
%                 & $\rightleftarrow$ & -d   
%\end{tabular} \xe

Table \ref{fgo} provides some token counts from the 100 million word British National Corpus \citep{BNC,BYUBNC} and the 421 million word Corpus of Contemporary American English \citep{COCA}. 

\begin{table}
\centering
\begin{tabular}{l@{} l r r l@{} l r r} 
\toprule
    &            & BNC   & COCA   &   &             & BNC & COCA  \\ 
\midrule
    & undergo    & 613   & 2,757  &   & forgo       & 203 & 1,213 \\
    & undergoes  & 123   & 594    &   & forgoes     & 4   & 89    \\
    & undergoing & 581   & 2,464  &   & forgoing    & 24  & 232   \\
    & underwent  & 550   & 2,440  & * & forwent     & 0   & 0     \\
$*$ & undergoed  & 0     & 0      & * & forgoed     & 0   & 0     \\
    & undergone  & 565   & 1,996  &   & forgone     & 64  & 190   \\ 
\midrule
    & foresee    & 271   & 1,020  &   & sightsee    & 4   & 55    \\
    & foresees   & 58    & 300    &   & sightsees   & 0   & 0     \\
    & foreseeing & 38    & 89     &   & sightseeing & 20  & 5     \\
    & foresaw    & 132   & 431    & * & sightsaw    & 0   & 0     \\
$*$ & foreseed   & 0     & 0      & * & sightseed   & 0   & 0     \\
    & foreseen   & 275   & 568    & * & sightseen   & 0   & 0     \\ 
\midrule
    & ride       & 1,607 & 10,904 &   & stride      & 102 & 402   \\
    & rides      & 230   & 2,495  &   & strides     & 54  & 682   \\
    & riding     & 1,696 & 10,994 &   & striding    & 253 & 715   \\
    & rode       & 1,079 & 5,997  &   & strode      & 641 & 1,752 \\
$*$ & rided      & 1     & 0      & * & strided     & 0   & 2     \\
    & ridden     & 456   & 1,303  & * & stridden    & 1   & 2     \\ 
\bottomrule
\end{tabular}
\caption{Token counts the BNC and COCA for English verbs with inflectional gaps}
\label{fgo}
\end{table}

\citep{Pater2000,Pater2005b,Pater2005a,Pater2006,Pater2009,Becker2009b}

``the smaller set of indexed morphemes could be built in'' \citep[][?]{Pater2009}

This fails in the case of the 






\footnote{This observation is due to Charles Yang (personal communication).}

There are more seriosu failures:

if as standardly assumed 
(CITATION)
the initial state of the child's grammar ranks all markedness constraints above faithfulness constraints, then 

it is hard to see how even regular inflection is applied to nonce words.

Consider the case of 
/bl\textsci k/

Consider separately a nonce word such 
\emph{vuwt} [v\textupsilon t]

/t-z/
a marked contrast in voicing, so clearly markedesss concerns favor [ts] is favored. As it happens, however, word-final [ts] is a marked structure in Latin,
where it is resolved by deleting the final consonant.

\ex \label{para} Latin plosive-final third declension nouns \citep[from][]{Gorman2011b}: 

\begin{tabular}{ r l l l l}
       & \textsc{UR}  & {nom.sg.} & {gen.sg.}         & {gloss} \\ \cmidrule{2-5}
    a. & /stirp-/   & \emph{stirps}   & \emph{stirpis}  & `scion'  \\
       & /urb-/     & \emph{urbs}\footnote{}   & \emph{urbis}    & `city'   \\
       & /fak-/     & \emph{fax}   & \emph{facis}    & `torch' \\
%       & /re\lm g-/ & \emph{r\emac x} & \emph{r\emac gis}  & `king' \\
%    b. & /ko\lm t-/ & \emph{c\omac s}    & \emph{c\omac tis}  & `flint' \\
%       & /ped-/     & \emph{p\emac s}    & \emph{pedis}    & `foot'
\end{tabular} \xe

\footnotetext[\value{footnote}]{This is also written \emph{urps} in Classical texts, and it is generally accepted that the surface form was [urps] %\citep[21]{Lindsay1895} 
\citep[][137f.]{Neue1866}.}




