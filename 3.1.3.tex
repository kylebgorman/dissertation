\subsection{Labial attraction}

The final rule 

\subsubsection{Phonological description}

\citet[][36]{Lees1966a} describes \textsc{Labial Attraction} as a process by which ``a high, short harmonic vowel is rounded in the second syllable of a disyllabic word whose first vowel is /a/, and whose medial consonant cluster contains a labial /p, b, m, v/, and then it is de-harmonified''. This description is transalted into an autosegmental rule below.

\begin{example}
\textsc{Labial Attraction} (after \citealt[][286]{Lees1966b}, \citealt[][171]{Inkelas2001}): 

\xymatrix@R=24pt@C=8pt{
\txt{[$-$\textsc{Round}]} &                                         & \txt{[$-$\textsc{High}]} & \txt{[$+$\textsc{Labial}]}    & \txt{[$+$\textsc{High}]}\ar@{-}[dr] &         & \txt{[$+$\textsc{Round}]}\ar@{..}[dl] \\
                         & \txt{V}\ar@{-}[ul]\ar@{-}[ur]\ar@{-}[d] &                           & \txt{C$_0$ C C$_0$}\ar@{-}[u] &                                      & \txt{V} & \\
                         & \txt{[$+$\textsc{Back}]}\ar@{-}[urrrr] 
}
\end{example}

The formalization of this rule is naturally complex, and perhaps obscures the fact that \textsc{Labial Attraction} generates exceptions to \textsc{Roundness Harmony}; it produces \emph{aCu} sequences (e.g., \emph{çapul} `raid', \emph{sabur} `patient', \emph{şaful} `wooden honey tub', \emph{avuç} `palm of hand', and \emph{samur} `sable' \citep[][285]{Lees1966b}) instead of the otherwise expected \emph{aCı}. \citeauthor{Lees1966b} notes the existence of exceptions (e.g., \emph{tavır} `mode') but writes that they are they are ``surprisingly rare'' (ibid., 286); \citet[][225]{Clements1982} note additional exceptions.  

There is reason to believe that \textsc{Labial Attraction} is at best a sequence structure constraint as it never applies in derived environments. If, contrary to fact, there was a \textsc{Labial Attraction} alternation, then one would expect, for example, that the gen.sg. of \emph{sap} `stalk' would be *\emph{sapun} instead of the observed \emph{sapın}.

\citet{Lees1966a}
\citet{Zimmer1969}

\begin{quote}
\ldots decisive evidence against a rule of Labial Attraction is the existence of a further, much larger set of roots containing /\ldots~aCu~\ldots/ sequences in which the intervening consonant or consonant cluster does not contain a labial\ldots We conclude that there is no systematic restriction on the set of consonants that may occur medially in rotos of the form /\ldots~aCu~\ldots/. \citep[][225]{Clements1982}
\end{quote}

\noindent
\citeauthor{Clements1982} appear to be suggesting that \textsc{Labial Attraction} implies that \emph{aTu}, where \emph{T} represents a non-labial consonant, should be infrequent, but this fact is not inconsistent with the formulation of the constraint by \citet{Lees1966a,Lees1966b} and \citet{Zimmer1969}; \emph{aTu} clusters do not meet \textsc{Labial Attraction}'s structural description. This is less than conclusive, since the fluctuating vowel alternation also fails to undergo harmony, and might just indicate that the flucutating vowel is epenthesized relatively late in the derivation; the frequency of \emph{aTu} provides only indirect information about \textsc{Labial Attraction} in that it provides a baseline estimate for just how frequent this disharmonic sequence of vowels is.

\citet{Inkelas1997}
\citet{Inkelas2001}

\begin{quote}
Lee's rule of \textsc{Labial Attraction}\ldots is not a real generalization about the Turkish lexicon. It is not true synchronically, either of native or nonnative items; nor, according to the historical and dialectical literature, does \textsc{Labial Attraction} appear to have been true at any stage going back as far as Old Turkic. \citep[][196]{Inkelas2001}
\end{quote}

zimmer: 319-320 
inkelas et al.: 196

\subsubsection{Psycholinguistic evidence}

To the author's knowledge, there are no psycholinguistic studies investigating \textsc{Labial Attraction} beyond the experimental results of \citet{Zimmer1969} reviewed in the next section. 
