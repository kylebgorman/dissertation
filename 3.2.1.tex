% 3.2.1: Lexical statistics

This choice of statistical test is justified in Section \ref{stattech}.

\subsubsection{Roundness harmony}

%\ex Lexical effects of \textsc{Roundness Harmony} \citep{TELL}: \vspace{6pt} \\
%\begin{tabular}{l r r r r r}
%\toprule              & 
%Corpus                & $p$-value \\
%\midrule
%Full TELL             &  
%Elicited TELL         & 
%TELL with etymologies & 
%\end{tabular}
%\xe

%\citet{Harrison2004} 73\% 
%of lexical types are harmonic (both back and round)

\subsubsection{Labial attraction}

%\ex Lexical effects of \textsc{Labial Attraction} \citep[][186]{Inkelas2001}: \vspace{6pt} \\
%\begin{tabular}{l r r r r r}
%\toprule
%Corpus                & aPu & aPı & aTu & aTı   & $p$-value   \\ % & corpus size
%\midrule
%Full TELL             & 378 & 248 & 446 & 1,140 & 2.83\e{-44} \\ % & 31,236 \\
%Elicited TELL         & 152 & 265 & 101 & 1,839 & 9.84\e{-60} \\ % & 16,541 \\
%TELL with etymologies & 128 & 109 &  79 &   470 & 6.56\e{-32} \\ 
%\bottomrule
%\end{tabular}
%\xe

%Etymological subset of TELL
%Native    & Foreign   \\
%aBu & aBI & aBu & aBI \\
%12  & 11  & 84  & 28  \\
%p = 0.0417
