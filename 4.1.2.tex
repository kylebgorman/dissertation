\subsection{Syllabification}

For this task, it is necessary to separate medial consonant clusters into codas and onsets. While CELEX provides syllabified transcriptions, no description of the syllabification procedure is given in the documentation, and inspection of these syllabifications suggests the procedure used is not fully systematic. For instance, \emph{chemistry} is syllabified as [ˈkɛ.mɪ.strɪ] but \emph{ministry} as [ˈmɪ.nɪs.trɪ].\footnote{Note that word-final \emph{y} is generally lax [ɪ] in Received Pronunciation \citep[][294]{AOE2}.} This putative [ɪ.strɪ $\sim$ ɪs.trɪ] contrast, and many other such contrasts in the CELEX data, are dubious simply because there is no clear evidence that contrastive syllabification occurs in any language. Apparent counterexamples appear to be effects of vowel or consonant length \citep[e.g.,][]{Elfner2006}, word stress, or morphological structure, none of which explain the \emph{chemistry}/\emph{ministry} syllabification contrast.

An automated syllabification system was constructed and applied to medial clusters in the CELEX data. This system need only deal with simplex English words that contain the medial consonant clusters which are the focus of this study.  There is, for instance, no need to address the status of ``ambisyllabic'' consonants, i.e., singleton medial consonants preceded by a stressed lax vowel \citep[][219f.]{Rubach1996}, or to address morphological effects on syllabification. The technique used is a variation on the theme of onset maximization \citep[42f.]{Kahn1976}, which parses word-medial clusters so that the onset is as large as possible. Medial clusters in words like \emph{neu}[.tr]\emph{on} or \emph{bi}[.str]\emph{o} are found in word-initial position (e.g., [tr]\emph{ansit}, [str]\emph{ike}), and onset maximization leaves the coda of the first syllable empty. In contrast, the [nstr] cluster in \emph{mi}[n.str]\emph{el} does not occur word-initially; here the maximal onset is [str], and the [n] is assigned to the coda.

There is one case where unchecked maximization of medial onsets produces incorrect syllabifications. When a medial consonant cluster is preceded by a stressed lax vowel, as in words like \emph{propaga}[n.d]\emph{a}, \emph{whi}[s.p]\emph{er}, \emph{vi}[s.t]\emph{a}, or \emph{bi}[s.k]\emph{uit}, the first consonant of the cluster checks the lax vowel \citep[e.g.,][3]{Hammond1997}. \citet[][55]{Harris1994} notes that onset maximization makes the wrong predictions when the medial cluster is a valid onset: in \emph{whisper}, \emph{vista}, and \emph{biscuit}, it incorrectly assigns [sp, st, sk] to the medial onset, leaving the coda empty. The approach adopted here is to first apply onset maximization, then to reassign the first consonant of a medial onset to the preceding coda if it is immediately preceded by a stressed lax vowel.

If the English affricates [tʃ, dʒ] are treated as sequences and not individual segments, this addition to the standard onset maximization technique would itself make the wrong prediction for medial affricates preceded by lax stressed vowels in words like \emph{ra}[.tʃ]\emph{et} or \emph{a}[.dʒ]\emph{ile}, incorrectly assigning the stop and fricative portions to separate syllables. For this reason, an additional constraint is assumed which prevents the two components of the affricates from being split by syllabification. This is motivated by the tendency of affricates to pattern with single segments in many languages. For instance, the only complex onsets in Classical Nahua are the affricates [ts, tʃ, tɬ] \citep[][9]{Launey2011}.

%In fact, [t.ʃ, d.ʒ] clusters are not found in simplex English words, despite the fact that that affricates occur as medial onsets in clusters (e.g., \emph{tru}[n.tʃ]\emph{eon}, \emph{so}[l.dʒ]\emph{er}).
