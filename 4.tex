\label{clusters}



The term \emph{lexicon} has many senses: throughout, I use it to refer simply to the set of underlying representations.
%\citep[][269]{LANGUAGE}.

In the preceding chapter, it was proposed that 

Reconsider the null hypothesis presented in the preceding chapter:

\begin{example}
The null hypothesis: \\
Universal grammar does not countenance constraints on the contents of URs
\end{example}

The idea 

relationship to proposal

One primary way which this null hypothesis has been critiqued is by demonstrating that the lexicon of some language 
exhibits dispreferences or gaps corresponding to.

argue by trends in the lexicon

possibility of sparsity (C\&H, Fischer/Vogt and recent stats)

Of course, to obtain the desired results, we must guarantee that each sequence structure rule reflects a general systematic fact about the language, and not a fact which is due merely to the existence of accidental gaps in the lexicon. \citep[][401, fn.~8]{Stanley1967}

In the preceding chapter, it was proposed that 

Lexicon

This appears to be exceptionlessly true of Arabic

That even gradient patterns 

One of the first study of this type was an analysis of co-occurrence restrictions on consonants in Javanese roots by \citet{Mester1988}, recently revisited by \citet{Graff2011}. 
Another Austronesian language studied in this fashion is Muna \citep{Coetzee2008a,Anttila2008}.
This technique has been applied to consonants in Semitic, especially Arabic \citep{McCarthy1988,McCarthy1994,Pierrehumbert1993,Frisch1996,Frisch2004,Coetzee2008a} but also in Tigrinya \citep{Buckley1997}, Hebrew \citep{Berent2003}, and Amharic \citep{Colavin2010}, and to various co-occurrence restrictions in English \citep{Berkley1994b,Berkley1994a,Pierrehumbert1994,Dmitrieva2008a,Dmitrieva2008b,Coetzee2008b}, Russian \citep{Padgett1992}, Dutch \citep{Graff2011}, Navajo \citep{Martin2007,Martin2011} and Gitksan \citep{Brown2010}.

The absence of 
The claim here is that however stark the absence or underrepresentation of some form is, it 

This hypothesis admits the possibility that there may be accidental gaps in the lexicon, a possibility which predates generative thinking:

\begin{quotation}
\ldots the fact that some [clusters--KG] are not found must be due to accidental gaps in the inventory of signs, and cannot be explained by structural laws of the language. \citep[][16]{Fischer-Jorgensen1952}
\end{quotation}

\noindent
Hans \citeauthor{Vogt1954} makes a similar observation in his study of Georgian clusters

\begin{quotation}
Although my material is drawn from a fairly extensive corpus---all accessible dictionaries and vocabularies, printed texts of tens of thousands of pages as well as ordinary speech---there is every reason to believe, as experience has shown, that additional material would yield new clusters. The material will never be complete. It will always contain accidental gaps \ldots partly because some clusters by pure chance do not occur in the vocabulary. \citep[][30]{Vogt1954}
\end{quotation}

\citet{Chomsky1965}

This chaper demonstrates that apparently systematic gaps may arise even in the absence of phonotactic preferences, and that this possibility compromises attempts to show that such preferences shape the lexicon. 

% 4.1: Aspects in the theory of wordlikeness

% 4.2: Evaluation

The remainder of this chapter is dedicated to evaluating two claim

\section{Conclusions}

\citet{Borowsky1989} on peripherality.

productive \citet{Duanmu2008}

The above is a stark reminder.


A reasonable objection to the arguments presented in this chapter is to view the results as little more than an indictment of the results of \citealt{Pierrehumbert1994}.

However, the fact that state-of-the-art models are not capable of providing large improvements to the predictive accuracy indicates


 do not have a statistically significant effect on the shape of the English lexicon, but that experiments might turn up evidence that speakers have internalized 


shown to be aware of static constraints if they reached statistical significance. 



 statistically reliable static constraints could be identified 

The of



Regarding the historical developments,

\citet{Martin2007}

On the other hand, patterns created by sound change are not guaranteed to persist over time. 
One example of non-persistence is discussed by \citet{Iverson2005}.  
Around 1100 CE, Old English \emph{sk} became [ʃ]. 
This sound change introduced no alternations.
Since long vowels were not found before tautosyllabic syllable clusters at this time, there were no \emph{V\lm sk\#} words when the
 change was actuated, and \emph{V\lm sh\#} continues to be rare in Modern English. 
What \citeauthor{Iverson2005} observe, however, is that there is nothing apparently peripheral about words like \emph{leash} or \emph{whoosh}, and loanwords and coinages have readily filled the gap.

A third pattern is that a historically inherited pattern

\citet[][140]{Frisch2004} suggest that the strong tendency for the first and second consonants of the Arabic root to be non-identic
al is the ``a diachronic result of a processing constraint that disfavors repetition.'' 
Unfortunately, there is no evidence that this pattern is diachronic other than in the sense that it appears to be inherited from the proto-language: there is simply no Proto-Semitic verb roots with identical first and second consonants \citep[][178]{Greenberg1950}. 
In other Semitic languages, the inherited patern has experienced considerable erosion. 

\begin{example}
Tigrinya roots with identical first and second consonants \citep{Buckley1990a}: \\
\begin{tabular}{l l l l}
a. & lʌlʌw     & `scorch'                   & (< Ge'ez \emph{lʌwlʌw} `inflame')     \\
   & mʌmʌy     & `winnow'                   & (< Ge'ez \emph{mʌymʌy} `distinguish') \\
%   & mʌmʌt & `pick out loot' & (< 
b. & s’ʌs’ʌw   & `finish off a drink'       & (cf. \emph{s’ʌws’ʌw} `gulp down')           \\
   & t’ʌt’ʌf   & `prune tree'               & (cf. \emph{t’ʌft’ʌf} `smear wall with mud') \\
c. & kʷakʷkʷʌr & `waste away, be emaciated' & (cf. \emph{kʷarkʷʌr} `interrogate')         \\
   & kakʷkʷɨʕ  & `clean wax from ears'      & (cf. \emph{kaʕkʷɨʕ} `start to form pods')   \\
\end{tabular}
\end{example}

Similar exceptions are found in 
%Amharic (\citealp[][?]{Broselow1984}, \citealp[][?]{McCarthy1985}) and 
Hebrew \citep[][29]{Bat-El2005}.

The next two chapters return to the question of synchrony, addressing the relationship between statistical patterns in the lexicon and speakers' behaviors when presented with underrepresented sequences.

accidental gaps
\citet[][419f.]{Hayes2008a}

