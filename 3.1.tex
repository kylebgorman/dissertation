\section{Turkish vowel sequence structure constraints}

%\citet{Lees1966a,Lees1966b} 
\citet{Lees1966b,Lees1966a} 
proposes three constraints on Turkish vowel sequences which have been the focus of many subsequent studies. In this section, these constraints are formalized, and where possible, related to phonological alternations in Turkish and to evidence that speakers have internalized these restrictions. The following feature specification for the Turkish vowels is assumed throughout.

\begin{example}
Turkish vowel features:

\vspace{0.5\baselineskip}
\begin{tabular}{c | c c c c}
                       & \multicolumn{2}{c}{[$-$\textsc{Back}]} & \multicolumn{2}{c}{[$+$\textsc{Back}]} \\
                       & [$-$\textsc{Round}] & [$+$\textsc{Round}] & [$-$\textsc{Round}] & [$+$\textsc{Round}] \\ \midrule
\buf[$+$\textsc{High}] & {i} & {ü} [y] & {ı} [ɯ] & {u} \\
\buf[$-$\textsc{High}] & {e} & {ö} [ø] & {a} [ɑ] & {o} \\
\end{tabular}
\end{example}

\subsection{Backness harmony}

One of the most salient properties of Turkish is its harmony system, which \citeauthor{Lees1966b} (\citeyear[][35]{Lees1966b}, \citeyear[][284]{Lees1966a}) derives by means of feature spreading rules. The least constrained of these processes is the rule of backness harmony described in this section. 

\subsubsection{Phonological description}

Backness harmony spreads the value of \textsc{Back} rightward over any intervening consonants onto the next vowel. 

%\citep[after][229]{Clements1982}: \\
\begin{example}
\textsc{Backness Harmony}: 

\xymatrix@R=24pt@C=24pt{
\txt{V}                                        & \txt{C$_0$} & \txt{V} \\
\txt{[α \textsc{Back}]}\ar@{-}[u]\ar@{..}[urr] \\
}
\end{example}

Along with the rule of \textsc{Roundness Harmony} discussed in the next section, this rule accounts for a great deal of the suffix allomorphy found in Turkish. The nominative plural (nom.pl.) is formed by adding the suffixes \emph{-ler} or \emph{-lar} to the nominative singular (nom.sg.). 

\begin{example}
The Turkish nominative: 

\vspace{0.5\baselineskip}
\begin{tabular}{l l l l l}
   & \emph{nom.sg.} & \emph{nom.pl.} \\ 
a. & ip             & ipler          & `rope' & \citep[][216]{Clements1982} \\
   & el             & eller          & `hand'    \\
   & köy            & köyler         & `village' \\
   & yüz            & yüzler         & `face'    \\
   & kız            & kizlar         & `girl'    \\
   & sap            & saplar         & `stalk'   \\
   & son            & sonlar         & `end'     \\
   & pul            & pullar         & `stamp'   \\
b. & neden          & nedenler       & `reason'  & (TELL) \\
   & boğaz          & boğazlar       & `throat'  \\
   & kiler          & kilerler       & `pantry'  \\
   & pelür          & pelürler       & `tissue paper' \\
   & sapık          & sapıklar       & `pervert' \\
\end{tabular}
\end{example}

\noindent
\textsc{Backness Harmony} correctly predicts the choice of suffix. There are a few complications, however. First, while most polysyllabic roots conform to \textsc{Backness Harmony} (\lastx b), not all do. In this case, suffix vowels generally harmonize with the final root vowel (\nextx a). Yet there is also a small class of nouns which exhibit [$-$\textsc{Back}] suffix vowels despite the fact that their final root vowel is [$+$\textsc{Back}] (\nextx b).

\begin{example}
Exceptional Turkish nominatives: 

\vspace{0.5\baselineskip}
\begin{tabular}{l l l l l}
   & \emph{nom.sg.} & \emph{nom.pl.} \\
a. & mezar          & mezarlar       & `grave' & \citep{TELL} \\
   & model          & modeller       & `model' \\
   & silah          & silahlar       & `weapon'     \\
   & memur          & memurlar       & `bureaucrat' \\
   & sabun          & sabunlar       & `soap'       \\
b. & saat           & saatler        & `hour, clock' \\
   & harf           & harfler        & `(alphabetic) letter' \\ %& \citep{Goksel2005}
   & etol           & etoller        & `fur stole' \\
\end{tabular}
\end{example}

While it is uncontroversial that disharmonic suffixes (\lastx b) are no more than sporadic exceptions to \textsc{Backness Harmony}, the status of root disharmony (\lastx a) has been the subject of much debate. \citet[][212, 289]{Anderson1974} views the fact that disharmonic roots still trigger suffix harmony as evidence that suffix harmony is distinct from root harmony, the latter being a sequence structure constraint. 

%Once additional source of evidence on root (dis)harmony is inconclusive. There is a small class of bisyllabic words in which the second vowel, always [$+$\textsc{High}, $-$\textsc{Back}], alternates with zero. 

%\ex High-vowel/zero alternations \citep[][243]{Clements1982}: \\
%\begin{tabular}{l l l l}
%   & nom.sg. & gen.sg. \\
%a. & fikir   & fikri  & `idea' \\
%   & hüküm   & hükmün & `judgement' \\
%%  & filim   & filmi & `film' & \citep[][178]{Inkelas2001} \\
%b. & vakit   & vaktin & `time' \\
%   & rahim   & rahmin & `womb' \\
%\end{tabular} \xe
%
%\noindent
%It is possible that \textsc{Backness Harmony} might produce a fluctuating \emph{ı} after root \emph{a}, but this does not obtain (\lastx b). However, this might simply indicate that the fluctuating vowel is epenthetic and that harmony applies before epenthesis (see \citealt{Clements1982} for both sides of this argument), making it less than a counterexample. 

Underspecification provides an alternative under which the exceptionality of disharmonic roots is only apparent. While it is clear that the first vowel of Turkish roots are contrastively specified for backness (e.g., \emph{kül} `ash' vs.  \emph{kul} `servant', \emph{kepek} `bran' vs. \emph{kapak} `lid'), there are also some minimal pairs with regards to backness (dis)harmony (e.g., \emph{deve} `camel' vs. \emph{deva} `medicine', \emph{sene} `year' vs. \emph{sena} `praise'). This supports the possibility, first suggested by \citet{Clements1982}, that harmonic roots may be underspecified and disharmonic roots fully specified, exemplified below.

%There is some further evidence that individual vowels may differ in specification for this feature even within individual roots or affixes. For instance, the present continuous suffix has harmony-determined allomorphs \emph{-iyor}, \emph{-üyor}, \emph{-ıyor}, \emph{-uyor}, but the \emph{o} of the suffix is invariant. A similar situation might obtain in Turkish roots. 

\begin{example}
Underlying specification of \textsc{Back} for (dis)harmonic roots: 

\xymatrix@R=24pt@C=24pt{
\txt{a.} & \txt{s} & \txt{V} & \txt{n} & \txt{V}  & \txt{\emph{sene} `year'} \\
&        & \txt{[$-$\textsc{Back}]}\ar@{-}[u]     \\
\txt{b.} & \txt{s} & \txt{V} & \txt{n} & \txt{V} & \txt{\emph{sena} `praise'} \\
         &         & \txt{[$-$\textsc{Back}]}\ar@{-}[u] & & \txt{[$+$\textsc{Back}]}\ar@{-}[u]
}
\end{example}

\noindent
Harmonizing suffix vowels will also be underspecified for \textsc{Back}. Of course, \textsc{Backness Harmony} needs to be prevented from overwriting the [$+$\textsc{Back}] specification of disharmonic roots, one option being the use of a a \textsc{Structure Preservation} condition \citep{Kiparsky1985}. Unfortunately, this detail reintroduces the duplication alluded to above. Any condition on the rule of \textsc{Backness Harmony} which prevents overwriting entails that it has no control over the distribution of harmonic and disharmonic roots, and thus fails to account for predominance of harmonic roots.

However, the theory of exceptionality presented in \emph{SPE} (p.~374f.) provides a direct account of suffix harmony in disharmonic roots without duplication. In \emph{SPE}, the specification of the target (i.e., the segment to be changed) of every rule $R$ must be [$+$rule $R$] by convention.  A root or affix which fails to undergo $R$ despite otherwise matching the structural description is simply said to be marked [$-$rule $R$], and thus failing to meet the full structural description. If disharmonic roots are [$-$\textsc{Backness Harmony}], then suffix vowels are correctly predicted to undergo \textsc{Backness Harmony}, since the [$-$\textsc{Backness Harmony}] root is no longer the target but rather the environment, which is not required to be [$+$rule $R$]. 




\subsubsection{Psycholinguistic evidence}

The evidence from alternations leaves open the question of whether speakers internalize a generalization regarding the tendency of roots to conform to \textsc{Backness Harmony}. One piece of evidence that speaks in favor of root-internal harmony comes from a language game discussed by \citet{Harrison2001}.\footnote{Thanks to Bert Vaux for bringing this study to my attention.} This game is not indigenous to Turkish, but it corresponds to a morphological process native to the related language Tuvan, in which it conveys a sense of ``vagueness or jocularity'', and \citeauthor{Harrison2001} report that it can be quickly taught to even young Turkish speakers. The game consists of reduplication of the base and replacement of the first vowel in the reduplicant with \emph{a} or \emph{u}. 

In both Tuvan and Turkish, the unchanged reduplicant vowel is also affected. Reduplication interacts with root harmony in both Tuvan and Turkish. When the base is harmonically [$-$\textsc{Back}], the insertion of a [$+$\textsc{Back}] results in what \citeauthor{Harrison2001} call ``reharmonization'' (\nextx a). 

\begin{example}
\label{redupgame}
Turkish reduplication game \citep[][231]{Harrison2001}: 

\vspace{0.5\baselineskip}
\begin{tabular}{l l l l}
a. & kibrit & kibrit-kabrıt & `match'    \\
   & bütün  & bütün-batın   & `whole'    \\
b. & mali   & mali-muli     & `Mali'     \\
   & butik  & butik-batik   & `boutique' \\
\end{tabular}
\end{example}

\noindent
Under the underspecification hypothesis, non-initial harmonic vowels lack an underlying \textsc{Back} feature, so it comes as no surprise that changing the \textsc{Back} specification of the root-initial vowel results in reharmonization. And the full specification of disharmonic roots correctly predicts that they will be exempt from reharmonization (\lastx b), which is also borne out in Tuvan and in an unrelated Finnish language game \citep{Campbell1986}.

A number of studies have found that speakers of Finnish use disharmony in speech processing experiments.\footnote{Thanks to Charles Yang for pointing out the relevance of these studies to me.} \citet{Suomi1997} and \citet{Vroomen1998} generate nonce trisyllabic words by adding a monosyllabic pseudo-prefix to real and nonce disyllabic words, all of which are harmonic for the feature \textsc{Back}. These stimuli are auditorily presented to subjects who are asked to press a button when the nonce trisyllable ends with a target nonce disyllable, or a real disyllabic word. Speakers are quicker to press the button when the prefix and disyllabic word disagree for \textsc{Back}. These results suggest that speakers are attuned to the fact that disharmonic transisitions are good predictors of word boundaries. If speakers have also internalized the converse generalization, that harmonic transistions are more likely to be root-internal, then there is additional evidence that harmony is active not just in Finnish affix alternations but also in roots. 

\citet{Kabak2010} report that Turkish \textsc{Backness Harmony} has the same effect on word-spotting as it does in Finnish: speakers are quicker and more accurate at the task of spotting the nonce target word \emph{pavo} when preceded by the pseudo-prefix \emph{gölü-}, a disharmonic transistion, than when it is preceded by the pseudo-prefix \emph{golu-}, a harmonic transition. \citeauthor{Kabak2010} find that effect of harmony does not obtain for speakers of French, a language which lacks vowel harmony. As in Finnish, the results imply speakers have internalized the predominance of root-internal harmony.

It seems that the root-internal harmony bias is in fact learned by Turkish speakers very early. The pseudoword spotting experiment has been adapted for 9-month-old Turkish infants by \citet{Kampen2008}. Infants are familiarized with harmonic disyllabic pseudowords bearing a pseudo-prefix, which may be harmonic or disharmonic. At test time, the infants are played the disyllabic pseudowords in isolation using the head turn preference paradigm. Infants show a preference to listen to those pseudowords which were familiarized with a disharmonic pseudo-prefix over those which were familiarized with a harmonic pseudo-prefix. This preference is not observed in 9-month-old infants learning German, which also lacks vowel harmony. Similarly, \citeauthor{Kampen2008} report that Turkish 6-month-old infants prefer to listen to harmonic pseudowords such as \emph{paroz} over disharmonic pseudowords like \emph{nelok}, but German 6-month-old infants show no such preference.

\subsection{Roundness Harmony}

The rule of roundness harmony and the data that motivates it overlaps with the preceding evidence for \textsc{Backness Harmony}. 

\subsubsection{Phonological description}

The environment for roundness harmony differs from \textsc{Backness Harmony} in that it requiers the target to be [$+$\textsc{High}]. 

\begin{example}
\textsc{Roundness Harmony}:

\xymatrix@R=24pt@C=24pt{
\txt{[α \textsc{Round}]}\ar@{-}[d]\ar@{..}[drr] &             & \txt{[$+$\textsc{High}]} \\
\txt{V}                                         & \txt{C$_0$} & \txt{V}\ar@{-}[u] \\
}
\end{example}

This rule, in concert with \textsc{Backness Harmony}, accounts for the forms of the dative singular (dat.sg.) and genitive singular (gen.sg.), among other suffixes.

\begin{example}
Turkish nominal suffix allomorphy: 

\vspace{0.5\baselineskip}
\begin{tabular}{l l l l l l l}
   & \emph{nom.sg.} & \emph{dat.sg.} & \emph{gen.sg.}  \\
a. & ip             & ipi            & ipin           & `rope' & (CS:216) \\
   & el             & eli            & elin           & `hand'    \\
   & kız            & kızı           & kızın          & `girl'    \\
   & sap            & sapı           & sapın          & `stalk'   \\
   & yüz            & yüzü           & yüzün          & `face'    \\
   & köy            & köyü           & köyün          & `village' \\
   & pul            & pulu           & pulun          & `stamp'   \\
   & son            & sonu           & sonun          & `end'     \\
b. & boğaz          & boğazı         & boğazın        & `throat'  & (TELL) \\
   & pelür          & pelürü         & pelürün        & `tissue paper' \\
   & döviz          & dövizi         & dövizin        & `currency' \\
   & yamuk          & yamuğu         & yamuğun        & `trapezoid' \\
   & ümit           & ümiti          & ümitin         & `hope'     \\
\end{tabular}
\end{example}

\noindent
Much as was the case for \textsc{Backness Harmony}, there are disharmonic roots which participate in suffix harmony. Once again, underspecification of harmonic roots and full specification of disharmonic roots allows for the use of the \emph{SPE} exception convention. 

\subsubsection{Psycholinguistic evidence}

The only psycholinguistic evidence for \textsc{Roundness Harmony} comes from \citeauthor{Harrison2001}'s language game (\ref{redupgame}). As shown above, the second vowel in the reduplicated form of \emph{bütün} `whole' is \emph{bütün-batın}. The second vowel in the base, \emph{ü}, is reharmonized as [$+$\textsc{Back}, $-$\textsc{Round}], indicating that \textsc{Roundness Harmony} also participates in reharmonization.

\subsection{Labial attraction}

%\begin{quote}
%\ldots{}these generalizations\ldots{}have no observable consequences in the course of the normal use of the language. \citep[][320]{Zimmer1969}
%\end{quote}

\subsubsection{Phonological description}

\citet[][36]{Lees1966a} describes \textsc{Labial Attraction} as a process by which ``a high, short harmonic vowel is rounded in the second syllable of a disyllabic word whose first vowel is /a/, and whose medial consonant cluster contains a labial /p, b, m, v/, and then it is de-harmonified''. This description is transalted into an autosegmental rule below.

\begin{example}
\textsc{Labial Attraction} (after \citealt[][286]{Lees1966b}, \citealt[][171]{Inkelas2001}): 

\xymatrix@R=24pt@C=8pt{
\txt{[$-$\textsc{Round}]} &                                         & \txt{[$-$\textsc{High}]} & \txt{[$+$\textsc{Labial}]}    & \txt{[$+$\textsc{High}]}\ar@{-}[dr] &         & \txt{[$+$\textsc{Round}]}\ar@{--}[dl] \\
                         & \txt{V}\ar@{-}[ul]\ar@{-}[ur]\ar@{-}[d] &                           & \txt{C$_0$ C C$_0$}\ar@{-}[u] &                                      & \txt{V} & \\
                         & \txt{[$+$\textsc{Back}]}\ar@{-}[urrrr] 
}
\end{example}

The formalization of this rule is naturally complex, and perhaps obscures the fact that \textsc{Labial Attraction} generates exceptions to \textsc{Roundness Harmony}; it produces \emph{aCu} sequences (e.g., \emph{çapul} `raid', \emph{sabur} `patient', \emph{şaful} `wooden honey tub', \emph{avuç} `palm of hand', and \emph{samur} `sable' \citep[][285]{Lees1966b}) instead of the otherwise expected \emph{aCı}. \citeauthor{Lees1966b} notes the existence of exceptions (e.g., \emph{tavır} `mode') but writes that they are they are ``surprisingly rare'' (ibid., 286); \citet[][225]{Clements1982} note additional exceptions.  

There is reason to believe that \textsc{Labial Attraction} is at best a sequence structure constraint as it never applies in derived environments. If, contrary to fact, there was a \textsc{Labial Attraction} alternation, then one would expect, for example, that the gen.sg. of \emph{sap} `stalk' would be *\emph{sapun} instead of the observed \emph{sapın}.

\citet{Lees1966a}
\citet{Zimmer1969}

\begin{quote}
\ldots decisive evidence against a rule of Labial Attraction is the existence of a further, much larger set of roots containing /\ldots~aCu~\ldots/ sequences in which the intervening consonant or consonant cluster does not contain a labial\ldots We conclude that there is no systematic restriction on the set of consonants that may occur medially in rotos of the form /\ldots~aCu~\ldots/. \citep[][225]{Clements1982}
\end{quote}

\noindent
\citeauthor{Clements1982} appear to be suggesting that \textsc{Labial Attraction} implies that \emph{aTu}, where \emph{T} represents a non-labial consonant, should be infrequent, but this fact is not inconsistent with the formulation of the constraint by \citet{Lees1966a,Lees1966b} and \citet{Zimmer1969}; \emph{aTu} clusters do not meet \textsc{Labial Attraction}'s structural description. This is less than conclusive, since the fluctuating vowel alternation also fails to undergo harmony, and might just indicate that the flucutating vowel is epenthesized relatively late in the derivation; the frequency of \emph{aTu} provides only indirect information about \textsc{Labial Attraction} in that it provides a baseline estimate for just how frequent this disharmonic sequence of vowels is.

\citet{Inkelas1997}
\citet{Inkelas2001}

\begin{quote}
Lee's rule of \textsc{Labial Attraction}\ldots is not a real generalization about the Turkish lexicon. It is not true synchronically, either of native or nonnative items; nor, according to the historical and dialectical literature, does \textsc{Labial Attraction} appear to have been true at any stage going back as far as Old Turkic. \citep[][196]{Inkelas2001}
\end{quote}

zimmer: 319-320 
inkelas et al.: 196

\subsubsection{Psycholinguistic evidence}

To the author's knowledge, there are no psycholinguistic studies investigating \textsc{Labial Attraction} beyond the experimental results of \citet{Zimmer1969} reviewed in the next section. 

