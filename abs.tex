This dissertation uses the status of unattested words to provide evidence for the architecture of Universal Grammar (UG).

The first part of the dissertation considers constraints on phonological underlying representations (URs). 
It is proposed that UG countenances no language-specific constraints on URs. 
Putative counterexamples are dismissed as either diachronic contingencies or as induced by synchronic constraints not coterminous with the morph.
A corpus study of the inventory of medial consonant clusters in English simplex words reveals that a number of gaps in this system are accidental in nature, and do not require the structural explanations proposed in prior studies. 
A reanalysis of a study of Turkish wordlikeness judgements reveals that speakers are sensitive to constraints on underlying representations if and only if they are induced by phonological alternations.
A systematic review of English wordlikeness studies reveals once the presence or absence of gross phonotactic violations is controlled for, current gradient phonotactic models are superfluous.

The second part investigates inflectional gaps, complex words judged ineffable.
It is proposed that UG countenances no language-specific constraints on phonological output representations which give rise to ineffability. 
Putative counterexamples are dismissed.
%This indicates that Universal Grammar does not incorporate any inviolable constraints on surface structure. 
It is proven that a model of inviolable constraint learning fails to converge on the proper grammar. 
Models which store complex words are shown to result in extensive undergeneration. 
An analysis of inflectional gaps as reflexes of unproductive morphophonological generalizations is put forth and applied to cases from English, Spanish, Greek, and Russian.
