This dissertation argues for that phonotactic knowledge is primarily derived not from static generalizations but rather from phonological alternations and prosodic parsing. Chapter \ref{intro} outlines the proposal, historical precursors, and predictions. 
%especially with regards to order of acquisition.

Chapter \ref{gradience} considers evidence from wordlikeness rating tasks. It is argued that intermediate well-formedness ratings are obtained whether or not the categories in question are graded. A primitive categorical model of wordlikeness using prosodic representations is outlined, and shown to predict English speakers' wordlikeness judgements at least as accurately as state-of-the-art gradient wellformedness models. Once categorical effects are controlled for, gradient models are uncorrelated with well-formedness ratings.

Chapter \ref{turkish} considers the relationship between lexical generalizations, phonological alternations, and speakers' nonce word judgements with a focus on Turkish vowel patterns. It is shown that even exception-filled phonological generalizations have a robust effect on wellformedness judgements, but that statistically reliable phonotactic generalizations go unlearned when they are not derived from phonological alternations.

Chapter \ref{clusters} investigates the role of phonological alternations in constraining lexical entries, focusing specifically on medial consonant clusters in English. Static phonotactic constraints previously proposed to describe gaps in the inventory of medial clusters are shown to be statistically unsound, whereas phonological alternations impose robust restrictions on the cluster inventory. The remaining gaps in the cluster inventory are attributed to the sparse nature of the lexicon, not static phonotactic restrictions.
