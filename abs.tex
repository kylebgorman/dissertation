This dissertation outlines a program for the theory of phonotactics---the theory of speakers' knowledge of possible and impossible (or likely and unlikely) words---minimizing undue \emph{duplication} (in the sense of \citealt{KK77}:136f.) of phonotactics and phonology, and argues that the alternative view of phonotactics as stochastic, and of phonotactic learning as probabilistic inference, is not capable of accounting for the facts of this domain.
Chapter \ref{intro} outlines the proposal, precursors, and predictions. 
%especially with regards to order of acquisition.

Chapter \ref{gradience} considers evidence from wordlikeness rating tasks. It is argued that intermediate well-formedness ratings are obtained whether or not the categories in question are graded. A primitive categorical model of wordlikeness using prosodic representations is outlined, and shown to predict English speakers' wordlikeness judgements as accurately as state-of-the-art gradient wellformedness models. Once categorical effects are controlled for, gradient models are uncorrelated with well-formedness ratings.

Chapter \ref{turkish} considers the relationship between lexical generalizations, phonological alternations, and speakers' nonce word judgements with a focus on Turkish vowel patterns. It is shown that even exception-filled phonological generalizations have a robust effect on wellformedness judgements, but that statistically reliable phonotactic generalizations go unlearned when they are not derived from phonological alternations.

Chapter \ref{gaps} investigates the role of phonological alternations in constraining lexical entries, focusing specifically on medial consonant clusters in English. Static phonotactic constraints previously proposed to describe gaps in the inventory of medial clusters are shown to be statistically unsound, whereas phonological alternations impose robust restrictions on the cluster inventory. The remaining gaps in the cluster inventory are attributed to the sparse nature of the lexicon, not static phonotactic restrictions.
