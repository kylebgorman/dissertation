This dissertation considers two types of gaps in the lexicon. In both cases, it is argued that current generative models can be modified minimally to account for the apparent overgeneration of missing structures. 

The first part of the dissertation considers the underlying phonological structure of morphs, specifically medial consonant clusters in English. Many prior studies have argued for morpheme structure constraints, generalizations that account for the underlying shape of morphs. It is shown, however, that many of the gaps in the inventory of medial consonant clusters are due to sampling from sparse data, cannot be reliably predicted by current computational models of phonotactic learning, and can thus be regarded as accidental gaps. In contrast, phonological alternations induce reliable structural gaps in the inventory of medial clusters. This suggests that language-specific phonotactics can be derived without a separate phonotactic module (\citealt{Halle1962}). 
%This model is validated with a meta-analysis of 5 English wordlikeness studies.

The second part considers the inflectional patterns in a number of languages. There are many languages in which a certain class of words unexpectedly fails to 
Hungarian and Kinande. 

The use of elsewhere rules to account for allomorphy fail to 

This model is validated using data on inflectional gaps in English, Spanish, Russian, Swedish, and Greek.

% Keywords: Gaps, sparsity, Zipf's Law, acquisition, phonotactics, wordlikeness, syllable contact, inflectional gaps, defectivity, lexical exceptions
% Languages: Turkish, English, Latin, Hungarian, Kinande, Spanish, Russian, Swedish, Greek
