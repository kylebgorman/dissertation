It is my intention that this document is far more than an existence proof.

Perhaps the most important fact for the study of speakers' internal knowledge of language is Wilhelm von Humboldt's observation that speakers make ``infinite use of finite means'' CITE \citep{Chomsky1966}. 

An enduring demonstration of ``infinite use'' is the famous emph{wug-test} \citep{Berko1958}. Jean Berko-Gleason introduced English children aged ?-? to pictures of concrete objects (such as a small bird), labeled them with a nonce word (``that's a \emph{wug}''), and then asked them to finish the sentence

The children in the study produced the same response you may have: the plural is \emph{wugs} (where $\langle$s$\rangle$ writes a final [z]). 

Of course, children have never heard the plural of these nonce words, but they extend /-z/ to these new words. 

But what of the ``finite means'' from which grammars spring? On this, a good deal less has been said.
a study of spaekers' knowledge of natural
The most important fact for 

In this dissertation, I critique two such epicyclic proposals, and present solutions using independently-motivated grammatical machinery to account for the data. 

The first half of this dissertation began with a suggestion from Charles Yang that the sparse nature of the lexicon might itself account for at least some phonotactic gaps. The success of the resulting study, in chapter four, led me to question the received wisdom that speakers' knowledge of phonotactic generalizations is gradient and potentially independent of the phonological system. 

These results support a null hypothesis in which phonotactic knowledge is derivative of the language acquisition device and phonological processes. While I present this theory as a falsifiable scientific hypothesis, an alternative would treat this proposal as a \emph{negative heuristic}, a 

guides us away from ``phonotactic grammar'' as an explanatory device. 

even those who are not inclined to accept this proposal should note how much of the coverage traditionally attributed to
