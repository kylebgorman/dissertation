
Perhaps the most important fact for the study of speakers' internal knowledge of language is Wilhelm von Humboldt's observation that speakers make ``infinite use of finite means'' CITE \citep{Chomsky1966}. 

An enduring demonstration of ``infinite use'' is the famous emph{wug-test} \citep{Berko1958}. Jean Berko-Gleason introduced English children aged ?-? to pictures of concrete objects (such as a small bird), labeled them with a nonce word (``that's a \emph{wug}''), and then asked them to finish the sentence

The children in the study produced the same response you may have: the plural is \emph{wugs} (where $\langle$s$\rangle$ writes a final [z]). 

Of course, children have never heard the plural of these nonce words, but they extend /-z/ to these new words. 

But what of the ``finite means'' from which grammars spring? On this, a good deal less has been said.
a study of spaekers' knowledge of natural
The most important fact for 

In this dissertation, I critique two such epicyclic proposals, and present solutions using independently-motivated grammatical machinery to account for the data. 
