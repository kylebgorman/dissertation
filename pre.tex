
Perhaps the most important fact for the study of speakers' internal knowledge of language is Wilhelm von Humboldt's observation that speakers make ``infinite use of finite means'' CITE \citep{Chomsky1966}. 

An enduring demonstration of ``infinite use'' is the famous emph{wug-test} \citep{Berko1958}. Jean Berko-Gleason introduced English children aged ?-? to pictures of concrete objects (such as a small bird), labeled them with a nonce word (``that's a \emph{wug}''), and then asked them to finish the sentence

The children in the study produced the same response you may have: the plural is \emph{wugs} (where $\langle$s$\rangle$ writes a final [z]). 

Of course, children have never heard the plural of these nonce words, but they extend /-z/ to these new words. 

But what of the ``finite means'' from which grammars spring? On this, a good deal less has been said.
a study of spaekers' knowledge of natural
The most important fact for 

In this dissertation, I critique two such epicyclic proposals, and present solutions using independently-motivated grammatical machinery to account for the data. 

The first half of this dissertation began with a suggestion from Charles Yang that the sparse nature of the lexicon might itself account for at least some phonotactic gaps. The success of the resulting study, in chapter four, led me to question the received wisdom that speakers' knowledge of phonotactic generalizations is gradient and potentially independent of the phonological system. 

These results support a null hypothesis in which phonotactic knowledge is derivative of the language acquisition device and phonological processes. While I present this theory as a falsifiable scientific hypothesis, an alternative would treat this proposal as a \emph{negative heuristic}, a 

guides us away from ``phonotactic grammar'' as an explanatory device. 

even those who are not inclined to accept this proposal should note how much of the coverage traditionally attributed to

can be accounted for by the nu

The second half of the dissertation develops a theory of inflectional gaps. I was pleased to hear (from Greville Corbett) that Stephen Anderson regarded inflectional gaps as ``the biggest scandal in morphology'', and even more so when Prof. Anderson agreed to serve as the external member of my committee.

In developing this theory, I have focused primarily on how inflectional gaps could be learned, and thus my approach is does not directly bear on competing morphological theories. At the same time, this theory is consistent with a proposal by \citet{Pesetsky1977} that inflectional gaps provide novel evidence for the ordering of various components of morphology. As an example, consider a few Russian verbs exhibiting inflectional gaps in the 1sg non-past. 

\begin{example}[Russian 1sg. imperfect gaps (after \citealp{Pesetsky1977})]
\begin{tabular}{l l l@{}l@{}l@{}l@{}l@{} l}
   & infinitive &   & 1sg. non-past \\
a. & pobedit{j} & * & pobeʒu  & / & * & pobeʒdu  & `conquer'   \\
   & ubedit{j}  & * & ubeʒu   & / & * & ubeʒdu   & `persuade'  \\
b. & tʃudit{j}  & * & tʃuʒu   & / & * & tʃuʒdu & `act oddly'   \\
   & otʃudit{j} & * & natʃuʒu & / & * & natʃuʒdu & `alienate'  \\
\end{tabular}
\end{example}

\noindent As \citeauthor{Pesetsky1977} notes, the ill-formedness of the 1sg. non-past forms is shared by verbs related by prefixation. The opposite pattern can be seen in English. 

\begin{example}pEnglish preterite gaps]
\begin{tabular}{l l l@{}l@{}l@{}l@{}l@{}}
   & present  &   & preterite &   &   &           \\
a. & foresee  &   & foresaw   &   &   &           \\
   & sightsee & * & sightsaw  & / & * & sightseed \\
b. & undergo  &   & underwent &   &   &           \\
   & forgo    & * & forwent   & / & * & forgoed   \\
\end{tabular}
\end{example}

Unlike the pattern observed in Russian, verbs that appear to share the same root do not show the same patterns of defectivity. If we simply label the process of verbal inflection as the source of inflectional gaps, we can understand this contrast as follows: in Russian, inflection precedes and feeds prefixation; in English, the formation of complex verb stems and feeds the inflectional system. 

While at this time I do not have any insights derived from these facts, there exists a potential to use inflectional gaps for evidence of ordered operations in morphology. These inferences are available to any account that recognizes inflectional gaps as something about the inflectional system, including the one presented here, but not if inflectional gaps are due to phonological properties of the derived objects.

The two parts of this dissertation were conceived and executed separately, but some uniting principles can be identified. Both parts argue against inviolable, language-specific constraints in favor of a derivational explanation for the facts at hand. Both parts derive the phenomena at hand using novel acquisitional strategies (\emph{occultation} and \emph{tolerance}, respectively) and independently motivated components of the grammatical architecture (alternation phonology and the elsewhere principle). And the accounts presented in both parts are inspired by concerns for the sparsity of the primary linguistic data.
