\subsection{Data sources}

While these are by no means the only wordlikeness s studies in Englihs, these studies all have the following characteristics: 
the item-averaged ratings are reported in the document,
the stimuli are presented auditorily,
and at least some stimuli are not phonotactically valid in English.

(some caveats about subjects

\citep{Shademan2007}

cost of prior averaging
\citep{Baayen2004}

\subsubsection{\citealt{Greenberg1964}}

\citet{Greenberg1964} investigate English wordlikeness using the technique of free magnitude estimation, a mechanism which has become increasingly popular in syntax \citep[e.g.,][]{Bard1996,Sprouse2011}. At the beginning of each run of the experiment, a subject heard a recording of the word \emph{stick}. In each subsequent trial, the subject heard a new item and was asked to assign a value for ``How far would you say that is from English?'', with \emph{stick} representing ``1''. They were explicitly instructed that a word which was ``twice as far from English'' as \emph{stick} should be given ``a number that is twice as large'' (op. cit., 160). The data used here are from their ``experiment B'', in which 17 undergraduates were presented 13 items. This is the only study used here in which actual English words (\emph{stick}, \emph{grass}, \emph{spell}, and \emph{truck}) were intentionalyl included among the stimuli.

\subsubsection{\citealt{Scholes1966}}

\citet{Scholes1966} conducts a number of English wordlikeness judgement with middle school children. The data used here come from his ``experiment 5'', in which 66 items were presented to 33 seventh-grade students. For each stimulus, the subjects produced forced choice ``yes''/``no'' answers to the question of whether the item ``is likely to be usable as a word of English''. \citet{Hayes2008a} and \citet{Albright2009a} analyse this data, which is binary at the trial level, as gradient by performing an item averaging using the fraction of ``yes'' answers for each stimulus.

%mrupation problem

\subsubsection{\citealt{Albright2003b}}

\citet{Albright2003b} gather wordlikeness data to provide norms for the \emph{wug}-task that is the focus of their study. 87 items were presented to 20 undergraduate subjects, who rated each word on a seven-point Likert scale; though \citeauthor{Albright2003b} do not provide the precise instructions the subjects were given, they do report that ``1'', the lowest point on the scale, was labeled
``completely bizarre, impossible as an English word'', and that the highest point, ``7'', was named ``complete normal, would make a fine English word''. 

\subsubsection{\citet{Albright2007}}

Wordlikeness is the focus of an unpublished study by \citet{Albright2007}. \citeauthor{Albright2007} presetns an unknown number of subjects with 40 nonce words in addition to 170 fillers, which are rated on the same seven-point Likert scale used by \citet{Albright2003b}.
