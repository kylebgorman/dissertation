\label{intro}


\section{A program for phonotactic theory}



Theories of phonotactic knowledge should be evaluated by the same stringentcriteria applied to other formal arenas: neither overgeneration nor undergeneration should be permitted. It will be shown that the orthodoxy that phonotactic knowledge is in some sense ``probabilistic'' suffers from quite severe over- and undergeneration. 

This heuristic has the greatest impact regarding debates about ``abstractness'' of underlying representations. 
It cannot be said, precisely, that it either excludes or requires any particular type of abstractness. 

Proposed restrictions on UR abstractness have been faulted for a number of reasons. In some cases, they preclude otherwise-desirable analyses of alternations. Another objection that could reasonably made is that any restriction on abstract URs that introduces an assymetry in URs is bad.
\citep[~chap.~1]{KK77}

I would like to suggest that there are two types of problems that arise in developing a theory of phonotactics free of duplication. The first type of problem consists of conflicts between theoretical assumptions and the desire to eliminate duplication. If we continue to view duplication as a sort of negative heuristic, then it may be the case that the theoretical assumptions are wrong. Such a case arises in Chapter \ref{turkish} in the discussion of archiphonemic underspecification analysis of Turkish vowel harmony proposed by \citet{Clements1982}. \citeauthor{Clemenst1982} propose that all but the first vowel of a harmonic root is underspecified for backness (and in high vowels, roundness) and is filled in by rule. Since there are disharmonic roots, this rule must be ``structure-filling''. Consequently, this rule cannot account for the apparent markedness of disharmonic roots revealed by wordlikeness judgements (among other psycholinguistic tasks). If duplication is a pathology, this analysis is wrong.

Another type of problem is the ``conundrum'', an apparent fact which poses a problem for virtually any generative theorists. 

pure allophony

There are a few contexts which this interacts:

- prosody
- exceptionality
- wordlikeness

Syntax can hardly be described as a theory of word ``arrangement''; it would be inconceivable for syntacticians to engage in a lengthy debate about whether some word sequence is permitted in some language. Yet it is possible for two reasonable phonologists to dispute whether [k.p] is a permissible medial consonant cluster in English. While lexical entries are thought to consist of segments, and segments to consist of feature specifications, the phonological form of a word like \emph{spectre} is not ``generated'', in any relevant sense, by the concatenation of segments /s/, /p/, and so on, or by the concatenation of syllables \{spɛk\}$_{\sigma}$ and \{tɚ\}$_{\sigma}$. It would be possible to posit ``lexical concatenation rules'', on analogy with the phrase structure rules of syntax, but it is unclear how the strings produced would be linked with meanings, or how such a system would distinguish between, e.g., \emph{brick} and meaningless \emph{blick}.

The more appealing alternative is that the phonological form of \emph{spectre}, whatever it is, is an atom of unit of lexical memory. It remains to account for any restrictions on the contents of underlying forms. The theory of \emph{phonotactics} (which is related to \emph{syntax} through the Greek root \emph{táxis} `order') is concerned with these restrictions and how they are related to the mappings between underlying and surface forms for which the phonological component is responsible. 

%\footnote{It is tempting, however, to describe binding principles as ``syntactics''.}

I believe that the boundaries of this knowledge

The boundaries of this knowledge
and what might be acquired.
These are not unrelated issues:

One question is: why bother?

It is certainly possible to imagine otherwise. \citet[][320]{Zimmer1969} writes that phonotactic generalizations ``have no observable consequences in the course of the normal use of the language'', and \citeauthor{PE} echo this sentiment more recently:

\begin{quote}
Even if we, as linguists, find some generalizations in our description of the lexicon, there is no reason to posit these generalizations as part of the speaker's knowledge of their language, since they are computationally inert and thus irrelevant to the input-output mapping that the grammar is responsible for. \citep[][18]{PE}
\end{quote}

This represents a principled null hypothesis, but can be quickly dismissed in light of speaker's ready judgements of possible and impossible words observed by \citeauthor{Chomsky1965}. Metalinguistic judgements of these sorts, reviewed in Chapter \ref{wordlikeness}, are not the only task which exemplify phonotactic knowledge; infants and adults are also thought to make use \emph{possible word constraint} in more quotidien tasks like recognizing words in running speech \citep[e.g.,][]{Brown1956,Mattys1999,McQueen1998b,Norris1997}.

\subsection{Units of phonotactic description}

\subsubsection{Segments}

Nothing will be said about what \citet{Stanley1967} calls \emph{segment structure rules}, or language-specific constraints on the inventory of (archi)phonemes, since there is a great deal of overlap between competing theories. Consider possible explanations for the apparent absense of ejectives in English, It might be that English does, in a certain sense, permit underlying ejectives, but such segments are uniformly realized as plain (i.e., pulmoni) voiceless stops: in other words, the inventory is an epiphenomenon of phonological processes. Or, perhaps English lacks the features needed to contrast ejectives with plain voiceless stops, in which case the inventory is an epiphenomenon of the system of contrast. Or, perhaps the absense of ejectives follows from no particular fact, and has the grammatical status of an accidental gap. Further assumptions are needed to distinguish these analyses.

\subsubsection{Underlying representations}

The term \emph{sequence structure constraint} is now somewhat imprecise given the apparent obsolescence of boundary segments: the ``sequence structure onstraint'' that rules out onset /bn/ in English must actually be stated as a constraint militating against /\#bn/. \citep[149]{Kenstowicz1977}

\subsubsection{Surface sequences}

% discussion of Ernestus & Baayen 2003, Becker et al. 2011

The notion of blocking in non-derived environments (\citealp[163]{Kiparsky1973a}, \citeyear[152]{Kiparsky1982a}, \citealp{Mascaro1976}) is well-known, but \citet{Hall2006} also amasses evidence for phonological processes restricted to non-derived environments. One of the most famous examples is the distribution of \emph{ich-laut} [ç] and \emph{ach-laut} [x] in German \citep{Bloomfield1930}. The dorsal fricative is [x] before back vowels and[ç] in other contexts.

\begin{example}[German \emph{ich}- and \emph{ach-laut}]
\begin{tabular}{l l l}
a. & [buːx]   & `book'           \\
   & [tɔxtər] & `daughter'       \\
   & [naxt]   & `night'          \\
b. & [siçt]   & `view'           \\
   & [ʃpeçt]  & `woodpecker'     \\
   & [ɡərʏçt] & `rumor'          \\
   & [knøçəl] & `ankle, knuckle' \\
   & [flɛçə]  & `surface'        \\
\end{tabular}
\end{example}

\emph{Umlaut}, the fronting (and raising) of back vowels in certain morphological contexts, produces the front variant of the dorsal fricative; e.g., [lɔx]-[løçər] `hole-holes'.
%\emph{B}[uːx]-\emph{B}[yːç]\emph{er} `book-books',

\emph{K}[uːx]\emph{en}
\emph{K}[uːç]\emph{en}

%\emph{Mas}[oːx]-\emph{Mas}
\footnote{Examples like \emph{Mas}[oːç]\emph{ist} `masochist', \emph{Eun}[uːç]\emph{ismus} `eunichism', first noted by \citet{Merchant1994}, suggest that this should also be restricted to assimilation within the same foot \citep[226f.]{Jensen2000}.}
%\emph{Mas}[oːx]-\emph{Mas}[oːç]\emph{ist}

\section{The model}

\subsection{Syllabification}

\subsection{Rule application and occultation}

\subsection{Detecting violations}

This is precisely the case for 

\subsection{Exceptionality}
% autosegments

\section{Predictions of the model}

\subsection{Gradience}
\subsubsection{Probabilistic well-formedness}
\subsubsection{External factors}

\subsection{Static phonotactics}
\subsubsection{Insensitivity to statistics}
\subsubsection{Saussurean arbitrariness}

\subsection{Order of acquisition}

A point of departure is a review of phonotactic acquisition by \citet{Hayes2004b}.

\emph{pure phonotactic learner}
%...
However, \citeauthor{Hayes2004b} ignores the considerable evidence that infants younger than nine months of age---the \emph{terminus ante quem} for the onset of phonotactic knowledge---already have relatively rich and detailed lexical knowledge

A word of caution is in order. An experiment which finds typically-developing infants of a certain age insensitive to an adult-like contrast using a certain task is by no means sufficient evidence to conclude that infants of this age lack this ability. A negative result may simply indicate that the task lacks ecological validity or requires domain-general cognitive resources beyond those of the infants. For instance, \citet{Werker2002} report that 14-month-old infants are insensitive to small phonological differences between novel words, but \citet{Fennell2006} shows that they are sensitive to the same contrasts when the novel words are situated in a more naturalistic naming scenario and cognitive demands are minimized. Other negative findings might be attributed to failure to control for properties of the stimulus, or to an experimental design which lacks sufficient statistical power.

\citet{Jusczyk1993b} native vs. non-native
\citet{Jusczyk1994} no preference for high prob
\citet{Friederici1993}

Finally, \citet{Mattys1999} find that 9-month-old infants treat hetero-organic nasal-stop clusters in nonce words as indications of word juncture, though this effect is small compared to the effect of other cues like primary stress: they do not test younger infants.

\subsubsection{Prosodic parsing}

Newborn infants are already able to distinguish between monosyllabic and bisyllabic words, but these infants are thought to recognize syllables ``holistically'' rather than as sequences at least until 4 months \citep[e.g.,]{Bertoncini1981,Eimas1999,Jusczyk1987}. The earliest evidence for segmental representations comes not from phonotactic preferences, but from infants' ability to dicriminate between familiar words like \emph{cup} and mispronunciations like \emph{*tup} at 7.5 months of age \citep{Jusczyk1995}.

\subsubsection{Phonological processes}

Very few studies have investigated young infants' knowledge of phonological alternations. One exception is a fascinating study by \citet{White2008}. Simplifying somewhat, the experimenters expose 8.5-month-old and 10-month-old infants to an artificial language in which the voicing of fricatives is contrastive, but voiced and voiceless variants of plosives are in a complementary distribution, appearing only after vowels (\emph{na-bevi}) and after voiceless consonants (\emph{rot-pevi}), respectively. After familiarization, infants at both ages prefer to listen to stimuli which preserve the complementary distribution over those which disrupt it (e.g., \emph{na-poli}, \emph{rot-boli}). As \citeauthor{White2008} note, this suggests that infants have both extracted the plosive voicing alternation and grouped \emph{pevi} and \emph{bevi} together.

\subsubsection{Lexical acquisition}

As first observed by \citet{Darwin1877}, the first words infants recognize are names, their own and those of caretakers. Recent studies suggest that infants learn these names as early as 4 months of age \citep{Bortfeld2005,Mandel1995,Tincoff1999}. Infants as young as 6 months of age have learned the meaning of familiar words \citep{Bergelson2012}. By 8 months of age, infants are able to locate familiar and novel words in utterances \citep{Jusczyk1997,Seidl2006}.

\section{Outline of the dissertation}

The remainder of the dissertation consists of three case studies which provide support for the novel and contentious predictions of the minimal phonotactic model sketched above.

Chapter \ref{gradience} considers evidence from wordlikeness rating tasks. It is argued that intermediate well-formedness ratings are obtained whether or not the categories in question are graded. A primitive categorical model of wordlikeness using prosodic representations is outlined, and shown to predict English speakers' wordlikeness judgements at least as acccurately as state-of-the-art gradient wellformedness models. Once categorical effects are controlled for, gradient models are uncorrelated with well-formedness ratings.

Chapter \ref{turkish} considers the relationship between lexical generalizations, phonological alternations, and speakers' nonce word judgements with a focus on Turkish vowel patterns. It is shown that even exception-filled phonological generalizations have a robust effect on wellformedness judgements, but that statistically reliable phonotactic generalizations go unlearned when they are not derived from phonological alternations.

Chapter \ref{clusters} investigates the role of phonological alternations in constraining lexical entries, focusing specifically on medial consonant clusters in English. Static phonotactic constraints previously proposed to describe gaps in the inventory of medial clusters are shown to be statistically unsound, whereas phonological alternations impose robust restrictions on the cluster inventory. The remaining gaps in the cluster inventory are attributed to the sparse nature of the lexicon, not static phonotactic restrictions.

%% NOTES HERE
%
%a. & \emph{B}[uːx]          & \emph{B}[yːç]\emph{er} & `book(s)' \\
%   & \emph{L}[ɔx]           & \emph{L}[øç]\emph{er}  & `hole(s)' \\
%   &\emph{T}[ɔx]\emph{ter} & \emph{T}[øç]\emph{ter}
%   & \emph{N}[ax]\emph{t}   & \emph{N}[ɛç]\emph{te} & `night' \\
%
% While the bulk of evidence for detailed phonological entries is found in infants over a year of age \citep[e.g.,]{Werker2002,Fennell2003,Fennell2006,Stager1997,Swingley2000,Zamuner2006},
%9-month-old infants already have learned the predominant stress pattern of their language \citep{Jusczyk1993a}.
%\emph{τάξις}
%Whether phonological computations or representations themselves are graded \citep[e.g.,][]{Lakoff1973} is besides the point, as metalinguistic judgements are behaviors, not mental states; they can no more be compared than can ``fear'' and ``flight response''.

\citet{Silverman2000}
\citet{Kiparsky1995}
\citet[352f.]{Hale2003a}
%$\begin{bmatrix} +\textsc{Dorsal} \\ +\textsc{Continuant \end{bmatrix}~\goesto~\begin{bmatrix} =\textsc{Back} \end{bmatrix}~/~\begin{bmatrix} =\textsc{Back} \end{bmatrix}~\gap$
\citeauthor{SPE} propose to derive gradience from the complexity (using a simple feature-counting metric) of the redundancy rules needed to map a nonce word to an attested word, and they show that this derives the increasing cline of wordlikeness running from ``consonant soup'' \emph{bnzk} to ill-formed \emph{bnick}, possible \emph{blick}, and finally lexical \emph{brick}.

Despite the considerable attention given to the proposals of \emph{SPE} in the wake of that book's publication in 1968, the \emph{SPE} wordlikeness model has received almost no further attention in the literature. At the risk of explaining what might be no more than an accidental gap in the literature, the novel aspects of \emph{SPE} model---gradience derived from similarity to existing lexical entries---may have been overshadowed by the many other contentious proposals in \emph{SPE}, and particularly by compelling arguments against the assumption that wordlikeness contrasts derive solely from properties of underlying forms. \citet{Shibatani1973} observes that there are some generalizations about surface forms which give rise to wordlikeness contrasts, but cannot be stated as constraints on underlying forms. An example from German is shown in (\ref{fd}) below.

\begin{example}[German final devoicing] 
\label{fd}
\begin{tabular}{l l l l}
   & nom.sg. & nom.pl.    \\
a. & [piːp]    & [piːpə]  & `cheep(s)'      \\
   & [diːp]    & [diːbə]  & `thief/thieves' \\
b. & [ɡʀaːt]   & [ɡʀaːtə] & `ridge(s)'      \\
   & [ɡʀaːt]   & [ɡʀaːdə] & `degree(s)'     \\
\end{tabular}
\end{example}

\noindent The plural appears to preserve a contrast in final obstruent voicing which is absent in the singular:\footnote{While there is a contentious debate as to whether devoicing in German is completely neutralizing \citep[e.g.,][]{Fourakis1984} or not \citep[e.g.,][]{Port1985}, it is irrelevant to the discussion at hand.} word-final voiced stops are never found in German, and \citet[95]{Shibatani1973} reports that ``it is easy to show that a native speaker of German rejects those forms \emph{on the ground that they end in voiced obstruents}'' (emphasis in original). However, given that root-final voicing is not predictable (e.g., /ɡʀaːt \alt{} ɡʀaːd/), the process of \textsc{Final Devoicing} cannot be any kind of lexical redundancy and is mysterious under the \emph{SPE} account. The alternative proposed by \citeauthor{Shibatani1973} and by \citet{Clayton1976}, is that to say that nonce words like [ɡʀaːd] are ill-formed not because of any underlying property, but since they fail to undergo an otherwise-exceptionless phonological process, \textsc{Final Devoicing}, or equivalently, the surface-true generalization it derives.

\citet{Sommerstein1974} further notes that the computation of hypothetical underlying forms from nonce surface forms which is implied by the \emph{SPE} theory is non-trivial in the presence of phonological opacity \citep[see][528f.]{Anderson1988a}, and the rejection of the biuniqueness principle means that there is not always a unique solution, as is illustrated by the two underlying forms corresponding to [ɡʀaːt] in (\ref{fd}) above.

\subsubsection{Autosegmental phonology and beyond}

The arguments of \citeauthor{Shibatani1973} and others led theorists to focus their attention on properties of surface representations as determinants of wordlikeness. Though syllabification plays no role in \emph{SPE}, it is crucial to many earlier studies \citep[for a review, see][]{Goldsmith2011b}, and it received particular attention in the 1970s. \citet{Hooper1973} and \citet{Kahn1976} argue that the syllable is useful for defining wordlikeness generalizations.\footnote{\citet{Steriade1999} and \citet{Blevins2003}, however, argue that a number of phonotactic generalizations previously stated in syllabic terms can be reanalyzed without making reference to syllables.} \citeauthor{Hooper1973} argues, for instance, that [bn], impossible as an English onset, is unobjectionable as a syllable contact cluster in nonce words like \emph{stabnik} (or in names like \emph{Abner}), and that this demonstrates the superiority of syllable-based wordlikeness generalizations. This already signals further trouble for alternative accounts which focus on underlying forms. Syllabification may span morphs, is generally predictable, and is universally non-contrastive, and as a consequence, few posit in to be present in underlying representations \citep[though see, e.g.,][]{Vaux2003}. \citeauthor{Hooper1973} also points to loanword adaptations which produce native syllable structure \citep[e.g.,][]{Carlisle1991} as evidence that syllabification is part of the phonological computation. Further enrichments to the theory are provided by the autosegmental theory of the syllable \citep{McCarthy1979b}, which envisions the syllable as an articulated tree structure \citep[as first envisioned by][]{Pike1947a}, and theories like prosodic licensing \citep{Ito1989a}, in which syllabification triggers phonological repairs.

While none of these authors discuss gradience, the syllable plays a role in the definition of a gradient measure, ``positional probability'', though to correlate closely with human judgements of wordlikeness. The positional probability of a nonce monosyllabic word is derived from the combined probabilities at which segments occur in onset, nucleus, and coda positions in the lexicon of a given language. 

\citet{Shibatani1973}, however, argues that morpheme structure constraints cannot account for all wordlikeness constraints.
In German, for instance, final obstruents devoice, and as a result, there a
re pairs such  [ɡʀaːt]-[ɡʀaːtə] `ridge(s)' and [ɡʀaːt]-[ɡʀaːdə] `degree(s)'
 differing only in the plural shape of the root.
Since the voicing of final obstruents is contrastive, the restriction on th
e voicing of obstruents cannot be a morpheme structure constraint. 
Yet, \citeauthor{Shibatani1973} claims that native speakers reject nonce wo
rds ending in final voiced obstruents ``\emph{on the ground that they end i
n voiced obstruents}'' (95). 
Thus, not all constrasts in possible wordhood can be stated as morpheme str
ucture constraints.

Whereas \citeauthor{Shibatani1973} maintains that morpheme structure constraints are insufficient to account for speakers' knowledge of possible wordhood, others argue that morpheme structure constraints are also unnecessary. \citet[297]{Hale1965}, \citet{Kisseberth1970b}, and \citet[212f.]{Postal1968} observe that the structural descriptions of phonological processes often are often reflected in the lexical redundancies in the same language. In Russian, there are alternations implicating a process of obstruent voice assimilation, and tautomorphemic obstruent clusters have uniform voicing \citep[283]{A74}. \citet[205f.]{Dell1973} and \citet[28f.]{Stampe1973} argue that morpheme structure constraints are otiose, as phonological rules triggering alternations impose restrictions on the contents of URs, an effect known as \emph{Stampean occultation}.

Since at least \citet{Pike1947b}, linguists have posited language-specific constraints on the contents of underlying representations.
These \emph{morpheme structure constraints} were at one time thought to fully account for speakers' knowledge of possible and impossible words (e.g., \citealt{Chomsky1965}, \citeyear[382]{SPE}, \citealt[22f.]{SPR}, \citeyear{
Halle1962}, \citealt{Stanley1967}).
