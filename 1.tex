%\chapter{Sequence structure constraints in generative grammar}
\label{msc}

The first part of this text concerns the empirical substance of the theory of \textsc{Morpheme Structure Constraints} (MSCs), that is, language-specific constraints on the shapes of morphs stored in lexical memory.
Evidence is presented to support the claim that such constraints are solely derived by other components of grammar, and do not have an ontology of their own.

The first section of this chapter outlines a brief history of MSCs.
The following section draws upon this material to present the above claim as a null hypothesis, and presents conditions for falsifying it.
The final section outlines the remainder of Part I (chapters 2--4), which present evidence in support of this hypothesis.

% it is not clear what follows from a phonotactic description
% pater on NC
% can't be right tho, see stanley, kisseberth, clayton, shibatani, hooper, etc.

\section{A brief history of MSCs}

\subsection{Structuralism}

Since, with the possible exception of Roman Jakobson (who might be labeled an early generativist), the mentalistic interpretation of structural linguistic analyses was anathema, one must take care to not assume that a description of MSCs is something that these linguists would posit as part of speakers' knowledge of language.

\subsubsection{Morpheme boundaries in phonology}

bloomfield

%While \citeauthor{Bloomfield1930} is known for introducing morphemic structure into phonology, a member of the Prague circle developed an arguably more sophisticated approach only two years later. 

\citet{Jakobson1932} attributes a number of phonological properites of the Russian imperative to morpheme juncture. The vast majority of Russian infinitives consist of the stem followed by a theme vowel and /t\pal/. The reflexive is marked wih a final /-sə/ which feeds a general process of regressive voicing assimilation. In the reflexive infinitive, /\ldots t\pal-s\ldots/ is realized [ts], with no palatalization. However, palatalization of a stem-final labial or coronal---including /t\pal/, as in (\ref{russian}c)---is preserved in the reflexive imperative.\footnote{I have taken a number of liberties with \citeauthor{Jakobson1932}'s presentation of the data, which uses an abstract phonemic transcription. Thanks to Lev Blumenfeld (p.c.) for help with the transcription of this data.}

\begin{example}[Russian reflexives (after \citealt[][]{Jakobson1932}] \label{russian}
\begin{tabular}{l l l l} %\toprule
   &  infinitive      & imperative \\ %\midrule
a. & [slav\pal itsə]  & [slaf\pal s\pal ə]  & `be glorious'    \\
   & [upram\pal itsə] & [upram\pal s\pal ə] & `be stubborn'    \\
b. & [kras\pal itsə]  & [kras\pal s\pal ə]  & `put on makeup'  \\
   & [ʒar\pal itsə]   & [ʒar\pal s\pal ə]   & `roast'          \\
c. & [zəbytsə]        & [zəbut\pal s\pal ə] & `forget' \\ %\bottomrule
%   & ab\'utsa      & ab\'ujsa     & `put on shoes'   \\
\end{tabular}
\end{example}

\noindent The forms in (\ref{russian}c) have different outcomes for their /t\pal-s/ clusters. \citeauthor{Jakobson1932} proposes that the preservation of palatalization is a special property of the imperative. %A few years later, \citet{Trnka1936}, another member of the Prague Circle, makes the connection between \citeauthor{Jakobson1932}'s hypothesis and phonotactic generalizations.

\subsubsection{The morph as a constraint domain}

A corrolary of this hypothesis is that the phoneme sequences found within morphs may be distinct from those which span multiple morphs. 

This was apparent to structuralists 

the phonologists 
This is apparent from 
a discussion of constraints on vowel sequences in Mixteco given by \citet{Pike1947b}.
It is apparent that \citeauthor{Pike1947b}'s constraints on vowel sequences are not intended to hold across morph boundaries, 

according to the phonological analysis of a Mixteco text published a few years earlier \citep{Pike1944}.

\begin{example}
Mixteco MSCs \citep{Pike1947b} and complex words \citep{Pike1944}:

\begin{tabular}{r l l l} %\toprule
   & MSC & complex exception \\ %\midrule
%a. & *{C}a{C}e & [k\'a-\textsuperscript{n}dee] & `kept \ldots inside' \\
%b. & *{C}\textipa{@}{C}e & [n\`i-k\`ə bə-de] & `who entered'        \\
%c. & *{C}e{C}i & [te-n\'i-ke\textsuperscript{n}da] & `was walking         \\
%d. & *{C}i{C}e & [te-n\`i-kee-t\`ə] & `and went away'      \\ %\toprule
%e. & *{C}e{C}o & b\'e\textglotstopvari e-\v{z}\'o & `our house'          \\
%f. & *{C}eo & ke-o-d\'e & `we eat him'         \\
a. & *{C}a{C}e & [ká\textsuperscript{n}dee] & `kept \ldots inside' \\
b. & *{C}ə{C}e & [nìk\`əbəde] & `who entered'        \\
c. & *{C}e{C}i & [teníke\textsuperscript{n}da] & `was walking         \\
d. & *{C}i{C}e & [tenìkeet\`ə] & `and went away'      \\ 
\end{tabular}
\end{example}

\noindent \citet[][166]{Pike1947b} affirms that the morpheme is ``marked'' by the violation of morpheme-internal sequence restrictions, an idea further developed by \citet{Harris1955} as a morpheme discovery routine.

\subsubsection{Biuniqueness and neutralization}

joos
harris
chomsky

\subsection{Early generativism}

\subsubsection{\citealt{Stanley1967}}

inventory constraints
sequence constraints

\subsubsection{Surface constraints and occultation}

shibatani
stampe
dell

\emph{A}[mt] `office', \emph{Anbau} `cultivation'
\emph{ei}[n.g]\emph{reifen} `to intervene',

\subsubsection{The duplication problem}

\citeauthor{Anderson1974} sees the duplication of \textsc{Backness Harmony} as both sequence structure constraint and phonological rule as a foregone conclusion
.The tendency of sequence structure constraints to duplicate phonological rules was noted by \citet[][401f.]{Stanley1967} and in \emph{SPE} (p.~382) in the discussion of morpheme structure constraints, and is labeled ``the duplication problem'' by \citet[][?]{Kenstowicz1977}.

inventory problems 
dell, chung, but also problems

conspiracies

\subsection{Autosegmentalism}

also ``supersegmentalism''

The proposals during this period are very diverse. 

\subsubsection{Syllable structure}

hooper
noske

\subsubsection{Morph features}

Leben
Kaye

\subsubsection{The Obligatory Contour Principle}

\subsection{Classic Optimality Theory}

%(page numbers here are taken from the published version)

\subsubsection{Richness of the Base}

some quotes from OT doc
\citet{PE}
\citet{Bye2001}

\subsubsection{Freedom of Analysis}

\citet{Smolensky1996}
\citet{PE}

\subsubsection{Undominated constraints}

% complaints about maceahern, gallagher
% bit on loanword adaptation

\subsection{Recent developments}

\subsubsection{Language acquisition}
% jusczyk forever
% white et al.
% early word learning
% syllables

\subsubsection{Psycholinguistic tasks in adults}

% wordlikeness
% spotting
% error correction, etc.

\subsubsection{Non-native word adaptation}

% perceptual factors

\section{The null hypothesis}

\subsection{The empirical proposal}

\subsection{The acquisition principle}

\subsection{Falsifiability}

%\subsection{Universal constraints?}    % 1.2.4: Universal constraints?
% This may be cut.

(see \citealt[][323f.]{Hockett1947}, \citealt[][415f.]{Nida1948}, \citealt[][50f.]{Anderson1992}, \citealt[][36f.]{Stump2001a}). Irrespective of one's position on this debate, it does not seem correct to extend this hypothesis to lexical roots. For instance, there  

imperfect subjunctive  & īrem     & īrēs     & īret     & īrēmus     & īrētis     & īrent \\
pluperfect subjunctive & issem    & issēs    & isset    & issēmus    & issētis    & issent \\
imperfect subjunctive  & venīrem  & venīrēs  & venīret  & venīrēmus  & venīrētis  & venīrent \\
pluperfect subjunctive & vēnissem & vēnissēs & vēnisset & vēnissēmus & vēnissētis & vēnissent \\

That this is not the 

While it might be possible to analyze \emph{ueni:re} as ``athematic'' /weni:-re/ and \emph{i:re} as /i:-re/, the frequentive \emph{uentita:re} `come often, be wont to come', formed with the \emph{-tit-} frequentive suffix (see \citet[][\S263]{Allen1903}), which selects for the first conjugation (cf.~\emph{agere} `act, make' vs. \emph{actita:re} `act, make often/repeatedly') suggests /wen-/.

It is also interesting to note that the same pattern has emerged in the history of French for a set of verbs which do not share this property in Latin. and the future and conditional indicative forms of \emph{aller} `go' and \emph{cuire} `to cook' in French.\footnote{\emph{Cuire}, interestingly, is not descended from the same conjugation as Latin \emph{i:re}, but rather from Latin \emph{coquere}; thus this syncretism is not simply an etymological relic of Latin. The same holds for other French verbs that inflect in the same manner, such as \emph{conduire} `drive (a vehicle); behave' and \emph{d\'etruire} `destroy'.}

future indicative      & irai   & iras   & ira      & irons    & irez    & iront \\
conditional indicative & irais  & irais  & irait    & irions   & iriez   & iraient \\
future indicative      & cuirai & cuiras  & cuira   & cuirons  & cuirez  & cuiront \\
conditional indicative & curias & cuirais & cuirait & cuirions & cuiriez & cuiront \\

One could even conceive of a largely vacuous principle on morpheme structure which simply requires that some subset of morphs not be null. 

%Anttila 2008
%Dmitrieva et al. 2008a,b
%Duanmu 2009
%Berkley 1994a,b, 2000
%Buckley 1997
%Coetzee 2008, Coetzee and Pater 2008
%Goad 2011
%Graff & Jaeger in press,
%Hammond 1999
%Colavin 2010,
%Frisch 1996, Frisch et al. 2004
%Hayes & Wilson 2008
%Kessler et al. 1997
%Martin 2007
%McGowan 2011
%Mester 1986
%Fudge 1969
%Padgett 1992
%Pierrehumbert 1993, 1994


\section{Outline of the dissertation}

%Chapter 2
%Chapter 3
%Chapter 4
