\label{intro}

This dissertation has two distinct but complementary aims. 
The first is to outline the empirical scope of \emph{phonotactic theory}, the theory of speakers' knowledge of possible and impossible sounds, sound sequences, and words.
The second is to suggest that the core core facts of this domain are compatible with a traditional view of phonotactic knowledge as independent of the lexicon, categorical, and closely related to phonological processes, but inconsistent with an increasingly popular view of phonotactics and phonotactic learning as a type of probabilistic inference over the lexicon, and therefore gradient and independent of phonological processes.\footnote{
    Throughout, the term \emph{lexicon} is used in a specific sense of the set underlying representations in some language; this is not meant to imply a position on the possibility that larger, composite linguistic representations are also stored in lexical memory.
    For a review of recent experimental evidence on this question, see \citealt{LignosInPressa}.}

Despite a recent surge of interest in phonotactic theory, the empirical scope of the theory remains poorly defined.
%and consequently, it is not uncommon to find suggestions that speakers do not have any type of phonotactic knowledge at all \citep[e.g.,][17f.]{PE}.
The first task in developing a theory of phonotactic knowledge, then, is to outline the types of facts that the theory should account for.

\section{The empirical scope of phonotactic theory}
\label{s:espt}

Phonotactic knowledge is evidenced in quite different ways than phonological knowledge.
In Russian, for instance, a process of anticipatory assimilation ensures that derived clusters of obstruents agree in voice.\footnote{
    For sake of discussion, the complex behavior of [v] is ignored here.}

\begin{example}[Russian voice assimilation alternations (adapted from \citealt{SPR}:22f.)]
\label{rovs}
\begin{tabular}{l ll ll}
a. & [ˈʒedʒbɨ] & `were one to burn'      & [ˈʒetʃl\pal{}i] & `should one burn?'      \\
b. & [ˈmoɡbɨ]  & `were (he) getting wet' & [ˈmokl\pal{}i]  & `was (he) getting wet?' \\
\end{tabular}
\end{example}

There is only one concise explanation for the [dʒ]-[tʃ] and [k]-[ɡ] alternations, namely that Russian speakers have internalized a process of voice assimilation.
It is also the case that voicing in underlying representations in Russian is ``nondistinctive in all but the last member of a cluster of obstruents'' \citep[283]{A74}.
It is possible to deny, however, that this restriction on underlying hetero-voiced obstruent clusters (and many other constraints on underlying representations that have been proposed) is part of the grammatical knowledge of Russian speakers, as some linguists have done.

\begin{quote}
MSCs are merely artifacts of the grammar, and thus play no part in the phonological component of a language. \citep[302]{Clayton1976}

Even if we, as linguists, find some generalizations in our description of the lexicon, there is no reason to posit these generalizations as part of the speaker's knowledge of their language, since they are computationally inert and thus irrelevant to the input-output mappings that the grammar is responsible for. \citep[17f.]{PE}
%[MSCs] have no observable consequences in the course of the normal use of the language\ldots{} \citep[320]{Zimmer1969}
\end{quote}

\noindent
\citeauthor{PE} go on to label constraints on underlying representations as ``extensional'', and thus irrelevant to generative grammar, in contrast to ``intensional'' statements like phonological rules.
Regarding the facts about Russian obstruents discussed so far, it seems to be clear that the alternation facts are in some sense more privileged.
Were it the case that Russian speakers had not internalized a process of obstruent voicing assimilation, the only alternative explanation for the forms in (\ref{rovs}) is a massive system of phonologically conditioned suppletive allomorphy.
In contrast, were there no constraint on underlying obstruent voicing, however, it is not immediately obvious that anything would be different.
Presumably, this is what \citeauthor{PE} mean when they refer to constraints on underlying representations as ``computationally inert''.
There are also many cases where restrictions on underlying representations are specific to underlying representations and do not apply to surface representations.
For instance, hiatus is exceptionally rare in native Turkish roots; of the handful of examples in the Turkish Electronic Living Lexicon \citep{TELL}, many appear to be compounds (e.g., \emph{ısıalan} `endothermic', cf.~\emph{ısı} `heat', \emph{alan} `taker').
Yet numerous phonological processes give rise to hiatus in derived environments \citep[e.g.,][]{Kabak2007b}.
\citet{Silverman2000} observes that operations like reduplication and truncation also tend to introduce violations of restrictions that hold of underlying representations.
However, there are many facts which strongly suggest that speakers internalize phonotactic constraints.

\subsection{Wordlikeness judgements}

As \citet{Halle1962} and \citet{Chomsky1965} note, speakers can distinguish between a well-formed and an ill-formed word, neither of which is an actual word.
Neither [blɪk] \emph{blick}\footnote{
    In fact, philospher R.M.~\citet{Hare1955} uses the term \emph{blik} to describe \emph{a priori}, unfalsifiable frames of reference through which experiences are filtered; for instance, materialism can be thought of as the \emph{blik} interpreting all experience in terms of the properties of matter. 
    Since this term has some currency in the philosophy of science and religion, this is likely a ``real word'' rather than a lexical gap for a significant English-speaking minority. 
    Thanks to Steve Anderson for brining this to my attention.}
nor [bnɪk] \emph{bnick} is a word of English, yet speakers immediately report that only the former is ``possible'', an \emph{accidental} gap in the lexicon, whereas the latter is judged to be impossible (i.e., structurally excluded).
There can be no question that it is part of speakers' knowledge.
Elicited in a controlled fashion, these \emph{wordlikeness judgements} are perhaps the most important (and least controversial) source of phonotactic evidence, and they play a major role in Chapters \ref{gradience}--\ref{turkish} of this dissertation.

\subsection{Word production and recognition}
\label{ss:wpr}

Speakers have difficulty producing \citep[e.g.,][]{Davidson2006a,Davidson2010,Rose2007,Vitevitch1997} and perceiving \citep{Dupoux1999,Kabak2007a,Massaro1983} certain types of nonce words judged to be phonotactically illicit.
For instance, \citet{GallagherInPress} finds that speakers of Quechua have difficulty repeating nonce words with multiple ejectives (e.g., [k'ap'i]), which do not occur in the language; in other words, multiple ejective sequences are not merely absent, but also difficult for these speakers to repeat.

In English, sequences of adjacent obstruents which do not also agree in voice (e.g., \emph{a}[b.s]\emph{inth}) are quite rare within a word, and therefore a hetero-voiced obstruent cluster is a clue to the presence of a word boundary in running speech.
Infants \citep[e.g.,][]{Mattys2001b} and adults \citep{McQueen1998b,Norris1997} are thought to use this heuristic for word recognition in experimental settings.\footnote{
    There is reason to distinguish the computations involved in using this word segmentation heuristic from those implicated by other phonotactic behaviors.
    Whatever the locus of [bnɪk]'s ill-formedness, for example, no segmentation into multiple words or morphs renders it a well-formed sequence.
    See \S\ref{ss:bh} for further discussion.}

\citet{Berent2001b} and \citet{Coetzee2008b} claim that non-word recognition latencies in lexical decision tasks reflect speakers' phonotactic knowledge, the hypothesis being that a phonotactically illicit nonce word will be rejected more quickly than a well-formed nonce word.
However, phonotactic constraints are often confounded with independent predictors of lexical decision latencies.
For instance, \citet{Coetzee2008b} finds that English speakers recognize [sp\ldots{}p] and [sk\ldots{}k] nonce monosyllables faster than [st\ldots{}t] nonce monosyllables in an auditory lexical decision task.
\citeauthor{Coetzee2008b} attributes this to ad hoc phonotactic constraints against the former sequences, but another explanation is available.
Even at an early stage of recognition, [stVt] is distinguished from [spVp, skVk] by its higher \emph{cohort density}: there are far more English words starting with initial [st] than with [sp] or [sk].
High cohort density is known to inhibit auditory processing of non-words \citep[e.g.,][]{Marslen-Wilson1978} and this alone could account for the processing difference.

\subsection{Loanword adaptation}

Loanword adaptation may provide further evidence for the grammatical relevance of phonotactic knowledge \citep[e.g.,][]{Fischer-Jorgensen1952}.
In Desano \citep{Kaye1971,Kaye1974}, for instance, all underlying representations (URs) are either totally oral (e.g., [jaha] `to hear') or totally nasal (e.g., [ñãhã] `to enter'), and loanwords are made to conform to this generalization: Portuguese \emph{martelo} `hammer' is adapted as [barateru] and Spanish \emph{naranja} `orange' as [nãnãnã].\footnote{
    \citeauthor{Kaye1971}'s transcriptions have been converted to IPA notation.}
It is not difficult to imagine that some component of the synchronic grammar is responsible for the fact that the restriction over native vocabulary is extended to loanwords.
Furthermore, many Desano affixes have have distinct oral and nasal allomorphs. 
For instance, a noun classifier suffix for round objects takes the form [ru] after oral roots (e.g., [goru] `ball') and [nũ] after nasal roots (e.g., [sẽnãnũ] `pineapple').
%\citep[38]{Kaye1971}. 
As \citet[38]{Kaye1971} notes, this distribution implies the existence of a process of nasal harmony, and it is not implausible that the same process is responsible for the above-mentioned adaptations.

However, it is not possible to equate all patterns of adaptation with phonological alternations targeting native vocabulary.
\citet{Peperkamp2005} highlights several cases where native alternations are distinct from loanword adaptations.
For instance, in Korean, [s] does not surface in codas.
As \citet{Kenstowicz2001} report, native /s/ is realized as [t] in codas (e.g., [nat]-[nasɨl] `sickle \textsc{nom.}-\textsc{acc.}').
In loanwords, however, final [s] becomes an onset by epenthesis (e.g., \emph{boss} $>$ [posɨ]).
Evidence of this sort has lead many \citep[e.g.,][]{Dupoux1999,Peperkamp2003,Peperkamp2005} to suggest that loanword adaptations are the result of speech perception, not phonological computations.

There are other cases in which loanword adaptations are not easily identified with any synchronic process.
Consider the case of loanwords which begin in onset clusters and which are borrowed into languages which do not permit complex onsets.
In the Wikchamni dialect of Yokuts, for instance, Spanish loanwords with complex onsets are adapted either via anaptyxis or deletion of the first consonant.

\begin{example}[Wikchamni Yokuts adaptations \citep{Gamble1989}]
\begin{tabular}{l lll l}
a. & \emph{cruz}     & $>$ & k\asp{}uluʃ & `cross' \\
   & \emph{frijoles} & $>$ & pilhoːliʃ   & `beans' \\
b. & \emph{plato}    & $>$ & laːto       & `plate' \\
   & \emph{clavo}    & $>$ & laːwu       & `nail'  \\
\end{tabular}
\end{example}

\noindent
However, there is no reason to regard these adaptations as a product of the synchronic phonology: no Yokuts root begins with a consonant, and there is no way to derive an initial CC cluster.
Furthermore, as \citet{Gamble1989} notes, there is no reason to believe that the adaptations actually occurred within Yokuts at all, since there was at best quite limited contact between Spanish and Yokuts speakers; the adaptations may have occurred in another language altogether.

%\begin{example}[Adaptation of onset clusters in Finnish loanwords]
%\begin{tabular}{l llll}
%a. & toːri    & `store'   & ($<$ \emph{store})              & \citep[English:][89]{Hellstrom1976}   \\
%   & rosseri  & `grocery' & ($<$ \emph{grocery})            &                             \\
%b. & ranska   & `French'  & ($<$ \emph{Franska})            & \citep[Swedish:][67]{Campbell2004}    \\
%   & ruːvi    & `screw'   & ($<$ \emph{skruv})              &                             \\
%c. & risti    & `cross'   & ($<$ \emph{kristĭ})             & \citep[Old Russian:][60]{Bjornflaten2006} \\
%   & raːmattu & `bible'   & ($<$ \emph{gramota} `document') &                             \\
%\end{tabular}
%\end{example}

Finally, there are many cases where putative phonotactic restrictions are not extended to loanwords (e.g., \citealt{Clements1982}, \citealt[75]{Davidson1997}, \citealt{Fries1949}, \citealt{Holden1976}, \citealt{Ito1995a,Ito1995b}, \citealt[95]{Shibatani1973}, \citealt{Ussishkin2003}, \citealt{Vogt1954}; additional examples can be found throughout this dissertation).
For instance, native words in San Mateo Huave all end in a consonant, but 
\citet{Davidson1997} note that final unstressed syllables in Spanish loanwords are never repaired by epenthesis (e.g., \emph{verde} `green' $>$ [beɾde], *[beɾdej]).
Given the considerable disagreement about the nature of loanword adaptation at the present juncture, it may be premature to regard this inertness as strong evidence against the constraints in question, though it may be a useful diagnostic.

\subsection{Alternate phonologies}

Language games or speech disguises may also provide evidence for phonotactic knowledge \citep[e.g.,][]{Vaux2011}, assuming that these ``alternative phonologies'' are not qualitatively different from naturally-occurring language processes \citep[e.g.,][697]{Bagemihl1995}.
An example of language game evidence bearing on phonotactic questions can be found in \S\ref{ss:bh}, where a language game is used to argue for the hypothesis that root-internal vowel sequences in Turkish are subject to vowel harmony.

\subsection{Lexical statistics}

Finally, phonotactic gaps or tendencies in the lexicon are often taken as evidence for phonotactic knowledge, under the hypothesis that grammatical constraints are the cause of these lexical generalizations.
Chapters \ref{turkish}--\ref{gaps} consider in detail the evidentiary status of these lexical tendencies.

\section{The grammatical basis of phonotactic knowledge}

With the primary evidence for phonotactic theory now established, it is possible to consider the grammatical architecture that underlies this knowledge.

\subsection{The insufficiency of morpheme structure constraints}

Early generative phonologists posited that phonotactic ill-formedness derives solely from \emph{morpheme structure constraints}, restrictions on underlying representations \citep{Chomsky1965,SPE,SPR,Halle1962}.
\citet{Stanley1967} distinguishes between two types.
\emph{Segment structure constraints} impose restrictions on the underlying segment inventory.
For example, voicing is non-contrastive for /ts, tʃ, x/ in Russian \cite[22]{SPR}: [dz, dʒ, ɣ] appear in surface, but not underlying, representations.
This dissertation will have little to say about the nature of segment structure constraints.
Of more interest here are \emph{Sequence structure constraints} apply to underlying sequences; an example is given below.

\begin{example}[An English MSC (adapted from \citealt{Chomsky1965}:100)]
$\begin{bmatrix} +\textsc{Cons} \end{bmatrix}~\goesto~\begin{bmatrix} +\textsc{Liquid} \end{bmatrix}~/~\#~\begin{bmatrix} -\textsc{Cont} \end{bmatrix}~\gap$
\end{example}

\noindent
This sequence structure constraint specifies that a consonant immediately after a word-initial stop is a liquid; this would preclude underlying */bnɪk/, for example.

However, \citet{Shibatani1973} shows quite convincingly that not all wordlikeness contrasts can be expressed as constraints on URs.
In German, for instance, obstruent voicing is contrastive, but neutralizes finally: e.g., [ɡʀaːt]-[ɡʀaːtə] `ridge(s)' vs.~[ɡʀaːt]-[ɡʀaːdə] `degree(s)'.
By hypothesis, the latter root ends in /d/, so the constraint against final voiced obstruents is specific to surface representations.
\citeauthor{Shibatani1973} claims, however, that German speakers judge voiced obstruent-final nonce words to be ill-formed.\footnote{
    Furthermore, voicing of final obstruents is usually lost in German loanword adaptation: e.g., English \emph{hot dog} becomes [hɑt dɔk] \citep[506]{Ussishkin2003}.}

\subsection{The duplication problem}
\label{ss:dp}

Whereas \citeauthor{Shibatani1973} argues that morpheme structure constraints are insufficient to account for speakers' phonotactic knowledge, other authors observe the tendency for structural descriptions of phonological processes to reappear among the morpheme structure constraints on the same language (e.g., \citealt{SPE}:382, \citealt{Hale1965}:297, \citealt{Kisseberth1970b}, 
%\citealt{PasterInPress},
\citealt{Postal1968}:212f., \citealt{Stanley1967}:401).
Russian obstruent voice assimilation, discussed above, provides an example of this type: there are no hetero-voiced obstruent clusters in either underlying or surface representations.
These two facts are tantalizingly similar, but are treated as separate if a distinction between morpheme structure constraints and phonological processes is drawn.
This is a special case of what \citet{Kisseberth1970b} calls \emph{conspiracies} and what \citet[136]{KK77} term the \emph{duplication problem}.
\citet[205f.]{Dell1973} and \citet[28f.]{Stampe1973} argue that the distinction between constraints on URs and alternations is artificial, and that these different levels of description are related by a principle now known as \emph{Stampean occultation} \citep[54]{OT}.
In a language like Russian, in which surface obstruent clusters exceptionlessly agree in voicing, there is simply no reason for the language acquisition device to posit underlying hetero-voiced obstruent clusters: obstruent voice assimilation ``occults'' underlying */kb/, for instance. 
Were such an underlying form posited, it would surface as [gb] in all contexts.\footnote{
    \emph{Lexicon Optimization} \citep[209]{OT} implements a form of Stampean occultation notable in that it projects all non-alternating surface segments directly into URs.
    For instance, in English, Lexicon Optimization demands underlying /ŋ/ in words like \emph{bank}, though [ŋ] could be otherwise be analyzed as an allophone of /n/ before velar consonants (e.g., \citealt[65f.]{Borowsky1986}, \citealt[85]{SPE}, \citealt[62]{Halle1985a}), eliminating /ŋ/ from the phoneme inventory.
    However, this is not core to \citeauthor{Dell1973} and \citeauthor{Stampe1973}'s insight about the relationship between surface and underlying sequence structure restrictions.
    For instance, the hypothetical \{bæŋk\} posited by Lexicon Optimization could be revised to /bænk/, and take a \emph{free ride} (in the sense of \citealt{Zwicky1970}) on the process of nasal place assimilation found elsewhere in English (see \S\ref{npa}).
    Indeed, this seems desirable, since Lexicon Optimization forces a duplication between underlying and surface constraints.
%this point is made by \citet{Chomsky1957} and \citet[69f.]{Harris1960} with respect to the principle of biuniqueness, which enforces a similar restriction.
    For instance, [ŋ] does not appear word-initially and English speakers have considerable difficulty producing it in this position \citep{Rusaw2009}.
    The allophonic analysis of [ŋ] predicts this fact, since there is no way to derive the [ŋ] allophone in onset position.
    Future work will consider the how purely allophonic relationships might be acquired.}
    %Assuming Lexicon Optimization, the only alternative is to stipulate an ad hoc constraint against onset [ŋ] \citep[39]{Jusczyk2002}.}
In Chapter \ref{gaps}, it is argued that constraints described in terms of non-contrastive prosodic structures can also be derived by Stampean occultation.
In an architecture like Lexical Phonology, it is even possible to apply a process to individual underlying representations (i.e., at the ``lexical level'') to enforce constraints which are not surface-true.
Consequently, it is impossible to construct an argument which would distinguish morpheme structure constraints from other types of phonological computations.

This is in ways similar to \citeauthor{SPR}'s famous argument (\citeyear{SPR}) against the morphophonemic/phonemic distinction.\footnote{
    This is not to imply that \citeauthor{SPR} was the first to make this argument, or to otherwise disregard this distinction: similar ideas can be found in work by \citet{Bloch1941}, \citeauthor{Bloomfield1926} (\citeyear{Bloomfield1926}:160, \citeyear{Bloomfield1962}:5f.), \citet{Chao1934}, \citet[244f.]{Hamp1953}, and \citet[47f.]{Sapir1930a}, among others (see \citealt{Anderson1985} \emph{passim}).}
The principle of biuniquness in vogue at that time separates neutralizing (morphophonemic) and non-neutralizing (phonemic) processes.
In Russian, obstruents participate in voice assimilation whether this neutralizes a phonemic distinction (\ref{rovs}a) or not (\ref{rovs}b): recall that there is no underlying /dʒ/ in Russian.
Under biuniqueness, there must be separate, non-overlapping variants of this process, one applying in neutralizing contexts and another in non-neutralizing contexts.
From this, \citeauthor{SPR} argues that biuniqueness (and the distinction between the morphophonemic and phonemic levels that follows from it) entails ``a significant increase in the complexity of the representation\ldots{}an unwarranted complication which has no place in a scientific description of language'' (24).
While \citeauthor{Anderson1985} (\citeyear{Anderson1985}:110f., \citeyear{Anderson2000}) correctly observes that it is in principle possible to sketch an analysis of Russian obstruent voicing which preserves biuniqueness without the morphophonemic/phonemic duplication \citeauthor{SPR} objects to (see also \citealt{Kiparsky1985}), this requires further contested assumptions---contrastive underspecification (against which, see \citealt{Steriade1995}) and a Duke-of-York derivation.
This does not seem inconsistent with \citeauthor{SPR}'s claim that biuniquness imposes unnecessary additional complexities: under \citeauthor{Anderson2000}'s analysis the complexity is not two variants of a single process, but rather a dependency on contentious theoretical assumptions.

\subsection{Static constraints}

Stampean occultation provides a mechanism for many sequence structure constraints constraints on underlying representations to be expressed as phonological processes. 
However, not all restrictions on underlying representation have an obvious reflex in the system of alternations.
Of the residue that remains once these \emph{derived constraints} are identified, many can be attributed to the language-specific prosodic inventory.
For instance, a language which does not permit onset clusters may do so without there being evidence for processes that simplify initial consonant clusters; Yokuts, discussed above, is a clear example.
There is a sense in which prosodic parsing operations like syllabification or footing can be thought of as phonological computations, and the boundary between prosodic restrictions and alternations is not always clear.
For instance, Latin does not permit a geminate consonant after a long vowel; since geminates always span syllable boundaries, this appears at first glance to be a restriction on the Latin syllable template.
However, Latin seems to enforce this restriction via a process of degemination \citep{GormanInPressc}; assibilation of \emph{t, d} in perfect passive participles produces a geminate \emph{ss} (e.g., 
\emph{fossus} `dug', cf.~\emph{fodere} `to dig') except after diphthongs and long monophthongs, where a singleton \emph{s} appears (e.g., \emph{lūsus} `played', cf.~\emph{lūdere} `to play').
This restriction is simultaneously a component of the prosodic inventory and of the alternation system.

A question which is central to phonotactic theory is whether there are additional types of sequence structure constraints beyond those which are derived from alternations or restrictions on the prosodic inventory.
Recent answers to the affirmative have generally been presented with proposals for how such constraints are learned.

One possibility will be carefully considered in this dissertation.
According to this view, much of phonotactic learning is accomplished by statistical inference performed over the lexicon, and phonotactic knowledge is the sum of these lexical statistical patterns.
Two predictions immediately follow from this position.
First, since statistical generalizations may be more or less true, phonotactic knowledge may be gradient, reflecting the strength of the many competing patterns involved.
Secondly, lexical statistical patterns need not be reflected in the system of alternations or in the prosodic system at all: they may be ``static''.
These predictions will be taken up in some detail in Chapters \ref{gradience} and \ref{turkish}, respectively.

Another possibility is posed by traditional thinking in Optimality Theory.
According to a standard proposal, at the ``initial state'' all markedness constraints are ranked above all faithfulness constraints \citep[e.g.,][]{Smolensky1996a}.
If learning proceeds via constraint demotion, markedness constraints which are not implicated by alternations will remain undominated \citep[e.g.,][]{Coetzee2008b}.
It is difficult to make concrete predictions from this proposal in the absence of a complete proposal for the contents of the universal constraint set \textsc{Con}, but it has the potential to blur the distinction between accidental and structural phonotactic gaps, a distinction which is at the heart of phonotactic theory and is the subject of Chapter \ref{gaps}.
If the constraint set is in fact universal, any exceptionless gap which corresponds to a markedness constraint \emph{in any langauge} will be accorded the status of an inviolable phonotactic restriction.
If \textsc{Con} is sufficiently rich to incorporate constraints like the *[spVp] proposed by \citet{Coetzee2008b}, it seems quite likely that it will also contain markedness constraints rulling out English syllable contact clusters which are regarded as accidental gaps in Chapter \ref{gaps}.
For instance, a constraint *[s.w] seems plausible as a subcomponent of the so-called \emph{syllable contact law} \citep[e.g.,][]{Gouskova2004,Murray1983} which disfavors syllable contact clusters with increasing sonority.
This *[s.w] constraint is without exception in English, yet nonce words like \emph{teeswa} [tes.wa] seem quite unobjectionable to native speakers.
Numerous other examples of this type could be adduced, and pose a serious problem for the markedness-over-faithfulness model of phonotactic learning.

The merits of these two models are the subject of the rest of this dissertation.
Whatever these merits, there is some value also in the null hypothesis, that there are no static phonotactic constraints at all, if it can be maintained.
Specifically, the principle of \emph{no static phonotactics} has interesting ramifications for evaluating certain competing phonological analyses.
Consider Sanskrit aspiration alternations such as \emph{bodhati}-\emph{bhotsyati} `he wakes-he will wake'.
According to one analysis, which has precedents as far back as the grammar of Pāṇini, the root /budh/ undergoes a process shifting aspiration leftward in certain contexts (e.g., \citealt{Borowsky1983}, \citealt{Hoenigswald1965}, \citealt{Kaye1985}, \citealt{Sag1974}, \citeyear{Sag1976}, \citealt{Schindler1976}, \citealt{Stemberger1980}, \citealt{Whitney1889}:\S141f.).
An alternative analysis posits an underlying /bhudh/ and a process of aspirate dissimilation, a synchronic analogue of Grassman's Law (e.g., \citealt{Anderson1970}, \citealt{Hoard1975}, \citealt{Kiparsky1965}:\S3.2, \citealt{Phelps1973}, \citealt{Phelps1975b}, \citealt{Zwicky1965}:109f.).
Under the latter analysis, multiple surface aspirates (e.g., hypothetical *\emph{bhodhati}) are phonotactically marked; the former account makes no such prediction.
The principle of no static phonotactic constraints, if it can be maintained, could in theory adjudicate between these competing analyses: if the postulated constraint on multiple surface aspirates is active in wordlikeness judgements or loanword adaptation, for instance, this rules out the former account.

\section{Outline of the dissertation}

The remainder of the dissertation consists of three case studies which provide support for the novel and contentious predictions of the minimal phonotactic model sketched above.

Chapter \ref{gradience} considers evidence from wordlikeness rating tasks. It is argued that intermediate well-formedness ratings are obtained whether or not the categories in question are graded. 
A primitive categorical model of wordlikeness using prosodic representations is outlined, and shown to predict English speakers' wordlikeness judgements as accurately as state-of-the-art gradient wellformedness models. 
Once categorical effects are controlled for, gradient models are largely uncorrelated with well-formedness ratings.

Chapter \ref{turkish} considers the relationship between lexical generalizations, phonological alternations, and speakers' nonce word judgements with a focus on Turkish vowel patterns. 
It is shown that even exception-filled phonological generalizations have a robust effect on wellformedness judgements, but that statistically reliable phonotactic generalizations may go unlearned when they are not derived from phonological alternations.

Chapter \ref{gaps} investigates the role of phonological alternations in constraining lexical entries, focusing specifically on medial consonant clusters in English. 
Static phonotactic constraints previously proposed to describe gaps in the inventory of medial clusters are shown to be statistically unsound, whereas phonological alternations impose robust restrictions on the cluster inventory. 
The remaining gaps in the cluster inventory are attributed to the sparse nature of the lexicon, not static phonotactic restrictions.

Chapter \ref{conclusions} summarizes the findings, considers their relation to order of acquisition, and proposes directions for future research.

%Whereas \citeauthor{Shibatani1973} maintains that morpheme structure constraints are insufficient to account for speakers' knowledge of possible wordhood, others argue that morpheme structure constraints are also unnecessary. \citet[297]{Hale1965}, \citet{Kisseberth1970b}, and \citet[212f.]{Postal1968} observe that the structural descriptions of phonological processes often are often reflected in the lexical redundancies in the same language. In Russian, there are alternations implicating a process of obstruent voice assimilation, and tautomorphemic obstruent clusters have uniform voicing \citep[283]{A74}. \citet[205f.]{Dell1973} and \citet[28f.]{Stampe1973} argue that morpheme structure constraints are otiose, as phonological rules triggering alternations impose restrictions on the contents of URs, an effect known as \emph{Stampean occultation}.
%\citet{Silverman2000}
%\citet{Kiparsky1995}
%\citet[352f.]{Hale2003a}
%Theories of phonotactic knowledge should be evaluated by the same stringentcriteria applied to other formal arenas: neither overgeneration nor undergeneration should be permitted. It will be shown that the orthodoxy that phonotactic knowledge is in some sense ``probabilistic'' suffers from quite severe over- and undergeneration. 
%This heuristic has the greatest impact regarding debates about ``abstractness'' of underlying representations. 
%It cannot be said, precisely, that it either excludes or requires any particular type of abstractness. 
%Proposed restrictions on UR abstractness have been faulted for a number of reasons. In some cases, they preclude otherwise-desirable analyses of alternations. Another objection that could reasonably made is that any restriction on abstract URs that introduces an assymetry in URs is bad.
%\citep[~chap.~1]{KK77}
%I would like to suggest that there are two types of problems that arise in developing a theory of phonotactics free of duplication. The first type of problem consists of conflicts between theoretical assumptions and the desire to eliminate duplication. If we continue to view duplication as a sort of negative heuristic, then it may be the case that the theoretical assumptions are wrong. Such a case arises in Chapter \ref{turkish} in the discussion of archiphonemic underspecification analysis of Turkish vowel harmony proposed by \citet{Clements1982}. \citeauthor{Clemenst1982} propose that all but the first vowel of a harmonic root is underspecified for backness (and in high vowels, roundness) and is filled in by rule. Since there are disharmonic roots, this rule must be ``structure-filling''. Consequently, this rule cannot account for the apparent markedness of disharmonic roots revealed by wordlikeness judgements (among other psycholinguistic tasks). If duplication is a pathology, this analysis is wrong.
%The arguments of \citeauthor{Shibatani1973} and others led theorists to focus their attention on properties of surface representations as determinants of wordlikeness. Though syllabification plays no role in \emph{SPE}, it is crucial to many earlier studies \citep[for a review, see][]{Goldsmith2011b}, and it received particular attention in the 1970s. \citet{Hooper1973} and \citet{Kahn1976} argue that the syllable is useful for defining wordlikeness generalizations.\footnote{\citet{Steriade1999} and \citet{Blevins2003}, however, argue that a number of phonotactic generalizations previously stated in syllabic terms can be reanalyzed without making reference to syllables.} \citeauthor{Hooper1973} argues, for instance, that [bn], impossible as an English onset, is unobjectionable as a syllable contact cluster in nonce words like \emph{stabnik} (or in names like \emph{Abner}), and that this demonstrates the superiority of syllable-based wordlikeness generalizations. This already signals further trouble for alternative accounts which focus on underlying forms. Syllabification may span morphs, is generally predictable, and is universally non-contrastive, and as a consequence, few posit in to be present in underlying representations \citep[though see, e.g.,][]{Vaux2003}. \citeauthor{Hooper1973} also points to loanword adaptations which produce native syllable structure \citep[e.g.,][]{Carlisle1991} as evidence that syllabification is part of the phonological computation. Further enrichments to the theory are provided by the autosegmental theory of the syllable \citep{McCarthy1979b}, which envisions the syllable as an articulated tree structure \citep[as first envisioned by][]{Pike1947a}, and theories like prosodic licensing \citep{Ito1989a}, in which syllabification triggers phonological repairs.
%Despite the considerable attention given to the proposals of \emph{SPE} in the wake of that book's publication in 1968, the \emph{SPE} wordlikeness model has received almost no further attention in the literature. At the risk of explaining what might be no more than an accidental gap in the literature, the novel aspects of \emph{SPE} model---gradience derived from similarity to existing lexical entries---may have been overshadowed by the many other contentious proposals in \emph{SPE}, and particularly by compelling arguments against the assumption that wordlikeness contrasts derive solely from properties of underlying forms. \citet{Shibatani1973} observes that there are some generalizations about surface forms which give rise to wordlikeness contrasts, but cannot be stated as constraints on underlying forms. An example from German is shown in (\ref{fd}) below.

%\emph{K}[uːx]\emph{en}
%\emph{K}[uːç]\emph{en}
%\footnote{Examples like \emph{Mas}[oːç]\emph{ist} `masochist', \emph{Eun}[uːç]\emph{ismus} `eunichism', first noted by \citet{Merchant1994}, suggest that this should also be restricted to assimilation within the same foot \citep[226f.]{Jensen2000}.}

%Whether phonological computations or representations themselves are graded \citep[e.g.,][]{Lakoff1973} is besides the point, as metalinguistic judgements are behaviors, not mental states; they can no more be compared than can ``fear'' and ``flight response''.

%\begin{example}[German \emph{ich}- and \emph{ach-laut}]
%\begin{tabular}{l l l}
%a. & [buːx]   & `book'           \\
%   & [tɔxtər] & `daughter'       \\
%   & [naxt]   & `night'          \\
%b. & [siçt]   & `view'           \\
%   & [ʃpeçt]  & `woodpecker'     \\
%   & [ɡərʏçt] & `rumor'          \\
%   & [knøçəl] & `ankle, knuckle' \\
%   & [flɛçə]  & `surface'        \\
%\end{tabular}
%\end{example}
%
%\emph{Umlaut}, the fronting (and raising) of back vowels in certain morphological contexts, produces the front variant of the dorsal fricative; e.g., [lɔx]-[løçər] `hole-holes', \emph{B}[uːx]-\emph{B}[yːç]\emph{er} `book-books',
% 
%Throughout the term \emph{lexicon} is used in a specific sense of the set underlying representations in some language; this is not meant to imply a position on the possibility that larger, composite linguistic representations are also stored in lexical memory (thought see \citealt{LignosInPressa} for some recent experimental evidence bearing on this question).
%I reject the possibility that phonotactics do exist altogether.
