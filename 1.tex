\chapter{A generative theory of phonotactic knowledge and acquisition}
\label{intro}

Syntax can hardly be described as a theory of word ``arrangement''; it would be inconceivable for syntacticians to engage in a lengthy debate about whether some word sequence is permitted in some language. Yet it is possible for two reasonable phonologists to dispute whether [k.p] is a permissible medial consonant cluster in English. While lexical entries are thought to consist of segments, and segments to consist of feature specifications, the phonological form of a word like \emph{spectre} is not ``generated'', in any relevant sense, by the concatenation of segments /s/, /p/, and so on, or by the concatenation of syllables \{spɛk\}$_{\sigma}$ and \{tɚ\}$_{\sigma}$. It would be possible to posit ``lexical concatenation rules'', on analogy with the phrase structure rules of syntax, but it is unclear how the strings produced would be linked with meanings, or how such a system would distinguish between, e.g., \emph{brick} and meaningless \emph{blick}.

The more appealing alternative is that the phonological form of \emph{spectre}, whatever it is, is an atom of unit of lexical memory. It remains to account for any restrictions on the contents of underlying forms. The theory of \emph{phonotactics} (which is related to \emph{syntax} through the Greek root \emph{táxis} `order') is concerned with these restrictions and how they are related to the mappings between underlying and surface forms for which the phonological component is responsible. 

%\footnote{It is tempting, however, to describe binding principles as ``syntactics''.}

I believe that the boundaries of this knowledge

The boundaries of this knowledge
and what might be acquired.
These are not unrelated issues:

One question is: why bother?

It is certainly possible to imagine otherwise. \citet[][320]{Zimmer1969} writes that phonotactic generalizations ``have no observable consequences in the course of the normal use of the language'', and \citeauthor{PE} echo this sentiment more recently:

\begin{quote}
Even if we, as linguists, find some generalizations in our description of the lexicon, there is no reason to posit these generalizations as part of the speaker's knowledge of their language, since they are computationally inert and thus irrelevant to the input-output mapping that the grammar is responsible for. \citep[][18]{PE}
\end{quote}

This represents a principled null hypothesis, but can be quickly dismissed in light of speaker's ready judgements of possible and impossible words observed by \citeauthor{Chomsky1965}. Metalinguistic judgements of these sorts, reviewed in Chapter \ref{wordlikeness}, are not the only task which exemplify phonotactic knowledge; infants and adults are also thought to make use \emph{possible word constraint} in more quotidien tasks like recognizing words in running speech \citep[e.g.,][]{Brown1956,Mattys1999,McQueen1998b,Norris1997}.

\subsection{Units of phonotactic description}

\subsubsection{Segments}

\subsubsection{Underlying representations}

\subsubsection{Surface sequences}

% discussion of Ernestus & Baayen 2003, Becker et al. 2011

The notion of blocking in non-derived environments (\citealp[163]{Kiparsky1973a}, \citeyear[152]{Kiparsky1982a}, \citealp{Mascaro1976}) is well-known, but \citet{Hall2006} also amasses evidence for phonological processes restricted to non-derived environments. One of the most famous examples is the distribution of \emph{ich-laut} [ç] and \emph{ach-laut} [x] in German \citep{Bloomfield1930}. The dorsal fricative is [x] before back vowels and[ç] in other contexts.

\begin{example}[German \emph{ich}- and \emph{ach-laut}]
\begin{tabular}{l l l}
a. & [buːx]   & `book'           \\
   & [tɔxtər] & `daughter'       \\
   & [naxt]   & `night'          \\
b. & [siçt]   & `view'           \\
   & [ʃpeçt]  & `woodpecker'     \\
   & [ɡərʏçt] & `rumor'          \\
   & [knøçəl] & `ankle, knuckle' \\
   & [flɛçə]  & `surface'        \\
\end{tabular}
\end{example}

\emph{Umlaut}, the fronting (and raising) of back vowels in certain morphological contexts, produces the front variant of the dorsal fricative; e.g., [lɔx]-[løçər] `hole-holes'.
%\emph{B}[uːx]-\emph{B}[yːç]\emph{er} `book-books',

\emph{K}[uːx]\emph{en}
\emph{K}[uːç]\emph{en}

%\emph{Mas}[oːx]-\emph{Mas}
\footnote{Examples like \emph{Mas}[oːç]\emph{ist} `masochist', \emph{Eun}[uːç]\emph{ismus} `eunichism', first noted by \citet{Merchant1994}, suggest that this should also be restricted to assimilation within the same foot \citep[226f.]{Jensen2000}.}
%\emph{Mas}[oːx]-\emph{Mas}[oːç]\emph{ist}

\section{The model}

\subsection{Syllabification}

\subsection{Rule application and occultation}

\subsection{Detecting violations}

This is precisely the case for 

\subsection{Exceptionality}
% autosegments

\section{Predictions of the model}

\subsection{Gradience}
\subsubsection{Probabilistic well-formedness}
\subsubsection{External factors}

\subsection{Static phonotactics}
\subsubsection{Insensitivity to statistics}
\subsubsection{Saussurean arbitrariness}

\subsection{Order of acquisition}

A point of departure is a review of phonotactic acquisition by \citet{Hayes2004b}.

\emph{pure phonotactic learner}
%...
However, \citeauthor{Hayes2004b} ignores the considerable evidence that infants younger than nine months of age---the \emph{terminus ante quem} for the onset of phonotactic knowledge---already have relatively rich and detailed lexical knowledge

A word of caution is in order. An experiment which finds typically-developing infants of a certain age insensitive to an adult-like contrast using a certain task is by no means sufficient evidence to conclude that infants of this age lack this ability. A negative result may simply indicate that the task lacks ecological validity or requires domain-general cognitive resources beyond those of the infants. For instance, \citet{Werker2002} report that 14-month-old infants are insensitive to small phonological differences between novel words, but \citet{Fennell2006} shows that they are sensitive to the same contrasts when the novel words are situated in a more naturalistic naming scenario and cognitive demands are minimized. Other negative findings might be attributed to failure to control for properties of the stimulus, or to an experimental design which lacks sufficient statistical power.

\citet{Jusczyk1993b} native vs. non-native
\citet{Jusczyk1994} no preference for high prob
\citet{Friederici1993}

Finally, \citet{Mattys1999} find that 9-month-old infants treat hetero-organic nasal-stop clusters in nonce words as indications of word juncture, though this effect is small compared to the effect of other cues like primary stress: they do not test younger infants.

\subsubsection{Prosodic parsing}

Newborn infants are already able to distinguish between monosyllabic and bisyllabic words, but these infants are thought to recognize syllables ``holistically'' rather than as sequences at least until 4 months \citep[e.g.,]{Bertoncini1981,Eimas1999,Jusczyk1987}. The earliest evidence for segmental representations comes not from phonotactic preferences, but from infants' ability to dicriminate between familiar words like \emph{cup} and mispronunciations like \emph{*tup} at 7.5 months of age \citep{Jusczyk1995}.

\subsubsection{Phonological processes}

Very few studies have investigated young infants' knowledge of phonological alternations. One exception is a fascinating study by \citet{White2008}. Simplifying somewhat, the experimenters expose 8.5-month-old and 10-month-old infants to an artificial language in which the voicing of fricatives is contrastive, but voiced and voiceless variants of plosives are in a complementary distribution, appearing only after vowels (\emph{na-bevi}) and after voiceless consonants (\emph{rot-pevi}), respectively. After familiarization, infants at both ages prefer to listen to stimuli which preserve the complementary distribution over those which disrupt it (e.g., \emph{na-poli}, \emph{rot-boli}). As \citeauthor{White2008} note, this suggests that infants have both extracted the plosive voicing alternation and grouped \emph{pevi} and \emph{bevi} together.

\subsubsection{Lexical acquisition}

As first observed by \citet{Darwin1877}, the first words infants recognize are names, their own and those of caretakers. Recent studies suggest that infants learn these names as early as 4 months of age \citep{Bortfeld2005,Mandel1995,Tincoff1999}. Infants as young as 6 months of age have learned the meaning of familiar words \citep{Bergelson2012}. By 8 months of age, infants are able to locate familiar and novel words in utterances \citep{Jusczyk1997,Seidl2006}.

\section{Outline of the dissertation}

The remainder of the dissertation consists of three case studies which provide support for the novel and contentious predictions of the minimal phonotactic model sketched above.

Chapter \ref{gradience} considers evidence from wordlikeness rating tasks. It is argued that intermediate well-formedness ratings are obtained whether or not the categories in question are graded. A primitive categorical model of wordlikeness using prosodic representations is outlined, and shown to predict English speakers' wordlikeness judgements at least as acccurately as state-of-the-art gradient wellformedness models. Once categorical effects are controlled for, gradient models are uncorrelated with well-formedness ratings.

Chapter \ref{turkish} considers the relationship between lexical generalizations, phonological alternations, and speakers' nonce word judgements with a focus on Turkish vowel patterns. It is shown that even exception-filled phonological generalizations have a robust effect on wellformedness judgements, but that statistically reliable phonotactic generalizations go unlearned when they are not derived from phonological alternations.

Chapter \ref{clusters} investigates the role of phonological alternations in constraining lexical entries, focusing specifically on medial consonant clusters in English. Static phonotactic constraints previously proposed to describe gaps in the inventory of medial clusters are shown to be statistically unsound, whereas phonological alternations impose robust restrictions on the cluster inventory. The remaining gaps in the cluster inventory are attributed to the sparse nature of the lexicon, not static phonotactic restrictions.

%% NOTES HERE
%
%a. & \emph{B}[uːx]          & \emph{B}[yːç]\emph{er} & `book(s)' \\
%   & \emph{L}[ɔx]           & \emph{L}[øç]\emph{er}  & `hole(s)' \\
%   &\emph{T}[ɔx]\emph{ter} & \emph{T}[øç]\emph{ter}
%   & \emph{N}[ax]\emph{t}   & \emph{N}[ɛç]\emph{te} & `night' \\
%
% While the bulk of evidence for detailed phonological entries is found in infants over a year of age \citep[e.g.,]{Werker2002,Fennell2003,Fennell2006,Stager1997,Swingley2000,Zamuner2006},
%9-month-old infants already have learned the predominant stress pattern of their language \citep{Jusczyk1993a}.
%\emph{τάξις}
%Whether phonological computations or representations themselves are graded \citep[e.g.,][]{Lakoff1973} is besides the point, as metalinguistic judgements are behaviors, not mental states; they can no more be compared than can ``fear'' and ``flight response''.

\citet{Silverman2000}
\citet{Kiparsky1995}
\citet[352f.]{Hale2003a}
%$\begin{bmatrix} +\textsc{Dorsal} \\ +\textsc{Continuant \end{bmatrix}~\goesto~\begin{bmatrix} =\textsc{Back} \end{bmatrix}~/~\begin{bmatrix} =\textsc{Back} \end{bmatrix}~\gap$
