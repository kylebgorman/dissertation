\label{intro}

This dissertation has two distinct but complementary aims. 
The first is to outline the empirical scope of \emph{phonotactic theory}, the theory of speakers' knowledge of possible and impossible sounds, sound sequences, and words.
The second is to show that the facts of this domain are consonant with a traditional view of phonotactic knowledge as categorical, but inconsistent with an increasingly popular view of phonotactics and phonotactic learning as a type of statistical inference over lexical entries.

If MSCs are epiphenomenal, there are no substantive constraints on underlying representations which are not derived from phonological processes.
This principle---\emph{no static phonotactic constraints}---has interesting ramifications for evaluating certain competing phonological analyses.
Consider Sanskrit aspiration alternations such as \emph{bodhati}-\emph{bhotsyati} `he wakes-he will wake'.
%According to one analysis, which has precedents going back to Pāṇini, the root /budh/ undergoes a process shifting aspiration leftward in certain contexts (e.g., \citeboth{Borowsky1983}, \citeboth{Hoenigswald1965}, \citeboth{Kaye1985}, \citeboth{Sag1974}, \citeyear{Sag1976}, \citeboth{Schindler1976}, \citeboth{Stemberger1980}, \citeboth{Whitney1889}:\S141f.).
%An alternative analysis posits an underlying /bhudh/ and a process of aspirate dissimilation, a synchronic analogue of Grassman's Law (e.g., \citeboth{Anderson1970}, \citeboth{Hoard1975}, \citeboth{Kiparsky1965}:\S3.2, \citeboth{Phelps1973}, \citeboth{Phelps1975b}, \citeboth{Zwicky1965}:109f.).
%Under the latter analysis, multiple surface aspirates (e.g., hypothetical *\emph{bhodhati}) are phonotactically marked; the former account makes no such prediction.
%If the principle of no static phonotactic constraints can be maintained, psycholinguistic tasks could in theory be used to decide between these two accounts.

%While lexical entries are thought to consist of segments, and segments to consist of feature specifications, the phonological form of a word like \emph{spectre} is not ``generated'', in any relevant sense, by the concatenation of segments /s/, /p/, and so on, or by the concatenation of syllables \{spɛk\}$_{\sigma}$ and \{tɚ\}$_{\sigma}$. 
%It would be possible to posit ``lexical concatenation rules'', on analogy with the phrase structure rules of syntax, but it is unclear how the strings produced would be linked with meanings, or how such a system would distinguish between, e.g., \emph{brick} and meaningless \emph{blick}.
%The more appealing alternative is that the phonological form of \emph{spectre}, whatever it is, is an atom of unit of lexical memory. 
%It remains to account for any restrictions on the contents of underlying forms. 
%The theory of \emph{phonotactics} (which is related to \emph{syntax} through the Greek root \emph{táxis} `order') is concerned with these restrictions and how they are related to the mappings between underlying and surface forms for which the phonological component is responsible. 
%\footnote{It is tempting, however, to describe binding principles as ``syntactics''.}

\section{The empirical scope of phonotactic theory}

\section{The architectural basis of phonotactic knowledge}

With the primary evidence for phonotactic theory now established, it is possible to consider the grammatical architecture that underlies this knowledge.

\subsection{The insufficiency of morpheme structure constraints}

Early generative phonologists posited that phonotactic ill-formedness derives solely from \emph{morpheme structure constraints}, restrictions on underlying representations \citep[e.g.,][]{Chomsky1966,SPE,SPR,Halle1962}.
These come in two types \citep{Stanley1967}.
\emph{Segment structure constraints} impose restrictions on the underlying segment inventory: e.g., in Russian, voicing is contrastive for all obstruents except /ts, tʃ, x/ \citep[22]{SPR}: [dz, dʒ, ɣ] appear in surface, but not underlying, representations.
\emph{Sequence structure constraints} apply to underlying sequences: an example is given below.

\begin{example}[An English MSC (adapted from \citealt{Chomsky1965}:100)]
$\begin{bmatrix} +\textsc{Cons} \end{bmatrix}~\goesto~\begin{bmatrix} +\textsc{Liquid} \end{bmatrix}~/~\#~\begin{bmatrix} +\textsc{Cons} \end{bmatrix}~\gap$
\end{example}

\noindent
This sequence structure constraint specifies the second of a sequence of word-initial consonants as a liquid, so as to preclude underlying /bnɪk/, for example. 

However, \citet{Shibatani1973} argues convincingly that not all wordlikeness contrasts can be expressed as constraints on URs.
In German, for instance, obstruent voicing is contrastive, but neutralizes finally: e.g., [ɡʀaːt]-[ɡʀaːtə] `ridge(s)' vs.~[ɡʀaːt]-[ɡʀaːdə] `degree(s)'
.
By hypothesis, the non-alternating `ridge(s)' indicates a root-final /t/, and the [t]-[d] alternation in `degree(s)' a final /d/. 
Since the voicing of root-final obstruents is contrastive, the surface restriction on obstruent voicing is not also a sequence structure constraint.
Yet, \citeauthor{Shibatani1973} claims, German speakers judge voiced obstruent-final nonce words to be ill-formed.\footnote{
    Voicing of final obstruents is also lost in German loanword adaptation: e.g., English \emph{hot dog} becomes [hɑt dɔk] \cite[506]{Ussishkin2003}.}

\subsection{The duplication problem}
\label{ss:dp}

Whereas \citeauthor{Shibatani1973} argues that morpheme structure constraints are insufficient to account for speakers' knowledge of possible words, other authors suggest that the distinction btween morpheme structure constraints and phonological processes may be artifactual.
\citet[297]{Hale1965}, \citet{Kisseberth1970b}, and \citet[212f.]{Postal1968} all observe the tendency for structural descriptions of phonological processes to reappear among the morpheme structure constraints on the same language (e.g., \citeboth{Hale1965}:297, \citeboth{Kisseberth1970b}, \citeboth{Postal1968}:212f.).
In Russian, for instance, a process of anticipatory assimilation ensures that derived clusters of obstruents agree in voice.\footnote{For sake of discussion, the somewhat more complex behavior of of [v] is ignored here.}

\begin{example}[Russian voice assimilation alternations (adapted from \citealt{SPR}:22f.)]
\label{rovs}
\begin{tabular}{l ll ll}
a. & [ˈʒed.ʒbɨ] & `were one to burn'      & [ˈʒet.ʃl\pal{}i] & `should one burn?'      \\
b. & [ˈmoɡ.bɨ]  & `were (he) getting wet' & [ˈmok.l\pal{}i]  & `was (he) getting wet?' \\
\end{tabular}
\end{example}

\noindent
Similarly, underlying voicing is ``nondistinctive in all but the last member of a cluster of obstruents'' (\citeboth{A74}:283): e.g., [proz\pal{}bə] `request'.
Despite their tantalizing similarity, these two facts are treated as distinct under the traditional view.
This is what 
\citet[?]{KK77} call the \emph{duplication} problem, 


In some prior sense, the alternation facts are more privileged.
It is certainly possible to deny that the restrictions on underlying representations are psychologically real since they are ``computationally inert and thus irrelevant to the input-output mapping that the grammar is responsible for'' \citep[18]{PE}.
%Somewhat more to the point, the derivations will come out the same whether or not the grammar includes any statement of the redundancy of obstruent voicing. 
On the other hand, voice assimilation is essential to a concise statement of the surface forms of Russian; the only alternative would result in a huge number of allomorph; the only alternative would be to posit an enormous amount of phonologically sensitive suppletive allomorphy, so that, for instance, /moɡ/ occurs before voiced obstruents and /mok/ elsewhere.
\namecite[205f.]{Dell1973} and \namecite[28f.]{Stampe1973} suggest that the problem is that the distinction between constraints on URs and alternations is unmotivated, and that these different levels are related by a principle now known as \emph{Stampean occultation} \cite[54]{OT}.
In a language like Russian, in which surface obstruent clusters exceptionlessly agree in voicing, there is simply no reason for the language acquisition device to posit underlying hetero-voiced obstruent clusters: obstruent voice assimilation ``occults'' underlying */kb/, for instance.
Were such an underlying form posited, it would surface as [gb] in all contexts.
Unless there is some alternation (e.g., epenthesis) which would reveal /kb/, there is simply no reason for learners to posit this abstract underlying representation.\footnote{
    \emph{Lexicon Optimization} \cite[209]{OT} implements a form of Stampean occultation notable in that it projects all non-alternating surface segments directly into URs.
    For instance, in English, Lexicon Optimization demands underlying /ŋ/ in words like \emph{bank}, even though [ŋ] can be analyzed as an allophone of /n/ before velar consonants (e.g., \citeboth[65f.]{Borowsky1986}:65f.), simplifying the phoneme inventory.
    However, this is not a necessary component of an implementation of \citeauthor{Dell1973} and \citeauthor{Stampe1973}'s insight about the relationship between surface and underlying sequence structure restrictions.
    For instance, the hypothetical /bæŋk/ posited by Lexicon Optimization could be revised to /bænk/, and take a \emph{free ride} (in the sense of \citealt{Zwicky1970}) on the process of nasal place assimilation found elsewhere in English.
    Indeed, this seems desirable, since Lexicon Optimization forces a duplication between underlying and surface constraints: for instance, [ŋ] does not appear word-initially and English speakers have considerable difficulty producing it in this position \cite{Rusaw2009}.
    The allophonic analysis of [ŋ] predicts this fact, since there is no way to derive the [ŋ] allophone in onset position.
    Assuming Lexicon Optimization, the only alternative is to posit an ad-hoc constraint against onset [ŋ] \citep[39]{Jusczyk2002}.
    Future work will consider how purely allophonic relationships could be acquired in a framework like Lexicon Optimization.}
In an architecture like Lexical Phonology, it is even possible to apply a process to individual underlying representations (i.e., at the ``lexical level'') to enforce constraints which are not surface-true.
Consequently, MSCs are otiose, their empirical converage entirely subsumed by phonological processes.

This is similar in principle to the context in which Russian obstruent voicing is discussed by \citet{SPR}.
The principle of biuniquness in vogue at that time distinguishes between, and separates, neutralizing (morphophonemic) and non-neutralizing (phonemic) phonological processes.
In Russian, obstruents participate in voice assimilation whether this neutralizes a phonemic distinction (\ref{rovs}a) or not (\ref{rovs}b): recall that there is no underlying /dʒ/ in Russian.
\citeauthor{SPR} famously argues that biuniqueness (and the distinction between the morphophonemic and phonemic levels that follows from it) entails ``a significant increase in the complexity of the representation'' (24).
While \namecite{Anderson2000} sketches an analysis which preserves biuniqueness without morphophonemic/phonemic duplication, this requires further contested assumptions---contrastive underspecification (against which, see \citeboth{Steriade1995}) and a Duke-of-York derivation, and so is consistent with \citeauthor{SPR}'s claim that biuniquness imposes unnecessary complexities.

Even constraints on underlying representations may be elegantly described in terms of non-contrastive prosodic structures like the syllable \citep[e.g.,][]{Hooper1973,Kahn1976}.
For instance, as \citet{Haugen1956} notes, many restrictions on medial consonant clusters follow from the requirement that word-medial consonant clusters be decomposable into a (possibly null) well-formed coda and a well-formed onset.
Assuming Stampean occultation, syllable structure need not be underlyingly present to derive this constraint, contrary to what has sometimes been claimed (e.g., \citeboth{A74}:255).



If some underlying cluster does not surface faithfully, it 



: an underlying representation containing a cluster which could not be decomposed in this fashion would be occulted.

Chapter \ref{gaps} FIXME
Similarly, initial /bn/ in English is occulted by FIXME

\section{Problems for probabilistic phonotactics}

\section{Outline of the dissertation}

The remainder of the dissertation consists of three case studies which provide support for the novel and contentious predictions of the minimal phonotactic model sketched above.

Chapter \ref{gradience} considers the status of intermediate well-formedness ratings in wordlikeness tasks. 
It is argued that intermediate wellformedness ratings are obtained whether or not the categories being rated are graded or not.
A primitive categorical model of wordlikeness using simple prosodic representations is outlined, and shown to predict English speakers' wordlikeness judgements as well as state-of-the-art gradient wellformedness models; a simple model of similarity to existing words is also a reliable predictor of wellformedness ratings.
Once categorical effects are controlled for, current gradient phonotactic models are largely uncorrelated with wellformedness ratings.

Chapter \ref{turkish} considers the relationship between lexical generalizations, phonological alternations, and speakers' nonce word judgements with a focus on Turkish vowel patterns.
It is shown that even exception-filled phonological generalizations may have a robust effect on wellformedness judgements.
However, phonotactic generaliztaions go unlearned when they are not derived from phonological alternations; this fact is independent of ``naturalness''.

Chapter \ref{clusters} investigates medial consonant clusters in English.
Static phonotactic constraints previously proposed by \citet{Pierrehumbert1994} to describe gaps in the inventory of medial clusters are shown to be statistically unsound.
Neutralizing phonological alternations impose robust restrictions on the cluster inventory and the remaining gaps can be attributed to the sparse nature of the lexicon. 

% JUNK

%Whereas \citeauthor{Shibatani1973} maintains that morpheme structure constraints are insufficient to account for speakers' knowledge of possible wordhood, others argue that morpheme structure constraints are also unnecessary. \citet[297]{Hale1965}, \citet{Kisseberth1970b}, and \citet[212f.]{Postal1968} observe that the structural descriptions of phonological processes often are often reflected in the lexical redundancies in the same language. In Russian, there are alternations implicating a process of obstruent voice assimilation, and tautomorphemic obstruent clusters have uniform voicing \citep[283]{A74}. \citet[205f.]{Dell1973} and \citet[28f.]{Stampe1973} argue that morpheme structure constraints are otiose, as phonological rules triggering alternations impose restrictions on the contents of URs, an effect known as \emph{Stampean occultation}.
%\citet{Silverman2000}
%\citet{Kiparsky1995}
%\citet[352f.]{Hale2003a}
%Theories of phonotactic knowledge should be evaluated by the same stringentcriteria applied to other formal arenas: neither overgeneration nor undergeneration should be permitted. It will be shown that the orthodoxy that phonotactic knowledge is in some sense ``probabilistic'' suffers from quite severe over- and undergeneration. 
%This heuristic has the greatest impact regarding debates about ``abstractness'' of underlying representations. 
%It cannot be said, precisely, that it either excludes or requires any particular type of abstractness. 
%Proposed restrictions on UR abstractness have been faulted for a number of reasons. In some cases, they preclude otherwise-desirable analyses of alternations. Another objection that could reasonably made is that any restriction on abstract URs that introduces an assymetry in URs is bad.
%\citep[~chap.~1]{KK77}
%I would like to suggest that there are two types of problems that arise in developing a theory of phonotactics free of duplication. The first type of problem consists of conflicts between theoretical assumptions and the desire to eliminate duplication. If we continue to view duplication as a sort of negative heuristic, then it may be the case that the theoretical assumptions are wrong. Such a case arises in Chapter \ref{turkish} in the discussion of archiphonemic underspecification analysis of Turkish vowel harmony proposed by \citet{Clements1982}. \citeauthor{Clemenst1982} propose that all but the first vowel of a harmonic root is underspecified for backness (and in high vowels, roundness) and is filled in by rule. Since there are disharmonic roots, this rule must be ``structure-filling''. Consequently, this rule cannot account for the apparent markedness of disharmonic roots revealed by wordlikeness judgements (among other psycholinguistic tasks). If duplication is a pathology, this analysis is wrong.
%The arguments of \citeauthor{Shibatani1973} and others led theorists to focus their attention on properties of surface representations as determinants of wordlikeness. Though syllabification plays no role in \emph{SPE}, it is crucial to many earlier studies \citep[for a review, see][]{Goldsmith2011b}, and it received particular attention in the 1970s. \citet{Hooper1973} and \citet{Kahn1976} argue that the syllable is useful for defining wordlikeness generalizations.\footnote{\citet{Steriade1999} and \citet{Blevins2003}, however, argue that a number of phonotactic generalizations previously stated in syllabic terms can be reanalyzed without making reference to syllables.} \citeauthor{Hooper1973} argues, for instance, that [bn], impossible as an English onset, is unobjectionable as a syllable contact cluster in nonce words like \emph{stabnik} (or in names like \emph{Abner}), and that this demonstrates the superiority of syllable-based wordlikeness generalizations. This already signals further trouble for alternative accounts which focus on underlying forms. Syllabification may span morphs, is generally predictable, and is universally non-contrastive, and as a consequence, few posit in to be present in underlying representations \citep[though see, e.g.,][]{Vaux2003}. \citeauthor{Hooper1973} also points to loanword adaptations which produce native syllable structure \citep[e.g.,][]{Carlisle1991} as evidence that syllabification is part of the phonological computation. Further enrichments to the theory are provided by the autosegmental theory of the syllable \citep{McCarthy1979b}, which envisions the syllable as an articulated tree structure \citep[as first envisioned by][]{Pike1947a}, and theories like prosodic licensing \citep{Ito1989a}, in which syllabification triggers phonological repairs.
%Despite the considerable attention given to the proposals of \emph{SPE} in the wake of that book's publication in 1968, the \emph{SPE} wordlikeness model has received almost no further attention in the literature. At the risk of explaining what might be no more than an accidental gap in the literature, the novel aspects of \emph{SPE} model---gradience derived from similarity to existing lexical entries---may have been overshadowed by the many other contentious proposals in \emph{SPE}, and particularly by compelling arguments against the assumption that wordlikeness contrasts derive solely from properties of underlying forms. \citet{Shibatani1973} observes that there are some generalizations about surface forms which give rise to wordlikeness contrasts, but cannot be stated as constraints on underlying forms. An example from German is shown in (\ref{fd}) below.

%\emph{K}[uːx]\emph{en}
%\emph{K}[uːç]\emph{en}
%\footnote{Examples like \emph{Mas}[oːç]\emph{ist} `masochist', \emph{Eun}[uːç]\emph{ismus} `eunichism', first noted by \citet{Merchant1994}, suggest that this should also be restricted to assimilation within the same foot \citep[226f.]{Jensen2000}.}

%Whether phonological computations or representations themselves are graded \citep[e.g.,][]{Lakoff1973} is besides the point, as metalinguistic judgements are behaviors, not mental states; they can no more be compared than can ``fear'' and ``flight response''.

%\begin{example}[German \emph{ich}- and \emph{ach-laut}]
%\begin{tabular}{l l l}
%a. & [buːx]   & `book'           \\
%   & [tɔxtər] & `daughter'       \\
%   & [naxt]   & `night'          \\
%b. & [siçt]   & `view'           \\
%   & [ʃpeçt]  & `woodpecker'     \\
%   & [ɡərʏçt] & `rumor'          \\
%   & [knøçəl] & `ankle, knuckle' \\
%   & [flɛçə]  & `surface'        \\
%\end{tabular}
%\end{example}
%
%\emph{Umlaut}, the fronting (and raising) of back vowels in certain morphological contexts, produces the front variant of the dorsal fricative; e.g., [lɔx]-[løçər] `hole-holes', \emph{B}[uːx]-\emph{B}[yːç]\emph{er} `book-books',
