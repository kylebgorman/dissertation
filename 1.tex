\label{intro}

This dissertation has two distinct but complementary aims. 
The first is to outline the empirical scope of \emph{phonotactic theory}, the theory of speakers' knowledge of possible and impossible sounds, sound sequences, and words.
The second is to show that the core facts of this domain are compatible with a traditional view of phonotactic knowledge as independent of the lexicon, categorical, and closely related to phonological processes, but inconsistent with an increasingly popular view of phonotactics and phonotactic learning as a type of probabilistic inference over the lexicon, and therefore gradient and independent of phonological processes.\footnote{
    Throughout, the term \emph{lexicon} is used in a specific sense of the set underlying representations in some language; this is not meant to imply a position on the possibility that larger, composite linguistic representations are also stored in lexical memory.
    For a review of recent experimental evidence on this question, see \citealt{LignosInPressa}.}

Despite a recent surge of interest in phonotactic theory, the empirical scope of the theory remains poorly defined.
%and consequently, it is not uncommon to find suggestions that speakers do not have any type of phonotactic knowledge at all \citep[e.g.,][17f.]{PE}. 
The first task in developing a theory of phonotactic knowledge, then, is to outline the types of facts that the theory should account for.

\section{The empirical scope of phonotactic theory}
\label{s:espt}

Phonotactic knowledge is evidenced in quite different ways than phonological knowledge.
In Russian, for instance, a process of anticipatory assimilation ensures that derived clusters of obstruents agree in voice.\footnote{
    For sake of discussion, the complex behavior of [v] is ignored here.}

\begin{example}[Russian voice assimilation alternations (adapted from \citealt{SPR}:22f.)]
\label{rovs}
\begin{tabular}{l ll ll}
a. & [ˈʒedʒbɨ] & `were one to burn'      & [ˈʒetʃl\pal{}i] & `should one burn?'      \\
b. & [ˈmoɡbɨ]  & `were (he) getting wet' & [ˈmokl\pal{}i]  & `was (he) getting wet?' \\
\end{tabular}
\end{example}

There is only one concise explanation for the [dʒ]-[tʃ] and [k]-[ɡ] alternations, namely that Russian speakers have internalized a process of voice assimilation.
It is also the case that voicing in underlying representations in Russian is ``nondistinctive in all but the last member of a cluster of obstruents'' \citep[283]{A74}.
It is possible to deny, however, that this restriction on underlying hetero-voiced obstruent clusters is part of the grammatical knowledge of Russian speakers, as some linguists have done.

\begin{quote}
MSCs are merely artifacts of the grammar, and thus play no part in the phonological component of a language. \citep[302]{Clayton1976}

Even if we, as linguists, find some generalizations in our description of the lexicon, there is no reason to posit these generalizations as part of the speaker's knowledge of their language, since they are computationally inert and thus irrelevant to the input-output mappings that the grammar is responsible for. \citep[17f.]{PE}

[MSCs] have no observable consequences in the course of the normal use of the language\ldots{} \citep[320]{Zimmer1969}
\end{quote}

\noindent
\citeauthor{PE} go on to label constraints on underlying representations as ``extensional'', and thus irrelevant to generative grammar, in contrast to ``intensional'' statements like phonological rules.
Regarding the facts about Russian obstruents discussed so far, it seems to be clear that the alternation facts are in some sense more privileged.
Were it the case that Russian speakers had not internalized a process of obstruent voicing assimilation, the only alternative explanation for the forms in (\ref{rovs}) is a massive system of phonologically conditioned suppletive allomorphy.
In contrast, were there no constraint on underlying obstruent voicing, however, it is not immediately obvious that \emph{anything} would be different. 
Presumably, this is what \citeauthor{PE} mean when they refer to constraints on underlying representations as ``computationally inert''.
There are also many cases where restrictions on underlying representations are specific to underlying representations and do not apply to surface representations.
For instance, hiatus is exceptionally rare in native Turkish roots; of the handful of examples in the Turkish Electronic Living Lexicon \citep{TELL}, many appear to be compounds (e.g., \emph{ısıalan} `endothermic', cf.~\emph{ısı} `heat', \emph{alan} `taker').
Yet numerous phonological processes give rise to hiatus in derived environments \citep[e.g.,][]{Kabak2007b}.
\citet{Silverman2000} observes that operations like reduplication and truncation also tend to introduce violations of constraints on underlying representations.
However, there are many facts which strongly suggest that speakers internalize phonotactic constraints.

\subsection{Wordlikeness judgements}

As \citet{Halle1962} and \citet{Chomsky1965} note, speakers can distinguish between a well-formed and an ill-formed word, neither of which is an actual word.
Neither [blɪk] \emph{blick} nor [bnɪk] \emph{bnick} is a word of English, yet English speakers immediately report that only the former is ``possible'', an \emph{accidental} gap in the lexicon, whereas the latter is judged to be impossible (i.e., structurally excluded).
There can be no question that it is part of speakers' knowledge.
Elicited in a controlled fashion, these \emph{wordlikeness judgements} are perhaps the most important (and least controversial) source of phonotactic evidence, and they play a major role in chapters \ref{gradience}--\ref{turkish} of this dissertation.

\subsection{Word production and recognition}
\label{ss:wpr}

Speakers have difficulty producing \citep[e.g.,][]{Davidson2006a,Davidson2010,Rose2007,Vitevitch1997} and perceiving \citep{Dupoux1999,Kabak2007a,Massaro1983} certain types of nonce words judged to be phonotactically illicit.
For instance, \citet{GallagherInPress} finds that speakers of Quechua have difficulty repeating nonce words with multiple ejectives (e.g., [k'ap'i]), which do not occur in the language;
in other words, multiple ejective sequences are not merely absent, but also difficult for these speakers to repeat.

In English, sequences of adjacent obstruents which do not also agree in voice (e.g., \emph{a}[b.s]\emph{inth}) are quite rare within a word, and therefore a hetero-voiced obstruent cluster is a clue to the presence of a word boundary in running speech.
Infants \citep[e.g.,][]{Mattys2001b} and adults \citep{McQueen1998b,Norris1997} are thought to use this heuristic for word recognition in experimental settings.\footnote{
    There is reason to distinguish the computations involved in using this word segmentation heuristic from those implicated by other phonotactic behaviors.
    Whatever the locus of [bnɪk]'s ill-formedness, for example, no segmentation into multiple words or morphs renders it a well-formed sequence.
    See \S\ref{ss:bh} for further discussion.}

\citet{Berent2001b} and \citet{Coetzee2008b} claim that non-word recognition latencies in lexical decision tasks reflect speakers' phonotactic knowledge, the hypothesis being that a phonotactically illicit nonce word will be rejected more quickly than a well-formed nonce word.
However, phonotactic constraints are often confounded with independent predictors of lexical decision latencies.
For instance, \citet{Coetzee2008b} finds that English speakers recognize [sp\ldots{}p] and [sk\ldots{}k] nonce monosyllables faster than [st\ldots{}t] nonce monosyllables in an auditory lexical decision task.
\citeauthor{Coetzee2008b} attributes this to ad hoc phonotactic constraints against the former sequences, but another explanation is available.
Even at an early stage of recognition, [stVt] is distinguished from [spVp, skVk] by its higher \emph{cohort density}: there are far more English words starting with initial [st] than with [sp] or [sk].
High cohort density is known to inhibit auditory processing of non-words \citep[e.g.,][]{Marslen-Wilson1978} and this alone could account for the processing difference.

\subsection{Loanword adaptation}

Loanword adaptation may provide further evidence for the grammatical relevance of phonotactic knowledge.
In Desano \citep{Kaye1974}, for instance, all underlying representations (URs) are either totally oral (e.g., [yaha] `to hear') or totally nasal (e.g., [ñãhã] `to enter'), and loanwords are made to conform to this generalization: Portuguese \emph{martelo} `hammer' is adapted as [barateru] and Spanish \emph{naranja} `orange' as [nãnãnã].
It is not a stretch to imagine that some component of the synchronic grammar is responsible for the fact that the restriction over native vocabulary is extended to loanwords.
% FIXME discuss evidence that Desano has nasal spreading
% FIXME mention \citet{Jacobs2000}

\begin{example}[Adaptation of onset clusters in Finnish loanwords]
\begin{tabular}{l llll}
a. & toːri    & `store'   & ($<$ \emph{store})              & \citep[English:][89]{Hellstrom1976}   \\
   & rosseri  & `grocery' & ($<$ \emph{grocery})            &                             \\
b. & ranska   & `French'  & ($<$ \emph{Franska})            & \citep[Swedish:][67]{Campbell2004}    \\
   & ruːvi    & `screw'   & ($<$ \emph{skruv})              &                             \\
c. & risti    & `cross'   & ($<$ \emph{kristĭ})             & \citep[Old Russian:][60]{Bjornflaten2006} \\
   & raːmattu & `bible'   & ($<$ \emph{gramota} `document') &                             \\
\end{tabular}
\end{example}

There are many other cases, however, where it is claimed that phonotactic restrictions are not extended to loanwords (e.g., \citealt{Clements1982}, \citealt[75]{Davidson1997}, \citealt{Fries1949}, \citealt{Holden1976}, \citealt{Ito1995a,Ito1995b}, \citealt[95]{Shibatani1973}, \citealt{Ussishkin2003}, \citealt{Vogt1954}; additional examples can be found throughout this dissertation).
For instance, native words in San Mateo Huave all end in a consonant, but final unstressed syllables in Spanish loanwords are never repaired by epenthesis (e.g., \emph{verde} `green' $>$ [beɾde], *[beɾdej]).
%FIXME example here
Given the limited understanding of loanword adaptation at the present juncture, it may be premature to regard this inertness as strong evidence against the constraints in question, though it may be a useful diagnostic.
% FIXME mention hyman
% peperkamp & dupoux 2003
% peperkamp 2005

\subsection{Alternate phonologies}

Language games or speech disguises may also provide evidence for phonotactic knowledge \citep[e.g.,][]{Vaux2011}; an example appears in \S\ref{ss:bh} below.

\subsection{Lexical statistics}

Finally, phonotactic gaps or tendencies in the lexicon are often taken as evidence for phonotactic knowledge, under the hypothesis that grammatical constraints are the cause of these lexical generalizations.
Chapters \ref{turkish}--\ref{gaps} consider in detail the evidentiary status of these lexical tendencies.

\section{The grammatical basis of phonotactic knowledge}

With the primary evidence for phonotactic theory now established, it is possible to consider the grammatical architecture that underlies this knowledge.

\subsection{The insufficiency of morpheme structure constraints}

Early generative phonologists posited that phonotactic ill-formedness derives solely from \emph{morpheme structure constraints}, restrictions on underlying representations \citep{Chomsky1965,SPE,SPR,Halle1962}.
\citet{Stanley1967} distinguishes between two types.
\emph{Segment structure constraints} impose restrictions on the underlying segment inventory.
For example, voicing is non-contrastive for /ts, tʃ, x/ in Russian \cite[22]{SPR}: [dz, dʒ, ɣ] appear in surface, but not underlying, representations.
\emph{Sequence structure constraints} apply to underlying sequences; an example is given below.

\begin{example}[An English MSC (adapted from \citealt{Chomsky1965}:100)]
$\begin{bmatrix} +\textsc{Cons} \end{bmatrix}~\goesto~\begin{bmatrix} +\textsc{Liquid} \end{bmatrix}~/~\#~\begin{bmatrix} +\textsc{Cons} \end{bmatrix}~\gap$
\end{example}

\noindent
This sequence structure constraint specifies the second of a sequence of word-initial consonants as a liquid, so as to preclude underlying */bnɪk/, for example.

However, \citet{Shibatani1973} shows quite convincingly that not all wordlikeness contrasts can be expressed as constraints on URs.
In German, for instance, obstruent voicing is contrastive, but neutralizes finally: e.g., [ɡʀaːt]-[ɡʀaːtə] `ridge(s)' vs.~[ɡʀaːt]-[ɡʀaːdə] `degree(s)'.
By hypothesis, the latter root ends in /d/, so the constraint against final voiced obstruents is specific to surface representations.
\citeauthor{Shibatani1973} claims, however, that German speakers judge voiced obstruent-final nonce words to be ill-formed.\footnote{
    Voicing of final obstruents is usually lost in German loanword adaptation: e.g., English \emph{hot dog} becomes [hɑt dɔk] \citep[506]{Ussishkin2003}.}

\subsection{The duplication problem}
\label{ss:dp}

Whereas \citeauthor{Shibatani1973} argues that morpheme structure constraints are insufficient to account for speakers' phonotactic knowledge, other authors observe the tendency for structural descriptions of phonological processes to reappear among the morpheme structure constraints on the same language (e.g., \emph{SPE}:382, \citealt{Hale1965}:297, \citealt{Kisseberth1970b}, 
%\citealt{PasterInPress},
\citealt{Postal1968}:212f., \citet{Stanley1967}:401).
Russian obstruent voice assimilation, discussed above, provides an example of this type: there are no hetero-voiced obstruent clusters in either underlying or surface representations.
These two facts are tantalizingly similar, but are treated as separate if a distinction between morpheme structure constraints and phonological processes is drawn.
This is a special case of what \citet{Kisseberth1970b} calls \emph{conspiracies}, which \citet[136]{KK77} term the \emph{duplication problem}.
\citet[205f.]{Dell1973} and \citet[28f.]{Stampe1973} argue that the distinction between constraints on URs and alternations is artificial, and that these different levels of description are related by a principle now known as \emph{Stampean occultation} \citep[54]{OT}.
In a language like Russian, in which surface obstruent clusters exceptionlessly agree in voicing, there is simply no reason for the language acquisition device to posit underlying hetero-voiced obstruent clusters: obstruent voice assimilation ``occults'' underlying */kb/, for instance. Were such an underlying form posited, it would surface as [gb] in all contexts.\footnote{
    \emph{Lexicon Optimization} \citep[209]{OT} implements a form of Stampean occultation notable in that it projects all non-alternating surface segments directly into URs.
    For instance, in English, Lexicon Optimization demands underlying /ŋ/ in words like \emph{bank}, though [ŋ] could be otherwise be analyzed as an allophone of /n/ before velar consonants (e.g., \citealt[65f.]{Borowsky1986}, \citealt[85]{SPE}, \citealt[62]{Halle1985a}), eliminating /ŋ/ from the phoneme inventory.
    However, this is not core to \citeauthor{Dell1973} and \citeauthor{Stampe1973}'s insight about the relationship between surface and underlying sequence structure restrictions.
    For instance, the hypothetical \{bæŋk\} posited by Lexicon Optimization could be revised to /bænk/, and take a \emph{free ride} (in the sense of \citealt{Zwicky1970}) on the process of nasal place assimilation found elsewhere in English (see \S\ref{npa}).
    Indeed, this seems desirable, since Lexicon Optimization forces a duplication between underlying and surface constraints.
%this point is made by \citet{Chomsky1957} and \citet[69f.]{Harris1960} with respect to the principle of biuniqueness, which enforces a similar restriction.
    For instance, [ŋ] does not appear word-initially and English speakers have considerable difficulty producing it in this position \citep{Rusaw2009}.
    The allophonic analysis of [ŋ] predicts this fact, since there is no way to derive the [ŋ] allophone in onset position.
    Future work will consider the how purely allophonic relationships might be acquired.}
    %Assuming Lexicon Optimization, the only alternative is to stipulate an ad hoc constraint against onset [ŋ] \citep[39]{Jusczyk2002}.}
In chapter \ref{gaps}, it is argued that constraints described in terms of non-contrastive prosodic structures can also be derived by Stampean occultation.
In an architecture like Lexical Phonology, it is even possible to apply a process to individual underlying representations (i.e., at the ``lexical level'') to enforce constraints which are not surface-true.
Consequently, morpheme structure constraints are otiose, their empirical converage entirely subsumed by phonological processes.

This argument against the distinction between morpheme structure constraints and phonological processes is quite similar to the famous argument (\citeyear{SPR}) against the morphophonemic/phonemic distinction.\footnote{
    This is not to imply that \citeauthor{SPR} was the first to make this argument, or to otherwise disregard this distinction: similar ideas can be found in work by \citet{Bloch1941}, \citeauthor{Bloomfield1926} (\citeyear{Bloomfield1926}:160, \citeyear{Bloomfield1962}:5f.), \citet{Chao1934}, \citet[244f.]{Hamp1953}, and \citet[47f.]{Sapir1930a}, among others (see \citealt{Anderson1985} \emph{passim}).}
The principle of biuniquness in vogue at that time separates neutralizing (morphophonemic) and non-neutralizing (phonemic) processes.
In Russian, obstruents participate in voice assimilation whether this neutralizes a phonemic distinction (\ref{rovs}a) or not (\ref{rovs}b): recall that there is no underlying /dʒ/ in Russian.
Under biuniqueness, there must be separate neutralizing and non-neutralizing variants of this process, however.
From this, \citeauthor{SPR} argues that biuniqueness (and the distinction between the morphophonemic and phonemic levels that follows from it) entails ``a significant increase in the complexity of the representation\ldots{}an unwarranted complication which has no place in a scientific description of language'' (24).
While \citeauthor{Anderson1985} (\citeyear{Anderson1985}:110f., \citeyear{Anderson2000}) sketches an analysis which preserves biuniqueness without morphophonemic/phonemic duplication (see also \citealt{Kiparsky1985}), this requires further contested assumptions---contrastive underspecification (against which, see \citealt{Steriade1995}) and a Duke-of-York derivation.
This does not seem inconsistent with \citeauthor{SPR}'s claim that biuniquness imposes unnecessary additional complexities: under \citeauthor{Anderson2000}'s analysis the complexity is not a duplication, but rather a dependency on contentious theoretical assumptions.

\subsection{Static constraints}

Under the assumptions of Stampean occultation, there is no way to express substantive constraints on underlying representations except via phonological process.
In lexical phonology, such constraints are the consequences of phonological processes operating at the ``lexical'' level, and in an Optimality Theory grammar, these are high ranked markedness constraints.\footnote{In Optimality Theory, it is often assumed that all markedness constraints are ranked above all faithfulness constraints in the ``initial state'' of the learner \citep[e.g.,][]{Smolensky1996a}.
    This admits the possibility that markedness constraints not implicated by alternations will remain undominated \citep[e.g.,][]{Coetzee2008b}.
    It is difficult to evaluate this proposal in the absence of any complete proposal for the contents of \textsc{Con}, the universal constraint sets, but this has the potential to blur the traditional distinction between accidental and structural phonotactic gaps: any exceptionless gap which corresponds to a markedness constraint \emph{in any language} will be accorded structural, inviolable status, a quite powerful prediction.
    If \textsc{Con} is sufficiently elaborate to incorporate constraints like the *[spVp] proposed by \citet{Coetzee2008b}, it might also incorporate a constraint like *[s.w], perhaps as a subcomponent of the so-called \emph{syllable contact law} \citep[e.g.,][]{Gouskova2004,Murray1983}, which disfavors syllable contact clusters with increasing sonority 
    The *[s.w] constraint is without exception in English, yet nonce words like \emph{teeswa} [tes.wa] seem quite unobjectionable.
    % FIXME tilwa?
    Numerous other examples of this type could be adduced.
    Consequently, this specific model is rejected here.}
A corrolary is that if some phonotactic constraint is not (or cannot be) identified with a phonological process, it does not exist in the minds of speakers at all.

This principle---\emph{no static phonotactic constraints}---has interesting ramifications for evaluating certain competing phonological analyses.
Consider Sanskrit aspiration alternations such as \emph{bodhati}-\emph{bhotsyati} `he wakes-he will wake'.
According to one analysis, which has precedents as far back as the grammar of Pāṇini, the root /budh/ undergoes a process shifting aspiration leftward in certain contexts (e.g., \citealt{Borowsky1983}, \citealt{Hoenigswald1965}, \citealt{Kaye1985}, \citealt{Sag1974}, \citeyear{Sag1976}, \citealt{Schindler1976}, \citealt{Stemberger1980}, \citealt{Whitney1889}:\S141f.).
An alternative analysis posits an underlying /bhudh/ and a process of aspirate dissimilation, a synchronic analogue of Grassman's Law (e.g., \citealt{Anderson1970}, \citealt{Hoard1975}, \citealt{Kiparsky1965}:\S3.2, \citealt{Phelps1973}, \citealt{Phelps1975b}, \citealt{Zwicky1965}:109f.).
Under the latter analysis, multiple surface aspirates (e.g., hypothetical *\emph{bhodhati}) are phonotactically marked; the former account makes no such prediction.
The principle of no static phonotactic constraints, if it can be maintained, could in theory adjudicate between these competing analyses: if the postulated constraint on multiple surface aspirates is active in wordlikeness judgements or loanword adaptation, for instance, this rules out the former account.

\subsection{Acquisition and the (in)dependence of phonotactic knowledge}
\label{ss:aerbpp}

\citet{Hayes2004b} argues that phonotactic learning occurs before lexical or phonological acquisition, and it therefore must be independent of other types of grammatical knowledge.
This is not fully consistent with the available experimental evidence, however.

Typically-developing infants know their names and the names of their caretakers as early as 4 months of age \citep{Bortfeld2005,Mandel1995,Tincoff1999}.
While infants are born able to distinguish between monosyllabic and bisyllabic words, the youngest infants are thought to recognize syllables ``holistically'' rather than as segment sequences \citep{Bertoncini1981,Eimas1999,Jusczyk1987}.
As early as 6 months of age, infants know the visual referents of familiar words presented auditorily \citep{Bergelson2012}.
At 7.5 months, phonological representations are sufficiently detailed to allow infants to discriminate between words like \emph{cup} and mispronunciations like \emph{*tup} \citep{Jusczyk1995}.
By 8 months, infants are able to locate both familiar and novel words in continuous speech \citep{Jusczyk1997,Seidl2006}.
Perception of non-native phonetic contrasts for vowels and consonants begins to decline at this age \citep{Best1994,Polka1994,Werker1981,Werker1984,Werker1988}.

While very few studies have investigated young infants' knowledge of phonological alternations, this is the subject of a fascinating study by \citet{White2008}.
Simplifying somewhat, the experimenters expose 8.5-month-old infants to an artificial language in which fricative voicing is contrastive, but voiced and voiceless variants of plosives are in complementary distribution, appearing only after vowels (\emph{na-bevi}) and after voiceless consonants (\emph{rot-pevi}), respectively.
After familiarization, infants prefer to listen to nonce words preserving this complementary distribution of stops over nonce words which disrupt this distribution (e.g., \emph{na-poli}, \emph{rot-boli}), suggesting that the infants have extracted an allophonic generalization concerning plosive voicing.

Crucially, all of this occurs \emph{before} there is any evidence for phonotactic knowledge, for which the earliest evidence begins at approximately 9 months of age \citep{Friederici1993,Jusczyk1993b,Jusczyk1994}.
% Sketch what this is.
While there are many gaps in current understanding (and keeping in mind the usual caveats about a strong interpretation of negative results), there is no reason to think that phonotactic knowledge is acquired before non-trivial amounts of lexical and phonological acquisition.
If anything, infants's phonotactic knowledge appears to come online at approximately the same point as highly specific phonological representations including subsyllabic units and the ability to extract allophonic generalizations, as predicted by the firm link between phonological and phonotactic knowledge argued for here.

\section{Outline of the dissertation}

The remainder of the dissertation consists of three case studies which provide support for the novel and contentious predictions of the minimal phonotactic model sketched above.

Chapter \ref{gradience} considers evidence from wordlikeness rating tasks. It is argued that intermediate well-formedness ratings are obtained whether or not the categories in question are graded. A primitive categorical model of wordlikeness using prosodic representations is outlined, and shown to predict English speakers' wordlikeness judgements as accurately as state-of-the-art gradient wellformedness models. Once categorical effects are controlled for, gradient models are uncorrelated with well-formedness ratings.

Chapter \ref{turkish} considers the relationship between lexical generalizations, phonological alternations, and speakers' nonce word judgements with a focus on Turkish vowel patterns. It is shown that even exception-filled phonological generalizations have a robust effect on wellformedness judgements, but that statistically reliable phonotactic generalizations go unlearned when they are not derived from phonological alternations.

Chapter \ref{gaps} investigates the role of phonological alternations in constraining lexical entries, focusing specifically on medial consonant clusters in English. Static phonotactic constraints previously proposed to describe gaps in the inventory of medial clusters are shown to be statistically unsound, whereas phonological alternations impose robust restrictions on the cluster inventory. The remaining gaps in the cluster inventory are attributed to the sparse nature of the lexicon, not static phonotactic restrictions.

Chapter \ref{conclusions} summarizes the findings and describes future directions.

%Whereas \citeauthor{Shibatani1973} maintains that morpheme structure constraints are insufficient to account for speakers' knowledge of possible wordhood, others argue that morpheme structure constraints are also unnecessary. \citet[297]{Hale1965}, \citet{Kisseberth1970b}, and \citet[212f.]{Postal1968} observe that the structural descriptions of phonological processes often are often reflected in the lexical redundancies in the same language. In Russian, there are alternations implicating a process of obstruent voice assimilation, and tautomorphemic obstruent clusters have uniform voicing \citep[283]{A74}. \citet[205f.]{Dell1973} and \citet[28f.]{Stampe1973} argue that morpheme structure constraints are otiose, as phonological rules triggering alternations impose restrictions on the contents of URs, an effect known as \emph{Stampean occultation}.
%\citet{Silverman2000}
%\citet{Kiparsky1995}
%\citet[352f.]{Hale2003a}
%Theories of phonotactic knowledge should be evaluated by the same stringentcriteria applied to other formal arenas: neither overgeneration nor undergeneration should be permitted. It will be shown that the orthodoxy that phonotactic knowledge is in some sense ``probabilistic'' suffers from quite severe over- and undergeneration. 
%This heuristic has the greatest impact regarding debates about ``abstractness'' of underlying representations. 
%It cannot be said, precisely, that it either excludes or requires any particular type of abstractness. 
%Proposed restrictions on UR abstractness have been faulted for a number of reasons. In some cases, they preclude otherwise-desirable analyses of alternations. Another objection that could reasonably made is that any restriction on abstract URs that introduces an assymetry in URs is bad.
%\citep[~chap.~1]{KK77}
%I would like to suggest that there are two types of problems that arise in developing a theory of phonotactics free of duplication. The first type of problem consists of conflicts between theoretical assumptions and the desire to eliminate duplication. If we continue to view duplication as a sort of negative heuristic, then it may be the case that the theoretical assumptions are wrong. Such a case arises in Chapter \ref{turkish} in the discussion of archiphonemic underspecification analysis of Turkish vowel harmony proposed by \citet{Clements1982}. \citeauthor{Clemenst1982} propose that all but the first vowel of a harmonic root is underspecified for backness (and in high vowels, roundness) and is filled in by rule. Since there are disharmonic roots, this rule must be ``structure-filling''. Consequently, this rule cannot account for the apparent markedness of disharmonic roots revealed by wordlikeness judgements (among other psycholinguistic tasks). If duplication is a pathology, this analysis is wrong.
%The arguments of \citeauthor{Shibatani1973} and others led theorists to focus their attention on properties of surface representations as determinants of wordlikeness. Though syllabification plays no role in \emph{SPE}, it is crucial to many earlier studies \citep[for a review, see][]{Goldsmith2011b}, and it received particular attention in the 1970s. \citet{Hooper1973} and \citet{Kahn1976} argue that the syllable is useful for defining wordlikeness generalizations.\footnote{\citet{Steriade1999} and \citet{Blevins2003}, however, argue that a number of phonotactic generalizations previously stated in syllabic terms can be reanalyzed without making reference to syllables.} \citeauthor{Hooper1973} argues, for instance, that [bn], impossible as an English onset, is unobjectionable as a syllable contact cluster in nonce words like \emph{stabnik} (or in names like \emph{Abner}), and that this demonstrates the superiority of syllable-based wordlikeness generalizations. This already signals further trouble for alternative accounts which focus on underlying forms. Syllabification may span morphs, is generally predictable, and is universally non-contrastive, and as a consequence, few posit in to be present in underlying representations \citep[though see, e.g.,][]{Vaux2003}. \citeauthor{Hooper1973} also points to loanword adaptations which produce native syllable structure \citep[e.g.,][]{Carlisle1991} as evidence that syllabification is part of the phonological computation. Further enrichments to the theory are provided by the autosegmental theory of the syllable \citep{McCarthy1979b}, which envisions the syllable as an articulated tree structure \citep[as first envisioned by][]{Pike1947a}, and theories like prosodic licensing \citep{Ito1989a}, in which syllabification triggers phonological repairs.
%Despite the considerable attention given to the proposals of \emph{SPE} in the wake of that book's publication in 1968, the \emph{SPE} wordlikeness model has received almost no further attention in the literature. At the risk of explaining what might be no more than an accidental gap in the literature, the novel aspects of \emph{SPE} model---gradience derived from similarity to existing lexical entries---may have been overshadowed by the many other contentious proposals in \emph{SPE}, and particularly by compelling arguments against the assumption that wordlikeness contrasts derive solely from properties of underlying forms. \citet{Shibatani1973} observes that there are some generalizations about surface forms which give rise to wordlikeness contrasts, but cannot be stated as constraints on underlying forms. An example from German is shown in (\ref{fd}) below.

%\emph{K}[uːx]\emph{en}
%\emph{K}[uːç]\emph{en}
%\footnote{Examples like \emph{Mas}[oːç]\emph{ist} `masochist', \emph{Eun}[uːç]\emph{ismus} `eunichism', first noted by \citet{Merchant1994}, suggest that this should also be restricted to assimilation within the same foot \citep[226f.]{Jensen2000}.}

%Whether phonological computations or representations themselves are graded \citep[e.g.,][]{Lakoff1973} is besides the point, as metalinguistic judgements are behaviors, not mental states; they can no more be compared than can ``fear'' and ``flight response''.

%\begin{example}[German \emph{ich}- and \emph{ach-laut}]
%\begin{tabular}{l l l}
%a. & [buːx]   & `book'           \\
%   & [tɔxtər] & `daughter'       \\
%   & [naxt]   & `night'          \\
%b. & [siçt]   & `view'           \\
%   & [ʃpeçt]  & `woodpecker'     \\
%   & [ɡərʏçt] & `rumor'          \\
%   & [knøçəl] & `ankle, knuckle' \\
%   & [flɛçə]  & `surface'        \\
%\end{tabular}
%\end{example}
%
%\emph{Umlaut}, the fronting (and raising) of back vowels in certain morphological contexts, produces the front variant of the dorsal fricative; e.g., [lɔx]-[løçər] `hole-holes', \emph{B}[uːx]-\emph{B}[yːç]\emph{er} `book-books',
% 
%Throughout the term \emph{lexicon} is used in a specific sense of the set underlying representations in some language; this is not meant to imply a position on the possibility that larger, composite linguistic representations are also stored in lexical memory (thought see \citealt{LignosInPressa} for some recent experimental evidence bearing on this question).
%I reject the possibility that phonotactics do exist altogether.
