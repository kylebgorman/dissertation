\chapter{Static and derived phonotactic preferences in Turkish}

While \citet{Chomsky1965} argue that speakers internalize facts about the shapes of possible words in their language, it is certainly possible to imagine that speakers would not bother at all. \citet[][320]{Zimmer1969} claims these generalizations ``have no observable consequences in the course of the normal use of the language'' and a similar sentiment is echoed more recently by \citeauthor{PE}:

\begin{quote}
Even if we, as linguists, find some generalizations in our description of the lexicon, there is no reason to posit these generalizations as part of the speaker's knowledge of their language, since they are computationally inert and thus irrelevant to the input-output mapping that the grammar is responsible for. \citep[][18]{PE}
\end{quote}

This represents a principled null hypothesis, but it can be quickly dismissed. 
This might represent  


This would be a principled null hypothesis, but it can be quickly rejected in light of the indisputable contrast between non-words like \emph{blick} and \emph{bnick} noted by \citeauthor{Chomsky1965}. It is also possible to show speakers access this knowledge not just in metalinguist tasks like wordlikeness judgements, but also in such naturalistic and quotidien tasks as segmenting spontaneous speech into words \citep{Mattys2001b,McQueen1998}.

\citeauthor{Chomsky1965} were not the first to take an interest in characterizing the lexicon, which was also a major concern for the scholars of the Copenhagen school. In two sophisticated studies dealing with possible and impossible onsets in Georgian, \citet{Fischer-Jorgensen1952} and \citet{Vogt1954} recognize the inherent difficulty of distinguishing between structural and accidental phonotactic gaps in the lexicon on the basis of statistical criteria alone. What distinguishes the early generativist approach to phonotactics is the mentalistic concern with what generalizations speakers have internalized. By grounding the study of phonotactics in the minds of speakers, it is possible to give a precise definition to the previously vague notions; an accidental phonotactic gap is simply one which speakers do not attend to or internalize. Yet few studies of phonotactics are explicitly concerned with what speakers have internalized, despite the increasing sophistication of the phonological representations used. In a seminal paper, \citet{McCarthy1988} uses a simple statistical technique, a chi-square test, in an attempt to validate a generalization about Arabic root consonant co-occurrence faces originally due to \citet{Greenberg1950}. While this particular generalization was ultimately also evaluated using a wordlikeness task \citep{Frisch2004}, there is now an extensive literature in which phonotactic knowledge is inferred directly from lexical statistics, with no psycholinguistic component of any kind 
\citep[e.g.,][]{Anttila2008,
Berkley1994b,Berkley1994a,Berkley2000,
Brown2010,
Buckley1997,
Coetzee2008a,
Dmitrieva2008a,Dmitrieva2008b,
Elmedlaoui1995,
Graff2011,
Padgett1992,Padgett1995,
MacEachern1999,
Kinney2005,
Kawahara2006,
Martin2007,Martin2011,
Mester1988,
Miller-Ockhuizen2003,
Pozdniakov2007,
Yip1989}.
%find amanda miller-ockhuizen on ju|'hoansi
The (usually tacit) assumption of most of this work is made explicit by \citet{Brown2010} in a study of consonant cluster co-occurrence restrictions in Gitksan: he claims that \emph{any} statistically significant pattern in the lexicon is one that is internalized by speakers. 

\begin{quote}
\ldots{}the patterns outlined above are statististically significant. Given this, it stands that these sound patterns should be explained by some linguistic mechanism. \citep[][48]{Brown2010}
\end{quote}

In the loanword adaptation literature, it has recently been recognized that not all statistically reliable generalizations about the native lexicon take part in nativization of borrowings \citep[e.g.,][]{Ito1995a,Ito1995b,Ussishkin2003}. In this chapter, data from Turkish is used to show that a widely reported, statistically reliable phonotactic generalization (\textsc{Labial Attraction}) has no effect on wordlikeness judgements. This undermines the widespread use of statistic criteria to determine which of the vast number of the overlapping, competing phonotactic generalizations are internalized by speakers and which are ignored. Comparison of \textsc{Labial Attraction} to related constraints in Turkish suggests a novel necessary condition for identifying psychologically real phonotactic constraints: namely, speakers attend to phonotactic generalizations only when they are derived from phonological processes (including syllabification).

There are a number of precedents for 

This hypothesis was anticipated by \cite[][283]{Anderson1974}.

\begin{quote}
It would be a result of the greatest interest if it were to turn out that every intramorphemic regularity was necessarily reflected in an alternation, and thus in a rule; but this position cannot seriously be maintained. \citep[][283]{Anderson1974}
\end{quote}



This hypothesis is not ahdered to by early research on sequence structure constraints \citep[e.g.,][]{SPR,Chomsky1965,SPE,Stanley1967}. 


Work in this vein sought to use sequence structure constraints to compress the lexicon as much as possible 


This work has a second interest, namely in ``compressing'' the lexicon as much as possible, 

It is important to know that this wor It is important to know that this wor It is important to know that this wor It is important to know that this work

% motivations for the development of a theory of static co-occurrence restrictions as developed by \citet{SPR}, \citet{Stanley1967} and others is to account for speakers' abilities to distinguish between accidental and structural gaps. \citet{Chomsky1965} propose that a theory of phonology must account for speakers' abilities to distinguish between accidental and structural gaps in their language, citing a contrast between \emph{blick}, which is judged to be a possible but unattested word of English, and \emph{bnick}, which is judged to be impossible. \emph{SPE} (p.~3810f.) places this contrast in a system of a system of sequence structure rules that also are used to eliminate lexical redundancy. In the phonological theory of the era, there is no process that could generate the \emph{blick} $\sim$ \emph{bnick} contrast or numerous other lexical regularities 

This condition (henceforth, the \textsc{derived} condition) 
On a historical note, this hypothesis is more constrained than what was proposed
such constraints must be derived from phonological processes in the language in question
synchronically motivated phonological processes such as alternation or syllabification.

%It is important to note that the nature of the \emph{blick}/\emph{bnick} contrast is no longer a mystery. 
%Generative phonological theory at the time of \emph{SPE} lacked the representational vocabulary to directly distinguish between word-initial /bn/, which is disallowed, and /bn/ as a licit syllable contact cluster (e.g., \emph{o}[b.n]\emph{oxious}), a point first made by \citep{Hooper1973}. It is not the introduction of prosodic primitives into generative phonology that explains the contrast between \emph{blick} and \emph{bnick}, though, but rather the principle that surface forms must be syllabified \citep[e.g.,][63f.]{Kiparsky1982b} which \emph{bnick} fails. If, as is standardly assumed, phonology repairs unsyllabifiable outputs \citep[e.g.][]{Ito1989a,Noske1992}, then \emph{bnick} is an impossible output, doomed to be modified (perhaps to \emph{nick}; \citealt[][19f.]{Wolf2009}), and /bnɪk/ is an impossible UR by Stampean occultation. 

%\footnote{I note a potential problem with this account: \citet{Davidson2005,Davidson2006a} argue that this ``excrescent'' schwa is distinct from the both in acoustics and articulation from ``lexical schwa'': English speakers' [z${^\textrm{ə}}$k] is more similar to \emph{scum} than \emph{succumb}.}

%This sequence of arguments has already been critiqued in Chapter \ref{1}, where it was shown that prosodic licensing and Stampean occultation places this contrast in core phonology, and that the principles by which speakers recognize impossible words must be stated on surface representations, not underlying representations. Further, \citet[][528f.]{Halle1975} rejects his own principle that lexical entries should be free of all redundancies (see \citealt[][201]{Reiss2003a} and \citealt{Vaux2003} for further discussion). This rejected principle of a redundancy-free lexicon is one of the two principles that motivates Morpheme Structure Constraints; the other is the subject of this and the following chapter.

%\citet{Lees1966b,Lees1966a}

%\citet{Berent2007a}
%\citet{Berent2007b}
%\citet{Berent2008a}
% 598: Highly marked inputs are repaired in perception to abide by the grammatical restrictions of the language
% Exps 1-2: syllable count task (misperception due to linguistic experience; "disyllables were more likely to be perceived as monosyllables if their monosyllabic counterparts had a less marked cluster" in both Russian and English, not what OT predicts)
% Exps 3-4: AB task
% Exps 5-6: priming effect (non-epenthetic and epenethetic things represented the same)
% Berent and Lennertz 2007: (640: If the high rate of monosyllabic errors to unmarked onsets is only due to phonetic failures to encode the input, then it is puzzling why the same onsets also yield the highest rate of accurate response.)
% Berent et al. 2008a:
%\citet{Peperkamp2007}
% Peperkamp 2007: during one stage of processing phonology is undone with help from the lexicon...it only takes licit inputs and is irrelevant (p. 633)
%\citet{Dupoux1999}
% Dupoux et al. 1999: Japanese speakers still can't distinguish between ebzo and ebuzo in an ABX task using the SAME a or b.
%Elliot Moreton (p.c.)

%\noindent A nuanced version of this alternative hypothesis would recognize that there may be other necessary conditions. Any explicit theory of static constraints will have to impose formal restrictions on possible co-occurrence restrictions. The set of possible constraints must respect results in formal learning theory, and may also be constrained by markedness considerations, for example.

%A further widespread assumption is that static co-occurrence restrictions validated statistically will be reflected by wordlikeness judgements. This assumption is so deeply entrenched that a number of recent studies of phonotactic learning forgo wordlikeness judgements altogether and instead model the lexical statistics themselves (e.g., \citealt{Coetzee2008a} and \citealt{Anttila2008} on Muna and Arabic, \citealt[][385]{Hayes2008a} on Shona and Wargamay, \citealt{Brown2010} on Gitksan). 

%Since the null hypothesis does not countenance statistical significance as evidence for the synchronic reality of a static lexical pattern, such patterns do not adjudicate between the null hypothesis and the alternative, and another source of evidence is needed. In the remainder of Part I, I turn to data from wordlikeness judgement tasks for these purposes. 

%In a wordlikeness task, a speaker is presented with nonce words and asked to report such intuitions about the ``possibility'' of a word. 
%The alternative hypothesis, then, is that a statistically reliable co-occurrence restriction will be reflected in native speakers' wordlikeness judgements. 
%The data below, drawn from Turkish, shows that 

%This chapter takes the argument one step farther by showing a contrast between static and derived constraints on URs in Turkish. Whereas the derived constraint has a robust effect on native speakers' wordlikeness judgements, the static constraint, which is equally statistically valid, has no effect on these judgements. Anticipating the conclusion, the simplest explanation for the non-effect of this static constraint on wordlikeness judgements is that speakers do not internalize static co-occurrence restrictions at all. 

%Classical Arabic adjectives often have stative verbs in which the root is imposed onto the template CaCuCa: 

\section{Turkish vowel sequence structure constraints}

%\citet{Lees1966a,Lees1966b} 
\citet{Lees1966b,Lees1966a} 
proposes three constraints on Turkish vowel sequences which have been the focus of many subsequent studies. In this section, these constraints are formalized, and where possible, related to phonological alternations in Turkish and to evidence that speakers have internalized these restrictions. The following feature specification for the Turkish vowels is assumed throughout.

\begin{example}[Turkish vowel features]
\begin{tabular}{c c c c c}
                       & \multicolumn{2}{c}{[$-$\textsc{Back}]} & \multicolumn{2}{c}{[$+$\textsc{Back}]} \\
                       & [$-$\textsc{Round}] & [$+$\textsc{Round}] & [$-$\textsc{Round}] & [$+$\textsc{Round}] \\ %\midrule 
\buf[$+$\textsc{High}] & {i} & {ü} [y] & {ı} [ɯ] & {u} \\
\buf[$-$\textsc{High}] & {e} & {ö} [ø] & {a} [ɑ] & {o} \\
\end{tabular}
\end{example}

\subsection{Backness harmony}

\citeauthor{Lees1966b} (\citeyear[][35]{Lees1966b}, \citeyear[][284]{Lees1966a}) models the Turkish vowel harmony systems by means of three feature spreading rules. Backness harmony is the the most general of these processes.  

\subsubsection{Phonological description}

\textsc{Backness Harmony} spreads the backness specification of a vowel rightward ignoring any intervening consonants.

\begin{example}[\textsc{Backness Harmony}]
$\begin{bmatrix} +\textsc{Vocalic} \end{bmatrix}~\goesto~\begin{bmatrix} =\textsc{Back} \end{bmatrix}~\big /~\begin{bmatrix} =\textsc{Back} \end{bmatrix}~\textrm{C}_0~\gap{}$
\end{example}

%\citep[after][229]{Clements1982}: \\
%\xymatrix@R=24pt@C=24pt{
%\txt{V}                                        & \txt{C$_0$~~~~~} & \txt{V} & \txt{(condition: rightward application)} \\
%\txt{[α \textsc{Back}]}\ar@{-}[u]\ar@{--}[urr] \\
%}

This rule accounts for the tendency of polysyllablic roots to either contain all [$+$\textsc{Back}] or all [$-$\textsc{Back}];
%, as can be seen in the nominative singulars in (\ref{turknom}b); 
this tendency is quantified in \S\ref{lexstats} below. \textsc{Backness Harmony} also accounts for the allomorphs of several inflectional suffixes. For instance, the nominative plural (nom.pl.) suffix is \emph{-ler}, when the final root vowel is [$-$\textsc{Back}], and \emph{-lar} when it is [$+$\textsc{Back}]. 

\begin{example}[The Turkish nominative] \label{turknom}
\begin{tabular}{l l l l@{ }l}
   & \emph{nom.sg.} & \emph{nom.pl.} \\
a. & {ip}           & {ipler}    & `rope'        & \citep[][216]{Clements1982} \\
   & {el}           & {eller}    & `hand'    \\
   & {köy}          & {köyler}   & `village' \\
   & {yüz}          & {yüzler}   & `face'    \\
   & {kız}          & {kızlar}   & `girl'    \\
   & {sap}          & {saplar}   & `stalk'   \\
   & {son}          & {sonlar}   & `end'     \\
   & {pul}          & {pullar}   & `stamp'   \\
b. & {neden}        & {nedenler} & `reason'       & (TELL) \\ % front
   & {kiler}        & {kilerler} & `pantry'       \\ % front
   & {pelür}        & {pelürler} & `tissue paper' \\ % back
   & {boğaz}        & {boğazlar} & `throat'       \\ % back
   & {sapık}        & {sapıklar} & `pervert'      \\ % back
\end{tabular}
\end{example}

There are a few complications, however. First, not all polysyllabic roots conform to \textsc{Backness Harmony}. In this case, as shown in (\ref{turkexcept}a), suffix vowels generally harmonize with the final root vowel. There is also a very small class of nouns, shown in (\ref{turkexcept}b), which take \emph{-ler} despite the fact that their final root vowel is [$+$\textsc{Back}].

\begin{example}[Exceptional Turkish nominatives] \label{turkexcept}
\begin{tabular}{l l l l@{ }l}
   & \emph{nom.sg.} & \emph{nom.pl.} \\
a. & {mezar}        & {mezarlar} & `grave' & \citep{TELL} \\
   & {model}        & {modeller} & `model' \\
   & {silah}        & {silahlar} & `weapon'     \\
   & {memur}        & {memurlar} & `bureaucrat' \\
   & {sabun}        & {sabunlar} & `soap'       \\
b. & {saat}         & {saatler}  & `hour, clock' \\
   & {harf}         & {harfler}  & `(alphabetic) letter' \\ %& \citep{Goksel2005}
   & {etol}         & {etoller}  & `fur stole' \\
\end{tabular}
\end{example}

While it is uncontroversial that the disharmonic suffixes of (\ref{turkexcept}b) are no more than very sporadic exceptions to \textsc{Backness Harmony}, root disharmony has ben the subject of much debate. As disharmonic roots still trigger suffix harmony, \citet[][212, 289]{Anderson1974} and \citet{Iverson1978} propose to separate suffix harmony, an alternation, from a sequence structure constraint governing root harmony.

The disadvantage of this account is that it introduces a sort of ``duplication'' (in the sense of \citealt{Kenstowicz1977}) of sequence structure and phonological generalizations, differning only in their patterns of exceptionality.
%\citep{Kisseberth1970b} (see also \citealp[][401]{Stanley1967} and \emph{SPE}:382)
However, \citet[][197f.]{Zonneveld1978} observes that the theory of exceptionality proposed in \emph{SPE}
% (p.~374f.)
can account for suffix harmony in disharmonic roots. \citeauthor{SPE} assume that the specification of the target (i.e., the segment or segments to be changed) of a rule \emph{R} must be marked [$+$\emph{R}] by convention. A root or affix which fails to undergo \emph{R} despite otherwise matching the structural description is simply said to be marked [$-$\emph{R}]. In other words, a form is never truly an ``exception'': it simply fails to match an extended structural description including the rule feature. If disharmonic roots are marked [$-$\textsc{Backness Harmony}], then the final vowel of disharmonic roots will still trigger \textsc{Backness Harmony}, since the [$-$\textsc{Backness Harmony}] root is no longer the target but rather the trigger, which is not subject to the [$+$\textsc{Backness Harmony}] requirement.\footnote{The definition of ``target of a rule'' is not obvious in autosegmental rules, but any useful formalization of this notion should, for example, identify a harmonic suffix vowel as the target of \textsc{Backness Harmony}.} With this in place, \textsc{Backness Harmony} is sufficient to account for both root and suffix harmony.

%Once additional source of evidence on root (dis)harmony is inconclusive. There is a small class of bisyllabic words in which the second vowel, always [$+$\textsc{High}, $-$\textsc{Back}], alternates with zero. 

%\ex High-vowel/zero alternations \citep[][243]{Clements1982}: \\
%\begin{tabular}{l l l l}
%   & nom.sg. & gen.sg. \\
%a. & fikir   & fikri  & `idea' \\
%   & hüküm   & hükmün & `judgement' \\
%%  & filim   & filmi & `film' & \citep[][178]{Inkelas2001} \\
%b. & vakit   & vaktin & `time' \\
%   & rahim   & rahmin & `womb' \\
%\end{tabular} \xe
%
%\noindent
%It is possible that \textsc{Backness Harmony} might produce a fluctuating \emph{ı} after root \emph{a}, but this does not obtain (\lastx b). However, this might simply indicate that the fluctuating vowel is epenthetic and that harmony applies before epenthesis (see \citealt{Clements1982} for both sides of this argument), making it less than a counterexample. 

Before moving on, it is necessary to dispense with an alternative analysis originally proposed by \citet{Clements1982} and further developed by \citet{Inkelas1997}. Root vowels exhibit a robust contrast for backness (e.g., \emph{kül} `ash' vs.  \emph{kul} `servant', \emph{kepek} `bran' vs. \emph{kapak} `lid'), whereas harmonic roots are those in which the backness of any remaining vowels is predictable. \citeauthor{Clements1982} propose that these vowels are underspecified for backness, whereas the non-initial vowels of disharmonic roots are fully specified. This is schematized below.

%This is exemplified below in (\ref{spec}).
%(e.g., \emph{deve} `camel' vs. \emph{deva} `medicine', \emph{sene} `year' vs. \emph{sena} `praise'). 
%There is some further evidence that individual vowels may differ in specification for this feature even within individual roots or affixes. For instance, the present continuous suffix has harmony-determined allomorphs \emph{-iyor}, \emph{-üyor}, \emph{-ıyor}, \emph{-uyor}, but the \emph{o} of the suffix is invariant. A similar situation might obtain in Turkish roots. 

\begin{example}[Autosegmental underspecification in harmonic roots (after \citealp{Clements1982})] \label{spec}
\xymatrix@R=24pt@C=24pt{
\txt{a.} & \txt{harmonic root:~~~~} & \txt{C} & \txt{V} & \txt{C} & \txt{V} & \txt{\ldots} \\
&   &    & \txt{[$-$\textsc{Back}]}\ar@{-}[u]\ar@{--}[urr] \\
\txt{b.} & \txt{disharmonic root:} & \txt{C} & \txt{V} & \txt{C} & \txt{V} & \txt{\ldots} \\
    &    &         & \txt{[$-$\textsc{Back}]}\ar@{-}[u] & & \txt{[$+$\textsc{Back}]}\ar@{-}[u]
}
\end{example}

Harmonizing suffix vowels will also be underspecified for backness.

Two additional details are needed to complete this analysis. First, there must be some kind of constraint which prevents speakers from positing redundant backness specifications for non-initial vowels in harmonic roots, a constraint on identical adjacent specifications in underlying representations, the lexical ``obligatory contour principle'' \citep[][OCP]{Leben1973}, on surface tonal representations \citep{Goldsmith1976}, or both (\citealp{Leben1978}, \citealp{McCarthy1986}; see \citealp{Odden1986,Odden1988} for criticism). Second, \textsc{Backness Harmony} needs to be prevented from overwriting the [$+$\textsc{Back}] specification of disharmonic roots, one option being the use of a structure preservation condition \citep{Kiparsky1985}. However, any condition which prevents \textsc{Backness Harmony} from overwriting underlying backness specifications will reintroduce the duplication of sequence structure and phonological generalizations; under this analysis, disharmonic roots are no longer exceptional, despite considerable evidence (reviewed below) that they are formally marked in Turkish.\footnote{On the other hand, if one interprets ``harmony'', that is, a single backness specification per root, as the learner's default (as \citealp{Odden1986} proposes for the tonal obligatory contour principle), this account is a mere notational variant of the rule exceptionality account.}
%\footnote{Kie Zuraw (p.c.) suggests that the markedness of disharmonic roots might be attributed to the presence of more structure, namely, multiple specifications for backness.}
I reject the underspecification analysis on these grounds. Rather, I assume a unitary phonological rule of \textsc{Backness Harmony} and assign [$-$\textsc{Backness Harmony}] to disharmonic roots.

\subsubsection{External evidence}
\label{backharmexternal}

While suffix harmony can be inferred from alternations, no evidence has yet been presented to show that speakers internalize in any way tendency for roots to exhibit backness harmony. 

%\begin{quote}
%\ldots{}these generalizations\ldots{}have no observable consequences in the course of the normal use of the language. \citep[][320]{Zimmer1969}
%\end{quote}

Indeed, \citeauthor{PE} adopt the null hypothesis that speakers do not internalize any generalizations about possible or likely sequences in underlying representations.

\begin{quote}
Even if we, as linguists, find some generalizations in our description of the lexicon, there is no reason to posit these generalizations as part of the speaker's knowledge of their language, since they are computationally inert and thus irrelevant to the input-output mapping that the grammar is responsible for. \citep[][18]{PE}
\end{quote}

Several independent pieces of external evidence argue that this is not the case for root harmony in Turkish. 
The adaptation of non-native word-initial onset clusters, discussed by \citet{Clements1982} and \citet{Kaun1999}, is one such case.\footnote{Thanks to Kie Zuraw for bringing this data to my attention.} While some speakers are said to be capable of pronouncing these non-native clusters, in normal speech the cluster is resolved by epenthesis of a [$+$\textsc{High}] vowel. In most cases, the epenthetic vowel matches the following root vowel for backness, suggesting that adapt foreign words in a way that conforms to the root harmony generalization.\footnote{At the same time, this shows that adaptation is not simply applying rules, like \textsc{Backness Harmony}, to foreign forms. Rather, foreign forms are restructured as non-exceptional Turkish lexical items.}

%the epenthetic vowel is always [$+$\textsc{High}],
%and in general, it conforms to \textsc{Backness Harmony}.
%For reasons that are 
%and when the first consonant of the cluster is non-dorsal, it shows backness harmony

\begin{example}[Adaptation of initial foreign clusters (\citealp{Clements1982}:247)] 
\begin{tabular}{l l l l l l}
a. & {spiker}  & \alt{} & {sipiker}  & `announcer' \\
   & {fren}    & \alt{} & {firen}    & `break'     \\
b. & {brom}    & \alt{} & {burom}    & `bromide'   \\
   & {prusya}  & \alt{} & {purusya}  & `Prussia'   \\
%b. & grip    & \alt{} & gırip    & `grippe'    \\ % unexpectedly back
%   & kredi   & \alt{} & kıredi   & `credit'    \\
%   & trablus & \alt{} & tırablus & `Tripoli'   \\
%b. & kral    & \alt{} & kıral    & `king'      \\
%   & grup    & \alt{} & gurup    & `group'     \\
\end{tabular}
\end{example}

Similar evidence comes from a language game discussed by \citet{Harrison2001}.\footnote{Thanks to Bert Vaux for bringing this study to my attention.} This game is not indigenous to Turkish, but is identical to a morphological process in the related language Tuvan, in which it conveys a sense of ``vagueness or jocularity'', and \citeauthor{Harrison2001} report that it can be quickly taught to Turkish speakers. The game consists of reduplication of the base and replacement of the first vowel in the reduplicant with the [$+$\textsc{Back}] vowels \emph{a} or \emph{u}. Both in the native Tuvan process and in the Turkish game, the second vowel of the reduplicant (shown in braces below) is affected by this process: when the base is harmonically [$-$\textsc{Back}], the insertion of a [$+$\textsc{Back}] results in what \citeauthor{Harrison2001} call ``reharmonization'', shown in  (\ref{redupgame}a).

\begin{example}[Turkish reduplication game (\citealp{Harrison2001}:231)] \label{redupgame}
\begin{tabular}{l l l l}
a. & {kibrit} & {kibrit}-\{{kabrıt}\} & `match'    \\
   & {bütün}  & {bütün}-\{{batın}\}   & `whole'    \\
b. & {mali}   & {mali}-\{{muli}\}     & `Mali'     \\
   & {butik}  & {butik}-\{{batik}\}   & `boutique' \\
\end{tabular}
\end{example}

\noindent \citeauthor{Harrison2001} analyze this in terms of the \citeauthor{Clements1982} underspecification analysis, but it is equally consistent with full specification. Reharmonization is simply the spread of the backness specification of the \emph{a} or \emph{u} in the reduplicant. On the other hand, (\ref{redupgame}b) shows that this does not obtain for disharmonic roots; presumably, the [$-$\textsc{Backness Harmony}] exception feature is copied under reduplication as well. Similar results obtain both in Tuvan and in an unrelated Finnish language game \citep{Campbell1986}.

\label{harmonyandsegmentation} A number of studies on Finnish suggest that speakers of that language, which has a vowel harmony system similar to that of Turkish, make use of (dis)harmony to processing running speech.\footnote{Thanks to Charles Yang for bringing these studies to my attention.} \citet{Suomi1997} and \citet{Vroomen1998} generate nonce trisyllabic words by adding a monosyllabic pseudo-prefix to real and nonce disyllabic words, all of which conform to backness harmony. These stimuli are auditorily presented to subjects who are asked to press a button when the nonce trisyllable ends with a target nonce disyllable, or a real disyllabic word. Speakers are quicker to press the button when the prefix and disyllabic word have different backness specifications. These results suggest that speakers are attuned to the fact that disharmonic transisitions are good predictors of word boundaries. If speakers have also internalized the converse generalization, that harmonic transistions are more likely to be root-internal, then there is additional evidence that harmony is active not just in Finnish affix alternations but also in roots. 

\citet{Kabak2010} report that Turkish \textsc{Backness Harmony} has the same effect in word-spotting tasks as it does in Finnish: speakers are quicker and more accurate at the task of spotting the nonce target word \emph{pavo} when preceded by the pseudo-prefix \emph{gölü-}, a disharmonic transistion, than when it is preceded by the pseudo-prefix \emph{golu-}, a harmonic transition. \citeauthor{Kabak2010} find that effect of harmony does not obtain for speakers of French, a language which lacks vowel harmony. As in Finnish, the results imply speakers have internalized the predominance of root-internal harmony.

It seems that the root-internal harmony bias is in fact learned by Turkish speakers very early. The pseudoword spotting experiment has been adapted for 9-month-old Turkish infants by \citet{Kampen2008}. Infants are familiarized with harmonic disyllabic pseudowords bearing a pseudo-prefix, which may be harmonic or disharmonic. At test time, the infants are played the disyllabic pseudowords in isolation using the head turn preference paradigm. Infants show a preference to listen to those pseudowords which were familiarized with a disharmonic pseudo-prefix over those which were familiarized with a harmonic pseudo-prefix. This preference is not observed in 9-month-old infants learning German, which also lacks vowel harmony. Similarly, \citeauthor{Kampen2008} report that Turkish 6-month-old infants prefer to listen to harmonic pseudowords such as \emph{paroz} over disharmonic pseudowords like \emph{nelok}, but German 6-month-old infants show no such preference.

\subsection{Roundness Harmony}

Roundness harmony is a closely related process with a somewhat narrower scope. 

\subsubsection{Phonological description}

Roundness harmony is quite similar to \textsc{Backness Harmony}, with the additional stipulation that the target must be [$+$\textsc{High}]. 

%\begin{example}[\textsc{Roundness Harmony}]
%\xymatrix@R=24pt@C=24pt{
%\txt{[α \textsc{Round}]}\ar@{-}[d]\ar@{--}[drr] &             & \txt{[$+$\textsc{High}]} \\
%\txt{V}                                         & \txt{C$_0$} & \txt{V}\ar@{-}[u] \\
%}
%\end{example}

This rule, in concert with \textsc{Backness Harmony}, accounts for the forms of the dative singular (dat.sg.) and genitive singular (gen.sg.), among other suffixes.

\begin{example}[Turkish nominal suffix allomorphy]
\begin{tabular}{l l l l l l@{ }l}
   & \emph{nom.sg.} & \emph{dat.sg.} & \emph{gen.sg.}  \\
a. & {ip}           & {ipi}          & {ipin}         & `rope' & \citep[][216]{Clements1982} \\
   & {el}           & {eli}          & {elin}         & `hand'    \\
   & {kız}          & {kızı}         & {kızın}        & `girl'    \\
   & {sap}          & {sapı}         & {sapın}        & `stalk'   \\
   & {yüz}          & {yüzü}         & {yüzün}        & `face'    \\
   & {köy}          & {köyü}         & {köyün}        & `village' \\
   & {pul}          & {pulu}         & {pulun}        & `stamp'   \\
   & {son}          & {sonu}         & {sonun}        & `end'     \\
b. & {boğaz}        & {boğazı}       & {boğazın}      & `throat'  & (TELL) \\
   & {pelür}        & {pelürü}       & {pelürün}      & `tissue paper' \\
   & {döviz}        & {dövizi}       & {dövizin}      & `currency' \\
   & {yamuk}        & {yamuğu}       & {yamuğun}      & `trapezoid' \\
   & {ümit}         & {ümiti}        & {ümitin}       & `hope'     \\
\end{tabular}
\end{example}

\noindent As was the case for \textsc{Backness Harmony}, disharmonic roots are found, and the final vowel of disharmonic roots triggesr suffix harmony.

\subsubsection{External evidence}

Though few studies have directly sought external evidence for speaker's awareness of root \textsc{Roundness Harmony}, two of the findings reviewed in \S\ref{backharmexternal} above speak in support of this claim. First, the epenthetic vowel in non-native word-initial onsets
like \emph{trablus} \alt{} \emph{tırablus} `Tripoli' conform also to \textsc{Roundness Harmony}.
%\citealp[][247]{Clements1982}). 
Second, \textsc{Roundness Harmony} participates in reharmonization in \citeauthor{Harrison2001}'s language game: the reduplicated \emph{bütün-batın} exhibits replacement of the second \emph{ü}, a [$+$\textsc{Round}] vowel, with [$-$\textsc{Round}] \emph{ı}.

\subsection{Labial attraction}

%\begin{quote}
%\ldots{}these generalizations\ldots{}have no observable consequences in the course of the normal use of the language. \citep[][320]{Zimmer1969}
%\end{quote}

\subsubsection{Phonological description}

\citet[][36]{Lees1966a} describes \textsc{Labial Attraction} as a process by which ``a high, short harmonic vowel is rounded in the second syllable of a disyllabic word whose first vowel is /a/, and whose medial consonant cluster contains a labial /p, b, m, v/, and then it is de-harmonified''. This description is transalted into an autosegmental rule below.

%\begin{example}[\textsc{Labial Attraction}]
%%(after \citealt[][286]{Lees1966b}, \citealt[][171]{Inkelas2001}): 
%\xymatrix@R=24pt@C=8pt{
%\txt{[$-$\textsc{Round}]} &                                         & \txt{[$-$\textsc{High}]} & \txt{[$+$\textsc{Labial}]}    & \txt{[$+$\textsc{High}]}\ar@{-}[dr] &         & \txt{[$+$\textsc{Round}]}\ar@{--}[dl] \\
%%                         & \txt{V}\ar@{-}[ul]\ar@{-}[ur]\ar@{-}[d] &                           & \txt{C}\ar@{-}[u] &                                      & \txt{V} & \\
%                         & \txt{V}\ar@{-}[ul]\ar@{-}[ur]\ar@{-}[d] &                           & \txt{C$_0$ C C$_0$}\ar@{-}[u] &                                      & \txt{V} & \\
%                         & \txt{[$+$\textsc{Back}]}\ar@{-}[urrrr]
%}
%\end{example}

The formalization of this rule is naturally complex, and perhaps obscures the fact that \textsc{Labial Attraction} generates exceptions to \textsc{Roundness Harmony}, producing \emph{aCu} sequences (e.g., \emph{çapul} `raid', \emph{sabur} `patient', \emph{şaful} `wooden honey tub', \emph{avuç} `palm of hand', \emph{samur} `sable'; \citealp[285]{Lees1966b}) rather than the expected \emph{aCı}. \citet[286]{Lees1966b} and \citet[311]{Zimmer1969} note the existence of exceptions (e.g., \emph{tavır} `mode') but both agree that they are surprisingly rare. 

\citet[225]{Clements1982} and \citet{Inkelas2001}

There is reason to believe that \textsc{Labial Attraction} is at best a sequence structure constraint as it never applies in derived environments. If, contrary to fact, there was a \textsc{Labial Attraction} alternation, then one would expect, for example, that the gen.sg. of \emph{sap} `stalk' would be *\emph{sapun} instead of the observed \emph{sapın}.

\citet{Lees1966a}
\citet{Zimmer1969}

\begin{quote}
\ldots{}decisive evidence against a rule of Labial Attraction is the existence of a further, much larger set of roots containing /\ldots~aCu~\ldots/ sequences in which the intervening consonant or consonant cluster does not contain a labial\ldots We conclude that there is no systematic restriction on the set of consonants that may occur medially in roots of the form /\ldots~aCu~\ldots/. \citep[][225]{Clements1982}
\end{quote}

\noindent
\citeauthor{Clements1982} appear to be suggesting that \textsc{Labial Attraction} implies that \emph{aTu}, where \emph{T} represents a non-labial consonant, should be infrequent, but this fact is not inconsistent with the formulation of the constraint by \citet{Lees1966a,Lees1966b} and \citet{Zimmer1969}; \emph{aTu} clusters do not meet \textsc{Labial Attraction}'s structural description. This is less than conclusive, since the fluctuating vowel alternation also fails to undergo harmony, and might just indicate that the flucutating vowel is epenthesized relatively late in the derivation; the frequency of \emph{aTu} provides only indirect information about \textsc{Labial Attraction} in that it provides a baseline estimate for just how frequent this disharmonic sequence of vowels is.

\citet{Inkelas1997}
\citet{Inkelas2001}

\begin{quote}
Lee's rule of \textsc{Labial Attraction}\ldots is not a real generalization about the Turkish lexicon. It is not true synchronically, either of native or nonnative items; nor, according to the historical and dialectical literature, does \textsc{Labial Attraction} appear to have been true at any stage going back as far as Old Turkic. \citep[][196]{Inkelas2001}
\end{quote}

zimmer: 319-320 
inkelas et al.: 196

\subsubsection{Psycholinguistic evidence}

To the author's knowledge, there are no psycholinguistic studies investigating \textsc{Labial Attraction} beyond the experimental results of \citet{Zimmer1969} reviewed in the next section. 

\section{Evaluation}

The results reveal a dissociation between the lexical statistics and wordlikeness judgements, a dissociation that finds a natural explanation from the principle of ...

There is some disagreement in the literature about just what \citet{Zimmer1969} finds regarding \textsc{Backness Harmony} and \textsc{Labial Attraction}.

\subsection{Lexical statistics} \label{lexstats}

This choice of statistical test is justified in Section \ref{stattech}.

\subsubsection{Backness harmony}

\subsubsection{Roundness harmony}

%\ex Lexical effects of \textsc{Roundness Harmony} \citep{TELL}: \vspace{6pt} \\
%\begin{tabular}{l r r r r r}
%\toprule              & 
%Corpus                & $p$-value \\
%\midrule
%Full TELL             &  
%Elicited TELL         & 
%TELL with etymologies & 
%\end{tabular}
%\xe

%\citet{Harrison2004} 73\% 
%of lexical types are harmonic (both back and round)

\subsubsection{Labial attraction}

\begin{quote}
Vowel labialization following labials is not a synchronic alternation in Turkish, yet it (unlike \textsc{Labial Attraction} per se) \emph{is} a statistically supported tendency worthy of further research. \citep[][196, emphasis in original]{Inkelas2001} 
\end{quote}

%\ex Lexical effects of \textsc{Labial Attraction} \citep[][186]{Inkelas2001}: \vspace{6pt} \\
%\begin{tabular}{l r r r r r}
%\toprule
%Corpus                & aPu & aPı & aTu & aTı   & $p$-value   \\ % & corpus size
%\midrule
%Full TELL             & 378 & 248 & 446 & 1,140 & 2.83\e{-44} \\ % & 31,236 \\
%Elicited TELL         & 152 & 265 & 101 & 1,839 & 9.84\e{-60} \\ % & 16,541 \\
%TELL with etymologies & 128 & 109 &  79 &   470 & 6.56\e{-32} \\ 
%\bottomrule
%\end{tabular}
%\xe

%Etymological subset of TELL
%Native    & Foreign   \\
%aBu & aBI & aBu & aBI \\
%12  & 11  & 84  & 28  \\
%p = 0.0417

\subsection{Wordlikeness results}

There is some disagreement in the literature about just what \citeauthor{Zimmer1969}'s study reveals about \textsc{Labial Attraction}.

zimmer
clements
\citet{Ito1993} 
%and \citet{NiChiosain1993} 
argue that the data shows that it holds of native words only


\citet{Inkelas2001} refute this point


\citet{Zuraw2000}:4

\citet{Inkelas2001} present a different variation on the original rule of \textsc{Labial Attraction}


\begin{quote}
Vowel labialization following labials is not a synchronic alternation in Turkish, yet it (unlike \textsc{Labial Attraction} per se) \emph{is} a statistically supported tendency worthy of further research. \citep[][196, emphasis in original]{Inkelas2001}
\end{quote}

\noindent \citeauthor{Inkelas2001} persistently refer to ``statistical support'', but in fact their study contains no statistics (that is, quantities calculated from a set of data) beyond counts and percentages. 



``These results support our view that there is no rule of Labial Attraction'' \citep[][225]{Clements1982}.

%since the opposing are all broader versions than the original formulation by \citeauthor{Lees1966a}.

\citet{Goodman1954}


\subsubsection{Backness harmony}


\begin{example}[Backness harmony wordlikeness forced choices (\citealp{Zimmer1969}:314)]
\begin{tabular}{l r l r}
\toprule
\multicolumn{2}{l}{harmonic} & \multicolumn{2}{l}{disharmonic} \\
\midrule
{pemez} & 30            & {pemaz} & 2  \\
{tipez} & 24            & {tipaz} & 8  \\ 
{terüz} & 19            & {teruz} & 13 \\ % roundness violator
{teriz} & 28            & {terız} & 3  \\
{tokaz} & 26            & {tokez} & 6  \\ % roundness violator
\bottomrule
\end{tabular}
\end{example}

$\gamma = 0.597$, $p = 5.2$\e{-29}

\subsubsection{Roundness harmony}

\begin{example}[Wordlikeness comparisons, roundness harmony]
\begin{tabular}{l r l r}
\toprule
\multicolumn{2}{l}{harmonic} & \multicolumn{2}{l}{disharmonic} \\
\midrule
{pörüz} & 32            & {pöriz} & 0  \\
{tatız} & 20            & {tatuz} & 12 \\
{tüpüz} & 31            & {tüpiz} & 1  \\
{takız} & 22            & {takuz} & 10 \\
\bottomrule
\end{tabular}
\end{example}

$\gamma = 0.641$, $p = 7.7$\e{-27}

%tamaz & 26 & tamoz & 6  \\
%putoz & 25 & putaz & 7  \\

\subsubsection{Labial harmony}

\begin{example}[Wordlikeness comparisons]
\begin{tabular}{l r l r}
\toprule
\multicolumn{2}{l}{{a}P{u}} & \multicolumn{2}{l}{{a}P{ı}} \\
\midrule
{pamuz} & 17       & {pamız} & 15 \\
{tafuz} & 21       & {tafız} & 11 \\
{tapuz} & 17       & {tapız} & 15 \\
{mavuz} & 16       & {mavız} & 16 \\
{tabuz} & 16       & {tabız} & 16 \\
\bottomrule
\end{tabular}
\end{example}

$\gamma = 0.087$, $p = 0.101$

\section{Conclusions}

\citet{Becker2011}

It is 
%further, even long-distance co-occurrence restrictions frequently are ``conjunctive'': for instance, \citet{Gallagher2009} report that in Chol, ejectives must be totally identical, that is to say, 
%\citet{Bakovic2005b} 

This has an ad hoc nature to it; 
There are two senses in which this objection is ad hoc. 
First, \citeauthor{Becker2011} appeal to no particular theory of the naturalness of processes or static constraints which excludes \textsc{Labial Attraction}. 
Secondly, this appears to be a minority view: \textsc{Labial Attraction} was considered a true generalization by early specialists
\citep[e.g.,][]{Lees1966a}, and despite \citeauthor{Zimmer1969}'s suggestive psycholinguistic results, reviewed above, it also been treated as a plausible constraint by later theorists \citep[e.g.,][]{NiChiosain1993,Ito1993,Ito1995a,Zuraw2000}.
Further, there is a real danger that if the label ``unnatural'' 
%Labial Attraction} 
describes an impossible structural change or structural description, that one will fail to account for earlier forms of Turkish or sound changes therein.

%Classical Arabic adjectives often have stative verbs in which the root is imposed onto the template CaCuCa:
Classical Arabic adjectives often have stative verbs in which the root melody is rewritten with \emph{a\ldots{}u}.\footnote{Thanks to Uri Horesh (p.c.) for assistance with this data.}

\begin{example}[Arabic stative verbs]
\begin{tabular}{l l l l}
\buf{}[kabiːr]  & `big'      & [kabura]  & `become big'       \\
\buf{}[ħasan]   & `handsome' & [ħasuna]  & `become beautiful' \\
\buf{}[dʒadiːb] & `barren'   & [dʒaduba] & `become dry'       \\
\end{tabular}
%b. & saʁiir & }`become small'     & (cf. \emph{saʁiːr} `small') \\
\end{example}

% FORMER 3.3.2: Diachronic factors 
 
% etymological issues
\citet{Inkelas2001}

\begin{example}[\textsc{Labial harmony} and etymology in TELL \citealp{Inkelas2001}:187]
\begin{tabular}{l r r r}
        & {a}P{u} & {a}P{ı} & $p$-value \\
native  & 12                & 11                & \multirow{2}{*}{0.042} \\
foreign & 84                & 28                \\
\end{tabular}
\end{example}

\citet{NiChiosain1993} and \citet{Ito1995b} 

``sonority projection''
Ther

\citet{Daland2011b} finds that this can be learned easily from positive data but a number of different psych

%It is important to note that the nature of the \emph{blick}/\emph{bnick} contrast is no longer a mystery. 
%Generative phonological theory at the time of \emph{SPE} lacked the representational vocabulary to directly distinguish between word-initial /bn/, which is disallowed, and /bn/ as a licit syllable contact cluster (e.g., \emph{o}[b.n]\emph{oxious}), a point first made by \citep{Hooper1973}. It is not the introduction of prosodic primitives into generative phonology that explains the contrast between \emph{blick} and \emph{bnick}, though, but rather the principle that surface forms must be syllabified \citep[e.g.,][63f.]{Kiparsky1982b} which \emph{bnick} fails. If, as is standardly assumed, phonology repairs unsyllabifiable outputs \citep[e.g.][]{Ito1989a,Noske1992}, then \emph{bnick} is an impossible output, doomed to be modified (perhaps to \emph{nick}; \citealt[][19f.]{Wolf2009}), and /bnɪk/ is an impossible UR by Stampean occultation. 
%\footnote{I note a potential problem with this account: \citet{Davidson2005,Davidson2006a} argue that this ``excrescent'' schwa is distinct from the both in acoustics and articulation from ``lexical schwa'': English speakers' [z${^\textrm{ə}}$k] is more similar to \emph{scum} than \emph{succumb}.}
%\noindent A nuanced version of this alternative hypothesis would recognize that there may be other necessary conditions. Any explicit theory of static constraints will have to impose formal restrictions on possible co-occurrence restrictions. The set of possible constraints must respect results in formal learning theory, and may also be constrained by markedness considerations, for example.
%A further widespread assumption is that static co-occurrence restrictions validated statistically will be reflected by wordlikeness judgements. This assumption is so deeply entrenched that a number of recent studies of phonotactic learning forgo wordlikeness judgements altogether and instead model the lexical statistics themselves (e.g., \citealt{Coetzee2008a} and \citealt{Anttila2008} on Muna and Arabic, \citealt[][385]{Hayes2008a} on Shona and Wargamay, \citealt{Brown2010} on Gitksan). 
%Since the null hypothesis does not countenance statistical significance as evidence for the synchronic reality of a static lexical pattern, such patterns do not adjudicate between the null hypothesis and the alternative, and another source of evidence is needed. In the remainder of Part I, I turn to data from wordlikeness judgement tasks for these purposes. 
%In a wordlikeness task, a speaker is presented with nonce words and asked to report such intuitions about the ``possibility'' of a word. 
%The alternative hypothesis, then, is that a statistically reliable co-occurrence restriction will be reflected in native speakers' wordlikeness judgements. 
%The data below, drawn from Turkish, shows that 
%This chapter takes the argument one step farther by showing a contrast between static and derived constraints on URs in Turkish. Whereas the derived constraint has a robust effect on native speakers' wordlikeness judgements, the static constraint, which is equally statistically valid, has no effect on these judgements. Anticipating the conclusion, the simplest explanation for the non-effect of this static constraint on wordlikeness judgements is that speakers do not internalize static co-occurrence restrictions at all. 
%\begin{unlabeledexample}
%$\displaystyle \gamma = \frac{C - D}{C + D}$
%\end{unlabeledexample}
%\footnote{Any readers who are skeptical that \citeauthor{Chomsky1965}'s intuitions generalize will find support for their claim in the wordlikeness judgement data reported in Appendix \ref{albright}.}
%As a quick validation of this claim, consider the results reported by \citet{Albright2007} and discussed in the previous chapter. English speakers asked to rate non-words on a scale from 1-7, where 7 indicates ``most like English'', favor non-words [blʌs, blɑd] (average scores 4.67 and 5.13) over [bnʌs, bnɑd] (average scores 2.06, 2.00).
