\label{turkish}

It has long been suggested that statistical criteria could distinguish between \emph{accidental phonotactic gaps}, those which could arise without any antecedent cause, and those gaps which are \emph{structural} in nature \citep[e.g.,][]{Fischer-Jorgensen1952,Saporta1955,Saporta1958,Vogt1954}.
Following seminal work by \citet{Mester1988} and \citet{McCarthy1988}, it is now commonly assumed that the phonotactic patterns in lexical entries ``directly determine the mental representation of the phonotactic constraints'' \citep[180]{Frisch2004} acquired by speakers, and therefore phonotactic constraints, whether categorical or gradient, can be directly inferred from statistical analysis of the lexicon.
There is by now an enormous amount of research that adopts this assumption.
Semitic languages, especially Arabic \citep{Coetzee2008a,Frisch2004,McCarthy1988,Pierrehumbert1993} but also Berber \citep{Elmedlaoui1995} and Tigrinya \citep{Buckley1997}, and Austronesian languages, including Javanese \citep{GraffInPress,Mester1988}, Muna \citep{Anttila2008a,Coetzee2008a}, and Samoan \citep{AldereteInPress}, have been of particular interest to linguists adopting this hypothesis; Aymara \citep{GraffInPress}, Cantonese \citep{Yip1989}, English \citep{Berkley1994a,Berkley1994b,Berkley2000,Coetzee2008b,Davis1989,Martin2007,Martin2011}, Gitksan \citep{Brown2010}, Hungarian \citep{Grimes2010}, Japanese \citep{Kawahara2006}, Ju|'hoansi \citep{Kinney2005}, Navajo \citep{Martin2007,Martin2011}, Ofo \citep[38f.]{MacEachern1999}, Russian \citep{Padgett1991,Padgett1992}, Shona \citep{Hayes2008a}, and Wargamay \citep{Hayes2008a} have also received a statistical treatment. 
\citet{Pozdniakov2007} analyze lexical tendencies in a diverse sample of 30 languages.

\citet{Brown2010} presents a strong form of this hypothesis, implying that \emph{any} statistically significant pattern in the lexicon is one that is internalized by speakers:

\begin{quote}
\ldots{} the patterns outlined above are statistically significant. Given this, it stands that these sound patterns should be explained by some linguistic mechanism. \citep[][48]{Brown2010}
\end{quote}

It would be a result of great interest were it shown that statistical significance is both necessary and sufficient to identify linguistic generalizations which are internalized by speakers, but there is no reason to grant this assumption with respect to phonotactic knowledge.
Nothing demands that the antecedent cause of a statistically reliable lexical tendency be grammatical. 
There are reasons to suspect that many static phonotactic constraints identified in this manner have no synchronic reality.

By hypothesis, the phonological component may impose constraints on sound segments and sequence on underlying representations.
Beyond this, 
a principled null hypothesis (implied by the principle of \textsc{No Static Phonotactics}) is that there is no synchronic, grammatical explanation for underlying representations a language chooses to instantiate.
It is a practical necessity that numerous well-formed underlying representations will be uninstantiated: the lexicon is finite but there are an infinitude of possible URs.
The synchronic grammar cannot reasonably be expected to account for all non-existent underlying representations: for instance, the phonology of English has little to say about absence of */blɪk/ (cf.~\emph{flick}, \emph{brick}, \emph{block}, \emph{blink}).

This much seems uncontroversial.
However, what has not been appreciated is that even when a phonotactic generalization over underlying representations can be given a phonological characterization, there is a plausible alternative to the assumption that it is part of the synchronic grammar: namely, the generalization may be the result of now-complete diachronic change.
Since sound changes often begin as phonological processes, it is no surprise that phonotactic gaps or tendencies can be explained with reference to phonological representations.
But this suggests that the structural nature of a gap is not pertinent to determining whether the constraint is synchronically real.

Saussure (\citeyear{CLG}) presents an example of phonotactic underrepresentation caused by sound change.
With only sporadic exceptions, Old Latin intervocalic \emph{s} undergoes a conditioned phonemic merger with \emph{r}.
This has two consequences.
Second, it introduces many \emph{s}-\emph{r} alternations: e.g., \emph{honōs}-\emph{honōris} `honor'.
The traditional analysis (e.g., \citealt[62]{Foley1965}, \citealt[142]{Gruber2006}, \citealt[134]{Heslin1987}, \citealt[377]{Kenstowicz1996}, \citealt[314]{Klausenburger1976}, \citealt[19]{Matthews1972a}, \citealt[88]{Roberts2012}, \citealt[526]{Watkins1970}) treats \emph{r} as the intervocalic allophone of /s/, and derives \emph{honōris} from underlying /honoːs-is/.
 
\begin{example}[\textsc{Rhotacism}]
$\textrm{s}~\goesto~\textrm{r}~/~\begin{bmatrix} +\textsc{Voc} \end{bmatrix}~\gap~\begin{bmatrix} +\textsc{Voc} \end{bmatrix}$
\end{example}

\noindent
However, subsequent sound changes \citep[e.g.,][]{Baldi1994,Safarewicz1932}, particularly the degemination of Old Latin \emph{ss} after diphthongs and long monophthongs (e.g., \emph{caussa} $>$ \emph{causa} `cause'), introduce numerous exceptions to \textsc{Rhotacism}.
In Classical Latin, intervocalic \emph{s} is found root-internally (\emph{asellus} `donkey', \emph{casa} `hut'), in environments derived by inflectional suffixes (\emph{uās}-\emph{uāsis} `vase', \emph{uisēre} `to view'), prefixation (\emph{dēsecāre} `to cut off'; cf.~\emph{dē} `from', \emph{secāre} `to cut'), compounding (\emph{olusātrum} `parsnip'; cf.~\emph{olus} `vegetable', \emph{ātrum} `black'), and denominal adjective formation (\emph{uentōsus} `windy'; cf.~\emph{uentus} `wind'), and is tolerated in nativized loanwords from Celtic (\emph{omāsum} `tripe'), Germanic (\emph{glaesum} `amber'), and Greek (\emph{basis} `pedestal').
These facts lead Saussure to conclude that \textsc{Rhotacism} is no longer ``inhérente à la nature de la langue'' (202).
While some of the apparent exceptions may be the result of opaque phonological application \citep{Heslin1987}---an explanation not yet available in Saussure's time---any formulation of rhotacization will admit nearly as many lexical exceptions as there are roots exhibiting \emph{s}-\emph{r} alternations \citep{GormanInPressc}.
Any synchronic account of the underrepresentation of intervocalic \emph{s} must confront the unproductive nature (as indicated by the accumulation of exceptions) of the proximate explanation for this tendency.

% FIXME mention Fudge's analysis of short-a before voiceless fricatives
% \citet{Trager1930,Trager1934,Trager1940},
% \citet{Cohen1970}
% \citet{Ferguson1975}
% \citet{PLC1} 429f.
% \citet{Fudge1987}
% \citet{AOE} I.203f., 515f. for an attempt to write a rule
% \citet{Payne1976,Payne1980} who finds that whereas speakers from out of state acquire 5 vocalic sound changes, only one out of 34 interviewees acquired the core pattern
% Fudge(!)

Other cases show that constraints identified by statistical analysis are inert: that is, there is no evidence (of any of type described in \S\ref{s:espt} above) that they are internalized by speakers.
For instance, most instances of Modern English [ʃ] derive from Old English [sk] (e.g., \emph{fisc} `fish') via sound change.
Since Old English does not permit long vowels before complex codas, compared to similar segments like [s], [ʃ] is still rarely preceded by long vowels in word-final syllables.\footnote{
    This sample is drawn from the CMU pronunciation dictionary, filtered by excluding words with a token frequency of less than 1 per million words in the SUBTLEX-US frequency norms.
    Less restrictive samples give similar results.
    The 9 words ending with a long vowel-[ʃ] sequence are \emph{douche}, \emph{leash}, \emph{gosh}, \emph{josh}, \emph{posh}, \emph{squash}, \emph{unleash}, \emph{wash}, and \emph{woosh}.}
As can be seen, long vowels are twice as common before [s] as before [ʃ], a significant generalization according to the Fisher exact test.
Similarly, \citet{HayesInPress} report that this constraint is discovered by the \citet{Hayes2008a} phonotactic learning model, which uses a related statistical criterion to identify constraints.
Despite this, \citet{Iverson2005} label the constraint on long vowels before [ʃ] as ``phonologically accidental'', as a millennium of coinages (e.g., \emph{posh}) and loanwords (e.g., \emph{douche}) disregard this generalization.
\citet{HayesInPress} find that a variant of this restriction has little to no effect on wordlikeness judgements.

\begin{table}[t]
\centering
\begin{tabular}{l r r r r}
\toprule
          & \{ɪ, ɛ, æ, ʌ, ʊ\}\gap{}\# & \{i, e, ɑ, ɔ, u\}\gap{}\# & \% long & $p$-value \\
\midrule
\gap{}ʃ\# & 78                & 9                 & 8\%      & \multirow{2}{*}{.026} \\
\gap{}s\# & 410               & 107               & 16\%     & \\
\bottomrule
\end{tabular}
\caption{Type frequencies of short and long vowels before word-final [ʃ] and [s]}
\label{ssh}
\end{table}

This latter example makes it clear that a purely statistical criterion overgenerates in the sense that it predicts phonotactic constraints which speakers do not seem to internalize.
To account for the cases like the one above, \citet{HayesInPress} propose that speakers are biased in favor of ``natural generalizations'' in probabilistic phonotactic learning.\footnote{
    \citeauthor{HayesInPress} do not provide an operational definition of ``naturalness'', so it is difficult to evaluate their specific results or to determine whether they apply the classification of ``natural'' and ''unnatural'' is applied in an ad hoc (or post hoc) fashion.}
However, this chapter argues that this overgeneration gives the lie to the broader assumption that phonotactics are extracted from patterns in the lexicon.
Rather, the only restrictions which speakers internalize are those which derive from phonological processes in the language.
This is independent of ``naturalness'', since both ``natural'' and ``unnatural'' variants of statistically reliable static constraints are shown to be inert.

This chapter focuses on three phonotactic generalizations in Turkish, comparing lexical statistics and the results of a wordlikeness task performed by \citet{Zimmer1969}.
Both the lexical statistics and \citeauthor{Zimmer1969}'s experimental results merit reconsideration, because prior discussions do not relate the lexical statistics to experimental data or competing formalizations of the generalizations involved.

\section{Turkish vowel sequence structure constraints}

\citet{Lees1966b,Lees1966a} proposes three constraints on Turkish vowel sequences; these constraints are the focus of many subsequent studies. In this section, these constraints are formalized, and where possible, related to phonological alternations and to behavioral evidence bearing on speakers' knowlege of the restrictions. The following feature specification for the eight vowels of Turkish is assumed throughout.

\begin{example}[Turkish vowel features]
\begin{tabular}{c c c c c}
                       & \multicolumn{2}{c}{[$-$\textsc{Back}]} & \multicolumn{2}{c}{[$+$\textsc{Back}]} \\
                       & [$-$\textsc{Rnd}] & [$+$\textsc{Rnd}] & [$-$\textsc{Rnd}] & [$+$\textsc{Rnd}] \\ 
\cmidrule{2-5}
\buf[$+$\textsc{High}] & {i} & {ü} [y] & {ı} [ɯ] & {u} \\
\buf[$-$\textsc{High}] & {e} & {ö} [ø] & {a} [ɑ] & {o} \\
\end{tabular}
\end{example}

Two less familiar notations are used in this chapter. First, directional application conditions are assumed, so as to derive the left-to-right, iterative properties of the harmony rules. Since it was first proposed by \citet{Johnson1972}, considerable evidence for directional application has been adduced (e.g., \citealt{A74}: chap.~9, \citealt{GormanInPressc}, \citealt{Howard1972}:65f., \citealt{Kavitskaya2008}, \citealt{Kaye1982}, \citealt{KK77}:189f., \citeyear{KK79}:319f., \citealt{Piggott1975}, \citealt{Sohn1971}; see \citealt{McCarthy2003b} and \citealt{Wolf2011b} for recent reviews.) Further, \citet{Johnson1972} and \citet{Kaplan1994} prove directional application and simultaneous application are equivalent in terms of formal learnability.

Secondly, rather than the use of an unbounded number of Greek-letter variables ($\alpha$, $\beta$, etc.) over feature values $\{+, -\}$, only a single variable, denoted by `$=$', is used \citep{McCawley1973}. A structural description [$=$F]\ldots{}[$=$F] matches a string S$_i$\ldots{}S$_j$ if and only if S$_i$ and S$_j$ are both [$+$F] or both [$-$F]. This is more restrictive than Greek-letter notation, in that it prevents the value of one feature being applied to different feature; \citet{Odden2012} argues that the two counterexamples against this restriction presented in \emph{SPE} (352-353) are not probative.

\subsection{Backness harmony}
\label{ss:bh}

\citeauthor{Lees1966b} (\citeyear[35]{Lees1966b}, \citeyear[284]{Lees1966a}) models the Turkish vowel harmony system with three rules; the most general of these rules spreads the specification [\textsc{Back}] rightward.

\begin{example}[\textsc{Backness Harmony} (condition: rightward application)]
$\begin{bmatrix} -\textsc{Cons} \end{bmatrix}~\goesto~\begin{bmatrix} =\textsc{Back} \end{bmatrix}~\big /~\begin{bmatrix} =\textsc{Back} \end{bmatrix}~\textrm{C}_0~\gap$
\end{example}

\noindent
A vowel becomes [$+$\textsc{Back}] after a [$+$\textsc{Back}] vowel, and [$-$\textsc{Back}] after a [$-$\textsc{Back}] vowel, ignoring any intervening consonants. The application of this rule proceeds from left to right; no vowel may be skipped.

If permitted to apply in non-derived environments, this rule accounts for the tendency of polysyllabic roots to contain only [$+$\textsc{Back}] or [$-$\textsc{Back}] vowels, a tendency which will be quantified below. \textsc{Backness Harmony} also triggers alternations in inflectional suffix vowels. For instance, the nominative plural (nom.pl.) suffix is \emph{-ler} when the final root vowel is [$-$\textsc{Back}], and \emph{-lar} when it is [$+$\textsc{Back}].

\begin{example}[The Turkish nominative]
\label{turknom}
\begin{tabular}{lllll}
   & \emph{nom.sg.} & \emph{nom.pl.} \\
a. & {ip}           & {ipler}    & `rope'         & \citep[][216]{Clements1982} \\
   & {köy}          & {köyler}   & `village'      \\
   & {yüz}          & {yüzler}   & `face'         \\
   & {kız}          & {kızlar}   & `girl'         \\
   & {pul}          & {pullar}   & `stamp'        \\
b. & {neden}        & {nedenler} & `reason'       & \citep{TELL} \\
   & {kiler}        & {kilerler} & `pantry'       \\
   & {pelür}        & {pelürler} & `onionskin'    \\
   & {boğaz}        & {boğazlar} & `throat'       \\
   & {sapık}        & {sapıklar} & `pervert'      \\
\end{tabular}
\end{example}

A few complications arise, however. First, as shown in (\ref{turkexcept}a), not all polysyllabic roots conform to \textsc{Backness Harmony}. In this case, suffix vowels generally exhibit harmony with the final root vowel. There is also a very small class of nouns, shown in (\ref{turkexcept}b), which take \emph{-ler}, although their final root vowel is [$+$\textsc{Back}]. Interestingly, the roots themselves may be harmonic.

\begin{example}[Exceptional Turkish nominatives] 
\label{turkexcept}
\begin{tabular}{lllll}
   & \emph{nom.sg.} & \emph{nom.pl.}&                    \\
a. & {mezar}        & {mezarlar}    & `grave'       & \citep{TELL}       \\
   & {model}        & {modeller}    & `model'                            \\
   & {silah}        & {silahlar}    & `weapon'                           \\
   & {memur}        & {memurlar}    & `official'                         \\
   & {sabun}        & {sabunlar}    & `soap'                             \\
b. & {etol}         & {etoller}     & `fur stole'   & \citep{Goksel2005} \\
   & {saat}         & {saatler}     & `hour, clock' 	              \\
   & {kahabat}      & {kahabatler}  & `fault'       \\
\end{tabular}
\end{example}

\citet[212]{A74} and \citet{Iverson1978} argue that suffix harmony in disharmonic roots found in (\ref{turkexcept}a) requires the rule governing suffix harmony alternations to be distinguished from a sequence structure constraint governing root harmony. 
Since the rule and sequence structure constraint are otherwise identical, this constitutes a ``duplication''. 
However, the theory of exceptionality proposed in \emph{SPE} can account for these facts without duplication \citep[197f.]{Zonneveld1978}.\footnote{
    \citet[29f.]{Kiparsky1968} discusses a parallel case in Finnish, and proposes a similar separation between the exception-filled sequence structure constraint and an exceptionless suffix harmony process. 
    \citet[171f.]{Howard1972} correctly observes that this duplication is also unnecessary.}

\citeauthor{SPE} assume that the specification of the target (i.e., the segment or segments to be changed) of a rule \emph{R} must be marked [$+$\emph{R}] by convention. 
A root or affix which fails to undergo \emph{R} despite otherwise matching the structural description is simply said to be marked [$-$\emph{R}]. 
In other words, no representation is ever truly an exception; rather, some underlying representations have non-default features which do not match the extended structural description of $R$, which requires that the target be [$+R$]. 
If disharmonic roots are marked [$-$\textsc{Backness Harmony}], then the final vowel of disharmonic roots will still trigger \textsc{Backness Harmony}, since the [$-$\textsc{Backness Harmony}] root is no longer the target but rather the trigger, which is not subject to the [$+$\textsc{Backness Harmony}] requirement.

Under this account, root and suffix harmony are derived from \textsc{Backness Harmony}, but they do not have the same ontology: suffix harmony is direct result of phonological rule application, whereas any ``effects'' of the  generalization concerning harmonic roots arises from a dispreference for lexical exceptionality.

\citeauthor{A74} also notes that roots which fail to undergo suffix harmony, like (\ref{turkexcept}b), may themselves be harmonic. He takes this to be evidence for the necessity of duplication:

\begin{quote}
\ldots{}there are words which are exceptions to harmony across boundaries (e.g., \emph{kabahat} `fault', \emph{kabahatti} `his fault') but which are perfectly regular internally. Since the morpheme structure condition and the phonological rule in this case have distinct classes of exceptions, it is clear that they cannot be identified. \citep[289]{A74}
\end{quote}

\noindent
Under the assumptions so far, there is no reason to think that \emph{kabahat} is [$+$\textsc{Backness Harmony}], however: even roots with all back vowels need not undergo this rule. Whatever the ultimate analysis of those few words which fail to show suffix harmony, nothing depends on whether or not the roots in question are harmonic.

It is necessary to dispense with an alternative analysis proposed by \citet{Clements1982} and \citet{Inkelas1997}. 
This analysis has desirable properties, but it subtlely reintroduces the duplication of phonological process and sequence structure constraint. 
Root vowels exhibit a robust contrast for backness (e.g., \emph{kül} `ash' vs. \emph{kul} `servant', \emph{kepek} `bran' vs. \emph{kapak} `lid'), whereas backness of vowels in non-initial syllables is predictable in harmonic roots (there are no prefixes in Turkish).
\citeauthor{Clements1982} propose that these vowels, as well as harmonizing suffix vowels, are underspecified for backness, whereas the non-initial vowels of disharmonic roots and of certain exceptional suffixes are fully specified. 
This is schematized below.

\begin{example}[Autosegmental underspecification in harmonic roots (after \citealp{Clements1982})] 
\label{spec}
\xymatrix@R=24pt@C=24pt{
\txt{a.} & \txt{harmonic root:} & \txt{C} & \txt{V} & \txt{C} & \txt{V} & \txt{C} \\
         &                      &         & \txt{[$-$\textsc{Back}]}\ar@{-}[u]\ar@{--}[urr] \\
\txt{b.} & \txt{disharmonic root:} & \txt{C} & \txt{V} & \txt{C} & \txt{V} & \txt{C} \\
         &                      &         & \txt{[$-$\textsc{Back}]}\ar@{-}[u] & & \txt{[$+$\textsc{Back}]}\ar@{-}[u]
}
\end{example}

One crucial detail is missing from this analysis: \textsc{Backness Harmony} needs to be prevented from overwriting the [$+$\textsc{Back}] specification of disharmonic roots, perhaps by structure preservation condition \citep{Kiparsky1985}. 
However, any condition which prevents \textsc{Backness Harmony} from overwriting underlying backness specifications will reintroduce the duplication of sequence structure and phonological generalizations; under this analysis, disharmonic roots are no longer exceptional, despite considerable evidence (reviewed below) that they are marked in Turkish.\footnote{
    On the other hand, it is possible to interpret the presence of a single backness specification per root as a sort of default. 
    A precedent for this is the surface-oriented interpretation of the tonal Obligatory Contour Principle proposed by \citet[134]{Goldsmith1976} and \citet{Odden1986}, under which adjacent identical tones are automatically attributed to a single underlying tone. 
    However, this is merely a notational variant of the rule exceptionality account in which [$+$\textsc{Backness Harmony}] is the default.} 
    The underspecification analysis is rejected here on these grounds.

While harmony in non-derived environments can be inferred from the aforementioned suffix alternations, no evidence has yet been presented to show that Turkish speakers internalize the tendency for roots to conform to backness harmony. 
    If Turkish speakers do not attend to this generalization, there is no need for the grammar to account for it. Several other ``external'' facts suggest that this is not the case. 
    The discussion here is not intended to imply uncritical acceptance of evidence from loanword adaptation, language games, or psycholinguistic tasks as evidence for phonological grammar, but rather to illustrate additional evidence that is pertinent if the linking hypothesis is correct.\footnote{
    Thanks to Bert Vaux and Kie Zuraw for bringing these studies to my attention.}

The production of non-native word-initial onset clusters, discussed by \citet{Clements1982} and \citet{Kaun1999}, suggests that loanword adaptation respects \textsc{Backness Harmony}. 
    Some speakers pronounce these non-native clusters, but in fast speech the cluster is split by anaptyxis. 
    In the majority of cases, this vowel matches the following root vowel for backness.

\begin{example}[Variable non-native cluster adaptation (\citealp{Clements1982}:247)] 
\begin{tabular}{lllll}
a. & {spiker}  & \alt{} & {sipiker}  & `announcer' \\
   & {fren}    & \alt{} & {firen}    & `brake'     \\
b. & {trablus} & \alt{} & {tırablus} & `Tripoli'   \\
   & {kral}    & \alt{} & {kıral}    & `king'      \\
c. & {brom}    & \alt{} & {burom}    & `bromide'   \\
   & {prusya}  & \alt{} & {purusya}  & `Prussia'   \\
\end{tabular}
\label{spiker}
\end{example}

\noindent
It is unclear whether the cluster-splitting vowel is deleted in the non-native variant or epenthesized in the fast speech variant. 
    Under either analysis, there is no ready explanation for the tendency of the cluster-splitting vowel to have the backness features of following vowels; if anything, one might expect it to determine the backness features of following vowels. 
    All that can be said with certainty is that the adaptation of non-native onset clusters appears to proceed in such a fashion so that the lexical items in question are [$+$\textsc{Backness Harmony}].

Similar evidence comes from a language game discussed by \citet{Harrison2001}. 
    The game is native to the related language Tuvan, where it is used to convey a sense of ``vagueness or jocularity''; it is not indigenous to Turkish, but can be taught quickly to children or adults. 
    In this game, the base is reduplicated and the first vowel of the reduplicant replaced with a [$+$\textsc{Back}] vowel. 
    In (\ref{redupgame}a), the second [$-$\textsc{Back}] vowel of the base is, in the reduplicant, ``reharmonized'' with the inserted [$+$\textsc{Back}] vowel. 
    The disharmonic roots of (\ref{redupgame}b) do not reharmonize.\footnote{
    A similar contrast between harmonic and disharmonic roots is found in Tuvan \citep{Harrison2001} and in an unrelated Finnish language game \citep{Campbell1986}.}

\begin{example}[Turkish reduplication game (\citealp{Harrison2001}:231)] 
\label{redupgame}
\begin{tabular}{llll}
a. & {kibrit} & {kibrit}-\{{kabrıt}\} & `match'    \\
   & {bütün}  & {bütün}-\{{batın}\}   & `whole'    \\
b. & {mali}   & {mali}-\{{muli}\}     & `Mali'     \\
   & {butik}  & {butik}-\{{batik}\}   & `boutique' \\
\end{tabular}
\end{example}

\noindent 
In the full specification analysis adopted here, reharmonization is the result of \textsc{Backness Harmony} applying within the reduplicant. 
    On the other hand, the lack of reharmonization in the reduplicants of disharmonic roots suggests that the [$-$\textsc{Backness Harmony}] exception feature is copied under reduplication.\footnote{
    There is reason to suspect that reduplicants bear the lexical diacritics of the roots they are derived from.
    In Kinande, verb reduplication, documented by \citet{Mutaka1990} and further discussed by \citet{Downing2000}, requires a bisyllabic reduplicant. 
    This clearly has synchronic force in the gramar, since reduplicated monosyllabic roots contain three copies of the root so as to satisfy this requirement, and reduplicated forms of many trisyllabic verbs are ineffable. 
    However, a few trisyllabic verbs exceptionally show full reduplication.
    This is a lexical property, but presence or absense of this exceptionalty feature is only apparent in the form of the reduplicant.}
    Under this definition, root harmony is not static, and so this is not an exception to the claim that static phonotactic generalizations are not preserved by operations like reduplication and truncation \citep{Silverman2000}.

A number of studies have investigated the role of harmony in word-spotting tasks, thought to mimic auditory word recognition and segmentation in natural settings. 
    Many of these studies have been carried out in Finnish, which has a vowel harmony system similar to that of Turkish. 
    \citet{Suomi1997} and \citet{Vroomen1998} task Finnish speakers with identifying harmonic disyllables in an auditory stream. 
    When the syllable preceding the disyllabic target has a different backness specification than the target, recognition of the target is facilitated. Presumably, disharmony facilitates the recognition of word boundaries. 
    \citet{Kabak2010} find that Turkish \textsc{Backness Harmony} has a similar effect: Turkish speakers are quicker and more accurate at the task of spotting the nonce target word \emph{pavo} when preceded by a disharmonic juncture (e.g., \emph{gölü-PAVO}) than when preceded by a harmonic juncture (e.g., \emph{golu-PAVO}). 
    \citeauthor{Kabak2010} report that this effect does not obtain for speakers of French, a language which lacks vowel harmony. 
    This implies that Turkish speakers have internalized the tendency of harmonic sequences to be root-internal and of disharmonic transitions to cross word boundaries. 

The Turkish word-spotting experiment is adapted for infants by \citet{Kampen2008}.
In this study, 9-month-old infants are familiarized with recordings of harmonic, disyllabic nonce words presented in isolation.
At test time, the infants listen to the disyllabic nonce words in isolation using the head turn preference paradigm. 
Infants acquiring Turkish listen longer to nonce words preceded by a disharmonic juncture during familiarization (e.g., \emph{lo-NETIS}), whereas infants acquiring German, a language which lacks vowel harmony, do not exhibit this preference. 
Similarly, \citeauthor{Kampen2008} report that Turkish 6-month-old infants prefer to listen to harmonic nonce words such as \emph{paroz} over disharmonic nonce words like \emph{nelok}, but German 6-month-old infants show no such preference.
However, there are some caveats for identifying these effects with grammatical computations: the domain for segmentation effects is considerably larger than the domain for harmony, so the two cannot be easily identified.
All that can be said is that there is an obvious similarity between computing the contexts for vowel harmony and for the word segmentation heuristic.

\subsection{Roundness harmony}

\textsc{Roundness Harmony} is quite similar to \textsc{Backness Harmony}, but imposes an additional restriction, that targets be [$+$\textsc{High}]. 

\begin{example}[\textsc{Roundness Harmony} (condition: rightward application)]
$\begin{bmatrix} -\textsc{Cons} \\ +\textsc{High} \end{bmatrix}~\goesto~\begin{bmatrix} =\textsc{Rnd} \end{bmatrix}~/~\begin{bmatrix}~=\textsc{Rnd}~\end{bmatrix}~\textrm{C}_0~\gap$
\end{example}

\noindent
A [$+$\textsc{High}] vowel becomes [$+$\textsc{Rnd}] after a [$+$\textsc{Rnd}] vowel, and [$-$\textsc{Rnd}] after a [$-$\textsc{Rnd}] vowel, ignoring any intervening consonants, and applying from left to right. 

If permitted to apply in non-derived environments, this rule accounts for the tendency of polysyllabic roots to contain only [$+$\textsc{Rnd}] or only [$-$\textsc{Rnd}] [$+$\textsc{High}] vowels.
In concert with \textsc{Backness Harmony}, \textsc{Roundness Harmony} also triggers alternations which account for the shape of the dative singular (dat.sg.) and genitive singular (gen.sg.) suffixes, among others.suffixes, among others.
As was the case for \textsc{Backness Harmony}, disharmonic roots are found, and the final vowel of disharmonic roots triggers suffix harmony.

\begin{example}[Turkish nominal suffix allomorphy]
\begin{tabular}{lllllll}
   & \emph{nom.sg.} & \emph{dat.sg.} & \emph{gen.sg.}  \\
a. & {ip}           & {ipi}          & {ipin}         & `rope' & \citep[][216]{Clements1982} \\
   & {kız}          & {kızı}         & {kızın}        & `girl'    \\
   & {sap}          & {sapı}         & {sapın}        & `stalk'  \\
   & {köy}          & {köyü}         & {köyün}        & `village' \\
   & {son}          & {sonu}         & {sonun}        & `end'     \\
b. & {boğaz}        & {boğazı}       & {boğazın}      & `throat'  & \citep{TELL} \\
   & {pelür}        & {pelürü}       & {pelürün}      & `onionskin' \\
   & {döviz}        & {dövizi}       & {dövizin}      & `currency'  \\
   & {yamuk}        & {yamuğu}       & {yamuğun}      & `trapezoid' \\
   & {ümit}         & {ümiti}        & {ümitin}       & `hope'      \\
\end{tabular}
\end{example}

Few studies have directly investigated whether speakers are aware of the tendency for roots to conform to \textsc{Roundness Harmony}. 
However, two of the external sources of evidence for \textsc{Backness Harmony} also bear on this question.
First, the cluster-splitting vowel found in non-native onset clusters, discussed above, tends to agree in roundness with following high vowels (e.g., \emph{prusya}-\emph{purusya} `Prussia'). 
Secondly, \textsc{Roundness Harmony} participates in reharmonization in the language game described by \citeauthor{Harrison2001}: the second \emph{ü} in \emph{bütün} `whole' reharmonizes to the [$-$\textsc{Rnd}] vowel \emph{ı} in reduplicated \emph{bütün-batın}.

\subsection{Labial attraction}

\citet[35]{Lees1966b} notes the tendency of Turkish high back vowels to be round after \emph{a}-labial consonant sequences, and formalizes this as a phonological process.

\begin{example}[\textsc{Labial Attraction}]
$\begin{bmatrix} -\textsc{Cons} \\ +\textsc{Back} \\ +\textsc{High} \end{bmatrix}~\goesto~\begin{bmatrix} +\textsc{Rnd} \end{bmatrix}~/~\textrm{ɑ}~\textrm{C}_0~\begin{bmatrix} +\textsc{Cons} \\ +\textsc{Labial} \end{bmatrix}~\textrm{C}_0~\gap{}$
\end{example}

This rule is significantly more complex than the harmony rules, and this may obscure the fact that it produces exceptions to \textsc{Roundness Harmony}, producing \emph{a}C$_0$\emph{u} (e.g., \emph{çapul} `raid', \emph{sabur} `patient', \emph{şaful} `wooden honey tub', \emph{avuç} `palm of hand', \emph{samur} `sable'; \citealp[285]{Lees1966a}) rather than the expected \emph{a}C$_0$\emph{ı}. 
However, \textsc{Labial Attraction} does not apply in derived environments: the gen.sg. of \emph{sap} `stalk is \emph{sapın} rather than *\emph{sapun} that would be predicted if \textsc{Labial Attraction} triggered alternations. 
\citeauthor{Lees1966b} and \citet{Zimmer1969} cite roots which do not conform to \textsc{Labial Attraction} (e.g., \emph{tavır} `mode') but agree that they are surprisingly rare.
Others \citep[e.g.,][]{Clements1982} dispute this.

\section{Evaluation}
\label{3evaluation}

\citet[311]{Zimmer1969} administers two paired wordlikeness tasks designed to evaluate native speakers' knowledge of \textsc{Backness Harmony}, \textsc{Roundness Harmony}, and \textsc{Labial Attraction} in roots. 
Speakers are presented with a pair of nonce words, differing only in whether they obey or violate one of these three constraints, and then indicate the nonce word that is more Turkish-like.\footnote{
    Compared to the unpaired ratings tasks  commonly used in wordlikeness research, paired rating tasks have considerably more statistical power \citep[e.g.][]{Gigerenzer2004}, since there is little chance that any contrast between the phonotactically licit and illicit members of an otherwise-identical nonce words pair is caused by an omitted variable.
    Consequently, paired rating tasks are ideal for collecting wordlikeness judgements.
    The use of paired stimuli also makes more overt the purpose of the experiment from speakers, as opposed to the standard practice of concealing the purpose, the latter thought to increase response variability (see \citealt[398f.]{Hertwig2001} for discussion).}
\citeauthor{Zimmer1969} concludes that the former two rules are reflected in wordlikeness judgements, whereas \textsc{Labial Attraction} is not. 
Below, both lexical statistics and \citeauthor{Zimmer1969}'s wordlikeness results are analyzed statistically; \textsc{Labial Attraction} is shown to be a statistically robust generalization over the Turkish lexicon, but no variant of \textsc{Labial Attraction} is reflected in \citeauthor{Zimmer1969}'s wordlikeness study. 
In contrast, the two harmony processes have robust effects both on the lexicon and on wordlikeness.  
This dissociation between statistical tendencies  generalizations and wordlikeness results provides further evidence against the assumption that phonotactic knowledge can be inferred directly from lexical statistics.

\subsection{Lexical statistics}

Counts were computed by regular expression matching on a 9,601-root subset of the TELL database which consists of roots which show no surface variation in any inflected form.

To test for associations between the process (more specifically, the constraint that it imposes on roots) and type frequency in this database, each root was sorted into a $2 \times 2$ contingency table; the contents of each cell are specific to the process in question. The counts in this table are not expected to add up to 9,601, since many roots neither exemplify nor violate the process in question; for instance, monosyllabic words are irrelevant to root harmony. The Fisher exact test is used to compute a $p$-value representing the probability of the observed patterns arising under the null hypothesis that there is no association between the constraint and type frequency.

\subsubsection{Backness harmony}

\textsc{Backness Harmony} is exemplified in the lexicon insofar as there is a positive association between the backness of vowels in all adjacent syllables. Any disagreements on the backness specification of vowels in adjacent syllables are counted solely as exceptions, even if other vowel transitions in the root are harmonic; further assumptions are necessary to determine whether a root which has one disharmonic transition may in any sense ``obey'' harmony elsewhere.
%This follows from \emph{SPE} theory of lexical exceptionality (see also \citealt{Gouskova2012}), in which phonological exceptionality is a property of underlying representations, not individual segments. 
To construct the contingency table, roots are binned according to the backness specification of the first nucleus, and that of following nuclei. For example, the first two syllables of \emph{adalet} `justice' are harmonic, but it is coded as disharmonic because there is a \emph{a\ldots{}e} transition later in the word. The resulting counts are shown in Table \ref{bhs}. 61\% of roots conform to \textsc{Backness Harmony}, and the interaction between the backness of the first and of the subsequent vowels
%predicted by root-internal \textsc{Backness Harmony} 
is significant.

\begin{table}[t]
\centering
\begin{tabular}{lrrr}
\toprule
                             & [$+$\textsc{Back}]$_1$ & [$-$\textsc{Back}]$_1$ & $p$-value                     \\
\midrule
\buf{}[$+$\textsc{Back}]$_{2\ldots{}n}$ & 3,089                     & 1,704              & \multirow{2}{*}{$1.19$\e{-89}} \\
\buf{}[$-$\textsc{Back}]$_{2\ldots{}n}$ & 1,698                     & 2,250                                               \\
\bottomrule
\end{tabular}
\caption{TELL roots sorted according to \textsc{Backness Harmony}}
\label{bhs}
\end{table}

\subsubsection{Roundness harmony}

\textsc{Roundness Harmony} predicts correlation between the roundness of a vowel and the roundness of high vowels in the next syllable. Any root for which a vowel does not agree in roundness with a high vowel in the following syllable (e.g., \emph{ümit}) is considered to be an exception. Roots are binned according to the roundness of non-initial high vowels, and according to the roundness of the preceding vowel. 
%Roots lacking non-initial high vowels do not bear on the status of root \textsc{Roundness Harmony}. 
The resulting counts are shown in Table \ref{rhs}. 83\% of the roots conform to \textsc{Roundness Harmony}, and the interaction between the roundness of the $i$th vowel and the roundness of the $(i +1)$th high vowel is significant.

\begin{table}[t]
\centering
\begin{tabular}{lrrr}
\toprule
                                              & [$+$\textsc{Rnd}]$_i$ & [$-$\textsc{Rnd}]$_i$ & $p$-value                      \\
\midrule
\buf{}[$+$\textsc{High}, $+$\textsc{Rnd}]$_{i+1}$ & 613                   &   261                 & \multirow{2}{*}{$1.02$\e{-36}} \\
\buf{}[$+$\textsc{High}, $-$\textsc{Rnd}]$_{i+1}$ & 581                   & 2,841                                                  \\
\bottomrule
\end{tabular}
\caption{TELL roots sorted according to \textsc{Roundness Harmony}}
\label{rhs}
\end{table}

\noindent
The counts in the bottom row of Table \ref{rhs} contain a number of roots which are apparent exceptions to \textsc{Roundness Harmony} but conform to \textsc{Labial Attraction} (bottom left), and which conform to \textsc{Roundness Harmony} at the expense of \textsc{Labial Attraction} (bottom right). Excluding these types of roots would have the effect of slightly increasing the overall rate of \textsc{Roundness Harmony}, since the former is more common.

\subsubsection{Labial attraction}

\citet{Clements1982} also object to \textsc{Labial Attraction} and argue that it is not ``systematic''.

\begin{quote}
Even more decisive evidence against a rule of Labial Attraction is the existence of a further, much larger set of roots containing /\ldots{}aCu\ldots/ sequences in which the intervening consonant or consonant cluster does not contain a labial\ldots{}We conclude that there is no systematic restriction on the set of consonants that may occur medially in roots of the form /\ldots{}aCu\ldots/. \citep[225]{Clements1982}
\end{quote}

\noindent 
This claim can be evaluated using the Fisher exact test. Let P denote a sequence of one or more consonants, one of which is labial, and let T denote a sequence of one or more consonants none of which is labial. 
The null hypothesis is that \emph{a}P\emph{u} sequences, which conform to \textsc{Labial Attraction}, are no more likely than would be expected from other \emph{a}T\emph{u} sequences violating \textsc{Roundness Harmony}. 
The resulting counts are shown in Table \ref{las}.
Whereas the sequence \emph{a}P\emph{u} is more than twice as likely as \emph{a}P\emph{ı}, the sequence \emph{a}T\emph{u} is 5 times less likely than \emph{a}T\emph{ı}. 
This interaction is significant, as predicted by \textsc{Labial Attraction}, but contrary to \citeauthor{Clements1982}'s claim.
In fact, \emph{a\ldots{}u} sequences are less, not more, common than \emph{a\ldots{}ı} sequences, presumably a consequence of \textsc{Roundness Harmony}.

\begin{table}[t]
\centering
\begin{tabular}{lrrr}
\toprule
       & a\ldots{}u & a\ldots{}ı & $p$-value                      \\
\midrule
aP\ldots{} & 124    & 57     & \multirow{2}{*}{$1.02$\e{-36}} \\
aT\ldots{} & 136    & 590    &                                \\
\bottomrule
\end{tabular}
\caption{TELL roots sorted according to \textsc{Labial Attraction}}
\label{las}
\end{table}

\subsection{Wordlikeness ratings}

\citet{Zimmer1969} administers two variants of the paired nonce word rating task. 
The first used 23 native adult speakers who were permitted to select either nonce word as more like Turkish, or to indicate `no preference'. 
For the purposes below, `no preference' results are ignored. 
The second experiment used 32 native adults, none of whom appeared in the preceding study, and used a forced binary choice.

Each response is coded as \emph{concordant} if the nonce word conforming to the process is preferred, and \emph{discordant} if the disharmonic word is selected. 
To test for an association between the constraints, a non-parametric statistic, the Goodman-Kruskal (\citeyear{Goodman1954}) $\gamma$ is computed from the count of concordant ($c$) and discordant ($d$) pairs:

\begin{equation*}
\displaystyle \gamma = \frac{c - d}{c + d}
\end{equation*}

\noindent
The $\gamma$ statistic ranges between -1, in the case that all paired choices are discordant, and 1, if all paired choices are concordant.\footnote{It also is possible to perform statistical tests aggregating over items, but for small number of items, such tests have very poor power.}

\subsubsection{Backness harmony}

Both of the \citet{Zimmer1969} experiments include 5 pairs which differ in whether or not the nonce words conform to, or violate, \textsc{Backness Harmony}. As can be seen from Table \ref{bhw}, harmonic pairs are preferred approximately 6-to-1, and aggregating over speakers, no disharmonic member of a pair is favored. Speakers have a highly reliable preference for nonce words which exhibit \textsc{Backness Harmony} ($\gamma = 0.694$, $p = 1.7$\e{-59}). It is interesting to note that the disharmonic nonce word which has the highest rating is found in the pair \emph{terüz}-\emph{teruz}, both of which violate \textsc{Roundness Harmony}. While this is little more than an anecdote, this may be  indicative of a link between the two processes, and their exceptions, in the minds of native speakers. 

\begin{table}[t]
\centering
\begin{tabular}{lrlr|lrlr}
\toprule
\multicolumn{4}{c|}{Experiment 1} & \multicolumn{4}{c}{Experiment 2} \\
\multicolumn{2}{c}{\textsc{harmonic}} & \multicolumn{2}{c|}{\textsc{disharmonic}} & \multicolumn{2}{c}{\textsc{harmonic}} & \multicolumn{2}{c}{\textsc{disharmonic}} \\
\midrule
{temez} & 19            & {temaz} & 3 & {pemez} & 30            & {pemaz} & 2 \\
{teriz} & 23            & {terız} & 0 & {teriz} & 28            & {terız} & 3 \\
{tokaz} & 21            & {tokez} & 1 & {tokaz} & 26            & {tokez} & 6 \\
{tipez} & 21            & {tipaz} & 1 & {tipez} & 24            & {tipaz} & 8 \\
{terüz} & 20            & {teruz} & 1 & {terüz} & 19            & {teruz} & 13 \\
\bottomrule
\end{tabular}
\caption{Effects of \textsc{Backness Harmony} on wordlikeness \citep[from][]{Zimmer1969}}
\label{bhw}
\end{table}
% harmonic responses are favored approximately 6 to 1.
%$\gamma = 0.717$, $p = 7.5$\e{-69}

\subsubsection{Roundness harmony}

Both experiments include 5 pairs which differ in the presence or absence of \textsc{Roundness Harmony}. 
As shown in Table \ref{rhw}, there is an approximately 5-to-1 preference for harmonic nonce words, and as was the case above, no disharmonic member of any pair is preferred overall, across all speakers. 
Turkish speakers have a reliable preference for nonce words to conform to \textsc{Roundness Harmony} ($\gamma = 0.680$, $p = 1.1$\e{-47}).

\begin{table}[t]
\center
\begin{tabular}{lrlr|lrlr}
\toprule
\multicolumn{4}{c|}{Experiment 1} & \multicolumn{4}{c}{Experiment 2} \\
\multicolumn{2}{c}{\textsc{harmonic}} & \multicolumn{2}{c|}{\textsc{disharmonic}} & \multicolumn{2}{c}{\textsc{harmonic}} & \multicolumn{2}{c}{\textsc{disharmonic}} \\
\midrule
{törüz} & 19 & {töriz} & 1 & {pörüz} & 32 & {pöriz} & 0  \\
{tüpüz} & 22 & {tüpiz} & 0 & {tüpüz} & 31 & {tüpiz} & 1  \\
{takız} & 15 & {takuz} & 3 & {takız} & 22 & {takuz} & 10 \\
{tatız} & 12 & {tatuz} & 6 & {tatız} & 20 & {tatuz} & 12 \\
\bottomrule
\end{tabular}
\caption{Effects of \textsc{Roundness Harmony} on wordlikeness \citep[from][]{Zimmer1969}}
\label{rhw}
\end{table}

\subsubsection{Labial attraction}

Both experiments include 5 pairs which either conform to \textsc{Labial Attraction} and violate \textsc{Roundness Harmony}, or vice versa; the preferences are shown in Table \ref{law}. There is a small preference against \textsc{Labial Attraction} (and therefore in favor of \textsc{Roundness Harmony}, though this is non-significant ($\gamma = -0.043$, $p = 0.305$).

Speakers do not have the preferences predicted by \textsc{Labial Attraction}. 
%This result also extends for the variants of \textsc{Labial Attraction} proposed by \citet{Zimmer1969} and \citet{Inkelas2001}, since these variants have structural descriptions targeting a superset of the original formulation by \citet{Lees1966a}. 
It is of interest that speakers have no apparent preference at all: one might have expected that they would prefer nonce items conforming to \textsc{Roundness Harmony}.

\begin{table}[t]
\centering
\begin{tabular}{lrlr|lrlr}
\toprule
\multicolumn{4}{c|}{Experiment 1} & \multicolumn{4}{c}{Experiment 2} \\
\multicolumn{2}{c}{aPu} & \multicolumn{2}{c|}{aPı} & \multicolumn{2}{c}{aPu} & \multicolumn{2}{c}{aPı} \\
\midrule
{tamuz} & 3 & {tamız} & 16 & {pamuz} & 15 & {pamız} & 17 \\
{tafuz} & 3 & {tafız} & 17 & {tafuz} & 21 & {tafız} & 11 \\
{tavuz} & 9 & {tavız} & 4  & {mavuz} & 16 & {mavız} & 16 \\
{tapuz} & 7 & {tapız} & 9  & {tapuz} & 17 & {tapız} & 15 \\
{tabuz} & 5 & {tabız} & 12 & {tabuz} & 16 & {tabız} & 16 \\
\bottomrule
\end{tabular}
\caption{Effects of \textsc{Labial Attraction} on wordlikeness \citep[from][]{Zimmer1969}}
\label{law}
\end{table}

\subsection{Discussion}

It has been shown that while \textsc{Labial Attraction} is a highly reliable generalization about Turkish roots, it is not reflected in wordlikeness judgements.
In contrast, harmony processes have a similar statistical profile, but have large effects on wordlikeness.
The most plausible explanation for this is that  \textsc{Labial Attraction} does not trigger alternations; indeed, it is counter-exemplified by the effects of \textsc{Roundness Harmony} in suffix allomorphy.
As is noted by \citet[412f.]{Inkelas1997}, the lexicon of Turkish will, under the assumptions here, remain as it is whether or not \textsc{Labial Attraction} has a synchronic reality.
The law of parsimony suggests that it does not exist at all.

\citet{Ito1995a,Ito1995b} and \citet{NiChiosain1993} claim that \textsc{Labial Attraction} holds only over the native vocabulary.
This is a potential confound for \citeauthor{Zimmer1969}'s experiment, since it is not implausible that speakers in \citeauthor{Zimmer1969}'s study would treat nonce words much like loanwords; indeed, many wordlikeness studies include instructions to the participants to treat the stimuli much as if they were loanwords \citep[e.g.,][]{Hay2004a}.
However, \citet{Inkelas2001} find that foreign words are more---not less---likely to conform to \textsc{Labial Attraction} than native words.
One possible explanation is that many of the languages in contact with Turkish (including English,  Farsi, and French) lack the /ɯ/ (\emph{ı}) phoneme needed to contribute exceptions to a hypothetical \textsc{Labial Attraction}.

It is not obvious that it is desirable to exclude \textsc{Labial Attraction} as a possible rule, as \citet[394, fn. 2]{Inkelas1997} suggest that \textsc{Labial Attraction} may have even induced alternations at one point in the history of Turkish. 
However, \citeauthor{Becker2011} cite the inertness of \textsc{Labial Attraction} as evidence that naturalness constrains phonotactic learning.

\begin{quote}
This is clearly a complex and somewhat unnatural phonotactic, both in terms of the nonlocality of environment and the conjunction of features from two distinct triggers, and it is therefore a welcome result that not all speakers readily encoded it into a generalizable constraint. \citep[118]{Becker2011}
\end{quote}

If this is correct, it should be possible to show that a more ``natural'' variant of \textsc{Labial Attraction} is in fact reflected in wordlikeness judgements, assuming it too is statistically valid.
\citeauthor{Inkelas2001} make a similar observation.

\begin{quote}
Vowel labialization following labials is not a synchronic alternation in Turkish, yet it (unlike \textsc{Labial Attraction} per se) \emph{is} a statistically supported tendency worthy of further research. \citep[196]{Inkelas2001}
\end{quote}

\noindent
This proposal is formalized below as \textsc{Labial Attraction}$'$.\footnote{
    \citet{Zimmer1969} proposes another variant of \textsc{Labial Attraction} which ignores intermediate consonantal place but requires an \emph{a} trigger in the preceding syllable. 
    However, this is neither supported by lexical statistics or the results of his study.}

\begin{shortexample}[\textsc{Labial Attraction}$'$]
$\begin{bmatrix} -\textsc{Cons} \\ +\textsc{Back} \\ +\textsc{High} \end{bmatrix}~\goesto~\begin{bmatrix} +\textsc{Rnd} \end{bmatrix}~/~\begin{bmatrix} +\textsc{Lab} \\ +\textsc{Cons} \end{bmatrix}~\gap{}$
\end{shortexample}

\noindent
The environment is now strictly local.
Rounding of high vowels after labial consonants is also acoustically natural, as both are distinguished by low first and second formants.
Finally, the rounding of a high back vowel after a labial consonant is widely attested \citep[e.g.,][]{Vaux1993}.
Furthermore, as shown in Table \ref{lasi}, \textsc{Labial Attraction}$'$ is even more statistically reliable than \citeauthor{Lees1966a}'s original formulation.
Since this reformulation targets a superset of \citeauthor{Lees1966b}'s original rule, it should be reflected in the paired rating task, but this does not obtain.

\begin{table}[t]
\centering
\begin{tabular}{lrrr}
\toprule
       & \ldots{}u  & \ldots{}ı & $p$-value                      \\
\midrule
P\ldots{}  & 371    & 71        & \multirow{2}{*}{$6.98$\e{-49}} \\
T\ldots{}  & 811    & 922       &                                \\
\bottomrule
\end{tabular}
\caption{TELL roots sorted according to \textsc{Labial Attraction}$'$}
\label{lasi}
\end{table}

%Regarding the non-local condition holding between the \emph{a} portion of the trigger and the \emph{u} target, Classical Arabic has many adjectives which form corresponding inchoative verbs by overwriting the vocalic melody with /a\ldots{}u/ (the final /-a/ is an inflectional suffix).

%\begin{example}[Arabic derived inchoatives]
%\begin{tabular}{l l l l}
%\buf{}[kabiːr]  & `big'      & [kabura]  & `become big'       \\
%\buf{}[ħasan]   & `handsome' & [ħasuna]  & `become beautiful' \\
%\buf{}[dʒadiːb] & `barren'   & [dʒaduba] & `become dry'       \\
%\end{tabular}
%\end{example}

%\noindent
%In an Optimality Theory framework, for instance, constraints responsible for rounding of high back vowels before labials and for /a\ldots{}u/ overwriting can be conjoined to approximate the structural description of \textsc{Labial Attraction}.

\section{Conclusions}

Statistical reliability is neither necessary nor sufficient to predict what speakers know about possible and impossible words in their language.
Phonotactic constraints may go unlearned whether or not they are ``natural''.
In the case of Turkish vowel restrictions, at least, it is precisely those constraints which are derive from phonological processes---albeit processes with a considerable number of exceptions--- which are reflected in psycholinguistic tasks.
This should not be taken to imply that all phonotactic constraints inferred from the lexicon are illusory: for instance, \citet{Frisch2001} present psycholinguistic support for co-occurrence restrictions in Arabic posited by \citeauthor{Frisch2004} (\citeyear{Frisch2004} [1995]) on the basis of lexical data.
As a general principle, though, it should be apparent that lexical statistics do not contribute reliable evidence for the theory of phonotactics.
This is even more serious when a strong relationship between phonotactic and phonological representations is assumed (as it is here): contrary to common practices \citep[e.g.,][]{Mester1988,Padgett1991,Padgett1992}, lexical statistics do not provide principled evidence for the architecture of phonological grammar.

%\ex High-vowel/zero alternations \citep[][243]{Clements1982}: \\
%\begin{tabular}{l l l l}
%   & nom.sg. & gen.sg. \\
%a. & fikir   & fikri  & `idea' \\
%   & hüküm   & hükmün & `judgement' \\
%%  & filim   & filmi & `film' & \citep[][178]{Inkelas2001} \\
%b. & vakit   & vaktin & `time' \\
%   & rahim   & rahmin & `womb' \\
%\end{tabular} \xe
%This is not to imply uncritical acceptance of \citeauthor{Becker2011}'s account. \citeauthor{Becker2011} place this generalization ``in the grammar'' to account for their claim that ``naturalness'', which they equate with Universal Grammar, constrains the types of generalizations speakers extract in this task.
%It is difficult to evaluate their claim in the absense of a theory of phonotactic naturalness, something which has consisteny eluded formalization for decades. Even the informal definition used by \citeauthor{Becker2011} is flawed. \citeauthor{Becker2011} label generalizations unnatural if they lack phonetic precursors, or if they are not unattested---no description of the methodology used to determine attestedness is given. These two varieties of evidence are not fully independent, however: extragrammatical \emph{channel bias} effects are thought to be a first-order predictor of attestation \citep{Blevins2004,Moreton2008}. In light of extensive evidence for generalizations which are phonetically unnatural \citep[e.g.,]{Anderson1981,Bach1972,Buckley2000a}, even as revealed in nonce word studies \citep[e.g.,][]{Hayes2009}, the generalizations extracted in the \citet{Becker2011} study are phonetically natural is either accident or conundrum. It is certainly not something that can be explained by the tacit theory of possible gneeralizations the authors adhere to.
%Whereas the classic analysis of this phenomenon, by \citet{Inkelas1997}, does not make use of lexical exceptionality, \citeauthor{Becker2011}'s analysis of final aspirated stops and affricates as the underlying form requires that _every_ lexical entry with a final stop or affricate be treated as exceptional according to some phonological constraint. This projection of a lexical contrast into the grammar has another potential flaw: the constraints used to produce the voiced variants in, e.g., the dative, are not surface true. Consider the constraint *VpV used to induce alternations like \emph{kap}-\emph{kabı} `coat'. Ignoring the case of non-alternating root-final \emph{p} (e.g., \emph{ip}-\emph{ipi} `rope'), this is still not a surface-true generalization: \emph{p} occurs freely in intervocalic position: \emph{ahtapot} `octopus', \emph{köpük} `bubble', \emph{öpücük} `kiss'. More serious is the presence of root-internal intervocalic \emph{p} in ooots which simultaneously exhibit the root-final \emph{p}-\emph{b} alternation: \emph{supap}-\emph{supabı} `valve', \emph{hipermetrop}-\emph{hipermetrobu} `far-sightedness'. One prediction shared by virtually all theories of lexical exceptionality is that a lexical entry cannot simultaneously be a target for, and exception to, a phonological generalization, but this precisely what is needed to account for the behavior of \emph{supap}. 
% \citet{inkelas1997}
%küp-kübü `cube'
%kasap-kasabı `butcher'
%Further, the empirical status of the core instances of NDEB have recently been quite stridently disputed \citep{InkelasInPress}. There is some reason to suspect that NDEB is a symptom of interface restrictions on phonological application, rather than a diagnosis itself.
%\begin{example}[\textsc{Backness Harmony} lexical statistics]
%\begin{example}[Wordlikeness comparisons, backness harmony]
%\begin{example}[Wordlikeness comparisons, roundness harmony]
%\begin{example}[\textsc{Labial harmony} and etymology in TELL (counts from \citealp{Inkelas2001}:187)]
%\begin{example}[Wordlikeness comparisons, labial attraction]
%\begin{table}[ht]
%\centering
%\begin{tabular}{lrrrr}
%\toprule
%        & {a}P{u} & {a}P{ı} & \% aPu & $p$-value \\
%\midrule
%native  & 12      & 11      & 52     & \multirow{2}{*}{0.042} \\
%foreign & 84      & 28      & 75                              \\
%\bottomrule
%\end{tabular}
%\caption{?}
%\label{lae}
%\end{table}
%The loanword adaptation literature also contains many reports that statistically reliable phonotactic generalizations about the native lexicon are absent in processes of nativization \citep[e.g.,][]{Ito1995a,Ito1995b,Ussishkin2003}. It is not immediately clear that this is evidence that the generalizations in question are external to the synchronic grammar, however.
% 
%  & {harf}         & {harfler}  & `(alphabetic) letter' \\ 
%  & {el}           & {eller}    & `hand'         \\
%  & {sap}          & {saplar}   & `stalk'        \\
%  & {son}          & {sonlar}   & `end'          \\
%b. & grip    & \alt{} & gırip    & `grippe'    \\ % unexpectedly back
%   & kredi   & \alt{} & kıredi   & `credit'    \\
%  & {el}           & {eli}          & {elin}         & `hand'    \\
%  & {yüz}          & {yüzü}         & {yüzün}        & `face'    \\
%  & {pul}          & {pulu}         & {pulun}        & `stamp'   \\
%\begin{example}[Turkish progressives]
%\begin{tabular}{l l l l l}
%   & \emph{1sg. present}  & \emph{1sg. progressive} & \\
%a. & gelirim              & geliyorum               & `come'  \\
%b. & görürüm              & görüyorum               & `see'   \\
%c. & atarım               & atıyorum                & `throw' \\
%d. & bulurum              & buluyorum               & `find'  \\
%\end{tabular}
%\end{example}

