\label{turkish}

There are many precedents for the idea that statistical criteria could be used to differentiate \emph{accidental phonotactic gaps}, those which could arise without any antecedent cause, from \emph{structural phonotactic gaps} \citep[e.g.,][]{Fischer-Jorgensen1952,Saporta1955,Saporta1958,Vogt1954}.

``directly determine the mental representation of the phonotactic constraints'' \citep[180]{Frisch2004}.

In a study of consonant co-occurrence restrictions in Gitksan, \citet{Brown2010} claims that \emph{any} statistically significant pattern in the lexicon is one that is internalized by speakers:

\begin{quote}
\ldots{} the patterns outlined above are statistically significant. Given this, it stands that these sound patterns should be explained by some linguistic mechanism. \citep[][48]{Brown2010}
\end{quote}

Much of the latter-day literature on phonotactic knowledge, while adopting the representational machinery of generative phonology, takes less interest in the question of what generalizations speakers internalize. In a seminal paper, \citet{McCarthy1988} uses a simple statistical technique to validate a generalization about Arabic root consonant co-occurrence facts originally due to \citet{Greenberg1950}. The assumption appears to be that static co-occurrence restrictions which have a statistically significant trace in the lexicon will also be reflected in psycholinguistic tasks such as wordlikeness judgements. While the Arabic restriction discussed by \citeauthor{McCarthy1988} was ultimately investigated experimentally \citep{Frisch2004}, this is the case for only a small minority of the studies inspired by \citeauthor{McCarthy1988}'s study; in most cases, phonotactic generalizations are inferred directly from lexical statistics \citep[e.g.,][]{Anttila2008a,Berkley1994b,Berkley1994a,Berkley2000,Brown2010,Buckley1997,Coetzee2008a,Elmedlaoui1995,GraffInPress,MacEachern1999,Kinney2005,Kawahara2006,Martin2007,Martin2011,Mester1988,Miller-Ockhuizen2003,Padgett1992,Padgett1995,Pozdniakov2007,Yip1989}. 

There are several arguments against this inference. First, negative results in formal learnability theory (see \citealt{Yang2012} for a recent review) show that there are whole classes of rules and constraints which cannot be learned under standard assumptions about acquisition (e.g., that the data is finite, can appear in any order, and that speakers do not have to request grammatical judgements from adults to achieve adult-like competence). 

Secondly, there are a number of results which statistically reliable patterns are not learned by speakers \citep[e.g.,][]{Becker2011,Hayes2006,Hayes2009,HayesInPress}. This argues against a direct link between statistical criteria and the synchronic reality of phonotactic generalizations. The danger is that grammatical models which closely mimic lexical statistics but which lack psycholinguistic support (e.g., Gitksan: \citealt{Brown2010}; Muna: \citealt{Anttila2008a}, \citealt{Coetzee2008a}; Shona: \citealt[][385]{Hayes2008a}) may overstate the phonotactic generalizations internalized by speakers. 

Another type of non-learning occurs when generalizations are statistically reliable but demonstrably incorrect. \citet{Legate2012}, for instance, note that the overwhelming majority of English words have initial primary stress. While this suggests a quantity-insensitive stress system, English stress is in fact quantity-sensitive (see \citealt{Halle1998c} and references therein). 

Finally, there are many cases of statistical generalizations with a known diachronic cause but lacking any synchronic reality. One of the earliest arguments of this type was put forth by Saussure (\citeyear[202f.]{CLG}) in a discussion of rhotacism, a diachronic process which changed Old Latin intervocalic \emph{s} to \emph{r}. As a result, Classical Latin intervocalic \emph{s} is rare. However, Saussure denies that this generalization has any status in Classical Latin.\footnote{See \citet{GormanInPressc} and citations therein for a full explication of Saussure's position on rhotacism, and a restatement of the phonological argument in generative terms.} 

Similarly, [ʃ] in Modern English derives from Old English [sk] (e.g., \emph{fisc} `fish'), and since long vowels before complex codas are not found in Old English, [ʃ] is rarely preceded by long vowels in word-final syllables to this day according to \citet{Iverson2005}. The type frequencies of short and long vowels before final [ʃ] and a similar segment, [s] are shown in Table \ref{ssh}. As can be seen, long vowels before final [s] are twice as common before final [ʃ]; according to the appropriate statistical test (the Fisher exact test, described below), this is significant. Yet \citet{Iverson2005} label this constraint ``phonologically accidental'' since a millennium of speakers have produced coinages (e.g., \emph{posh}, \emph{swoosh}) and  borrowings (e.g., \emph{douche}, \emph{capiche}) that disregard this generalization.\footnote{In the loanword adaptation literature, there are many reports that statistically reliable generalizations over the native lexicon are ignored in nativization processes \citep[e.g.,][]{Ito1995a,Ito1995b,Ussishkin2003}. On the other hand, \citet{Frisch2001} suggest that novel borrowings in Egyptian Arabic rarely violate co-occurrence restrictions in that language; however, the generalization in question was exceptionless from Proto-Semitic \citep{Ehret1989} through Classical Arabic \citep{Greenberg1950}, so any violations of this generalization are unexpected.}

\begin{table}[ht]
\centering
\begin{tabular}{l r r r r}
\toprule
          & \{ɪ, ɛ, æ, ʌ, ʊ\}\gap{}\# & \{i, e, ɑ, ɔ, u\}\gap{}\# & \% long & $p$-value \\
\midrule
\gap{}ʃ\# & 78                & 9                 & 8\%      & \multirow{2}{*}{.026} \\
\gap{}s\# & 410               & 107               & 16\%     & \\
\bottomrule
\end{tabular}
\caption{Type frequencies of short and long vowels before word-final [ʃ] and [s] in a subset of the CMU pronunciation dictionary containing all words with token frequency greater than 1 per million words in the SUBTLEX-US frequency norms; the 9 words containing ending in a long vowel followed by ʃ are \emph{douche}, \emph{leash}, \emph{gosh}, \emph{josh}, \emph{posh}, \emph{squash}, \emph{unleash}, \emph{wash}, and \emph{woosh}.}
\label{ssh}
\end{table}

This chapter is a comprehensive investigation of three phonotactic generalizations in Turkish, comparing lexical statistics and the results of a wordlikeness task performed by \citet{Zimmer1969}. Both the lexical statistics and \citeauthor{Zimmer1969}'s experimental results merit reconsideration, because prior discussions do not relate the lexical statistics to competing formalizations of the generalizations involved, and there is widespread disagreement concerning the experimental findings.

One of the three generalizations, \textsc{Labial Attraction}, is shown to   state a statistically reliable generalization about the Turkish lexicon, but has no effect on speakers' wordlikeness judgements. This further undermines the widespread use of statistical criteria to determine which of the vast number of statistically reliable phonotactic generalizations are internalized by speakers and which are not. The simplest hypothesis is that \textsc{Labial Attraction} is inactive in wordlikeness tasks because it is static, corresponding to no phonological process in the language.

\section{Turkish vowel sequence structure constraints}

\citet{Lees1966b,Lees1966a} proposes three constraints on Turkish vowel sequences; these constraints are the focus of many subsequent studies. In this section, these constraints are formalized, and where possible, related to phonological alternations and to behavioral evidence bearing on speakers' knowlege of the restrictions. The following feature specification for the eight vowels of Turkish is assumed throughout.

\begin{example}[Turkish vowel features]
\begin{tabular}{c c c c c}
                       & \multicolumn{2}{c}{[$-$\textsc{Back}]} & \multicolumn{2}{c}{[$+$\textsc{Back}]} \\
                       & [$-$\textsc{Rnd}] & [$+$\textsc{Rnd}] & [$-$\textsc{Rnd}] & [$+$\textsc{Rnd}] \\ 
\cmidrule{2-5}
\buf[$+$\textsc{High}] & {i} & {ü} [y] & {ı} [ɯ] & {u} \\
\buf[$-$\textsc{High}] & {e} & {ö} [ø] & {a} [ɑ] & {o} \\
\end{tabular}
\end{example}

Two less familiar notations are used in this chapter. First, directional application conditions are assumed, so as to derive the left-to-right, iterative properties of the harmony rules. Since it was first proposed by \citet{Johnson1972}, considerable evidence for directional application has been adduced (e.g., \citealt{A74}: chap.~9, \citealt{GormanInPressc}, \citealt{Howard1972}:65f., \citealt{Kavitskaya2008}, \citealt{Kaye1982}, \citealt{KK77}:189f., \citeyear{KK79}:319f., \citealt{Piggott1975}, \citealt{Sohn1971}; see \citealt{McCarthy2003b} and \citealt{Wolf2011b} for recent reviews.) Further, \citet{Johnson1972} and \citet{Kaplan1994} prove directional application and simultaneous application are equivalent in terms of formal learnability.

Secondly, rather than the use of an unbounded number of Greek-letter variables ($\alpha$, $\beta$, etc.) over feature values $\{+, -\}$, only a single variable, denoted by `$=$', is used \citep{McCawley1973}. A structural description [$=$F]\ldots{}[$=$F] matches a string S$_i$\ldots{}S$_j$ if and only if S$_i$ and S$_j$ are both [$+$F] or both [$-$F]. This is more restrictive than Greek-letter notation, in that it prevents the value of one feature being applied to another. \citet{Odden2012} argues that the two counterexamples against this restriction adduced in \emph{SPE} (352-353) are not probative.

\subsection{Backness harmony}

\citeauthor{Lees1966b} (\citeyear[35]{Lees1966b}, \citeyear[284]{Lees1966a}) models the Turkish vowel harmony system with three rules. The most general of these rules spreads the specification [\textsc{Back}] rightward.

\begin{example}[\textsc{Backness Harmony} (condition: rightward application)]
$\begin{bmatrix} -\textsc{Cons} \end{bmatrix}~\goesto~\begin{bmatrix} =\textsc{Back} \end{bmatrix}~\big /~\begin{bmatrix} =\textsc{Back} \end{bmatrix}~\textrm{C}_0~\gap$
\end{example}

\noindent
A vowel becomes [$+$\textsc{Back}] after a [$+$\textsc{Back}] vowel, and [$-$\textsc{Back}] after a [$-$\textsc{Back}] vowel, ignoring any intervening consonants. The application of this rule proceeds from left to right; no vowel may be skipped.

If permitted to apply in non-derived environments, this rule accounts for the tendency of polysyllabic roots to contain only [$+$\textsc{Back}] or [$-$\textsc{Back}] vowels, a tendency which will be quantified below. \textsc{Backness Harmony} also triggers alternations in inflectional suffix vowels. For instance, the nominative plural (nom.pl.) suffix is \emph{-ler} when the final root vowel is [$-$\textsc{Back}], and \emph{-lar} when it is [$+$\textsc{Back}].

\begin{example}[The Turkish nominative]
\label{turknom}
\begin{tabular}{lllll}
   & \emph{nom.sg.} & \emph{nom.pl.} \\
a. & {ip}           & {ipler}    & `rope'         & \citep[][216]{Clements1982} \\
   & {köy}          & {köyler}   & `village'      \\
   & {yüz}          & {yüzler}   & `face'         \\
   & {kız}          & {kızlar}   & `girl'         \\
   & {pul}          & {pullar}   & `stamp'        \\
b. & {neden}        & {nedenler} & `reason'       & \citep{TELL} \\
   & {kiler}        & {kilerler} & `pantry'       \\
   & {pelür}        & {pelürler} & `onionskin'    \\
   & {boğaz}        & {boğazlar} & `throat'       \\
   & {sapık}        & {sapıklar} & `pervert'      \\
\end{tabular}
\end{example}

A few complications arise, however. First, as shown in (\ref{turkexcept}a), not all polysyllabic roots conform to \textsc{Backness Harmony}. In this case, suffix vowels generally exhibit harmony with the final root vowel. There is also a very small class of nouns, shown in (\ref{turkexcept}b), which take \emph{-ler}, although their final root vowel is [$+$\textsc{Back}]. Interestingly, they may themselves be harmonic.

\begin{example}[Exceptional Turkish nominatives] 
\label{turkexcept}
\begin{tabular}{lllll}
   & \emph{nom.sg.} & \emph{nom.pl.}&                    \\
a. & {mezar}        & {mezarlar}    & `grave'       & \citep{TELL}       \\
   & {model}        & {modeller}    & `model'                            \\
   & {silah}        & {silahlar}    & `weapon'                           \\
   & {memur}        & {memurlar}    & `official'                         \\
   & {sabun}        & {sabunlar}    & `soap'                             \\
b. & {etol}         & {etoller}     & `fur stole'   & \citep{Goksel2005} \\
   & {saat}         & {saatler}     & `hour, clock' 	              \\
   & {kahabat}      & {kahabatler}  & `fault'       \\
\end{tabular}
\end{example}

\citet[212]{A74} and \citet{Iverson1978} argue that suffix harmony in disharmonic roots found in (\ref{turkexcept}a) requires the rule governing suffix harmony alternations to be distinguished from a sequence structure constraint governing root harmony. Since the rule and sequence structure constraint are otherwise identical, this constitutes an undesirable ``duplication'' in the sense of \citealt{Kisseberth1970b}. However, the theory of exceptionality proposed in \emph{SPE} can account for these facts without duplication \citep[197f.]{Zonneveld1978}.\footnote{\citet[29f.]{Kiparsky1968} discusses a parallel case in Finnish, and proposes a similar separation between the exception-filled sequence structure constraint and an exceptionless suffix harmony process. \citet[171f.]{Howard1972} correctly observes that this duplication is unnecessary.}

\citeauthor{SPE} assume that the specification of the target (i.e., the segment or segments to be changed) of a rule \emph{R} must be marked [$+$\emph{R}] by convention. A root or affix which fails to undergo \emph{R} despite otherwise matching the structural description is simply said to be marked [$-$\emph{R}]. In other words, no form is never truly an ``exception''; rather, some underlying representations have non-default features which do not match the extended structural description of $R$ which requires that the target be [$+R$]. If disharmonic roots are marked [$-$\textsc{Backness Harmony}], then the final vowel of disharmonic roots will still trigger \textsc{Backness Harmony}, since the [$-$\textsc{Backness Harmony}] root is no longer the target but rather the trigger, which is not subject to the [$+$\textsc{Backness Harmony}] requirement.

Under this account, root and suffix harmony are derived from \textsc{Backness Harmony}, but they do not have the same ontology: suffix harmony is direct result of phonological rule application, whereas any ``effects'' of the  generalization concerning harmonic roots arises from a dispreference for lexical exceptionality.

\citeauthor{A74} also notes that roots which fail to undergo suffix harmony, like (\ref{turkexcept}b), may themselves be harmonic. He takes to be evidence for the necessity of duplication:

\begin{quote}
\ldots{}there are words which are exceptions to harmony across boundaries (e.g., \emph{kabahat} `fault', \emph{kabahattı} `his fault') but which are perfectly regular internally. Since the morpheme structure condition and the phonological rule in this case have distinct classes of exceptions, it is clear that they cannot be identified. \citep[289]{A74}
\end{quote}

\noindent
Under the assumptions so far, there is no reason to think that \emph{kabahat} is [$+$\textsc{Backness Harmony}], however: even roots with all back vowels need not undergo this rule. Whatever the ultimate analysis of those few words which fail to show suffix harmony, nothing depends on whether or not they are harmonic.

%\begin{example}[Turkish progressives]
%\begin{tabular}{l l l l l}
%   & \emph{1sg. present}  & \emph{1sg. progressive} & \\
%a. & gelirim              & geliyorum               & `come'  \\
%b. & görürüm              & görüyorum               & `see'   \\
%c. & atarım               & atıyorum                & `throw' \\
%d. & bulurum              & buluyorum               & `find'  \\
%\end{tabular}
%\end{example}

It is necessary to dispense with an alternative analysis proposed by \citet{Clements1982} and \citet{Inkelas1997}. This analysis has desirable properties, but a subtlety of it reintroduces the duplication of rule and sequence structure constraint. Root vowels exhibit a robust contrast for backness (e.g., \emph{kül} `ash' vs. \emph{kul} `servant', \emph{kepek} `bran' vs. \emph{kapak} `lid'), whereas backness of vowels in non-initial syllables is predictable in harmonic roots; there are no prefixes in Turkish. \citeauthor{Clements1982} propose that these vowels, as well as harmonizing suffix vowels, are underspecified for backness, whereas the non-initial vowels of disharmonic roots and of certain exceptional suffixes are fully specified. This is schematized below.

\begin{example}[Autosegmental underspecification in harmonic roots (after \citealp{Clements1982})] 
\label{spec}
\xymatrix@R=24pt@C=24pt{
\txt{a.} & \txt{harmonic root:} & \txt{C} & \txt{V} & \txt{C} & \txt{V} & \txt{C} \\
         &                      &         & \txt{[$-$\textsc{Back}]}\ar@{-}[u]\ar@{--}[urr] \\
\txt{b.} & \txt{disharmonic root:} & \txt{C} & \txt{V} & \txt{C} & \txt{V} & \txt{C} \\
         &                      &         & \txt{[$-$\textsc{Back}]}\ar@{-}[u] & & \txt{[$+$\textsc{Back}]}\ar@{-}[u]
}
\end{example}

One crucial detail is missing from this analysis: \textsc{Backness Harmony} needs to be prevented from overwriting the [$+$\textsc{Back}] specification of disharmonic roots, perhaps by structure preservation condition \citep{Kiparsky1985}. However, any condition which prevents \textsc{Backness Harmony} from overwriting underlying backness specifications will reintroduce the duplication of sequence structure and phonological generalizations; under this analysis, disharmonic roots are no longer exceptional, despite considerable evidence (reviewed below) that they are marked in Turkish.\footnote{On the other hand, it is possible to interpret the presence of a single backness specification per root as a sort of default. A precedent for this is the surface-oriented interpretation of the tonal Obligatory Contour Principle proposed by \citet[134]{Goldsmith1976} and \citet{Odden1986}, under which adjacent identical tones are automatically attributed to a single underlying tone. However, this is merely a notational variant of the rule exceptionality account in which [$+$\textsc{Backness Harmony}] is the default.} The the underspecification analysis is rejected here on these grounds.

While harmony in non-derived environments can be inferred from the aforementioned suffix alternations, no evidence has yet been presented to show that Turkish speakers internalize the tendency for roots to conform to backness harmony. If Turkish speakers do not attend to this generalization, there is no need for the grammar to account for it. Several other ``external'' facts suggest that this is not the case. The discussion here is not intended to imply uncritical acceptance of evidence from loanword adaptation, language games or psycholinguistic tasks as evidence for phonological grammar, but rather to illustrate additional evidence that is pertinent if the linking hypothesis is correct.\footnote{Thanks to Bert Vaux and Kie Zuraw for bringing these studies to my attention.}

The production of non-native word-initial onset clusters, discussed by \citet{Clements1982} and \citet{Kaun1999}, suggests that loanword adaptation respects \textsc{Backness Harmony}. Some speakers are said to be capable of pronouncing these non-native clusters, but in fast speech the cluster is split by anaptyxis. In most cases, this vowel matches the following root vowel for backness.

\begin{example}[Variable non-native cluster adaptation (\citealp{Clements1982}:247)] 
\begin{tabular}{lllll}
a. & {spiker}  & \alt{} & {sipiker}  & `announcer' \\
   & {fren}    & \alt{} & {firen}    & `brake'     \\
b. & {trablus} & \alt{} & {tırablus} & `Tripoli'   \\
   & {kral}    & \alt{} & {kıral}    & `king'      \\
c. & {brom}    & \alt{} & {burom}    & `bromide'   \\
   & {prusya}  & \alt{} & {purusya}  & `Prussia'   \\
\end{tabular}
\label{spiker}
\end{example}

\noindent
It is unclear whether the cluster-splitting vowel is deleted in the non-native variant or epenthesized in the fast speech variant. Under either analysis, there is no ready explanation for the tendency of the cluster-splitting vowel to have the backness features of following consonants; if anything, one might expect it to determine the backness features of following consonants. All that can be said with certainty is that the adaptation of non-native onset clusters appears to proceed in such a fashion so that the lexical items in question are [$-$\textsc{Backness Harmony}].

Similar evidence comes from a language game discussed by \citet{Harrison2001}. The game is native to the related language Tuvan, where it is used to convey a sense of ``vagueness or jocularity''; it is not indigenous to Turkish, but can be taught quickly to children or adults. In this game, the base is reduplicated and the first vowel of the reduplicant replaced with a [$+$\textsc{Back}] vowel. In (\ref{redupgame}a), the second [$-$\textsc{Back}] vowel of the base is, in the reduplicant, ``reharmonized'' with the inserted [$+$\textsc{Back}] vowel. The disharmonic roots of (\ref{redupgame}b) do not reharmonize.\footnote{A similar contrast between harmonic and disharmonic roots is found in Tuvan \citep{Harrison2001} and in an unrelated Finnish language game \citep{Campbell1986}.}

\begin{example}[Turkish reduplication game (\citealp{Harrison2001}:231)] 
\label{redupgame}
\begin{tabular}{llll}
a. & {kibrit} & {kibrit}-\{{kabrıt}\} & `match'    \\
   & {bütün}  & {bütün}-\{{batın}\}   & `whole'    \\
b. & {mali}   & {mali}-\{{muli}\}     & `Mali'     \\
   & {butik}  & {butik}-\{{batik}\}   & `boutique' \\
\end{tabular}
\end{example}

\noindent 
In the full specification analysis adopted here, reharmonization is the result of \textsc{Backness Harmony} applying within the reduplicant. On the other hand, the lack of reharmonization in the reduplicants of disharmonic roots suggests that the [$-$\textsc{Backness Harmony}] exception feature is copied under reduplication.\footnote{There is reason to believe that reduplicants bear lexical diacritics. Kinande reduplication, documented by \citet{Mutaka1990} and further discussed by \citet{Downing2000}, provides such a case. Verb reduplication requires a bisyllabic reduplicant. This clearly has synchronic force in the gramar, since reduplicated monosyllabic roots actually have three copies of the root, and since reduplicated forms of many trisyllable verbs are ineffable. However, a few trisyllabic verbs exceptionally show full reduplication. This is a lexical property, but one that is only seen on the reduplicant.} \citet{Silverman2000} notes that static phonotactic generalizations are not preserved by operations like reduplication and truncation: under this definition, root harmony is not static.

A number of studies have investigated the role of harmony in word-spotting tasks, though to mimic auditory word recognition and segmentation in natural settings. Many of these studies have been carried out in Finnish, which has a vowel harmony system similar to that of Turkish. \citet{Suomi1997} and \citet{Vroomen1998} task Finnish speakers with identifying harmonic disyllables in an auditory stream. When the syllable preceding the disyllabic target has a different backness specification than the target, recognition of the target is facilitated. Presumably, disharmony facilitates the recognition of word boundaries. \citet{Kabak2010} find that Turkish \textsc{Backness Harmony} has a similar effect: Turkish speakers are quicker and more accurate at the task of spotting the nonce target word \emph{pavo} when preceded by a disharmonic juncture (e.g., \emph{gölü-PAVO}) than when preceded by a harmonic juncture (e.g., \emph{golu-PAVO}). \citet{Kabak2010} find that this effect does not obtain for speakers of French, a language which lacks vowel harmony. This implies that Turkish speakers have internalized the tendency of harmonic sequences to be root-internal and of disharmonic transitions to cross word boundaries. 

The Turkish word-spotting experiment is adapted for infants by \citet{Kampen2008}. In this study, 9-month-old infants are familiarized with an auditory stream containing harmonic disyllabic nonce words. At test time, the infants listen to the disyllabic nonce words in isolation using the head turn preference paradigm. Infants acquiring Turkish listen longer to nonce words preceded by a disharmonic juncture during familiarization (e.g., \emph{lo-NETIS}), whereas infants acquiring German, a language which lacks vowel harmony, do not exhibit this preference. Similarly, \citeauthor{Kampen2008} report that Turkish 6-month-old infants prefer to listen to harmonic nonce words such as \emph{paroz} over disharmonic nonce words like \emph{nelok}, but German 6-month-old infants show no such preference.

It is not clear, however, that the segmentation effects are evidence for grammatical generalizations; for instance, while the effect is defined in terms of harmony, it operates over a domain much larger than the prosodic word over which harmony is computed.

\subsection{Roundness harmony}

This rule is quite similar to \textsc{Backness Harmony}, but imposes an additional restriction, that targets be [$+$\textsc{High}]. 

\begin{example}[\textsc{Roundness Harmony} (condition: rightward application)]
$\begin{bmatrix} -\textsc{Cons} \\ +\textsc{High} \end{bmatrix}~\goesto~\begin{bmatrix} =\textsc{Rnd} \end{bmatrix}~/~\begin{bmatrix}~=\textsc{Rnd}~\end{bmatrix}~\textrm{C}_0~\gap$
\end{example}

\noindent
A [$+$\textsc{High}] vowel becomes [$+$\textsc{Rnd}] after a [$+$\textsc{Rnd}] vowel, and [$-$\textsc{Rnd}] after a [$-$\textsc{Rnd}] vowel, ignoring any intervening consonants, and applying from left to right. 

If permitted to apply in non-derived environments, this rule accounts for the tendency of polysyllabic roots to contain only [$+$\textsc{Rnd}] or [$-$\textsc{Rnd}] if the non-initial vowels are [$+$\textsc{High}]. In concert with \textsc{Backness Harmony}, \textsc{Roundness Harmony} also triggers alternations which account for the shape of the dative singular (dat.sg.) and genitive singular (gen.sg.) among other suffixes. As was the case for \textsc{Backness Harmony}, disharmonic roots are found, and the final vowel of disharmonic roots triggers suffix harmony.

\begin{example}[Turkish nominal suffix allomorphy]
\begin{tabular}{lllllll}
   & \emph{nom.sg.} & \emph{dat.sg.} & \emph{gen.sg.}  \\
a. & {ip}           & {ipi}          & {ipin}         & `rope' & \citep[][216]{Clements1982} \\
   & {kız}          & {kızı}         & {kızın}        & `girl'    \\
   & {sap}          & {sapı}         & {sapın}        & `stalk'  \\
   & {köy}          & {köyü}         & {köyün}        & `village' \\
   & {son}          & {sonu}         & {sonun}        & `end'     \\
b. & {boğaz}        & {boğazı}       & {boğazın}      & `throat'  & \citep{TELL} \\
   & {pelür}        & {pelürü}       & {pelürün}      & `onionskin' \\
   & {döviz}        & {dövizi}       & {dövizin}      & `currency'  \\
   & {yamuk}        & {yamuğu}       & {yamuğun}      & `trapezoid' \\
   & {ümit}         & {ümiti}        & {ümitin}       & `hope'      \\
\end{tabular}
\end{example}

Few studies have directly investigated whether speakers are aware of the tendency for roots to conform to \textsc{Roundness Harmony}. However, two pieces of the external evidence on \textsc{Backness Harmony} bear on this question. First, the cluster-splitting vowel found in non-native onset clusters
, discussed above, tends to agree in roundness with following high vowels (e.g., \emph{prusya}-\emph{purusya} `Prussia'). Secondly, \textsc{Roundness Harmony} participates in reharmonization in the language game described by \citeauthor{Harrison2001}: the second \emph{ü} in \emph{bütün} `whole' reharmonizes to the [$-$\textsc{Rnd}] vowel \emph{ı} in reduplicated \emph{bütün-batın}.

\subsection{Labial attraction}

\citet{Lees1966b} describes \textsc{Labial Attraction} as a phonological process by which ``a high, short harmonic vowel is rounded in the second syllable of a disyllabic word whose first vowel is /a/, and whose medial consonant cluster contains a labial /p, b, m, v/, and then it is de-harmonified'' (36).

\begin{example}[\textsc{Labial Attraction}]
$\begin{bmatrix} -\textsc{Cons} \\ +\textsc{Back} \\ +\textsc{High} \end{bmatrix}~\goesto~\begin{bmatrix} +\textsc{Rnd} \end{bmatrix}~/~\textrm{ɑ}~\textrm{C}_0~\begin{bmatrix} +\textsc{Cons} \\ +\textsc{Labial} \end{bmatrix}~\textrm{C}_0~\gap{}$
\end{example}

This rule is significantly more complex than the harmony rules, and this may obscure the fact that it produces exceptions to \textsc{Roundness Harmony}, producing \emph{a}C$_0$\emph{u} (e.g., \emph{çapul} `raid', \emph{sabur} `patient', \emph{şaful} `wooden honey tub', \emph{avuç} `palm of hand', \emph{samur} `sable'; \citealp[285]{Lees1966a}) rather than the expected \emph{a}C$_0$\emph{ı}. However, \textsc{Labial Attraction} does not apply in derived environments: the gen.sg. of \emph{sap} `stalk is \emph{sapın} rather than *\emph{sapun} that would be predicted if \textsc{Labial Attraction} triggered alternations. \citet[286]{Lees1966a} and \citet[311]{Zimmer1969} note the existence of root-internal exceptions to this rule (e.g., \emph{tavır} `mode') but they agree that they are surprisingly rare.

\section{Evaluation}
\label{3evaluation}

Others dispute that \textsc{Labial Attraction} is a reliable generalization about Turkish roots. \citet{Clements1982} point to additional exceptions to the rule, and \citet{Inkelas2001} provide lexical counts from the Turkish Electronic Living Lexicon \citep[TELL;][]{TELL}, a database of words known to two native Turkish speakers. However, no prior study reports any inferential statistics.

\citet[311]{Zimmer1969} administers two paired wordlikeness tasks designed to evaluate native speakers' knowledge of \textsc{Backness Harmony}, \textsc{Roundness Harmony}, and \textsc{Labial Attraction} in roots. Speakers are presented with two nonce words and simply have to indicate the nonce word that is more Turkish-like.\footnote{Compared to the unpaired ratings tasks  commonly used in wordlikeness research, paired rating tasks have considerably more statistical power \citep{Gigerenzer2004}. When items under comparison differ only in whether or not they conform to the phonotactic generalizations, it is less likely that any contrast between the members of the pairs are caused by some omitted variable. Also, there is no unnecessary deception regarding the purpose of the experiment, which is known to reduce spurious response variability. Consequently, paired tasks are particularly sensitive to small phonotactic contrasts and are ideal for collecting wordlikeness judgements.} \citeauthor{Zimmer1969} concludes that the former two rules are reflected in wordlikeness judgements, whereas \textsc{Labial Attraction} is not. 

Below, both lexical statistics and \citeauthor{Zimmer1969}'s wordlikeness results are analyzed statistically; \textsc{Labial Attraction} is shown to be a statistically robust generalization over the Turkish lexicon, but no variant of \textsc{Labial Attraction} is reflected in \citeauthor{Zimmer1969}'s wordlikeness study. In contrast, the two harmony processes have robust effects both on the lexicon and on wordlikeness.  This dissociation between lexical generalizations and wordlikeness results provides further evidence against the common practice of inferring phonotactic knowledge directly from lexical statistics.

\subsection{Lexical statistics}

Counts were computed by regular expression matching on a 9,601-root subset of the TELL database which consists of words which show no surface variation between the two informants in any inflected form.

To test for associations between the process (more specifically, the constraint that it imposes on roots) and type frequency in this database, each root was sorted into a $2 \times 2$ contingency table; the contents of each cell are specific to the process in question. The counts in this table are not expected to add up to 9,601, since many roots neither could exemplify nor violate the process in question; for instance, monosyllabic words are irrelevant to root harmony. The Fisher exact test is used to compute a $p$-value representing the probability of the observed data arising under the null hypothesis that there is no association between the process and type frequency.

\subsubsection{Backness harmony}

\textsc{Backness Harmony} is exemplified in the lexicon insofar as there is an association between the backness of vowels in all adjacent syllables. Any disagreement on the backness specification of vowels in adjacent syllables are counted solely as exceptions, even if other vowel transitions in the root are harmonic. 
%This follows from \emph{SPE} theory of lexical exceptionality (see also \citealt{Gouskova2012}), in which phonological exceptionality is a property of underlying representations, not individual segments. 
To construct the contingency table, roots are binned according to the backness specification of the first nucleus, and that of following nuclei. For example, the first two syllables of \emph{adalet} `justice' are harmonic, but it is coded as disharmonic because there is a \emph{a\ldots{}e} transition later in the word. The resulting counts are shown in Table \ref{bhs}. 61\% of roots conform to \textsc{Backness Harmony}, and the interaction between the backness of the first and of the subsequent vowels
%predicted by root-internal \textsc{Backness Harmony} 
is significant.

\begin{table}[ht]
\centering
\begin{tabular}{lrrr}
\toprule
                             & [$+$\textsc{Back}]$_1$ & [$-$\textsc{Back}]$_1$ & $p$-value                     \\
\midrule
\buf{}[$+$\textsc{Back}]$_{2\ldots{}n}$ & 3,089                     & 1,704              & \multirow{2}{*}{$1.19$\e{-89}} \\
\buf{}[$-$\textsc{Back}]$_{2\ldots{}n}$ & 1,698                     & 2,250                                               \\
\bottomrule
\end{tabular}
\caption{TELL roots sorted according to \textsc{Backness Harmony}}
\label{bhs}
\end{table}

\subsubsection{Roundness harmony}

\textsc{Roundness Harmony} predicts correlation between the roundness of a vowel and the roundness of high vowels in the next syllable. Any root for which a vowel does not agree in roundness with a high vowel in the following syllable (e.g., \emph{ümit}) is considered to be an exception. Roots are binned according to the roundness of non-initial high vowels, and according to the roundness of the preceding vowel. 
%Roots lacking non-initial high vowels do not bear on the status of root \textsc{Roundness Harmony}. 
The resulting counts are shown in Table \ref{rhs}. 83\% of the roots conform to \textsc{Roundness Harmony}, and the interaction between the roundness of the $i$th vowel and the roundness of the $(i +1)$th high vowel is significant.

\begin{table}[ht]
\centering
\begin{tabular}{lrrr}
\toprule
                                              & [$+$\textsc{Rnd}]$_i$ & [$-$\textsc{Rnd}]$_i$ & $p$-value                      \\
\midrule
\buf{}[$+$\textsc{High}, $+$\textsc{Rnd}]$_{i+1}$ & 613                   &   261                 & \multirow{2}{*}{$1.02$\e{-36}} \\
\buf{}[$+$\textsc{High}, $-$\textsc{Rnd}]$_{i+1}$ & 581                   & 2,841                                                  \\
\bottomrule
\end{tabular}
\caption{TELL roots sorted according to \textsc{Roundness Harmony}}
\label{rhs}
\end{table}

\noindent
The counts in the bottom row of Table \ref{rhs} contain a number of roots which are apparent exceptions to \textsc{Roundness Harmony} but conform to \textsc{Labial Attraction} (bottom left), and which conform to \textsc{Roundness Harmony} at the expense of \textsc{Labial Attraction} (bottom right). Excluding these types of roots would have the effect of slightly increasing the overall rate of \textsc{Roundness Harmony}, since the former is more common.

\subsubsection{Labial attraction}

Reviewing the results of wordlikeness experiments, \citet{Zimmer1969} proposes a third variant of \textsc{Labial Attraction} which is insensitive to intervening consonant place. This is formalized below. 

\begin{example}[\textsc{Labial Attraction} (revised by \citealt{Zimmer1969})]
$\begin{bmatrix} -\textsc{Cons} \\ +\textsc{Back} \\ +\textsc{High} \end{bmatrix}~\goesto~\begin{bmatrix} +\textsc{Rnd} \end{bmatrix}~/~\textrm{ɑ}~\textrm{C}_0~\gap{}$
\end{example}

\noindent
\citet{Clements1982} also  to \textsc{Labial Attraction}, arguing that it is not ``systematic''.

\begin{quote}
Even more decisive evidence against a rule of Labial Attraction is the existence of a further, much larger set of roots containing /\ldots{}aCu\ldots/ sequences in which the intervening consonant or consonant cluster does not contain a labial\ldots{}We conclude that there is no systematic restriction on the set of consonants that may occur medially in roots of the form /\ldots{}aCu\ldots/. \citep[225]{Clements1982}
\end{quote}

\noindent 
This claim can be evaluated using the Fisher exact test. Let P denote a sequence of one or more consonants, one of which is labial, and let T denote a sequence of one or more consonants none of which is labial. The null hypothesis is that \emph{a}P\emph{u} sequences, which conform to \textsc{Labial Attraction}, are no more likely than would be expected from other \emph{a}T\emph{u} sequences violating \textsc{Roundness Harmony}. The resulting counts are shown in Table \ref{las}. Whereas the sequence \emph{a}P\emph{u} is more than twice as likely as \emph{a}P\emph{ı}, the sequence \emph{a}T\emph{u} is 5 times less likely than \emph{a}T\emph{ı}. This interaction is significant, as predicted by \textsc{Labial Attraction}, but contradicting a statistical interpretation of the \citeauthor{Clements1982}'s claim. In fact, \emph{a\ldots{}u} sequences are less, not more, common than \emph{a\ldots{}ı}.

\begin{table}[ht]
\centering
\begin{tabular}{lrrr}
\toprule
       & a\ldots{}u & a\ldots{}ı & $p$-value                      \\
\midrule
aP\ldots{} & 124    & 57     & \multirow{2}{*}{$1.02$\e{-36}} \\
aT\ldots{} & 136    & 590    &                                \\
\bottomrule
\end{tabular}
\caption{TELL roots sorted according to \textsc{Labial Attraction}}
\label{las}
\end{table}

\noindent
\citet{Inkelas2001} propose instead that the trigger must contain a labial consonant but does not require a preceding \emph{a} vowel.

\begin{quote}
Vowel labialization following labials is not a synchronic alternation in Turkish, yet it (unlike \textsc{Labial Attraction} per se) \emph{is} a statistically supported tendency worthy of further research. \citep[196]{Inkelas2001}
\end{quote}

\noindent
This is formalized below.

\begin{example}[\textsc{Labial Attraction} (revised by \citealt{Inkelas2001})]
$\begin{bmatrix} -\textsc{Cons} \\ +\textsc{Back} \\ +\textsc{High} \end{bmatrix}~\goesto~\begin{bmatrix} +\textsc{Rnd} \end{bmatrix}~/~\begin{bmatrix} +\textsc{Cons} \\ +\textsc{Labial} \end{bmatrix}~\gap{}$
\end{example}

Counts from TELL for this formulation of the process are shown in Table \ref{lasi}. This triples the number of roots which exemplify \textsc{Labial Attraction} (top left cell) but also slightly increases the number of exceptions. The interaction remains significant.

\begin{table}[ht]
\centering
\begin{tabular}{lrrr}
\toprule
       & \ldots{}u  & \ldots{}ı & $p$-value                      \\
\midrule
P\ldots{}  & 371    & 71        & \multirow{2}{*}{$6.98$\e{-49}} \\
T\ldots{}  & 811    & 922       &                                \\
\bottomrule
\end{tabular}
\caption{TELL roots sorted according to the reformulation of \textsc{Labial Attraction} proposed by \citet{Inkelas2001}}
\label{lasi}
\end{table}

\subsection{Wordlikeness ratings}

\citet{Zimmer1969} administers two variants of the paired nonce word rating task. The first used 23 native adult speakers who were permitted to select either nonce word as more like Turkish, or to indicate `no preference'. For the purposes below, `no preference' results are discarded. The second experiment used 32 native adults, none of whom appeared in the preceding study, and used a forced binary choice.

Each response is coded as \emph{concordant} if the nonce word conforming to the process is preferred, and \emph{discordant} if the disharmonic word is selected. To test for an association between the constraints, a non-parametric statistic, the Goodman-Kruskal (\citeyear{Goodman1954}) $\gamma$ is computed from the count of concordant ($c$) and discordant ($d$) pairs:

\begin{equation*}
\displaystyle \gamma = \frac{c - d}{c + d}
\end{equation*}

\noindent
The $\gamma$ statistic ranges between -1, in the case that all paired choices are discordant, and 1, if all paired choices are concordant.\footnote{It also is possible to perform statistical tests aggregating over items, but for this small number of items, such tests have very low statistical power.}

\subsubsection{Backness harmony}

Both of the \citet{Zimmer1969} experiments include 5 pairs which differ in whether or not the nonce words conform to, or violate, \textsc{Backness Harmony}. As can be seen from Table \ref{bhw}, harmonic pairs are preferred approximately 6-to-1, and aggregating over speakers, no disharmonic member of a pair is favored. Speakers have a highly reliable preference for nonce words which exhibit \textsc{Backness Harmony} ($\gamma = 0.694$, $p = 1.7$\e{-59}). It is interesting to note that the disharmonic nonce word which has the highest rating is found in the pair \emph{terüz}-\emph{teruz}, both of which violate \textsc{Roundness Harmony}. While this is little more than an anecdote, this may be  indicative of a link between the two processes, and their exceptions, in the minds of native speakers. 

\begin{table}[ht]
\centering
\begin{tabular}{lrlr|lrlr}
\toprule
\multicolumn{4}{c|}{Experiment 1} & \multicolumn{4}{c}{Experiment 2} \\
\multicolumn{2}{c}{\textsc{harmonic}} & \multicolumn{2}{c|}{\textsc{disharmonic}} & \multicolumn{2}{c}{\textsc{harmonic}} & \multicolumn{2}{c}{\textsc{disharmonic}} \\
\midrule
{temez} & 19            & {temaz} & 3 & {pemez} & 30            & {pemaz} & 2 \\
{teriz} & 23            & {terız} & 0 & {teriz} & 28            & {terız} & 3 \\
{tokaz} & 21            & {tokez} & 1 & {tokaz} & 26            & {tokez} & 6 \\
{tipez} & 21            & {tipaz} & 1 & {tipez} & 24            & {tipaz} & 8 \\
{terüz} & 20            & {teruz} & 1 & {terüz} & 19            & {teruz} & 13 \\
\bottomrule
\end{tabular}
\caption{Effects of \textsc{Backness Harmony} on wordlikeness \citep[from][]{Zimmer1969}}
\label{bhw}
\end{table}
% harmonic responses are favored approximately 6 to 1.
%$\gamma = 0.717$, $p = 7.5$\e{-69}

\subsubsection{Roundness harmony}

Both experiments include 5 pairs which differ in the presence or absence of \textsc{Roundness Harmony}. As shown in Table \ref{rhw}, there is an approximately 5-to-1 preference for harmonic nonce words, and as was the case above, no disharmonic member of any pair is overall preferred. Turkish speakers have a reliable preference for nonce words to conform to \textsc{Roundness Harmony} ($\gamma = 0.680$, $p = 1.1$\e{-47}).

\begin{table}[ht]
\center
\begin{tabular}{lrlr|lrlr}
\toprule
\multicolumn{4}{c|}{Experiment 1} & \multicolumn{4}{c}{Experiment 2} \\
\multicolumn{2}{c}{\textsc{harmonic}} & \multicolumn{2}{c|}{\textsc{disharmonic}} & \multicolumn{2}{c}{\textsc{harmonic}} & \multicolumn{2}{c}{\textsc{disharmonic}} \\
\midrule
{törüz} & 19 & {töriz} & 1 & {pörüz} & 32 & {pöriz} & 0  \\
{tüpüz} & 22 & {tüpiz} & 0 & {tüpüz} & 31 & {tüpiz} & 1  \\
{takız} & 15 & {takuz} & 3 & {takız} & 22 & {takuz} & 10 \\
{tatız} & 12 & {tatuz} & 6 & {tatız} & 20 & {tatuz} & 12 \\
\bottomrule
\end{tabular}
\caption{Effects of \textsc{Roundness Harmony} on wordlikeness \citep[from][]{Zimmer1969}}
\label{rhw}
\end{table}

\subsubsection{Labial attraction}

Both experiments include 5 pairs which either conform to \textsc{Labial Attraction} and violate \textsc{Roundness Harmony}, or vice versa; the preferences are shown in Table \ref{law}. There is a small preference against \textsc{Labial Attraction}, though this is non-significant ($\gamma = -0.043$, $p = 0.305$). Speakers do not have the preferences predicted by \textsc{Labial Attraction}. This result also extends for the variants of \textsc{Labial Attraction} proposed by \citet{Zimmer1969} and \citet{Inkelas2001}, since these variants have structural descriptions targeting a superset of the original formulation by \citet{Lees1966a}. It is interesting to note, however, that speakers do not have the clear opposite preference, as is predicted by \textsc{Roundness Harmony}. 

\begin{table}[ht]
\centering
\begin{tabular}{lrlr|lrlr}
\toprule
\multicolumn{4}{c|}{Experiment 1} & \multicolumn{4}{c}{Experiment 2} \\
\multicolumn{2}{c}{aPu} & \multicolumn{2}{c|}{aPı} & \multicolumn{2}{c}{aPu} & \multicolumn{2}{c}{aPı} \\
\midrule
{tamuz} & 3 & {tamız} & 16 & {pamuz} & 15 & {pamız} & 17 \\
{tafuz} & 3 & {tafız} & 17 & {tafuz} & 21 & {tafız} & 11 \\
{tavuz} & 9 & {tavız} & 4  & {mavuz} & 16 & {mavız} & 16 \\
{tapuz} & 7 & {tapız} & 9  & {tapuz} & 17 & {tapız} & 15 \\
{tabuz} & 5 & {tabız} & 12 & {tabuz} & 16 & {tabız} & 16 \\
\bottomrule
\end{tabular}
\caption{Effects of \textsc{Labial Attraction} on wordlikeness \citep[from][]{Zimmer1969}}
\label{law}
\end{table}

\subsection{Discussion}

It has been shown that while \textsc{Labial Attraction} is a highly reliable generalization about Turkish roots, it is not reflected in wordlikeness judgements. In contrast, harmony processes have a similar statistical profile, but have large effects on wordlikeness. The most plausible explanation for this is that  \textsc{Labial Attraction} does not trigger alternations; indeed, it is counter-exemplified by the effects of \textsc{Roundness Harmony} in suffixes. As is noted by \citet[412f.]{Inkelas1997}, the lexicon of Turkish will, under the assumptions here, remain as it is whether or not \textsc{Labial Attraction} has a synchronic reality. A principle of simplicity suggests that it does not exist at all.

\citet{NiChiosain1993} and \citet{Ito1995b} argue that \textsc{Labial Attraction} is true only of native vocabulary. This would be a potential confound for the experimental results, since it is not implausible that the subject's in \citeauthor{Zimmer1969}'s study treated nonce words as if they were loanwords. 
%\footnote{Indeed, many wordlikeness studies include instructions to the participants to treat the stimuli much as if they were loanwords \citep[e.g.,][]{Hay2004a}. This may influence the choice of task model used by participants.}
This was not the opinion of \citet[266]{Lees1966a}, though, and it is conclusively refuted by  \citet{Inkelas2001}. They observe that in the subset of the TELL database with etymological coding, a full 75\% of foreign vocabulary conform to \textsc{Labial Attraction}. Among native lexical items, only 52\% do, a significant trend but in an unexpected direction (Fisher exact test, $p = 0.042$). One possible explanation is that many of the languages in contact with Turkish lack the \emph{ı} phoneme, and therefore cannot contribute exceptions to \textsc{Labial Attraction}.

\citet{Becker2011} attribute the inactivity of \textsc{Labial Attraction} in wordlikeness to a claim that it is phonologically unnatural, and therefore goes unlearned. This claim is difficult to evaluate insofar as phonotactic naturalness has continually eluded formalization and \citeauthor{Becker2011} neither propose nor refer to any theory thereof. \citeauthor{Becker2011} use the lack of phonetic precursors as a diagnostic of unnaturalness; however, there is extensive formal and experimental evidence for the learning of ``unnatural'' generalizations \citep[e.g.,][]{Anderson1981,Bach1972,Blevins2003,Buckley2000a,Hayes2009,Pierrehumbert2006c,Seidl2005}. Furthermore, other functionally-oriented theorists have considered \textsc{Labial Attraction} sufficiently natural \citep[e.g.,][]{NiChiosain1993,Ito1993,Ito1995b}. 

It is not even obvious that it is desirable to exclude \textsc{Labial Attraction} as a possible rule. \citet[394, fn. 2]{Inkelas1997} suggest that \textsc{Labial Attraction} may have even induced alternations at one point in the history of Turkish. Beyond Turkish, the rounding of a high back vowel after a labial consonant, as in the alternative formulation of \textsc{Labial Attraction} proposed by \citet{Inkelas2001}, is widely attested \citep[e.g.,][]{Vaux1993}. It is also phonetically natural, as both labial consonants and high round vowels are distinguished by a very low first and second formant. As shown in Table \ref{lasi}, it is even a statistically reliable generalization about the Turkish lexicon. However, it is not reflected in alternations or in wordlikeness judgements.

%Regarding the non-local condition holding between the \emph{a} portion of the trigger and the \emph{u} target, Classical Arabic has many adjectives which form corresponding inchoative verbs by overwriting the vocalic melody with /a\ldots{}u/ (the final /-a/ is an inflectional suffix).

%\begin{example}[Arabic derived inchoatives]
%\begin{tabular}{l l l l}
%\buf{}[kabiːr]  & `big'      & [kabura]  & `become big'       \\
%\buf{}[ħasan]   & `handsome' & [ħasuna]  & `become beautiful' \\
%\buf{}[dʒadiːb] & `barren'   & [dʒaduba] & `become dry'       \\
%\end{tabular}
%\end{example}

%\noindent
%In an Optimality Theory framework, for instance, constraints responsible for rounding of high back vowels before labials and for /a\ldots{}u/ overwriting can be conjoined to approximate the structural description of \textsc{Labial Attraction}.

\section{Conclusions}

Statistical reliability is not a sufficient condition for a sequence structure generalization internalized by speakers or reflected in psycholinguistic tasks. This case suggests that it is a necessary condition that the constraint in question be derived from an alternation, however. Even alternations that have large numbers of lexical exceptions have robust effects on a constrained wordlikeness task. Purely statistical approaches to phonotactic knowledge fail to draw the distinction between static and derived phonotactic generalizations.

%\ex High-vowel/zero alternations \citep[][243]{Clements1982}: \\
%\begin{tabular}{l l l l}
%   & nom.sg. & gen.sg. \\
%a. & fikir   & fikri  & `idea' \\
%   & hüküm   & hükmün & `judgement' \\
%%  & filim   & filmi & `film' & \citep[][178]{Inkelas2001} \\
%b. & vakit   & vaktin & `time' \\
%   & rahim   & rahmin & `womb' \\
%\end{tabular} \xe
%This is not to imply uncritical acceptance of \citeauthor{Becker2011}'s account. \citeauthor{Becker2011} place this generalization ``in the grammar'' to account for their claim that ``naturalness'', which they equate with Universal Grammar, constrains the types of generalizations speakers extract in this task.
%It is difficult to evaluate their claim in the absense of a theory of phonotactic naturalness, something which has consisteny eluded formalization for decades. Even the informal definition used by \citeauthor{Becker2011} is flawed. \citeauthor{Becker2011} label generalizations unnatural if they lack phonetic precursors, or if they are not unattested---no description of the methodology used to determine attestedness is given. These two varieties of evidence are not fully independent, however: extragrammatical \emph{channel bias} effects are thought to be a first-order predictor of attestation \citep{Blevins2004,Moreton2008}. In light of extensive evidence for generalizations which are phonetically unnatural \citep[e.g.,]{Anderson1981,Bach1972,Buckley2000a}, even as revealed in nonce word studies \citep[e.g.,][]{Hayes2009}, the generalizations extracted in the \citet{Becker2011} study are phonetically natural is either accident or conundrum. It is certainly not something that can be explained by the tacit theory of possible gneeralizations the authors adhere to.
%Whereas the classic analysis of this phenomenon, by \citet{Inkelas1997}, does not make use of lexical exceptionality, \citeauthor{Becker2011}'s analysis of final aspirated stops and affricates as the underlying form requires that _every_ lexical entry with a final stop or affricate be treated as exceptional according to some phonological constraint. This projection of a lexical contrast into the grammar has another potential flaw: the constraints used to produce the voiced variants in, e.g., the dative, are not surface true. Consider the constraint *VpV used to induce alternations like \emph{kap}-\emph{kabı} `coat'. Ignoring the case of non-alternating root-final \emph{p} (e.g., \emph{ip}-\emph{ipi} `rope'), this is still not a surface-true generalization: \emph{p} occurs freely in intervocalic position: \emph{ahtapot} `octopus', \emph{köpük} `bubble', \emph{öpücük} `kiss'. More serious is the presence of root-internal intervocalic \emph{p} in ooots which simultaneously exhibit the root-final \emph{p}-\emph{b} alternation: \emph{supap}-\emph{supabı} `valve', \emph{hipermetrop}-\emph{hipermetrobu} `far-sightedness'. One prediction shared by virtually all theories of lexical exceptionality is that a lexical entry cannot simultaneously be a target for, and exception to, a phonological generalization, but this precisely what is needed to account for the behavior of \emph{supap}. 
% \citet{inkelas1997}
%küp-kübü `cube'
%kasap-kasabı `butcher'
%Further, the empirical status of the core instances of NDEB have recently been quite stridently disputed \citep{InkelasInPress}. There is some reason to suspect that NDEB is a symptom of interface restrictions on phonological application, rather than a diagnosis itself.
%\begin{example}[\textsc{Backness Harmony} lexical statistics]
%\begin{example}[Wordlikeness comparisons, backness harmony]
%\begin{example}[Wordlikeness comparisons, roundness harmony]
%\begin{example}[\textsc{Labial harmony} and etymology in TELL (counts from \citealp{Inkelas2001}:187)]
%\begin{example}[Wordlikeness comparisons, labial attraction]
%\begin{table}[ht]
%\centering
%\begin{tabular}{lrrrr}
%\toprule
%        & {a}P{u} & {a}P{ı} & \% aPu & $p$-value \\
%\midrule
%native  & 12      & 11      & 52     & \multirow{2}{*}{0.042} \\
%foreign & 84      & 28      & 75                              \\
%\bottomrule
%\end{tabular}
%\caption{?}
%\label{lae}
%\end{table}
%The loanword adaptation literature also contains many reports that statistically reliable phonotactic generalizations about the native lexicon are absent in processes of nativization \citep[e.g.,][]{Ito1995a,Ito1995b,Ussishkin2003}. It is not immediately clear that this is evidence that the generalizations in question are external to the synchronic grammar, however.
% 
%  & {harf}         & {harfler}  & `(alphabetic) letter' \\ 
%  & {el}           & {eller}    & `hand'         \\
%  & {sap}          & {saplar}   & `stalk'        \\
%  & {son}          & {sonlar}   & `end'          \\
%b. & grip    & \alt{} & gırip    & `grippe'    \\ % unexpectedly back
%   & kredi   & \alt{} & kıredi   & `credit'    \\
%  & {el}           & {eli}          & {elin}         & `hand'    \\
%  & {yüz}          & {yüzü}         & {yüzün}        & `face'    \\
%  & {pul}          & {pulu}         & {pulun}        & `stamp'   \\
