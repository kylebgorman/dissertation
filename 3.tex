% 3: Static and derived constraints on Turkish vowel harmony

In the preceding chapter, it was shown that apparent underrepresentation of certain sequences in the lexicon may occur 

This brief chapter argues that even statistically valid phonotactic generalizations are ignroed by speakers. Two phonotactic generalizations over the Turksih lexicon are shown to be equally statistically reliable, but only the generalization which is derived by a phonological alternation influence wordlikeness judgements. 

Research on gradient phonotactics has usually rejected this possibilty implicit
ly.

After reviewing co-occurrence restrictions in Gitksan roots, \citet{Brown2010} rejects this explicitly. He writes: 

\begin{quotation}
\ldots the patterns outlined above are statististically significant. Given this, it stands that these sound patterns should be explained by some linguistic mechanism. \citep[][48]{Brown2010}
\end{quotation}

%Classical Arabic adjectives often have stative verbs in which the root is imposed onto the template CaCuCa:

%\ex Arabic verbs of coming into being: \\
%\begin{tabular}{r l l l}
%a. & kabura & `become big'       & (cf. \emph{kabiːr} `big') \\
%b. & saʁiir & `become small'     & (cf. \emph{saʁiːr} `small') \\
%c. & ħasuna & `become beautiful' & (cf. \emph{ħasan} `handsome') \\
%\end{tabular}
%\xe 
