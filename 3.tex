\chapter{Static and derived phonotactic preferences in Turkish}
\label{turkish}

\citet{Chomsky1965} argue that speakers internalize language-specific generalizations about possible and impossible words, but it is certainly possible to imagine otherwise. \citet[][320]{Zimmer1969} writes that phonotactic generalizations ``have no observable consequences in the course of the normal use of the language'', and \citeauthor{PE} echo this sentiment more recently:

\begin{quote}
Even if we, as linguists, find some generalizations in our description of the lexicon, there is no reason to posit these generalizations as part of the speaker's knowledge of their language, since they are computationally inert and thus irrelevant to the input-output mapping that the grammar is responsible for. \citep[][18]{PE}
\end{quote}

This represents a principled null hypothesis, but can be quickly dismissed in light of speaker's ready judgements of possible and impossible words observed by \citeauthor{Chomsky1965}. As shown in Chapter \ref{gradience}, wordlikeness tasks implicate prosodic wellformedness and cannot be reduced to purely statistical generalizations. Metalinguistic judgements are not the only task in which phonotactic knowledge is on display; speakers are also thought to make use \emph{possible word constraint} in more quotidien tasks like recognizing words in running speech \citep[e.g.,][]{Brown1956,McQueen1998b,Norris1997}.


struct/act
copenhagen
pure statistical critique

\citet{Fischer-Jorgensen1952} and \citet{Vogt1954}
note the inherent difficulty of distinguishing between structural and accidental phonotactic gaps.

Even if the criteria differentiating structural and accidental generalizations were purely statistical, an accidental generalization is one which lacks an antecedent cause, and it falls to the linguist to delimit a theory of these causes. 

\citeauthor{Chomsky1965} were not the first linguists to take an interest in possible wordhood. 

The early generativists identify these causes in mentalism, grounding the study of accidental gaps in what patterns are internalized, and ultimately how they are learned and revealed in psycholinguistic tasks. Consequently, for the generativist an accidental phonotactic generalization can only be one which speakers do not learn, do not attend to, and thus a generalization which is without mental antecedent.

Much of the latter-day literature on phonotactic knowledge, while adopting the representational machinery of generative phonology, expresses limited interest in what generalizations speakers internalize. In a seminal paper, \citet{McCarthy1988} uses a simple statistical technique, a chi-square test, in an attempt to validate a generalization about Arabic root consonant co-occurrence facts, observations originally due to \citet{Greenberg1950}. The assumption appears to be that statistically valid static co-occurence restrictions will be reflected by wordlikeness judgements. While the Arabic restriction discussed by \citeauthor{McCarthy1988} was ultimately investigated experimentally \citep{Frisch2004}, this is the case for only a small minority of the studies in the \citeauthor{McCarthy1988}ian model; rather, the phonotactic generalizations are inferred directly from lexical statistics \citep[e.g.,][]{Anttila2008a,Berkley1994b,Berkley1994a,Berkley2000,Brown2010,Buckley1997,Coetzee2008a,Dmitrieva2008a,Dmitrieva2008b,Elmedlaoui1995,Graff2011,MacEachern1999,Kinney2005,Kawahara2006,Martin2007,Martin2011,Mester1988,Miller-Ockhuizen2003,Padgett1992,Padgett1995,Pozdniakov2007,Yip1989}. In a study of consonant co-occurrence restrictions in Gitksan, \citet{Brown2010} makes the tacit assumption of this approach explicit, claiming that \emph{any} statistically significant pattern in the lexicon is one that is internalized by speakers. 

\begin{quote}
\ldots{} the patterns outlined above are statististically significant. Given this, it stands that these sound patterns should be explained by some linguistic mechanism. \citep[][48]{Brown2010}
\end{quote}

There are several arguments against this inference. First, linguists since Saussure have noted diachronic causes for statistical generalizations which lack any synchronic reality. For instance, Saussure \citeyear[202f.]{CLG} recognizes the a diachronic process which changed Old Latin intervocalic \emph{s} to \emph{r}, and that this makes intervocalic \emph{s} rare in Classical Latin. However, Saussure denies that this generalization has any status in Classical Latin.\footnote{See \citet{Gorman2012e} and citations therein for a full explication of Saussure's position on rhotacism, and a restatement of the phonological argument in generative terms.} Secondly, wordlikeness generalizations are constrained by negative results in learnability (e.g., \citealp{Gold1967}; see \citealt{Yang2012} for a recent review), and may also be restricted by markedness constraints \citep[e.g.,][]{Becker2011,Hayes2009,Moreton2002}. Finally, there are a number of results which suggest patterns obvious to the linguist go unlearned \citep{Becker2011,Hayes2006,Zimmer1969}. This argues against a direct link between statistical criteria and the synchronic reality of phonotactic generalizations. Consequently, grammatical models which closely mimic lexical statistics but which lack psycholinguistic support (e.g., Gitksan: \citealt{Brown2010}; Muna: \citealt{Anttila2008a}, \citealt{Coetzee2008a}; Shona: \citealt[][385]{Hayes2008a}) may not be appropriate models of speakers' phonotactic knowledge.

This chapter is a comprehensive investigation of three well-known phonotactic generalizations in Turkish, as well as the results of a wordlikeness task designed to investigate the synchronic reality of these generalizations \citep{Zimmer1969}. This study merits reconsideration because there is widespread disagreement about the analysis of \citeauthor{Zimmer1969}, and prior discussions fail to relate the behavioral data to lexical statistics and competing formalizations of the constraints involved. One of these constraints, \textsc{Labial Attraction}, is shown to be highly statistically reliable but to be inactive in the wordlikeness task. This undermines the widespread use of statistical criteria to determine which of the vast number of phonotactic generalizations are internalized by speakers and which are not. The simplest hypothesis is that \textsc{Labial Attraction} is inactive in wordlikeness tasks because it is phonologically inactive a well.

\section{Turkish vowel sequence structure constraints}

\citet{Lees1966b,Lees1966a} proposes three constraints on Turkish vowel sequences, constraints which are the focus of many subsequent studies. In this setion, these constraints are formalized, and where possible, related to phonological alternations and to behavioral evidence suggesting that speakers internalize restrictions. The following feature specification for the eight vowels of Turkish vowels is assumed throughout.

\begin{example}[Turkish vowel features]
\begin{tabular}{c c c c c}
                       & \multicolumn{2}{c}{[$-$\textsc{Back}]} & \multicolumn{2}{c}{[$+$\textsc{Back}]} \\
                       & [$-$\textsc{Rnd}] & [$+$\textsc{Rnd}] & [$-$\textsc{Rnd}] & [$+$\textsc{Rnd}] \\ 
\cmidrule{2-5}
\buf[$+$\textsc{High}] & {i} & {ü} [y] & {ı} [ɯ] & {u} \\
\buf[$-$\textsc{High}] & {e} & {ö} [ø] & {a} [ɑ] & {o} \\
\end{tabular}
\end{example}

Two non-conventional notations are used in this chapter. First, directional application conditions are used when necessary to model the directionality of harmony, rather than ad hoc contextual restrictions (e.g., that the trigger be aligned with the left of a prosodic word). Directional application provides a panacea for the considerable body of evidence against simultaneous rule application adduced in the 1970s (e.g., \citealt{Howard1972}, \citeauthor{Kenstowicz1973} \citeyear[14f.]{Kenstowicz1973}, \citeyear[189f.]{Kenstowicz1977}, \citeyear[318f.]{Kenstowicz1979}, \citealt{Piggott1975}) and simplifies the analysis of harmony in Turkish \citep[209f.]{Anderson1974}. Further, \citet{Johnson1972} and \citet{Kaplan1994} prove that rules specified for directional application are formally learnable. Secondly, rather than the use of an unbounded number of Greek-letter variables ($\alpha$, $\beta$, etc.) over feature values $\{+, -\}$, only a single variable, denoted by `$=$', is used. Following \citet{McCawley1973}, a structural description [$=$F]\ldots{}[$=$F] matches a string S$_i$\ldots{}S$_j$ if and only if S$_i$ and S$_j$ are both [$+$F] or both [$-$F].\footnote{This is more restrictive than the Greek-letter variable notation, in that the value of $=$ for some feature F cannot be applied to F$'$ if F $\ne$ F$'$; \citet{Odden2012} dismisses the two potential counterexamples against this restrict, both adduced in \emph{SPE} (352-353).} 

\subsection{Backness harmony}

\citeauthor{Lees1966b} (\citeyear[35]{Lees1966b}, \citeyear[284]{Lees1966a}) models the Turkish vowel harmony system with three rules. The most general of these rules spreads the specification [\textsc{Back}] rightward.

\begin{example}[\textsc{Backness Harmony} (condition: rightward application)]
$\begin{bmatrix} -\textsc{Cons} \end{bmatrix}~\goesto~\begin{bmatrix} =\textsc{Back} \end{bmatrix}~\big /~\begin{bmatrix} =\textsc{Back} \end{bmatrix}~\textrm{C}_0~\gap$
\end{example}

\noindent
A vowel becomes [$+$\textsc{Back}] after a [$+$\textsc{Back}] vowel, and [$-$\textsc{Back}] after a [$-$\textsc{Back}] vowel, ignoring any intervening consonants. The application of this rule proceeds from left to right; no vowel may be skipped. 

If permitted to apply in non-derived environments, this rule accounts for the tendency of polysyllablic roots to contain only [$+$\textsc{Back}] or [$-$\textsc{Back}] vowels, a tendency which will be quantified below. \textsc{Backness Harmony} also triggers alternations in inflectional suffix vowels. For instance, the nominative plural (nom.pl.) suffix is \emph{-ler}, when the final root vowel is [$-$\textsc{Back}], and \emph{-lar} when it is [$+$\textsc{Back}]. 

\begin{example}[The Turkish nominative] \label{turknom}
\begin{tabular}{l l l l@{ }l}
   & \emph{nom.sg.} & \emph{nom.pl.} \\
a. & {ip}           & {ipler}    & `rope'         & \citep[][216]{Clements1982} \\
%   & {el}           & {eller}    & `hand'         \\
   & {köy}          & {köyler}   & `village'      \\
   & {yüz}          & {yüzler}   & `face'         \\
   & {kız}          & {kızlar}   & `girl'         \\
%   & {sap}          & {saplar}   & `stalk'        \\
%   & {son}          & {sonlar}   & `end'          \\
   & {pul}          & {pullar}   & `stamp'        \\
b. & {neden}        & {nedenler} & `reason'       & \citep{TELL} \\
   & {kiler}        & {kilerler} & `pantry'       \\ % front
   & {pelür}        & {pelürler} & `tissue paper' \\ % back
   & {boğaz}        & {boğazlar} & `throat'       \\ % back
   & {sapık}        & {sapıklar} & `pervert'      \\ % back
\end{tabular}
\end{example}

A few complications arise, however. First, as shown in (\ref{turkexcept}a), not all polysyllabic roots conform to \textsc{Backness Harmony}. In this case, suffix vowels generally exhibit harmony with the final root vowel. 

\begin{example}[Exceptional Turkish nominatives] \label{turkexcept}
\begin{tabular}{l l l l@{ }l}
   & \emph{nom.sg.} & \emph{nom.pl.} \\
a. & {mezar}        & {mezarlar} & `grave' & \citep{TELL} \\
   & {model}        & {modeller} & `model' \\
   & {silah}        & {silahlar} & `weapon'     \\
   & {memur}        & {memurlar} & `bureaucrat' \\
   & {sabun}        & {sabunlar} & `soap'       \\
b. & {saat}         & {saatler}  & `hour, clock' & \citep{Goksel2005} \\
   & {harf}         & {harfler}  & `(alphabetic) letter' \\ 
   & {etol}         & {etoller}  & `fur stole' \\
\end{tabular}
\end{example}

\noindent
There is also a very small class of nouns, shown in (\ref{turkexcept}b), which take \emph{-ler} despite the fact that their final root vowel is [$+$\textsc{Back}]. While it is uncontroversial that the disharmonic suffixes of (\ref{turkexcept}b) are no more than very sporadic exceptions to \textsc{Backness Harmony}, root disharmony (\ref{turkexcept}a) has been the subject of much debate. As disharmonic roots still trigger suffix harmony, \citet[212, 289]{Anderson1974} and \citet{Iverson1978} distinguish suffix harmony alternations from the sequence structure constraint governing root harmony.

The disadvantage of this account is that it introduces ``duplication'' (in the sense of \citealt{Kisseberth1970b} and \citealt{Kenstowicz1977}) of sequence structure and phonological generalizations, differning only in their patterns of exceptionality. However, \citet[][197f.]{Zonneveld1978} shows that the theory of exceptionality proposed in \emph{SPE} can account for suffix harmony in disharmonic roots.\footnote{An isomorphic analysis is available within less restrictive theories of phonological exceptionality \citep[e.g.,][]{Kisseberth1970,Pater2009}.} \citeauthor{SPE} assume that the specification of the target (i.e., the segment or segments to be changed) of a rule \emph{R} must be marked [$+$\emph{R}] by convention. A root or affix which fails to undergo \emph{R} despite otherwise matching the structural description is simply said to be marked [$-$\emph{R}]. In other words, no form is never truly an ``exception''; rather, some forms fail to match the extended structural description of $R$ which requires that the target be [$+R$]. If disharmonic roots are marked [$-$\textsc{Backness Harmony}], then the final vowel of disharmonic roots will still trigger \textsc{Backness Harmony}, since the [$-$\textsc{Backness Harmony}] root is no longer the target but rather the trigger, which is not subject to the [$+$\textsc{Backness Harmony}] requirement. Thus \textsc{Backness Harmony} is able to account for both root and suffix harmony.

It is necessary to dispense with an alternative analysis proposed by \citet{Clements1982} and \citet{Inkelas1997}. Root vowels exhibit a robust contrast for backness (e.g., \emph{kül} `ash' vs.  \emph{kul} `servant', \emph{kepek} `bran' vs. \emph{kapak} `lid'), whereas harmonic roots are those in which the backness of any remaining vowels is predictable. \citeauthor{Clements1982} propose that these vowels, as well as harmonizing suffix vowels, are underspecified for backness, whereas the non-initial vowels of disharmonic roots are fully specified. This is schematized below.


\begin{example}[Autosegmental underspecification in harmonic roots (after \citealp{Clements1982})] \label{spec}
\xymatrix@R=24pt@C=24pt{
\txt{a.} & \txt{harmonic root:} & \txt{C} & \txt{V} & \txt{C} & \txt{V} & \txt{C} \\
         &                      &         & \txt{[$-$\textsc{Back}]}\ar@{-}[u]\ar@{--}[urr] \\
\txt{b.} & \txt{disharmonic root:} & \txt{C} & \txt{V} & \txt{C} & \txt{V} & \txt{C} \\
         &                      &         & \txt{[$-$\textsc{Back}]}\ar@{-}[u] & & \txt{[$+$\textsc{Back}]}\ar@{-}[u]
}
\end{example}

One crucial detail is missing from this analysis: \textsc{Backness Harmony} needs to be prevented from overwriting the [$+$\textsc{Back}] specification of disharmonic roots, one option being a Structure Preservation condition \citep{Kiparsky1985}. However, any condition which prevents \textsc{Backness Harmony} from overwriting underlying backness specifications will reintroduce the duplication of sequence structure and phonological generalizations; under this analysis, disharmonic roots are no longer exceptional, despite considerable evidence (reviewed below) that they are marked in Turkish.\footnote{On the other hand, it is possible to interpret the presence of a single backness specification per root as a perceptual default. A precedent for this is the surface-oriented interpretation of the tonal Obligatory Contour Principle proposed by \citet[134]{Goldsmith1976} and \citet{Odden1986}, under which adjacent identical tones are automatically attributed to a single underlying tone. However, this interpretation makes the underspecification analysis a mere notational variant of the rule exceptionality account, in which [$+$\textsc{Backness Harmony}] is the default.} On these grounds, the underspecification analysis is rejected here in favor of the rule exceptionality account.

While harmony in non-derived environments can be inferred from the aforementioned suffix alternations, no evidence has yet been presented to show internalize the tendency for roots to conform to backness harmony. If Turkish speakers do not attend to this generalization, there is no need for the grammar to account for it. Several other ``external'' facts demonstrate that this is not the case. 

The production of non-native word-initial onset clusters, discussed by \citet{Clements1982} and \citet{Kaun1999}, suggests that loanword adaptation respects \textsc{Backness Harmony}.\footnote{Thanks to Kie Zuraw for bringing this data to my attention.} While some speakers are said to be capable of pronouncing these non-native clusters, in fast speech the cluster is reportedly split by anaptyxis. In most cases, this vowel matches the following root vowel for backness.

\begin{example}[Variable non-native cluster adaptation (\citealp{Clements1982}:247)] 
\begin{tabular}{l l l l l l}
a. & {spiker}  & \alt{} & {sipiker}  & `announcer' \\
   & {fren}    & \alt{} & {firen}    & `break'     \\
b. & {trablus} & \alt{} & {tırablus} & `Tripoli'   \\
   & {kral}    & \alt{} & {kıral}    & `king'      \\
c. & {brom}    & \alt{} & {burom}    & `bromide'   \\
   & {prusya}  & \alt{} & {purusya}  & `Prussia'   \\
%b. & grip    & \alt{} & gırip    & `grippe'    \\ % unexpectedly back
%   & kredi   & \alt{} & kıredi   & `credit'    \\
\end{tabular}
\end{example}

\noindent
It is worth noting that this cannot be merely the consequence of application of the native \textsc{Backness Harmony} rule, since the leftmost vowel of a prosodic word is not subject to harmonization. Rather, the adaptation of non-native onset clusters proceeds in such a fashion so that the lexical items in question are rarely [$-$\textsc{Backness Harmony}].

Similar evidence comes from a language game discussed by \citet{Harrison2001}.\footnote{Thanks to Bert Vaux for bringing this study to my attention.} This game is not indigenous to Turkish, but it is found in the grammar of the Turkic language Tuvan, where it is used to convey a sense of ``vagueness or jocularity''; \citeauthor{Harrison2001} report that it can be quickly taught to Turkish speakers. The game consists of reduplication of the base, with the first vowel of the reduplicant replaced with the [$+$\textsc{Back}] vowels \emph{a} or \emph{u}. Both in Tuvan and in the Turkish game, the second vowel of the reduplicant (shown in braces below) may also be effected by this process; when the base vowels are both [$-$\textsc{Back}], the insertion of a [$+$\textsc{Back}] vowel results in what \citeauthor{Harrison2001} call ``reharmonization'', shown in  (\ref{redupgame}a). On the other hand, as shown in (\ref{redupgame}b), disharmonic roots do not reharmonize.\footnote{A similar pattern is found in Tuvan, and in an unrelated Finnish language game \citep{Campbell1986}.}

\begin{example}[Turkish reduplication game (\citealp{Harrison2001}:231)] \label{redupgame}
\begin{tabular}{l l l l}
a. & {kibrit} & {kibrit}-\{{kabrıt}\} & `match'    \\
   & {bütün}  & {bütün}-\{{batın}\}   & `whole'    \\
b. & {mali}   & {mali}-\{{muli}\}     & `Mali'     \\
   & {butik}  & {butik}-\{{batik}\}   & `boutique' \\
\end{tabular}
\end{example}

\noindent 
\citeauthor{Harrison2001} present an analysis of this game couched in the vowel underspecification proposed by \citeauthor{Clements1982}, but the data is equally consistent with a full specification analysis. Reharmonization is the result of \textsc{Backness Harmony} in the reduplicant. On the other hand, the lack of reharmonization in the disharmonic roots indicates that the [$-$\textsc{Backness Harmony}] exception feature is copied under reduplication. 

A number of studies have investigated the role of harmony in word-spotting tasks, though to reveal how speakers identify word boundaries in running speech. Many of these studies have been carried out in Finnish, which has a vowel harmony system similar to Turkish. \citet{Suomi1997} and \citet{Vroomen1998} task speakers with identifying harmonic disyllables in an auditory stream. Finnish speakers are better able to complete this task when the preceding syllable has a different backness specification than the disyllabic target. \citet{Kabak2010} report that Turkish \textsc{Backness Harmony} has the same effect on word-spotting tasks as it does in Finnish: Turkish speakers are quicker and more accurate at the task of spotting the nonce target word \emph{pavo} when preceded by a disharmonic juncture (e.g., \emph{gölü-PAVO}) than when preceded by a harmonic juncture (e.g., \emph{golu-PAVO}), but this effect does not obtain for speakers of French, a language which lacks vowel harmony. These results imply that speakers of Turkish have internalized the tendency of harmonic sequences to be root-internal and of disharmonic transitions to mark word juncture.

The word-spotting experiment has been adapted for infants by \citep{Kampen2008}. In this study, 9-month-old infants are familiarized with an auditory stream containing harmonic disyllabic nonce words. At test time, the infants listen to the disyllabic nonce words in isolation using the head turn preference paradigm. Infants acquiring Turkish listen longer to nonce words preceded by a disharmonic juncture during familiarization (e.g., \emph{lo-NETIS}), whereas infants acquiring German, a language which lacks vowel harmony, do not exhibit this preference. Similarly, \citeauthor{Kampen2008} report that Turkish 6-month-old infants prefer to listen to harmonic pseudowords such as \emph{paroz} over disharmonic pseudowords like \emph{nelok}, but German 6-month-old infants show no such preference. 

\subsection{Roundness harmony}

\subsubsection{Phonological description}

This rule is quite similar to \textsc{Backness Harmony}, but imposes an additional restriction, that targets be [$+$\textsc{High}]. 

\begin{example}[\textsc{Roundness Harmony} (condition: rightward application)]
$\begin{bmatrix} -\textsc{Cons} \\ +\textsc{High} \end{bmatrix}~\goesto~\begin{bmatrix} =\textsc{Back} \end{bmatrix}~/~\begin{bmatrix}~=\textsc{Back}~\end{bmatrix}~\textrm{C}_0~\gap$
\end{example}

\noindent
A [$+$\textsc{High}] vowel becomes [$+$\textsc{Rnd}] after a [$+$\textsc{Rnd}] vowel, and [$-$\textsc{Rnd}] after a [$-$\textsc{Rnd}] vowel, ignoring any intervening consonants, and applying from left to right. 

If permitted to apply in non-derived environments, this rule accounts for the tendency of polysyllabic roots to contain only [$+$\textsc{Rnd}] or [$-$\textsc{Rnd}] if the non-initial vowels are [$+$\textsc{High}]. In concert with \textsc{Backness Harmony}, \textsc{Roundness Harmony} also triggers alternations which account for the shape of the dative singular (dat.sg.) and genitive singular (gen.sg.) among other suffixes. As was the case for \textsc{Backness Harmony}, disharmonic roots are found, and the final vowel of disharmonic roots triggers suffix harmony.

\begin{example}[Turkish nominal suffix allomorphy]
\begin{tabular}{l l l l l l@{ }l}
   & \emph{nom.sg.} & \emph{dat.sg.} & \emph{gen.sg.}  \\
a. & {ip}           & {ipi}          & {ipin}         & `rope' & \citep[][216]{Clements1982} \\
%  & {el}           & {eli}          & {elin}         & `hand'    \\
   & {kız}          & {kızı}         & {kızın}        & `girl'    \\
   & {sap}          & {sapı}         & {sapın}        & `stalk'   \\
%  & {yüz}          & {yüzü}         & {yüzün}        & `face'    \\
   & {köy}          & {köyü}         & {köyün}        & `village' \\
%  & {pul}          & {pulu}         & {pulun}        & `stamp'   \\
   & {son}          & {sonu}         & {sonun}        & `end'     \\
b. & {boğaz}        & {boğazı}       & {boğazın}      & `throat'  & \citep{TELL} \\
   & {pelür}        & {pelürü}       & {pelürün}      & `tissue paper' \\
   & {döviz}        & {dövizi}       & {dövizin}      & `currency' \\
   & {yamuk}        & {yamuğu}       & {yamuğun}      & `trapezoid' \\
   & {ümit}         & {ümiti}        & {ümitin}       & `hope'     \\
\end{tabular}
\end{example}

Few studies have directly investigated whether speakers are aware of the tendency for roots to conform to \textsc{Roundness Harmony}. However, two pieces of the external evidence on \textsc{Backness Harmony} bear on this question. First, the epenthetic vowel in non-native word-initial onsets like \emph{trablus} \alt{} \emph{tırablus} `Tripoli' conform both to \textsc{Backness Harmony} and to \textsc{Roundness Harmony}. Secondly, \textsc{Roundness Harmony} participates in reharmonization in the language game described by \citeauthor{Harrison2001}: the second \emph{ü} in \emph{bütün} `whole' reharmonizes to the [$-$\textsc{Rnd}] vowel \emph{ı} in reduplicated \emph{bütün-batın}.

\subsection{Labial attraction}

\citet{Lees1966b} describes \textsc{Labial Attraction} as a phonological process by which ``a high, short harmonic vowel is rounded in the second syllable of a disyllabic word whose first vowel is /a/, and whose medial consonant cluster contains a labial /p, b, m, v/, and then it is de-harmonified'' (36).

\begin{example}[\textsc{Labial Attraction}]
$\begin{bmatrix} -\textsc{Cons} \\ +\textsc{Back} \\ +\textsc{High} \end{bmatrix}~\goesto~\begin{bmatrix} +\textsc{Rnd} \end{bmatrix}~/~\begin{bmatrix} -\textsc{Cons} \\ -\textsc{Rnd} \\ +\textsc{Back} \\ -\textsc{High} \end{bmatrix}~\textrm{C}_0~\begin{bmatrix} +\textsc{Cons} \\ +\textsc{Labial} \end{bmatrix}~\textrm{C}_0~\gap{}$
\end{example}

The formalization of this rule is naturally complex, and perhaps obscures the fact that \textsc{Labial Attraction} generates exceptions to \textsc{Roundness Harmony}, producing \emph{a}P\emph{u} sequences (where P represents a labial consonant; e.g., \emph{çapul} `raid', \emph{sabur} `patient', \emph{şaful} `wooden honey tub', \emph{avuç} `palm of hand', \emph{samur} `sable'; \citealp[285]{Lees1966a}) rather than the expected \emph{a}P\emph{ı}. However, \textsc{Labial Attraction} does not apply in derived environments: for example, the gen.sg. of \emph{sap} `stalk is \emph{sapın} rather than *\emph{sapun}.

\citet[286]{Lees1966a} and \citet[311]{Zimmer1969} note the existence of root-internal exceptions to this rule (e.g., \emph{tavır} `mode') but agree that they are surprisingly rare; this is disputed in turn by  \citet{Clements1982} and \citet{Inkelas2001}. A more exhaustive review of the lexical statistics will be provided below.

\section{Evaluation}


by comparing lexical statistics with the results of wordlikeness task performed by \citet{Zimmer1969}.

This resolves the debate concerning the nature of \textsc{Labial Atttraction}: it is shown to be a statistically robust generalization over the Turkish lexicon (\citealt{Lees1966a}, \citealt{Zimmer1969}, \emph{pace} \citealt{Clements1982}, \citealt{Inkelas2001}); however, no variant of \textsc{Labial Attraction} is reflected in \citeauthor{Zimmer1969}'s wordlikeness study. This dissociation between lexical generalizations and wordlikeness results provides further evidence against the common practice of inferring phonotactic knowledge via purely statistical criteria.

\subsection{Lexical statistics}

%\citet{Becker2011,Hayes2009}

\subsubsection{Backness harmony}
\subsubsection{Roundness harmony}
\subsubsection{Labial attraction}

\begin{quote}
\ldots{}decisive evidence against a rule of Labial Attraction is the existence of a further, much larger set of roots containing /\ldots{}aCu\ldots/ sequences in which the intervening consonant or consonant cluster does not contain a labial\ldots{}We conclude that there is no systematic restriction on the set of consonants that may occur medially in roots of the form /\ldots{}aCu\ldots/. \citep[225]{Clements1982}
\end{quote}

but this fact is not inconsistent with the formulation of the constraint:
\emph{aTu} clusters do not meet \textsc{Labial Attraction}'s structural description. 

\noindent
\citeauthor{Clements1982} are suggesting that \textsc{Labial Attraction} implies that \emph{a}\ldots{}\emph{u} sequences separated by non-labial consonants should be infrequent. This is not immediately clear, since this is beyond the scope of \citeauthor{Lees1966a}'s description of \textsc{Labial Attraction}, but the frequency of these \emph{a}T\emph{u} sequences does provide an estimate of how often ???.

\begin{quote}
Lee's rule of \textsc{Labial Attraction}\ldots is not a real generalization about the Turkish lexicon. It is not true synchronically, either of native or nonnative items; nor, according to the historical and dialectical literature, does \textsc{Labial Attraction} appear to have been true at any stage going back as far as Old Turkic [\ldots] Vowel labialization following labials is not a synchronic alternation in Turkish, yet it (unlike \textsc{Labial Attraction} per se) \emph{is} a statistically supported tendency worthy of further research. \citep[][196, emphasis in original]{Inkelas2001} 
\end{quote}

\begin{example}[Lexical effects of \textsc{Labial Attraction}]
\begin{tabular}{l r r r r r}
\toprule
Corpus                & aPu & aPı & aTu & aTı   & $p$-value   \\ % & corpus size
\midrule
Full TELL             & 378 & 248 & 446 & 1,140 & 2.83\e{-44} \\ % & 31,236 \\
Elicited TELL         & 152 & 265 & 101 & 1,839 & 9.84\e{-60} \\ % & 16,541 \\
TELL with etymologies & 128 & 109 &  79 &   470 & 6.56\e{-32} \\ 
\bottomrule
\end{tabular}
\end{example}

%\citet{Harrison2004} 73\% of lexical types are harmonic (both back and round)

%Etymological subset of TELL
%Native    & Foreign   \\
%aBu & aBI & aBu & aBI \\
%12  & 11  & 84  & 28  \\
%p = 0.0417

\subsection{Wordlikeness ratings}



%\citet{Gigerenzer1990} argue that ratings tasks should be replaced with paired comparison to minimize error.

%\citet{Goodman1954}
%The results reveal a dissociation between the lexical statistics and wordlikeness judgements, a dissociation that finds a natural explanation from the principle of ...
%There is some disagreement in the literature about just what \citeauthor{Zimmer1969}'s study reveals about \textsc{Labial Attraction}.

%\citet{Ito1993} and \citet{NiChiosain1993} argue that the data shows that it holds of native words only, but \citet{Inkelas2001} refute this point

%\citet{Zuraw2000}:4
%\citet{Inkelas2001} present a different variation on the original rule of \textsc{Labial Attraction}

%\begin{quote}
%Vowel labialization following labials is not a synchronic alternation in Turkish, yet it (unlike \textsc{Labial Attraction} per se) \emph{is} a statistically supported tendency worthy of further research. \citep[][196, emphasis in original]{Inkelas2001}
%\end{quote}

%\noindent 
%\citeauthor{Inkelas2001} persistently refer to ``statistical support'', but in fact their study contains no statistics beyond counts and percentages.
%since the opposing are all broader versions than the original formulation by \citeauthor{Lees1966a}.

\subsubsection{Backness harmony}

\begin{example}[Wordlikeness comparisons, backness harmony]
\begin{tabular}{lrlr|lrlr}
\toprule
\multicolumn{4}{c|}{Experiment 1} & \multicolumn{4}{c}{Experiment 2} \\
\multicolumn{2}{c}{\textsc{harmonic}} & \multicolumn{2}{c|}{\textsc{disharmonic}} & \multicolumn{2}{c}{\textsc{harmonic}} & \multicolumn{2}{c}{\textsc{disharmonic}} \\
\midrule
{temez} & 19            & {temaz} & 3 & {pemez} & 30            & {pemaz} & 2 \\
{teriz} & 23            & {terız} & 0 & {teriz} & 28            & {terız} & 3 \\
{tokaz} & 21            & {tokez} & 1 & {tokaz} & 26            & {tokez} & 6 \\
{tipez} & 21            & {tipaz} & 1 & {tipez} & 24            & {tipaz} & 8 \\
{terüz} & 20            & {teruz} & 1 & {terüz} & 19            & {teruz} & 13 \\
\bottomrule
\end{tabular}
\end{example}

% harmonic responses are favored approximately 6 to 1.
%$\gamma = 0.717$, $p = 7.5$\e{-69}

Note that both of the pair \emph{terüz}-\emph{teruz} violate \textsc{Roundness Harmony}

\subsubsection{Roundness harmony}

\begin{example}[Wordlikeness comparisons, roundness harmony]
\begin{tabular}{lrlr|lrlr}
\toprule
\multicolumn{4}{c|}{Experiment 1} & \multicolumn{4}{c}{Experiment 2} \\
\multicolumn{2}{c}{\textsc{harmonic}} & \multicolumn{2}{c|}{\textsc{disharmonic}} & \multicolumn{2}{c}{\textsc{harmonic}} & \multicolumn{2}{c}{\textsc{disharmonic}} \\
\midrule
{törüz} & 19 & {töriz} & 1 & {pörüz} & 32 & {pöriz} & 0  \\
{tüpüz} & 22 & {tüpiz} & 0 & {tüpüz} & 31 & {tüpiz} & 1  \\
{takız} & 15 & {takuz} & 3 & {takız} & 22 & {takuz} & 10 \\
{tatız} & 12 & {tatuz} & 6 & {tatız} & 20 & {tatuz} & 12 \\
\bottomrule
\end{tabular}
\end{example}

% more than 5 to 1
% gamma = 0.680 p = 1.1e-47

\subsubsection{Labial attraction}

\begin{example}[Wordlikeness comparisons, labial attraction]
\begin{tabular}{lrlr|lrlr}
\toprule
\multicolumn{4}{c|}{Experiment 1} & \multicolumn{4}{c}{Experiment 2} \\
\multicolumn{2}{c}{aPu} & \multicolumn{2}{c|}{\textsc{disharmonic}} & \multicolumn{2}{c}{\textsc{harmonic}} & \multicolumn{2}{c}{\textsc{disharmonic}} \\
\midrule
{tamuz} & 3 & {tamız} & 16 & {pamuz} & 15 & {pamız} & 17 \\
{tafuz} & 3 & {tafız} & 17 & {tafuz} & 21 & {tafız} & 11 \\
{tavuz} & 9 & {tavız} & 4  & {mavuz} & 16 & {mavız} & 16 \\
{tapuz} & 7 & {tapız} & 9  & {tapuz} & 17 & {tapız} & 15 \\
{tabuz} & 5 & {tabız} & 12 & {tabuz} & 16 & {tabız} & 16 \\
\bottomrule
\end{tabular}
\end{example}



\subsection{Discussion}

etymological issues

%\citet{Inkelas2001}
%\citet{NiChiosain1993} and \citet{Ito1995b} 
%\begin{example}[\textsc{Labial harmony} and etymology in TELL \citealp{Inkelas2001}:187]
%\begin{tabular}{l r r r r}
%\toprule
%        & {a}P{u} & {a}P{ı} & \% aPu & $p$-value \\
%\midrule
%native  & 12      & 11      & 52.2\% & \multirow{2}{*}{0.042} \\
%foreign & 84      & 28      & 75.0\% \\
%\bottomrule
%\end{tabular}
%\end{example}

lack of theory of naturalness

%\citet{Becker2011}
%This is difficult to evaluate insofar as \citeauthor{Becker2011} neither propose nor refer to any theory of naturalness which might exclude \textsc{Labial Attraction}. 
%There are several other problems with this claim. Both early specialists \citep[e.g.,][]{Lees1966a} and later theorists working within a functionally-oriented framework \citep[e.g.,][]{NiChiosain1993,Ito1993,Ito1995a} have considered it plausible. Further, \citet{Hayes2009} claim that Hungarian speakers internalize similarly ``unnatural'' generalizations about exceptions to vowel harmony. 
%Finally, there is at least some reason to admit similar structural descriptions in other languages. For example, some Classical Arabic adjectives have stative verbs in which the root melody is overwritten with \emph{a\ldots{}u}. If the stative verbs are derived from the adjectives, as has recently been argued for Semitic \citep[e.g.,]{Ussishkin1999,Ussishkin2000,Tucker2010}, than there is little reason to believe that this sequence is somehow marked.
%\begin{example}[Arabic stative verbs]
%\begin{tabular}{l l l l}
%\buf{}[kabiːr]  & `big'      & [kabura]  & `become big'       \\
%\buf{}[ħasan]   & `handsome' & [ħasuna]  & `become beautiful' \\
%\buf{}[dʒadiːb] & `barren'   & [dʒaduba] & `become dry'       \\
%\end{tabular}
%\end{example}
%\footnote{Thanks to Uri Horesh (p.c.) for assistance with this data.}

%\noindent
%Further, labial consonants frequently cause rounding of following vowels \citep[e.g.,][]{Vaux1993}. The relevant constraints, conjoined, are a close approximation of \textsc{Labial Attraction}. 

\section{Conclusions}



%Generative phonological theory at the time of \emph{SPE} lacked the representational vocabulary to directly distinguish between word-initial /bn/, which is disallowed, and /bn/ as a licit syllable contact cluster (e.g., \emph{o}[b.n]\emph{oxious}), a point first made by \citep{Hooper1973}. It is not the introduction of prosodic primitives into generative phonology that explains the contrast between \emph{blick} and \emph{bnick}, though, but rather the principle that surface forms must be syllabified \citep[e.g.,][63f.]{Kiparsky1982b} which \emph{bnick} fails. If, as is standardly assumed, phonology repairs unsyllabifiable outputs \citep[e.g.][]{Ito1989a,Noske1992}, then \emph{bnick} is an impossible output, doomed to be modified (perhaps to \emph{nick}; \citealt[][19f.]{Wolf2009}), and /bnɪk/ is an impossible UR by Stampean occultation. 
%A further widespread assumption is that static co-occurrence restrictions validated statistically will be reflected by wordlikeness judgements. This assumption is so deeply entrenched that a number of recent studies of phonotactic learning forgo wordlikeness judgements altogether and instead model the lexical statistics themselves (e.g., \citealt{Coetzee2008a} and \citealt{Anttila2008} on Muna and Arabic, \citealt[][385]{Hayes2008a} on Shona and Wargamay, \citealt{Brown2010} on Gitksan). 
%Since the null hypothesis does not countenance statistical significance as evidence for the synchronic reality of a static lexical pattern, such patterns do not adjudicate between the null hypothesis and the alternative, and another source of evidence is needed. In the remainder of Part I, I turn to data from wordlikeness judgement tasks for these purposes. 
%In a wordlikeness task, a speaker is presented with nonce words and asked to report such intuitions about the ``possibility'' of a word. 
%The alternative hypothesis, then, is that a statistically reliable co-occurrence restriction will be reflected in native speakers' wordlikeness judgements. 
%The data below, drawn from Turkish, shows that 
%This chapter takes the argument one step farther by showing a contrast between static and derived constraints on URs in Turkish. Whereas the derived constraint has a robust effect on native speakers' wordlikeness judgements, the static constraint, which is equally statistically valid, has no effect on these judgements. Anticipating the conclusion, the simplest explanation for the non-effect of this static constraint on wordlikeness judgements is that speakers do not internalize static co-occurrence restrictions at all.
%\begin{unlabeledexample}
%$\displaystyle \gamma = \frac{C - D}{C + D}$
%\end{unlabeledexample}
%\footnote{Any readers who are skeptical that \citeauthor{Chomsky1965}'s intuitions generalize will find support for their claim in the wordlikeness judgement data reported in Appendix \ref{albright}.}
%As a quick validation of this claim, consider the results reported by \citet{Albright2007} and discussed in the previous chapter. English speakers asked to rate non-words on a scale from 1-7, where 7 indicates ``most like English'', favor non-words [blʌs, blɑd] (average scores 4.67 and 5.13) over [bnʌs, bnɑd] (average scores 2.06, 2.00).
%This hypothesis is not ahdered to by early research on sequence structure constraints \citep[e.g.,][]{SPR,Chomsky1965,SPE,Stanley1967}. Work in this vein sought to use sequence structure constraints to compress the lexicon as much as possible. This work has a second interest, namely in ``compressing'' the lexicon as much as possible, 
%motivations for the development of a theory of static co-occurrence restrictions as developed by \citet{SPR}, \citet{Stanley1967} and others is to account for speakers' abilities to distinguish between accidental and structural gaps. \citet{Chomsky1965} propose that a theory of phonology must account for speakers' abilities to distinguish between accidental and structural gaps in their language, citing a contrast between \emph{blick}, which is judged to be a possible but unattested word of English, and \emph{bnick}, which is judged to be impossible. \emph{SPE} (p.~3810f.) places this contrast in a system of a system of sequence structure rules that also are used to eliminate lexical redundancy. In the phonological theory of the era, there is no process that could generate the \emph{blick} $\sim$ \emph{bnick} contrast or numerous other lexical regularities 
%Generative phonological theory at the time of \emph{SPE} lacked the representational vocabulary to directly distinguish between word-initial /bn/, which is disallowed, and /bn/ as a licit syllable contact cluster (e.g., \emph{o}[b.n]\emph{oxious}), a point first made by \citep{Hooper1973}. It is not the introduction of prosodic primitives into generative phonology that explains the contrast between \emph{blick} and \emph{bnick}, though, but rather the principle that surface forms must be syllabified \citep[e.g.,][63f.]{Kiparsky1982b} which \emph{bnick} fails. If, as is standardly assumed, phonology repairs unsyllabifiable outputs \citep[e.g.][]{Ito1989a,Noske1992}, then \emph{bnick} is an impossible output, doomed to be modified (perhaps to \emph{nick}; \citealt[][19f.]{Wolf2009}), and /bnɪk/ is an impossible UR by Stampean occultation. 
%This sequence of arguments has already been critiqued in Chapter \ref{1}, where it was shown that prosodic licensing and Stampean occultation places this contrast in core phonology, and that the principles by which speakers recognize impossible words must be stated on surface representations, not underlying representations. 
%Further, \citet[][528f.]{Halle1975} rejects his own principle that lexical entries should be free of all redundancies (see \citealt[][201]{Reiss2003a} and \citealt{Vaux2003} for further discussion). This rejected principle of a redundancy-free lexicon is one of the two principles that motivates Morpheme Structure Constraints; the other is the subject of this and the following chapter.

%\citet{Berent2007a}
%\citet{Berent2007b}
%\citet{Berent2008a}
% 598: Highly marked inputs are repaired in perception to abide by the grammatical restrictions of the language
% Exps 1-2: syllable count task (misperception due to linguistic experience; "disyllables were more likely to be perceived as monosyllables if their monosyllabic counterparts had a less marked cluster" in both Russian and English, not what OT predicts)
% Exps 3-4: AB task
% Exps 5-6: priming effect (non-epenthetic and epenethetic things represented the same)
% Berent and Lennertz 2007: (640: If the high rate of monosyllabic errors to unmarked onsets is only due to phonetic failures to encode the input, then it is puzzling why the same onsets also yield the highest rate of accurate response.)
% Berent et al. 2008a:
%\citet{Peperkamp2007}
% Peperkamp 2007: during one stage of processing phonology is undone with help from the lexicon...it only takes licit inputs and is irrelevant (p. 633)
%\citet{Dupoux1999}
% Dupoux et al. 1999: Japanese speakers still can't distinguish between ebzo and ebuzo in an ABX task using the SAME a or b.
%Elliot Moreton (p.c.)
%Once additional source of evidence on root (dis)harmony is inconclusive. There is a small class of bisyllabic words in which the second vowel, always [$+$\textsc{High}, $-$\textsc{Back}], alternates with zero. 

%\ex High-vowel/zero alternations \citep[][243]{Clements1982}: \\
%\begin{tabular}{l l l l}
%   & nom.sg. & gen.sg. \\
%a. & fikir   & fikri  & `idea' \\
%   & hüküm   & hükmün & `judgement' \\
%%  & filim   & filmi & `film' & \citep[][178]{Inkelas2001} \\
%b. & vakit   & vaktin & `time' \\
%   & rahim   & rahmin & `womb' \\
%\end{tabular} \xe
%
%\noindent
%It is possible that \textsc{Backness Harmony} might produce a fluctuating \emph{ı} after root \emph{a}, but this does not obtain (\lastx b). However, this might simply indicate that the fluctuating vowel is epenthetic and that harmony applies before epenthesis (see \citealt{Clements1982} for both sides of this argument), making it less than a counterexample. 
%duplication problem: \citep{Kisseberth1970b} (see also \citealp[][401]{Stanley1967} and \emph{SPE}:382)
%(e.g., \emph{deve} `camel' vs. \emph{deva} `medicine', \emph{sene} `year' vs. \emph{sena} `praise').

%This is not to imply uncritical acceptance of \citeauthor{Becker2011}'s account. \citeauthor{Becker2011} place this generalization ``in the grammar'' to account for their claim that ``naturalness'', which they equate with Universal Grammar, constrains the types of generalizations speakers extract in this task.
% 
%It is difficult to evaluate their claim in the absense of a theory of phonotactic naturalness, something which has consisteny eluded formalization for decades. Even the informal definition used by \citeauthor{Becker2011} is flawed. \citeauthor{Becker2011} label generalizations unnatural if they lack phonetic precursors, or if they are not unattested---no description of the methodology used to determine attestedness is given. These two varieties of evidence are not fully independent, however: extragrammatical \emph{channel bias} effects are thought to be a first-order predictor of attestation \citep{Blevins2004,Moreton2008}. In light of extensive evidence for generalizations which are phonetically unnatural \citep[e.g.,]{Anderson1981,Bach1972,Buckley2000a}, even as revealed in nonce word studies \citep[e.g.,][]{Hayes2009}, the fact that the generalizations extracted in the \citet{Becker2011} study are phonetically natural is either accident or conundrum. It is certainly not something that can be explained by the tacit theory of possible gneeralizations the authors adhere to.

%Whereas the classic analysis of this phenomenon, by \citet{Inkelas1997}, does not make use of lexical exceptionality, \citeauthor{Becker2011}'s analysis of final aspirated stops and affricates as the underlying form requires that _every_ lexical entry with a final stop or affricate be treated as exceptional according to some phonological constraint. This projection of a lexical contrast into the grammar has another potential flaw: the constraints used to produce the voiced variants in, e.g., the dative, are not surface true. Consider the constraint *VpV used to induce alternations like \emph{kap}-\emph{kabı} `coat'. Ignoring the case of non-alternating root-final \emph{p} (e.g., \emph{ip}-\emph{ipi} `rope'), this is still not a surface-true generalization: \emph{p} occurs freely in intervocalic position: \emph{ahtapot} `octopus', \emph{köpük} `bubble', \emph{öpücük} `kiss'. More serious is the presence of root-internal intervocalic \emph{p} in roots which simultaneously exhibit the root-final \emph{p}-\emph{b} alternation: \emph{supap}-\emph{supabı} `valve', \emph{hipermetrop}-\emph{hipermetrobu} `far-sightedness'. One prediction shared by virtually all theories of lexical exceptionality is that a lexical entry cannot simultaneously be a target for, and exception to, a phonological generalization, but this precisely what is needed to account for the behavior of \emph{supap}. 
% \citet{inkelas1997}
%küp-kübü `cube'
%kasap-kasabı `butcher'
%Further, the empirical status of the core instances of NDEB have recently been quite stridently disputed \citep{InkelasInPress}. There is some reason to suspect that NDEB is a symptom of interface restrictions on phonological application, rather than a diagnosis itself.
