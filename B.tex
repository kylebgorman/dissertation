\chapter{Syllabification and phonologization of CELEX transcriptions}

CELEX includes broad, syllabified Received Pronunciation transcriptions of individual wordforms. However, a casual inspection of the data reveals the unsystematic nature of these syllabifications. Any two words sharing a nucleus-medial cluster-nucleus sequence should be given the same parse, since syllabification is universally non-contrastive. Yet many putative syllabification contrasts can be found in CELEX; for instance, compare the parses given to the clusters in \emph{chemistry} [ˈkɛ.mɪ.stɹɪ] and \emph{ministry} [ˈmɪ.nɪs.tɹɪ].\footnote{Note that word-final \emph{y} is generally lax [ɪ] in Received Pronunciation \citep[][II.294]{AOE}.}

A systematic syllabification requires an automated procedure for delimiting medial clusters and parsing them into coda and onset. Since this procedure is applied only to medial clusters, this study remains agnostic on several contentious issues concerning syllabification of English; for instance, there is no need to address the status of so-called ``ambisyllabic'' consonants, since this does not occur with medial clusters, or morphological effects on syllabification, since no complex words are included. The technique used here is a variation on the theme of onset maximization (e.g., \citealp[42f.]{Kahn1976}, \citealp{Kurylowicz1948}, \citealp[75]{Pulgram1970}, \citealp[][358f.]{Selkirk1982b}) which favors parses of word-medial clusters in which as much of the cluster as possible is assigned to the onset. Initial onsets are used to define what is a ``possible'' medial onset \citep[though cf.][36]{Fischer-Jorgensen1952}. Medial clusters in words like \emph{neu}[.tɹ]\emph{on} or \emph{bi}[.stɹ]\emph{o} are found in word-initial position (e.g., [tɹ]\emph{ansit}, [stɹ]\emph{ike}), so onset maximization assigns the entire medial cluster to the onset, leaving the medial coda empty. In contrast, the cluster in \emph{mi}[n.stɹ]\emph{el} is not found word-initially; the maximal onset here is [stɹ], and the residual [n] is assigned to the coda.

\paragraph{Stressed lax vowels} When a medial consonant cluster is preceded by a stressed lax vowel, as in words like \emph{wh}[ɪs.p]\emph{er}, \emph{v}[ɛs.t]\emph{ige}, or \emph{m}[ʌs.k]\emph{et}, the first consonant of the cluster checks the lax vowel \citep[e.g.,][3]{Hammond1997}. \citet[][55]{Harris1994} notes that onset maximization produces an incorrect result when, in addition, the medial cluster is a valid onset: in \emph{whisper}, \emph{vestige}, and \emph{musket}, onset maximization incorrectly assigns [sp, st, sk] to the medial onset, leaving the lax vowel unchecked. Consequently, the first consonant of a medial cluster is assigned to the coda of a preceding syllable before a stressed lax vowel \citep[cf.][48]{Pulgram1970}, a minimal modification of the onset maximization parse.

\paragraph{Affricates} If the affricates [tʃ, dʒ] are treated as stop-fricative sequences \citep[e.g.,][]{Hualde1988,Lombardi1990}, the previous principle will incorrectly assign the two segments of the affricate to separate syllables before stressed lax vowels (e.g., *\emph{ra}[t.ʃ]\emph{et} or \emph{a}[d.ʒ]\emph{ile}) unless some ad hoc constraint \citep[e.g.,][]{Wells1990} is in place. A more traditional analysis of affricates, in which they are simple segments distinguished from pure stops by stridency \citep[24]{Jakobson1961} or delayed release (\emph{SPE}:321f.) is assumed here, following \citeauthor{Pierrehumbert1994}. This assumption can be motivated by the tendency of affricates to pattern with individual segments in phonotactic generalizations. For instance, Classical Nahua allows the affricate series [ts, tʃ, tɬ] in onsets, but bans all other onset clusters \citep[9]{Launey2011}.
%\citet[86]{Butskhrikidze2002} notes a similar generalization in the consonant clusters of Georgian.

\paragraph{The velar nasal} According to some \citep[e.g.,][]{Sapir1925}, [ŋ] in English is a phoneme in its own right, but others (e.g., \emph{SPE}:85, \citealp{Borowsky1986}:65f.) hold that it is simply the allophone of /n/ before underlying /k, ɡ/ (with later deletion of /ɡ/ in some contexts). The total absence of onset [ŋ], where it cannot be followed by a dorsal consonant which conditions the velar allophone, is strong evidence that [ŋ] is a pure allophone of /n/, the analysis assumed here.

\paragraph{[j] onglides} While some have argued that the front onglide in words such as \emph{val}[j]\emph{ue} is inserted by rule (\emph{SPE}:196, \citealp[][89]{Halle1985a}, \citealp[][217]{McMahon1990}), the very presence or absence of the glide is contrastive (e.g., \emph{coo}/\emph{coup} \alt{} \emph{queue}, \emph{booty} \alt{} \emph{beauty}) indicating that it is present in underlying representation (\citealp{Anderson1988b}, \citealp[278]{Borowsky1986}). The front onglide is further assumed to be assigned to the nucleus, except when the onset would otherwise be null (e.g., \emph{jun}[j]\emph{or}). There is considerable evidence for this assumption. When [j] is a simplex onset, it may be followed by any vowel \citep[][276]{Borowsky1986}, but when [j] is immediately preceded by an onset consonant (e.g., [bj]\emph{ugle}), the following vowel is always [u], suggesting that the glide is nuclear (\citealp{Davis1995}, \citealp[][61f.]{Harris1994}, \citealp[][232]{Hayes1980}). \citet[][42]{Clements1983} note that /m, v/ do not appear in onset clusters, though they may be followed by [ju] in words like [mj]\emph{use} or [vj]\emph{iew}, also suggesting the nuclear affiliation of the glide. There is also external support for this analysis: the [ju] in words like \emph{spew} may pattern together in Pig Latin \citep{Davis1995,Idsardi2005}, \emph{shm}-reduplication \citep{Nevins2003}, and speech errors (e.g., [kju]\emph{mor} [h]\emph{omponent}, intended [hju]\emph{mor component}; \citealp[130]{Shattuck-Hufnagel1986}).

\paragraph{[w] onglides} The phonotactic properties of the back onglide [w] are the opposite of the front onglide. Whereas the front onglide shows only limited selectivity for preceding tautosyllabic consonants \citep{Davis1995,Kaye1996}, the back onglide [w] is rarely preceded by tautosyllabic consonants other than [k] (e.g., \emph{tran}[kw]\emph{il}). Unlike the front glide, syllable-initial [kw] may be followed by nearly any vowel \citep[161]{Davis1995}. Unlike [juː], onglide [w] followed by a vowel does not pattern together in Pig Latin \citep[166]{Davis1995}. Taken together, these facts indicate that the back onglide is assigned to the onset.

\paragraph{Post-vocalic \emph{r}} Word-medial post-vocalic \emph{r} has been lost in Received Pronunciation, the standard accent of the CELEX transcriptions. Even in \emph{r}-ful dialects, though, there is reason to believe that post-vocalic \emph{r} is fact nuclear, and thus irrelevant to syllable contact clusters. Many vowel contrasts are suspended before \emph{r} (e.g., \citealt[269f.]{Fudge1969}, \citealt[][255]{Harris1994}): compare American English \emph{fern}, \emph{fir}, \emph{fur} to \emph{pet}, \emph{pit}, \emph{putt}. Post-vocalic \emph{r} patterns with vowels and not other consonants for several phonological processes: it is the only consonant which does not block variable glottalization of following /t/ in British dialects \citep[258]{Harris1994}.

\begin{example}[/t/-\textsc{Glottalization} in British English]
\begin{tabular}{l l l l@{} l l l}
a. & {des}[ɚt]    & \alt{} &   & {des}[ɚʔ]    \\
   & {c}[ɚt]{ain} & \alt{} &   & {c}[ɚʔ]{ain} \\
b. & {fi}[st]     & \alt{} & * & {fi}[sʔ]     \\
   & {mi}[st]{er} & \alt{} & * & {mi}[sʔ]{er} \\
\end{tabular}
\end{example}

\noindent Post-vocalic \emph{r} is the only consonant which does not trigger variable deletion of a following /t, d/ in American dialects \citep[8]{Guy1980}.

\begin{example}[/t, d/-\textsc{Deletion} in American English]
\begin{tabular}{l l l l@{} l l l}
a. & {be}[lt] & \alt{} &   & {be}[l] \\
   & {me}[nd] & \alt{} &   & {me}[n] \\
b. & {sh}[ɚt] & \alt{} & * & {sh}[ɚ] \\
   & {c}[ɚd]  & \alt{} & * & {c}[ɚ]  \\
\end{tabular}
\end{example}
